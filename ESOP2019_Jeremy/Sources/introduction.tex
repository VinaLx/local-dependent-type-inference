
\section{Introduction}
\label{sec:introduction}

Compositionality is a desirable property in programming
designs. Broadly defined, it is the principle that a
system should be built by composing smaller subsystems. For instance,
in the area of programming languages, compositionality is
a key aspect of \emph{denotational semantics}~\cite{scott1971toward, scott1970outline}, where
the denotation of a program is constructed from the denotations of its parts.
% For example, the semantics for a language of simple arithmetic expressions
% is defined as:
% 
% \[\begin{array}{lcl}
% \llbracket n \rrbracket_{E} & = & n \\
% \llbracket e_1 + e_2 \rrbracket_{E} & = & \llbracket e_1 \rrbracket_E + \llbracket  e_2 \rrbracket_E \\
% \end{array}\]
% 
% \bruno{Replace E by fancier symbol?}
% Here there are two forms of expressions: numeric literals and
% additions. The semantics of a numeric literal is just the numeric
% value denoted by that literal. The semantics of addition is the
% addition of the values denoted by the two subexpressions.
Compositional definitions have many benefits.
One is ease of reasoning: since compositional
definitions are recursively defined over smaller elements they
can typically be reasoned about using induction. Another benefit
is that compositional definitions are easy to extend,
without modifying previous definitions.
% For example, if we also wanted to support multiplication,
% we could simply define an extra case:
% 
% \[\begin{array}{lcl}
% \llbracket e_1 * e_2 \rrbracket_E & = & \llbracket e_1 \rrbracket_E * \llbracket  e_2 \rrbracket_E \\
% \end{array}\]

Programming techniques that support compositional
definitions include:
\emph{shallow embeddings} of
Domain Specific Languages (DSLs)~\cite{DBLP:conf/icfp/GibbonsW14}, \emph{finally
  tagless}~\cite{CARETTE_2009}, \emph{polymorphic embeddings}~\cite{hofer_polymorphic_2008} or
\emph{object algebras}~\cite{oliveira2012extensibility}. These techniques allow us to create
compositional definitions, which are easy to extend without
modifications. Moreover, when modeling semantics, both finally tagless and object algebras
support \emph{multiple interpretations} (or denotations) of
syntax, thus offering a solution to the well-known \emph{Expression Problem}~\cite{wadler1998expression}.
Because of these benefits these techniques have become
popular both in the functional and object-oriented
programming communities.

However, programming languages often only support simple compositional designs
well, while support for more sophisticated compositional designs is lacking.
For instance, once we have multiple interpretations of syntax, we may wish to
compose them. Particularly useful is a \emph{merge} combinator,
which composes two interpretations~\cite{oliveira2012extensibility,
oliveira2013feature, rendel14attributes} to form a new interpretation that,
when executed, returns the results of both interpretations. 

% For example, consider another pretty printing interpretation (or
% semantics) $\llbracket \cdot \rrbracket_P$ for arithmetic expressions, which
% returns the string that denotes the concrete syntax of the
% expression. Using merge we can compose the two interpretations to
% obtain a new interpretation that executes both printing and evaluation:
% \jeremy{Explain what is $E\,\&\,P$?}
% 
% \[\begin{array}{lcl}
% \llbracket \cdot \rrbracket_E \otimes \llbracket \cdot \rrbracket_P & = & \llbracket \cdot \rrbracket_{E\,\&\,P} \\
% \end{array}\]

The merge combinator can be manually defined in existing programming languages,
and be used in combination with techniques such as finally tagless or object
algebras. Moreover variants of the merge combinator are useful to
model more complex combinations
of interpretations. A good example are so-called \emph{dependent} interpretations,
where an interpretation does not depend \emph{only} on itself, but also on 
a different interpretation. These definitions with dependencies are quite
common in practice, and, although they are not orthogonal to the interpretation they
depend on, we would like to model them (and also mutually dependent interpretations)
in a modular and compositional style.

% For example consider the following two
% interpretations ($\llbracket \cdot \rrbracket_{\mathsf{Odd}}$ and
% $\llbracket \cdot \rrbracket_{\mathsf{Even}}$) over Peano-style natural numbers:

% \[\begin{array}{lclclcl}
% \llbracket 0 \rrbracket_{\mathsf{Even}}  & = & \mathsf{True} & ~~~~~~~~~~~~~~~~~~~~ & \llbracket 0 \rrbracket_{\mathsf{Odd}} & = & \mathsf{False} \\
% \llbracket S~e \rrbracket_{\mathsf{Even}} & = & \llbracket e \rrbracket_{\mathsf{Odd}} & ~~ & \llbracket S~e \rrbracket_{\mathsf{Odd}} & = & \llbracket e \rrbracket_{\mathsf{Even}}\\
% \end{array}\]

% \emph{Are these interpretations compositional or not?} Under
% a strict definition of compositionality they are not because
% the interpretation of the parts does not depend \emph{only} on the
% interpretation being defined. Instead both interpretations also depend
% on the other interpretation of the parts. In general,
% definitions with dependencies are quite common in practice.
% In this paper we consider these
% interpretations compositional, and we
% would like to model such dependent (or even mutually dependent)
% interpretations in a modular and compositional style.

Defining the merge combinator in existing
programming languages is verbose and cumbersome, requiring code for every
new kind of syntax. Yet, that code is essentially mechanical and ought to be
automated. 
While using advanced meta-programming techniques enables automating
the merge combinator to a large extent in existing programming
languages~\cite{oliveira2013feature, rendel14attributes}, those techniques have
several problems: error messages can be problematic, type-unsafe reflection
is needed in some approaches~\cite{oliveira2013feature} and
advanced type-level features are required in others~\cite{rendel14attributes}.
An alternative to the merge combinator that supports modular multiple
interpretations and works in OO languages with
support for some form of multiple inheritance and covariant
type-refinement of fields has also been recently
proposed~\cite{zhang19shallow}. 
While this approach is relatively simple, it still
requires a lot of manual boilerplate code for composition of interpretations.

This paper presents a calculus and polymorphic type system with
\emph{(disjoint) intersection types}~\cite{oliveira2016disjoint},
called \fnamee. \fnamee
supports our broader notion of compositional designs, and enables
the development of highly modular and reusable programs. \fnamee
has a built-in merge operator and a powerful subtyping relation that
are used to automate the composition of multiple (possibly dependent)
interpretations. In \fnamee subtyping is coercive and enables the
automatic generation of coercions in a \emph{type-directed} fashion. 
This process is similar to that of other type-directed code generation mechanisms
such as 
\emph{type classes}~\cite{Wadler89typeclasses}, which eliminate 
boilerplate code associated to the \emph{dictionary translation}~\cite{Wadler89typeclasses}.

\fnamee continues a line of
research on disjoint intersection types.
 Previous work on
\emph{disjoint polymorphism} (the \fname calculus)~\cite{alpuimdisjoint} studied the
combination of parametric polymorphism and disjoint intersection
types, but its subtyping relation does not support
BCD-style distributivity rules~\cite{Barendregt_1983} and the type system
also prevents unrestricted intersections~\cite{dunfield2014elaborating}. More recently the \name
calculus (or \namee)~\cite{bi_et_al:LIPIcs:2018:9227} introduced a system with \emph{disjoint
  intersection types} and BCD-style distributivity rules, but did not
account for parametric polymorphism. \fnamee is unique in that it
combines all three features in a single calculus:
\emph{disjoint intersection types} and a \emph{merge operator};
\emph{parametric (disjoint) polymorphism}; and a BCD-style subtyping
relation with \emph{distributivity rules}. The three features together
allow us to improve upon the finally tagless and object
algebra approaches and support advanced compositional designs.
Moreover previous work on disjoint intersection types has shown 
various other applications that are also possible in \fnamee, including: \emph{first-class
  traits} and \emph{dynamic inheritance}~\cite{bi_et_al:LIPIcs:2018:9214}, \emph{extensible records} and \emph{dynamic
  mixins}~\cite{alpuimdisjoint}, and \emph{nested composition} and \emph{family polymorphism}~\cite{bi_et_al:LIPIcs:2018:9227}. 


Unfortunately the combination of the three features has non-trivial
complications. The main technical challenge (like for most other
calculi with disjoint intersection types) is the proof of coherence
for \fnamee. Because of the presence of BCD-style distributivity
rules, our coherence proof is based on the recent approach employed in
\namee~\cite{bi_et_al:LIPIcs:2018:9227}, which uses a
\emph{heterogeneous} logical relation called \emph{canonicity}. To account for polymorphism,
which \namee's canonicity does not support, we originally wanted
to incorporate the relevant parts of System~F's logical relation~\cite{reynolds1983types}.
However, due to a mismatch between the two relations, this did not work. The
parametricity relation has been carefully set up with a delayed type
substitution to avoid ill-foundedness due to its impredicative polymorphism.
Unfortunately, canonicity is a heterogeneous relation and needs to account for
cases that cannot be expressed with the delayed substitution setup of the
homogeneous parametricity relation. Therefore, to handle those heterogeneous
cases, we resorted to immediate substitutions and 
% restricted \fnamee to
\emph{predicative instantiations}.
%other
%measures to avoid the ill-foundedness of impredicative instantiation.
%We have settled on restricting \fnamee to \emph{predicative polymorphism} to
%keep the coherence proof manageable. 
We do not believe that predicativity is a severe restriction in practice, since many source
languages (e.g., those based on the Hindley-Milner type system like Haskell and
OCaml) are themselves predicative and do not require the full generality of an
impredicative core language. Should impredicative instantiation be required,
we expect that step-indexing~\cite{ahmed2006step} can be used to recover well-foundedness, though
at the cost of a much more complicated coherence proof.

The formalization and metatheory of \fnamee are a significant advance over that of
\fname. Besides the support for distributive subtyping, \fnamee removes 
several restrictions imposed by the syntactic coherence
proof in \fname. In particular \fnamee supports unrestricted
intersections, which are forbidden in \fname. Unrestricted
intersections enable, for example, encoding certain forms of 
bounded quantification~\cite{pierce1991programming}.
Moreover the new proof method is more robust
with respect to language extensions. For instance, \fnamee supports the bottom
type without significant complications in the proofs, while it was a challenging
open problem in \fname.
A final interesting aspect is that \fnamee's type-checking is decidable. In the
design space of languages with polymorphism and subtyping, similar mechanisms
have been known to lead to undecidability. Pierce's seminal paper
``\emph{Bounded quantification is undecidable}''~\cite{pierce1994bounded} shows
that the contravariant subtyping rule for bounded quantification in
\fsub leads to undecidability of subtyping.  In \fnamee the
contravariant rule for disjoint quantification retains decidability. 
Since with unrestricted intersections \fnamee can express several
use cases of bounded quantification, \fnamee could be an interesting and
decidable alternative to \fsub.

\begin{comment}
Besides coherence, we show
several other important meta-theoretical results, such as type-safety, 
sound and complete algorithmic subtyping, and
decidability of the type system. Remarkably, unlike 
\fsub's \emph{bounded polymorphism}, disjoint polymorphism
in \fnamee supports decidable type-checking.
\end{comment}

In summary the contributions of this paper are:
\begin{itemize}

\item {\bf The \fnamee calculus,} which is the first calculus to combine 
disjoint intersection types, BCD-style distributive subtyping and 
disjoint polymorphism. We show several meta-theoretical results, such as \emph{type-safety}, \emph{sound and complete algorithmic subtyping},
\emph{coherence} and \emph{decidability} of the type system.
\fnamee includes the \emph{bottom type}, which was considered to be a
significant challenge in previous work on disjoint polymorphism~\cite{alpuimdisjoint}.

\item {\bf An extension of the canonicity relation with polymorphism,}
  which enables the proof of coherence of \fnamee. We show that the ideas of
  System F's \emph{parametricity} cannot be ported to
  \fnamee. To overcome the problem we use a technique based on
  immediate substitutions and a predicativity restriction.

% \item {\bf Disjoint intersection types in the presence of bottom:}
%   Our calculus includes the bottom type, which was considered to be a
% significant challenge in previous work on disjoint polymorphism~\cite{alpuimdisjoint}.

\item {\bf Improved compositional designs:} We show that \fnamee's combination of features
enables improved
compositional programming designs and supports automated composition
of interpretations in programming techniques like object algebras and
finally tagless.

\item {\bf Implementation and proofs:} All of the metatheory
  of this paper, except some manual proofs of decidability, has been
  mechanically formalized in Coq. Furthermore, \fnamee is
  implemented and all code presented in the paper is available. The
  implementation, Coq proofs and extended version with appendices can be found in
  \url{https://github.com/bixuanzju/ESOP2019-artifact}.

\end{itemize}

% \bruno{
% Still need to figure out how to integrate row types in the intro story
% Furthermore, we provide a detailed
% comparison between \emph{distributive disjoint polymorphism} and
% \emph{row types}.
% }

% Compositionality is a desirable property in programming
% designs. Broadly defined, compositionality is the principle that a
% system should be built by composing smaller subsystems.
% In the area of programming languages compositionality is
% a key aspect of \emph{denotational semantics}~\cite{scott1971toward, scott1970outline}, where
% the denotation of a program is constructed from denotations of its parts.
% For example, the semantics for a language of simple arithmetic expressions
% is defined as:
% 
% \[\begin{array}{lcl}
% \llbracket n \rrbracket_{E} & = & n \\
% \llbracket e_1 + e_2 \rrbracket_{E} & = & \llbracket e_1 \rrbracket_E + \llbracket  e_2 \rrbracket_E \\
% \end{array}\]
% 
% \bruno{Replace E by fancier symbol?}
% Here there are two forms of expressions: numeric literals and
% additions. The semantics of a numeric literal is just the numeric
% value denoted by that literal. The semantics of addition is the
% addition of the values denoted by the two subexpressions.
% Compositional definitions have many benefits.
% One is ease of reasoning: since compositional
% definitions are recursively defined over smaller elements they
% can typically be reasoned about using induction. Another benefit
% of compositional definitions is that they are easy to extend,
% without modifying previous definitions.
% For example, if we also wanted to support multiplication,
% we could simply define an extra case:
% 
% \[\begin{array}{lcl}
% \llbracket e_1 * e_2 \rrbracket_E & = & \llbracket e_1 \rrbracket_E * \llbracket  e_2 \rrbracket_E \\
% \end{array}\]
% 
% Programming techniques that support compositional
% definitions include:
% \emph{shallow embeddings} of
% Domain Specific Languages (DSLs)~\cite{DBLP:conf/icfp/GibbonsW14}, \emph{finally
%   tagless}~\cite{CARETTE_2009}, \emph{polymorphic embeddings}~\cite{} or
% \emph{object algebras}~\cite{oliveira2012extensibility}. All those techniques allow us to easily create
% compositional definitions, which are easy to extend without
% modifications. Moreover both finally tagless and object algebras
% support \emph{multiple interpretations} (or denotations) of
% the syntax, thus offering a solution to the infamous \emph{Expression Problem}~\cite{wadler1998expression}.
% Because of these benefits they have become
% popular both in the functional and object-oriented
% programming communities.
% 
% However, programming languages often only support simple
% compositional designs well, while language support for more sophisticated
% compositional designs is lacking. Once we have multiple
% interpretations of syntax, then we may wish to compose those
% interpretations. In particular, when multiple interpretations exist, a useful operation
% is a \emph{merge} combinator ($\otimes$) that composes two
% interpretations~\cite{oliveira2012extensibility, oliveira2013feature, rendel14attributes}, forming a
% new interpretation that, when executed, returns the results of both
% interpretations. For example, consider another pretty printing interpretation (or
% semantics) $\llbracket \cdot \rrbracket_P$ for arithmetic expressions, which
% returns the string that denotes the concrete syntax of the
% expression. Using merge we can compose the two interpretations to
% obtain a new interpretation that executes both printing and evaluation:
% \jeremy{Explain what is $E\,\&\,P$?}
% 
% \[\begin{array}{lcl}
% \llbracket \cdot \rrbracket_E \otimes \llbracket \cdot \rrbracket_P & = & \llbracket \cdot \rrbracket_{E\,\&\,P} \\
% \end{array}\]
% 
% Such merge combinator can be manually defined in existing programming 
% The merge combinator can be manually defined in existing programming
% languages, and be used in combination with techniques such as finally
% tagless or object algebras. Furthermore variants of the
% merge combinator can help express more complex combinations of multiple
% interpretations. For example consider the following two
% interpretations ($\llbracket \cdot \rrbracket_{\mathsf{Odd}}$ and
% $\llbracket \cdot \rrbracket_{\mathsf{Even}}$) over Peano-style natural numbers:
% 
% \[\begin{array}{lclclcl}
% \llbracket 0 \rrbracket_{\mathsf{Even}}  & = & \mathsf{True} & ~~~~~~~~~~~~~~~~~~~~ & \llbracket 0 \rrbracket_{\mathsf{Odd}} & = & \mathsf{False} \\
% \llbracket S~e \rrbracket_{\mathsf{Even}} & = & \llbracket e \rrbracket_{\mathsf{Odd}} & ~~ & \llbracket S~e \rrbracket_{\mathsf{Odd}} & = & \llbracket e \rrbracket_{\mathsf{Even}}\\
% \end{array}\]
% 
% \emph{Are these interpretations compositional or not?} Under
% a strict definition of compositionality they are not because
% the interpretation of the parts does not depend \emph{only} on the
% interpretation being defined. Instead both interpretations also depend
% on the other interpretation of the parts. In general,
% definitions with dependencies are quite common in practice.
% In this paper we consider these
% interpretations compositional, and we
% would like to model such dependent (or even mutually dependent)
% interpretations in a modular and compositional style.
% 
% However defining the merge combinator in existing programming
% languages is verbose and cumbersome, and requires code for every new
% kind of syntax. Yet, that code is essentially mechanical and
% ought to be automated. While using advanced meta-programming
% techniques enables automating the merge combinator to a large extent
% in existing programming languages~\cite{oliveira2013feature, rendel14attributes}, those techniques have
% several problems. For example, error messages can be problematic, some
% techniques rely on type-unsafe reflection, while other techniques
% require highly advanced type-level features.
% 
% This paper presents a calculus and polymorphic type system with
% \emph{(disjoint) intersection types}~\cite{oliveira2016disjoint}, called \fnamee, that
% supports our broader notion of compositional designs, and enables
% the development of highly modular and reusable programs. \fnamee
% has a built-in merge operator and a powerful subtyping relation that
% are used to automate the composition of multiple interpretations
% (including dependent interpretations). \fnamee continues a line of
% research on disjoint intersection types. Previous work on
% \emph{disjoint polymorphism} (the \fname calculus) studied the
% combination between parametric polymorphism and disjoint intersection
% types, but the subtyping relation did not support
% BCD-style distributivity rules~\cite{Barendregt_1983}. More recently the \name
% calculus (or \namee) studied a system with \emph{disjoint
%   intersection types} and BCD-style distributivity rules, but did not
% account for parametric polymorphism. \fnamee is unique in that it
% allows the combination of three useful features in a single calculus:
% \emph{disjoint intersection types} and a \emph{merge operator};
% \emph{parametric (disjoint) polymorphism}; and a BCD-style subtyping
% relation with \emph{distributivity rules}. All three features are
% necessary to use improved versions of finally tagless or object
% algebras that support improved compositional designs.
% 
% Unfortunatelly the combination of the three features has non-trivial
% complications. The main technical challenge (as often is the case for
% calculi with disjoint intersection types) is the proof of coherence
% for \fnamee. Because of the presence BCD-style distributivity
% rules, the proof of coherence is based on the approach using a
% \emph{heterogeneous} logical relation employed in
% \namee~\cite{bi_et_al:LIPIcs:2018:9227}. However the logical relation in
% \namee, which we call here \emph{canonicity}, does not
% account for polymorphism. To account for polymorphism we originally
% expected to simply borrow ideas from \emph{parametricity}~\cite{reynolds1983types} in
% System F~\cite{reynolds1974towards} and adapt them to fit with the canonicity relation.
% However, this did not work. The problem is partly due to the fact that
% canonicity (unlike parametricity) is an heterogenous relation and
% needs to account for heterogeneous cases that are not considered in an
% homogeneous relation such as parametricity. Those heterogeneous cases, combined
% with \emph{impredicative polymorphism}, resulted in an ill-founded logical
% relation. Fortunatelly it turns out that
% restricting the calculus to \emph{predicative polymorphism} and using
% an approach based on substitutions is
% sufficient to recover a well-founded canonicity relation.
% Therefore we
% adopted this approach in \fnamee.
% We do not view
% the predicativity restriction as being very severe in practice, since many
% practical languages have such restriction as well. For example languages based
% on Hindley-Milner style type systems (such as Haskell, OCaml or ML)
% \ningning{it's hard to say this is true. When we say Hindley-Milner type system,
%   or Haskell, we are referring to the source language. However, the core
%   language for, for example Haskell, which is System FC, is impredicative.
%   \fnamee is more close to a core language (which usually has explicit type
%   abstractions/applications). In this sense it's unfair to compare it with other
%   source languages.} all use predicative polymorphism. Furthermore with the
% predicativity restriction, the canonicity relation and corresponding proofs
% remain relatively simple and do not require emplying more complex approaches
% such as \emph{step-indexed logical relations}. \ningning{we should emphasize
%   that predicativity is not a restriction, rather it's choice we made in order
%   to prove coherence in Coq. Step-indexed logical relation might work for
%   impredicativity; it's just we don't know.}
% 
% In summary the contributions of this paper are:
% 
% \begin{itemize}
% 
% \item {\bf The \fnamee calculus,} which integrates disjoint intersection types,
%     distributivity and disjoint polymorphism. \fnamee
%     is the first calculus puts all three features together. The
%     combination is non-trivial, expecially with respect to the
%     coherence proof.
% 
% \bruno{improve text}
% \item {\bf The canonicity logical relation,} which enables the proof
%     of coherence of \fnamee. We show that the ideas of
%   System F's \emph{parametricity} cannot be ported to
%   \fnamee. To overcome the problem we develop a canonicity
%   relation that enables a proof of coherence.
% 
% \item {\bf Disjoint intersection types in the presence of bottom:}
%   Our calculus includes a bottom type, which was considered to be a
% significant challenge in previous work.
% 
% \item {\bf Improved compositional designs:} We show how \fnamee has all the
% features that enable improved
% compositional programming designs and support automated composition
% of interpretations in programming techniques like object algebras and
% finally tagless.
% 
% \item {\bf Implementation and proofs:} All proofs
% (including type-safety, coherence and decidability of the type system)
% are proved in the Coq theorem prover. Furthermore \sedel \ningning{where comes the name \sedel?} and
% \fnamee are implemented and all code presented in the paper is
% available. The implementation, proofs and examples can be found in:
% 
% \url{MISSING}
% 
% \end{itemize}
% 
% \bruno{
% Still need to figure out how to integrate row types in the intro story
% Furthermore, we provide a detailed
% comparison between \emph{distributive disjoint polymorphism} and
% \emph{row types}.
% }

% Local Variables:
% org-ref-default-bibliography: "../paper.bib"
% End:
