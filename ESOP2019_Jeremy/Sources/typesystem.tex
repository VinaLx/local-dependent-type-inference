
%%%%%%%%%%%%%%%%%%%%%%%%%%%%%%%%%%%%%%%%%%%%%%%%%%%%%%%%%%%%%%%%%%%%%%%%
\section{Semantics of the \fnamee Calculus}
\label{sec:typesystem}
%%%%%%%%%%%%%%%%%%%%%%%%%%%%%%%%%%%%%%%%%%%%%%%%%%%%%%%%%%%%%%%%%%%%%%%%

% \bruno{Regarding text about differences to \fname. I think that maybe
%   we can make a comparision in Related Work, rather than the technical
% section. That will not disrupt the flow of the technical section, and
% will be useful to add weight to the related work section.}

This section gives a formal account of \fnamee, the first typed calculus combining
disjoint polymorphism~\cite{alpuimdisjoint} (and disjoint intersection
types) with BCD
subtyping~\cite{Barendregt_1983}. The main differences to \fname are in the
subtyping, well-formedness and disjointness relations. \fnamee adds
BCD subtyping and unrestricted intersections, and also closes an open problem of
\fname by including the bottom type.
% through appropriate adjustments to the disjointness relation.
% \fnamee is a variant of Levant's
% predicative System F~\cite{leivant1991finitely}. \jeremy{Do we really need to
%   cite this? Leivant's System F has finitely stratified levels} The choice of
% predicativity is due to the coherence proof, which is
% discussed in \cref{sec:coherence:poly}.
The dynamic semantics
of \fnamee is given by elaboration to the target calculus \tnamee---a variant of
System F extended with products and explicit coercions.
% \jeremy{Should we summaries the differences from \fname here? }

% \subsection{Motivation}
% \label{sec:poly:motivation}

% Parametric polymorphism~\cite{reynolds1983types} is a well-beloved (and
% well-studied) programming feature. It enables a single piece of code to be
% reused on data of different types. So it is quite natural and theoretically
% interesting to study combining parametric polymorphism with disjoint
% intersection types, especially how it affects disjointness and coherence. On a
% more pragmatic note, the combination of parametric polymorphism and disjoint
% intersection types also reveals new insights into practical applications.
% Dynamically typed languages (such as JavaScript) usually embrace quite flexible
% mechanisms for class/object composition, e.g., mixin composition where objects
% can be composed at run time, and their structures are not necessarily statically
% known. The use of intersection types in TypeScript is inspired by such flexible
% programming patterns. For example, an important use of intersection types in
% TypeScript is the following function for mixin composition:
% \begin{lstlisting}[language=JavaScript]
% function extend<T, U>(first: T, second : U) : T & U {...}
% \end{lstlisting}
% which is analogous to our merge operator in that it takes two objects and
% produces an object with the intersection of the types of the argument objects.
% However, the types of the two objects are not known, i.e., they are generic. An
% important point is that, while it is possible to define such function in
% TypeScript (albeit using some low-level (and type-unsafe) features of
% JavaScript), it can also cause, as pointed out by \cite{alpuimdisjoint},
% run-time type errors! Clearly a well-defined meaning for intersection types with
% type variables is needed.


% \paragraph{Disjoint polymorphism.}

% Motivated by the above two points, \cite{alpuimdisjoint} proposed disjoint
% polymorphism, a variant of parametric polymorphism. The main novelty is
% \emph{disjoint (universal) quantification} of the form $[[ \ X ** A . B ]]$.
% Inspired by bounded quantification~\cite{cardelli1994extension} where a type
% variable is constrained by a type bound, disjoint quantification allows type
% variables to be associated with \emph{disjointness constraints}.
% Correspondingly, a term construct of the form $[[ \ X ** A. ee ]]$ is used to
% create values of disjoint quantification. We have seen some examples of disjoint
% polymorphism at work in \cref{bg:sec:disjoint_poly}. With disjointness constraints
% and a built-in merge operator, a type-safe and conflict-free \lstinline{extend}
% function can be naturally defined as follows:
% \begin{lstlisting}
% extend T [U * T] (first : T) (second : U) : T & U = first ,, second;
% \end{lstlisting}
% The disjointness constraint on the type variable \lstinline{U} ensures that no
% conflicts can occur when composing two objects, which is quite similar to
% the trait model~\cite{scharli2003traits} in object-orientated programming.
% We shall devote a whole chapter (\cref{chap:traits}) to further this point.


% \paragraph{Adding BCD subtyping.}

% While \cite{alpuimdisjoint} studied the combination of disjoint intersection
% types and parametric polymorphism, they follow the then-standard approach
% of \cite{oliveira2016disjoint} to ensure coherence, thus excluding the
% possibility of adding BCD subtyping. The combination of BCD subtyping and
% disjoint polymorphism opens doors for more expressiveness. For example, we can
% define the following function
% \begin{lstlisting}
% combine A [B * A] (f : {foo : Int -> A})
%                   (g : {foo : Int -> B}) : {foo : Int -> A & B} = f ,, g;
% \end{lstlisting}
% which ``combines'' two singleton records with parts of types unknown and returns
% another singleton record containing an intersection type. A variant of this
% function plays a fundamental role in defining Object Algebra combinators (cf.
% \cref{chap:case_study}).



\subsection{Syntax and Semantics}

\begin{figure}[t]
  \centering
\begin{tabular}{llll} \toprule
  Types & $[[A]], [[B]], [[C]]$ & $\Coloneqq$ & $[[nat]] \mid [[Top]] \mid [[Bot]] \mid [[A -> B]]  \mid [[A & B]] \mid [[{l : A}]] \mid [[X]] \mid [[\ X ** A . B]] $\\
  % Monotypes & $[[t]]$ & $\Coloneqq$ & $[[nat]] \mid [[Top]] \mid [[Bot]] \mid [[t1 -> t2]]  \mid [[t1 & t2]] \mid [[X]] \mid [[{l : t}]]$\\
  Expressions & $[[ee]]$ & $\Coloneqq$ & $[[x]] \mid [[ii]] \mid [[Top]] \mid [[\x . ee]] \mid [[ee1 ee2]] \mid [[ ee1 ,, ee2 ]] \mid [[ ee : A ]] \mid [[{l = ee}]] \mid [[ ee.l  ]] $ \\
        & & $\mid$ & $  [[\X ** A . ee]] \mid [[ ee A ]] $ \\
  Term contexts & $[[GG]]$ & $\Coloneqq$ &  $[[empty]] \mid [[GG , x : A]] $  \\
  Type contexts & $[[DD]] $ & $\Coloneqq$ & $ [[empty]] \mid [[DD , X ** A]] $  \\ \bottomrule
\end{tabular}
  \caption{Syntax of \fnamee}
  \label{fig:syntax:fi}
\end{figure}


\Cref{fig:syntax:fi} shows the syntax of \fnamee.
% For brevity of the
% meta-theoretic study, we do not consider primitive operations on primitive
% types. They can be easily added to the language, and our prototype
% implementation is indeed equipped with common primitive types and their
% operations.
Metavariables $[[A]], [[B]], [[C]]$ range over types. Types include
standard constructs from prior work~\cite{oliveira2016disjoint, alpuimdisjoint}:
integers $[[nat]]$, the top type $[[Top]]$, arrows $[[A -> B]]$, intersections $[[A & B]]$, single-field record types
$[[{l : A}]]$ and disjoint quantification $[[ \X ** A . B ]]$. One
novelty in \fnamee is the addition of the uninhabited bottom type $[[Bot]]$.
% Monotypes $[[t]]$ are the same, minus the disjoint quantification.
Metavariable $[[ee]]$ ranges over expressions. Expressions are
integer literals $[[ii]]$, the top value $[[Top]]$,
lambda abstractions $[[ \x . ee]]$, applications $[[ee1 ee2]]$, merges
$[[ee1 ,, ee2]]$, annotated terms $[[ ee : A ]]$, single-field records
$[[ {l = ee} ]]$, record projections $[[ ee . l ]]$, type abstractions
$[[ \X ** A . ee ]]$ and type applications $[[ee A]]$.
% We sometimes (ab)use $[[\X . A]]$ to abbreviate $[[ \X ** Top. A  ]]$, similarly for $[[ \ X . ee ]]$.

%To support
%polymorphism, we inherit from \fname two constructs: type abstractions $[[ \X ** A . ee ]]$
%of type $[[ \X ** A . B ]]$, and type applications $[[ee A]]$.

% We use $[[ A [X ~> B]  ]]$ to denote capture-avoiding substitution.

% \paragraph{Contexts.}

% In the traditional formulation of System F, there is a single context for
% tracking both type and term variables. Here we use another style of
% presentation~\cite[chap.~16]{Harper_2016} where contexts are split into
% \emph{term contexts} $[[GG]]$ and \emph{type contexts} $[[DD]]$. The former
% track term variables $[[x]]$ with their types $[[A]]$; and the latter track type
% variables $[[X]]$ with their disjointness constraints $[[A]]$. This formulation
% also turns out to be convenient for the presentation of logical relations in
% \cref{sec:coherence:poly}.

% \begin{figure}[t]
%   \begin{small}
%   \drules[swfte]{$[[ ||- DD]]$}{Well-formedness of type context}{empty, var}
%   \drules[swfe]{$[[DD ||- GG]]$}{Well-formedness of value context}{empty, var}
%   \drules[swft]{$[[DD |- A]]$}{Well-formedness of type}{top, bot, nat, var, arrow, all, and, rcd}
%   \end{small}
%   \centering
%   \caption{Well-formedness of types}
%   \label{fig:well-formedness:fi}
% \end{figure}


% \paragraph{Disjoint polymorphism.}
% \bruno{Will this be redundant with the text about disjoint
%   polymorphism in Section 2? Try to unify the the 2 texts perhaps and
%   write everything in Section 2?}
% Disjoint polymorphism~\cite{alpuimdisjoint} was recently proposed to combine
% (disjoint) intersection types with parametric polymorphism. The main novelty is
% \emph{disjoint quantification} $[[ \ X ** A . B ]]$.
% Inspired by bounded quantification~\cite{cardelli1994extension} where a type
% variable is constrained by a type bound, disjoint quantification associates type
% variables with \emph{disjointness constraints}.
% A term construct $[[ \ X ** A. ee ]]$ is used to
% create values. With disjointness quantification,
% it is possible to write a generic and conflict-free \lstinline{merge} function
% as follows:
% \begin{lstlisting}
% merge T [U * T] (first : T) (second : U) : T & U = first ,, second;
% \end{lstlisting}
% which allows calling \lstinline{merge Int Char 1 'a'} for example, but rejects
% \lstinline{merge Int Int 1 2} due to ambiguity. The type system ensures that no
% conflicts can occur when composing two objects. This is quite similar to the
% trait model~\cite{scharli2003traits} in object-oriented programming.


\paragraph{Well-formedness and unrestricted intersections.}

\renewcommand\ottaltinferrule[4]{
  \inferrule*[narrower=0.8,right=#1,#2]
    {#3}
    {#4}
}


\fnamee's well-formedness judgment of types $[[ DD |- A ]]$ is
standard, and only enforces well-scoping. This
is one of the key differences from \fname, which uses
well-formedness to also ensure that all intersection types are disjoint. 
In other words, while in \fname all valid intersection types must be
disjoint, in \fnamee unrestricted intersection types such as
$[[nat & nat]]$ are allowed.
More specifically,
the well-formedness of intersection types in \fnamee and \fname is:
{\small
\begin{mathpar}
  \ottaltinferrule{wf-\fnamee}{}{ [[DD |- A]] \\ [[DD |- B]]  }{ [[ DD |- A & B ]] } \and
  \ottaltinferrule{wf-\fname}{}{ [[DD |- A]] \\ [[DD |- B]] \\ \hlmath{[[DD |- A ** B]]}  }{ [[ DD |- A & B ]] }
\end{mathpar}
}%
Notice that \fname has an extra disjointness condition $[[ DD |- A ** B ]]$ in the premise.
This is crucial for \fname's syntactic method for proving coherence,
but also burdens the calculus with various syntactic restrictions and
complicates its metatheory. For example, it requires extra effort to
show that \fname only produces disjoint intersection types. As a consequence,
\fname features a \emph{weaker} substitution lemma (note the gray
part in \cref{lemma:weaker_lemma}) than \fnamee (\cref{lemma:general_lemma}).

\renewcommand\ottaltinferrule[4]{
  \inferrule*[narrower=0.6,lab=#1,#2]
    {#3}
    {#4}
}


\begin{proposition}[Type substitution in \fname] \label{lemma:weaker_lemma}
  If $[[DD |- A]]$, $[[DD |- B]]$, $[[ (X ** C) in DD  ]]$, $\hlmath{[[ DD |- B ** C ]]}$
  and well-formed context $[[  DD [ X ~> B ]   ]]$, then $[[   DD [ X ~> B ] |-  A [ X ~> B  ]     ]]$.
\end{proposition}

\begin{figure}[t]
  \centering
    \drules[S]{$ [[A <: B ~~> c]]  $}{Declarative subtyping}{refl,trans,top,rcd,andl,andr,arr,and,distArr,topArr,distRcd,topRcd,bot,forall,topAll,distAll}
  \caption{Declarative subtyping}
  \label{fig:subtype_decl:fi}
\end{figure}


\begin{lemma}[Type substitution in \fnamee] \label{lemma:general_lemma}
  If $[[DD |- A]]$, $[[DD |- B]]$, $[[ (X ** C) in DD  ]]$
  and well-formed context $[[  DD [ X ~> B ]   ]]$, then $[[   DD [ X ~> B ] |-  A [ X ~> B  ]     ]]$.
\end{lemma}
% \begin{proof}
%   By induction on the derivation of $[[ DD |- A  ]]$.
% \end{proof}


\paragraph{Declarative subtyping.}

\fnamee's subtyping judgment is another major difference to \fname, because it
features BCD-style subtyping and a rule for the bottom type.
The full set of subtyping rules are shown in
\cref{fig:subtype_decl:fi}. The reader is advised to ignore the gray parts for
now. Our subtyping rules extend the BCD-style subtyping rules from
\namee~\cite{bi_et_al:LIPIcs:2018:9227} with a rule for parametric
(disjoint) polymorphism (\rref{S-forall}). Moreover, we have three new rules:
\rref{S-bot} for the bottom type, and \rref{S-distAll,S-topAll} for distributivity of
disjoint quantification. The subtyping relation is a partial order
(\rref{S-refl,S-trans}). Most of the rules are quite standard. $[[Bot]]$ is a subtype of all types (\rref{S-bot}).
Subtyping of disjoint quantification is covariant in its body, and
contravariant in its disjointness constraints (\rref{S-forall}). Of particular interest are those
so-called ``distributivity'' rules: \rref{S-distArr} says intersections
distribute over arrows; \rref{S-distRcd} says intersections distribute over
records. Similarly, \rref{S-distAll} dictates that
intersections may distribute over disjoint quantifiers.
%It should be noted that \fnamee features explicit polymorphism, so type
%variables are neutral to subtyping, i.e., $[[X <: X]]$, which is already
%contained in \rref{S-refl}.

% \begin{remark}
%   In our Coq formalization, we require that the two types $[[A]]$ and $[[B]]$ are
%   well-formed relative to some type context, resulting in the subtyping
%   judgment $[[DD |- A <: B]]$. But this is not very important
%   for the purpose of presentation, thus we omit contexts.
% \end{remark}

\paragraph{Typing rules.}

\begin{figure}[t]
  \centering
    \drules[T]{$[[DD; GG |- ee => A ~~> e]]$}{Inference}{top, nat, var, app, merge, anno, rcd, proj, tabs, tapp}
    \drules[T]{$[[DD ; GG |- ee <= A ~~> e]]$}{Checking}{abs, sub}
  \caption{Bidirectional type system}
  \label{fig:typing:fi}
\end{figure}


\fnamee features a bidirectional type system inherited from \fname.
% \bruno{We can just focus on the
%   T-TApp, T-TAbs and T-Merge. Maybe you can drop subsumption and lambdas,
%   since those are standard.
% Refer to the full rules in the appendix.}
The full set of typing rules are shown in \cref{fig:typing:fi}. Again we ignore
the gray parts and explain them in \cref{sec:elaboration:fi}.
The inference judgment $[[ DD; GG |- ee => A ]]$ says
that we can synthesize the type $[[A]]$ under the contexts $[[DD]]$ and
$[[GG]]$. The checking judgment $[[ DD ; GG |- ee <= A ]]$ asserts that $[[ee]]$
checks against the type $[[A]]$ under the contexts $[[DD]]$ and $[[GG]]$.
Most of the rules are quite standard in the literature.
The merge expression $[[ee1 ,, ee2]]$ is well-typed if both sub-expressions are
well-typed, and their types are \textit{disjoint} (\rref{T-merge}).
The disjointness relation will be explained in \cref{sec:disjoint:fi}.
To infer a type abstraction (\rref{T-tabs}), we add disjointness constraints to the type
context. For a type
application (\rref{T-tapp}), we check that the type argument
satisfies the disjointness constraints.
\Rref{T-merge,T-tapp} are the only rules checking disjointness.


\subsection{Disjointness}
\label{sec:disjoint:fi}

\renewcommand{\rulehl}[1]{#1}

\begin{figure}[t]
  \centering
  \drules[TL]{$[[ A top  ]]$}{Top-like types}{top,and,arr,rcd,all}
  \drules[D]{$[[DD |- A ** B]]$}{Disjointness}{topL, topR, arr, andL, andR, rcdEq, rcdNeq, tvarL, tvarR, forall,ax}
  % \drules[Dax]{$[[A **a B]]$}{Disjointness axioms}{sym, intArr, intRcd,intAll,arrAll,arrRcd,allRcd}
  \caption{Selected rules for disjointness}
  \label{fig:disjoint:fi}
\end{figure}

We now turn to another core judgment of \fnamee---the disjointness relation,
shown in \cref{fig:disjoint:fi}. The disjointness rules are mostly inherited
from \fname~\cite{alpuimdisjoint}, but the new bottom type requires
a notable change regarding disjointness with \emph{top-like types}.

\paragraph{Top-like types.}

Top-like types are all types that are isomorphic to $[[Top]]$ (i.e.,
simultaneously sub- and supertypes of $[[Top]]$). Hence, they are inhabited by a
single value, isomorphic to the $[[Top]]$ value. \cref{fig:disjoint:fi} captures
this notion in a syntax-directed fashion in the $[[A top]]$ predicate.
As a historical note, the concept of top-like types was already known by
Barendregt et al.~\cite{Barendregt_1983}. The \oname calculus~\cite{oliveira2016disjoint}
re-discovered it and coined the term ``top-like types''; the \fname calculus~\cite{alpuimdisjoint} extended it
with universal quantifiers. Note that in both calculi,
top-like types are solely employed for enabling a syntactic method of
proving coherence, and due to the lack of BCD subtyping, they do not have a
type-theoretic interpretation of top-like types.





% \jeremy{mention previous work do not have good explanations of top-like types?}
% \ningning{that's a good idea. I think ``all types that are isomorphic to
%   $[[Top]]$'' is a more accurate definition of Top-like. A comparison with
%   previous work would be great (i.e. why \oname and \fname include top-like,
%   and \namee doesn't, while \fnamee does again?)}

\paragraph{Disjointness rules.}

%The disjointness judgment $[[DD |- A ** B]]$ plays a central role in \fnamee,
%and is used to ensure that the merge operator can only form intersections of
%disjoint types.

The disjointness judgment $[[DD |- A ** B]]$ is helpful to check whether the merge of two
expressions of type $[[A]]$ and $[[B]]$ preserves coherence. Incoherence
arises when both expressions produce distinct values for the same type, either
directly when they are both of that same type, or through implicit upcasting to
a common supertype. Of course we can safely disregard top-like types in this matter
because they do not have two distinct values. In short, it suffices to check that the
two types have only top-like supertypes in common.

% (As a precondition, $[[A]]$ and $[[B]]$ are
% required to be both well-formed under the context $[[DD]]$.)
% A guiding principle
% of checking whether $[[A]]$ and $[[B]]$ are disjoint is to ask if their
% supertypes are top-like types. Let us apply this principle to design
% disjointness rules for bottom types. Given $[[Bot]]$ and $[[A]]$, we know that
Because $[[Bot]]$ and any another type $[[A]]$ always have $[[A]]$ as
a common supertype,
it follows that $[[Bot]]$ is only disjoint to $[[A]]$ when
$[[A]]$ is top-like.
More generally, if $[[A]]$ is a top-like type, then $[[A]]$ is disjoint to any
type. This is the rationale behind the two rules \rref*{D-topL,D-topR}, which
generalize and subsume $[[ DD |- Top ** A ]]$ and $[[ DD |- A ** Top ]]$ from \fname, and
also cater to the bottom type.
Two other interesting rules are
\rref*{D-tvarL,D-tvarR}, which dictate that a type variable $[[X]]$ is disjoint
with some type $[[B]]$ if its disjointness constraints $[[A]]$ is a subtype of
$[[B]]$.
Disjointness axioms $[[ A **a B ]]$ (appearing in \rref{D-ax}) take
care of two types with different type constructors (e.g., $[[nat]]$ and records). Axiom rules can be found in the appendix.
Finally we note that the disjointness relation is symmetric.

% \begin{lemma}[Symmetry of disjointness]
%   \label{lemma:symmetry-disjoint}
%   If $[[ DD |- A ** B  ]]$, then $[[  DD |- B ** A   ]]$.
% \end{lemma}
% \begin{proof}
%   By induction on the disjointness derivation. In the case for \rref{FD-forall},
%   apply \cref{lemma:narrow:disjoint}.
% \end{proof}


\subsection{Elaboration and Type Safety}
\label{sec:elaboration:fi}

\begin{figure}[t]
  \centering
\begin{tabular}{llll} \toprule
  Types & $[[T]]$ & $\Coloneqq$ & $[[nat]] \mid [[Unit]] \mid [[T1 -> T2]]  \mid [[T1 * T2]] \mid [[X]] \mid [[\ X . T]]$\\
  Terms & $[[e]]$ & $\Coloneqq$ & $[[x]] \mid [[ii]] \mid [[unit]] \mid [[\x . e]] \mid [[e1 e2]] \mid [[< e1 , e2>]] \mid [[\X . e]] \mid [[ e T ]] \mid [[c e]]$ \\
  Coercions & $[[c]]$ & $\Coloneqq$ & $[[id]] \mid [[c1 o c2]] \mid [[top]] \mid [[bot]] \mid [[c1 -> c2]] \mid [[< c1 , c2 >]] \mid [[pp1]] \mid [[pp2]] $ \\
  & & $\mid$ & $ [[\ c]] \mid [[distArr]] \mid [[topArr]] \mid [[topAll]] \mid [[distPoly]] $ \\
  Values & $[[v]]$ & $\Coloneqq$ & $[[ii]] \mid [[unit]] \mid [[\x . e]] \mid [[< v1 , v2>]] \mid [[\X . e]] \mid [[ (c1 -> c2) v ]] \mid [[\c v]]  $ \\
  & & $\mid$ & $ [[distArr v]] \mid [[topArr v]] \mid [[topAll v]] \mid [[distPoly v]]  $ \\
  Term contexts & $[[gg]]$ & $\Coloneqq$ &  $[[empty]] \mid [[gg , x : T]] $ \\
  Type contexts & $[[dd]] $ & $\Coloneqq $ & $[[empty]] \mid [[dd , X]] $  \\
  Evaluation contexts & $[[EE]]$ & $\Coloneqq$ &  $  [[__]] \mid [[EE e]] \mid [[v EE]] \mid [[ < EE , e >  ]] \mid [[ < v , EE > ]] \mid [[ c EE  ]] \mid [[ EE T  ]]  $ \\ \bottomrule
\end{tabular}
\caption{Syntax of \tnamee}
\label{fig:syntax:fco}
\end{figure}


The dynamic semantics of \fnamee is given by elaboration into
a target calculus. The target calculus \tnamee is the standard call-by-value
System F extended with products and coercions. The syntax of \tnamee is shown in
\cref{fig:syntax:fco}.

\paragraph{Type translation.}

\cref{def:type:translate:fi} defines the type translation function $| \cdot |$
from \fnamee types $[[A]]$ to \tnamee types $[[T]]$. Most cases are
straightforward. For example, $[[Bot]]$ is mapped to an uninhabited
type $[[\X . X]]$; disjoint quantification is mapped to universal
quantification, dropping the disjointness constraints. $| \cdot |$ is
naturally extended to work on contexts as well.

\begin{definition} Type translation $| \cdot |$ is defined as follows:
  \label{def:type:translate:fi}
  \begin{center}
\begin{tabular}{rlllrlllrll} \toprule
  $| [[nat]] |$ &$=$ & $ [[nat]]$ & $\qquad$ &   $| [[Top]] |$ & $ = $ & $ [[ Unit  ]] $ & $\qquad$ &   $| [[A -> B]] |$ & $ = $ & $ [[ |A| -> |B| ]]$ \\
  $| [[A & B]] |$ &$=$ & $ [[ |A| * |B|    ]] $ & $\qquad$ &   $| [[{l : A}]] |$ & $ = $ & $ [[ | A | ]] $ & $\qquad$ &   $| [[X]] |$ & $ = $ & $ [[ X ]]$ \\
  $| [[ Bot  ]] |$ &$=$& $[[ \X . X   ]]$ & $\qquad$   & $| [[\X ** A . B]] |$ & $ = $ & $[[\X . |B|]]$ \\ \bottomrule
\end{tabular}
  \end{center}
\end{definition}


\paragraph{Coercions and coercive subtyping.}

We follow prior work~\cite{bi_et_al:LIPIcs:2018:9227, biernacki2015logical} by having a syntactic category
for coercions~\cite{Henglein_1994}. In \cref{fig:syntax:fco}, we have several new coercions:
$[[bot]]$, $[[ \ c ]]$, $[[distPoly]]$ and $[[topAll]]$ due to the addition of polymorphism and bottom type. As
seen in
\cref{fig:subtype_decl:fi} the coercive subtyping judgment has the form $[[ A <: B ~~> c ]]$, which
says that the subtyping derivation for $[[A <: B]]$ produces a coercion $[[c]]$ that converts terms of type $[[ |A|  ]]$
to $[[| B |]]$.
% Each subtyping rule has its own specific form of coercion.


% Now we go back
% to the coercion part in \rref{S-forall}. Since the disjointness constraints are
% erased during elaboration, they do not contribute to the overall coercion; we
% only need the coercion generated by the subtyping of the bodies $[[B1]]$ and
% $[[B2]]$.

% As a cognitive aid, it is
% instructive to mentally ``desugar'' the coercion $[[\ c]]$ to the regular term
% $[[ \f . \ X . c (f X)]]$, as shown in \cref{tab:coercion2}, then the expression
% $ [[\c v]] $ is ``equal'' to $[[ \X . c (v X) ]]$, which is why we can treat $[[ \c v]]$ as a value.

% \begin{figure}[t]
%   \centering

% \begin{tabular}{lllll} \toprule
%   \textbf{Coercion} & \textbf{Term} & & \textbf{Coercion} & \textbf{Term} \\ \midrule
%   $[[id]]$         & $[[ \x . x]]$  & & $[[ c1 o c2  ]]$    &  $[[  \x . c1 (c2 x) ]]$ \\
%   $[[top]]$         & $[[ \x . unit ]]$  & & $[[ c1 -> c2  ]]$    &  $[[  \f . \x . c2 (f (c1 x))  ]]$ \\
%   $[[  pp1   ]]$         & $[[ \x . pp1 x    ]]$  & & $[[ pp2  ]]$    &  $[[  \x . pp2 x  ]]$ \\
%   $[[  < c1 , c2 >   ]]$         & $[[ \x . < c1 x, c2 x >    ]]$  & & $[[ distArr  ]]$    &  $[[  \x . \y . < (pp1 x) y , (pp2 x) y > ]]$ \\
%   $[[  topArr   ]]$         & $[[ \x . \ y . unit    ]]$  & & $[[ \ c  ]]$ &  $[[   \f . \ X . c (f X)    ]]$   \\
%   $[[  topAll   ]]$         & $[[ \x . \ X . unit    ]]$  & & $[[ distPoly  ]]$ &  $[[   \f . \ X .< (pp1 f) X , (pp2 f) X >     ]]$   \\ \bottomrule
% \end{tabular}
%   \caption{Correspondence between coercions and terms}
%   \label{tab:coercion2}
% \end{figure}




\paragraph{\tnamee static semantics.}

The typing rules of \tnamee are quite standard. We have one rule \rref*{t-capp} regarding
coercion application, which uses the judgment $[[ c |- T tri T' ]]$ to
type coercions. We show two representative rules \rref*{ct-forall,ct-bot}.
{\small
  \begin{mathpar}
  \drule{t-capp} \and
  \drule{ct-forall}  \and \drule{ct-bot}
  \end{mathpar}
}%


\begin{figure}[t]
  \centering
  \drules[r]{$[[e --> e']]$}{Single-step reduction}{forall,topAll, distAll,tapp,app,ctxt}
  \caption{Selected reduction rules}
  \label{fig:red:fi}
\end{figure}

\paragraph{\tnamee dynamic semantics.}

The dynamic semantics of \tnamee is mostly unremarkable. We write $[[e --> e']]$
to mean one-step reduction. \Cref{fig:red:fi} shows selected reduction
rules. The first line shows three representative rules regarding coercion reductions.
They do not contribute to computation but merely rearrange coercions.
Our coercion reduction rules are quite standard but not efficient in terms of
space. Nevertheless, there is existing work on space-efficient
coercions~\citep{Siek_2015, herman2010space}, which should be applicable to our
work as well. \Rref{r-app} is the usual $\beta$-rule that performs actual
computation, and \rref{r-ctxt} handles reduction under an evaluation context.
As usual, $[[-->>]]$ is the reflexive, transitive closure of $[[-->]]$.
Now we can show that \tnamee is type safe:

\begin{theorem}[Preservation]
  If $[[empty; empty |- e : T]]$ and $[[e --> e']]$, then $[[empty; empty |- e' : T]]$.
\end{theorem}

\begin{theorem}[Progress]
  If $[[empty; empty |- e : T]]$, either $[[e]]$ is a value, or $\exists [[e']].\ [[e --> e']]$.
\end{theorem}


\paragraph{Elaboration.}

Now consider the translation parts in \cref{fig:typing:fi}. The key idea of
the translation follows the prior work~\cite{dunfield2014elaborating, oliveira2016disjoint,
  alpuimdisjoint, bi_et_al:LIPIcs:2018:9227}: merges are elaborated to pairs
(\rref{T-merge}); disjoint quantification and disjoint type applications
(\rref{T-tabs,T-tapp}) are elaborated to regular universal quantification and
type applications, respectively.
% Below we show an example translation:
% \[
%   [[ (\X ** nat . (\x . x) : X -> X)  : \ X ** nat . X & nat -> X ]] \hlmath{\rightsquigarrow [[\ (pp1 -> id)  (\ X . \x . x)]]}
% \]
% \begin{align*}
%   & [[ (\X ** nat . (\x . x) : X -> X)  : \ X ** nat . X & nat -> X ]] \\
%   \rightsquigarrow & \\
%   & [[\ (pp1 -> id)  (\ X . \x . x)]]
% \end{align*}
Finally, the following lemma connects \fnamee to \tnamee:

% \begin{lemma}[Coercions preserve types]
%   \label{lemma:sub-correct:fi}
% \end{lemma}
% \begin{proof}
%   By structural induction on the derivation of subtyping.
% \end{proof}
\begin{lemma}[Elaboration soundness] We have that:
  \begin{itemize}
  \item If $[[A <: B ~~> c]]$, then $[[c |-  |A| tri |B|]]$.
  \item If $[[DD ; GG |- ee => A ~~> e]]$, then $[[ |DD| ; |GG| |- e : |A | ]]$.
  \item If $[[DD ; GG |- ee <= A ~~> e]]$, then $[[ |DD| ; |GG| |- e : |A | ]]$.
  \end{itemize}
\end{lemma}
% \begin{proof}
%   By structural induction on the derivation of typing.
% \end{proof}




\section{Algorithmic System and Decidability}


The subtyping relation in \cref{fig:subtype_decl:fi} is highly non-algorithmic
due to the presence of a transitivity rule.
This section presents an alternative algorithmic formulation.
Our algorithm extends that of \namee, which itself was inspired by
Pierce's decision
procedure~\cite{pierce1989decision}, to handle
disjoint quantifiers and the bottom type. We then prove that the algorithm is sound and
complete with respect to declarative subtyping.

Additionally we prove that the subtyping and disjointness relations are
decidable. Although the proofs of this fact are fairly straightforward, it is
nonetheless remarkable since it contrasts with the subtyping
relation for (full) \fsub~\cite{cardelli1985understanding}, which is 
undecidable~\cite{pierce1994bounded}. Thus while bounded quantification is
infamous for its undecidability, disjoint quantification has the nicer property
of being decidable.

\subsection{Algorithmic Subtyping Rules}


While \cref{fig:subtype_decl:fi} is a fine specification of how subtyping
should behave, it cannot be read directly as a subtyping algorithm for two
reasons: (1) the conclusions of \rref{S-refl,S-trans} overlap with the other
rules, and (2) the premises of \rref{S-trans} mention a type that does not
appear in the conclusion. Simply dropping the two offending rules
from the system is not possible without losing
expressivity~\cite{Laurent12note}. Thus we need a
different approach. Following \namee, we intend the algorithmic judgment $[[ fs |- A <: B ]]$ to be
equivalent to $[[ A <: fs => B ]]$, where $[[fs]]$ is a queue used to track record labels, domain types and disjointness constraints.
The full rules of the algorithmic subtyping of \fnamee are shown \cref{fig:algo:sub:fi}.

\begin{definition}[$[[fs]] \Coloneqq [[ []   ]] \mid [[ l , fs  ]] \mid [[  B , fs ]] \mid [[ X ** B , fs  ]]$]
$[[fs => A]]$ is defined as follows:
  \begin{center}
  \begin{tabular}{rlllrll} \toprule
    $[[ [] => A]]$ &=& $[[A]]$ & $\qquad$ & $[[ (B , fs) => A]]$ &=& $[[B -> (fs => A)]]$  \\
    $[[ (l , fs) => A]]$ &=& $[[{l : fs => A}]]$ & $\qquad$ & $[[ (X ** B , fs) => A]]$ &=& $[[\ X ** B . fs => A]]$ \\  \bottomrule
  \end{tabular}
  \end{center}
\end{definition}


\renewcommand{\rulehl}[2][gray!40]{%
  \colorbox{#1}{$\displaystyle#2$}}

\begin{figure}[t]
  \centering
  \drules[A]{$[[fs |- A <: B ~~> c]]$}{Algorithmic subtyping}{top,and,arr,rcd,forall,const, bot,arrConst,rcdConst,andConst,allConst}
  \caption{Algorithmic subtyping}
  \label{fig:algo:sub:fi}
\end{figure}

\renewcommand{\rulehl}[1]{#1}



For brevity of the algorithm, we use metavariable $[[rho]]$ to mean type constants:
\[
  \text{Type Constants}\quad [[rho]] \Coloneqq  [[nat]] \mid [[Bot]] \mid [[X]]
\]
The basic idea of $[[ fs |- A <: B ]]$ is to perform a case analysis on $[[B]]$
until it reaches type constants. We explain new rules regarding disjoint
quantification and the bottom type.
When a quantifier is encountered in $[[B]]$, \rref{A-forall} pushes
the type variables with its disjointness constraints onto $[[fs]]$ and continue
with the body. Correspondingly, in \rref{A-allConst}, when a quantifier is
encountered in $[[A]]$, and the head of $[[fs]]$ is a type variable, this
variable is popped out and we continue with the body.
\Rref{A-bot} is similar to its declarative counterpart.
Two meta-functions $[[ < fs >1 ]]$
and $[[ < fs >2 ]]$ are meant to generate correct forms of coercions, and their
definitions are shown in the appendix. For other algorithmic rules, we refer to
\namee~\cite{bi_et_al:LIPIcs:2018:9227} for detailed explanations.


\paragraph{Correctness of the algorithm.}

We prove that the algorithm is sound and complete with respect to the
specification. We refer the reader to our Coq formalization for more details.
We only show the two major theorems:

\begin{theorem}[Soundness]
  If $[[ fs |- A <: B ~~> c]]$ then $ [[   A <: fs => B ~~> c  ]]   $.
\end{theorem}

\begin{theorem}[Completeness]
  If $[[A <: B ~~> c]]$, then $\exists [[c']].\ [[ [] |- A <: B ~~> c']]$.
\end{theorem}


\subsection{Decidability}

Moreover, we prove that our algorithmic type system is decidable. To see this,
first notice that the bidirectional type system is syntax-directed, so we only
need to show decidability of algorithmic subtyping and
disjointness. The full (manual) proofs for decidability can be found in
the appendix.

\begin{restatable}[Decidability of algorithmic subtyping]{lemma}{decidesub} \label{lemma:decide-sub}
  Given $[[fs]]$, $[[A]]$ and $[[B]]$, it is decidable whether there exists
  $[[c]]$, such that $[[fs |- A <: B ~~> c]]$.
\end{restatable}

\begin{restatable}[Decidability of disjointness checking]{lemma}{decidedis} \label{lemma:decide-dis}
  Given $[[DD]]$, $[[A]]$ and $[[B]]$, it is decidable whether $[[ DD |- A ** B ]]$.
\end{restatable}

% Given algorithmic subtyping and disjointness are decidable, it follows that our
% bidirectional type checking is decidable ($[[dirflag]]$ is a short-hand for $[[ => ]]$ and $[[<=]]$).

% \begin{restatable}[Decidability of typing]{theorem}{decidetyp} \label{lemma:decide-typing}
%   Given $[[DD]]$, $[[GG]]$, $[[ee]]$ and $[[A]]$, it is decidable whether $[[DD ; GG  |- ee dirflag A]]$.
% \end{restatable}

\renewcommand\ottaltinferrule[4]{
  \inferrule*[narrower=0.8,right=#1,#2]
    {#3}
    {#4}
}


One interesting observation here is that although our disjointness
quantification has a similar shape to bounded quantification $[[\/X<:A. B]]$ in
\fsub~\citep{cardelli1985understanding}, subtyping for \fsub~is
undecidable~\citep{pierce1994bounded}. In \fsub, the subtyping relation between
bounded quantification is:
{\small
\[
  \drule{fsub-forall}
\]
}%
Compared with \rref{S-forall}, both rules are contravariant on
bounded/disjoint types, and covariant on the body. However, with bounded
quantification it is fundamental to track the bounds in the
environment, which complicates the design of the rules and makes
subtyping undecidable with \rref{fsub-forall}.
Decidability can be recovered
by employing an invariant rule for bounded quantification
(that is by forcing $[[A1]]$ and $[[A2]]$ to be identical).
Disjoint quantification does not require such invariant rule for
decidability.

\renewcommand\ottaltinferrule[4]{
  \inferrule*[narrower=0.6,lab=#1,#2]
    {#3}
    {#4}
}


\begin{comment}
For example, in the original
type $[[\/X<:A1. B1]]$, the $[[X]]$ in $[[B1]]$ is thought of being bound to
$[[A1]]$, the premise $[[DD, X <: A2 |- B1 <: B2]]$ forces $[[X]]$ to be bound
to $[[A2]]$ in $[[B1]]$. This destroys the original connection and makes it
impossible to give a decision procedure.
\end{comment}

%%% Local Variables:
%%% mode: latex
%%% TeX-master: "../paper"
%%% org-ref-default-bibliography: "../paper.bib"
%%% End:
