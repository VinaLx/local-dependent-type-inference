\documentclass{elsarticle}

\usepackage[T1]{fontenc}
\usepackage{ottalt}
\usepackage{xcolor}
\usepackage{mathtools}
\usepackage{amsmath}
\usepackage{amsfonts}
\usepackage{amssymb}
\usepackage{setspace}
\usepackage{xspace}
\usepackage{mathpartir}
\usepackage{tikz-cd}
\usepackage{comment}
\usepackage{listings}
%\usepackage[english]{babel}

\inputott{dependent-subtyping.ott}

\newcommand{\rulehl}[2][gray!40]{
  \colorbox{#1}{$\displaystyle#2$}}

\newcommand{\extract}[1]{\lvert #1 \rvert}
\newcommand{\lift}[1]{\lceil #1 \rceil}
\newcommand{\castup}[0]{\mathrm{cast}_\Uparrow}
\newcommand{\castdn}[0]{\mathrm{cast}_\Downarrow}
\newcommand{\system}[0]{the dependent implicitly polymorphic calculus\xspace}
\newcommand{\System}[0]{The dependent implicitly polymorphic calculus\xspace}

\newcommand{\name}[0]{$\lambda_{I}^{\forall}$\xspace}

\newcommand{\an}[3]{{\color{#2} \textsc{#1}:#3}}
\newcommand{\bruno}[1]{\an{bruno}{blue}{#1}}
\newcommand{\alvin}[1]{\an{alvin}{red}{#1}}

\newtheorem{theorem}{Theorem}
\newtheorem{lemma}[theorem]{Lemma}
\newtheorem{corollary}[theorem]{Corollary}

\title{A Dependently Typed Calculus with Polymorphic Subtyping}

% \titlerunning{Dependently Typed Polymorphic Subtyping}

\author{Mingqi Xue}
\ead{mqxue@cs.hku.hk}
\address{The University of Hong Kong}

\author{Bruno C. d. S. Oliveira}
\ead{bruno@cs.hku.hk}
\address{The University of Hong Kong}

% \authorrunning{M. Xue and B. C. d. S. Oliveira}

% \author{Mingqi Xue \and Bruno C. d. S. Oliveira}
% \institute{
%     The University of Hong Kong \\
%     \email{\{mqxue,bruno\}@cs.hku.hk}
% }

% \Copyright{Mingqi Xue and Bruno C. d. S. Oliveira}

% \ccsdesc{Theory of computation~Type theory}
% \ccsdesc{Software and its engineering~Functional languages}



\begin{document}

\begin{abstract}
  A \emph{polymorphic subtyping} relation, which relates more general types
  to more specific ones, is at the core of many modern functional languages.
  As those languages start moving towards dependently typed programming a
  natural question is how can polymorphic subtyping be adapted to such settings.

  This paper presents \system (\name): a simple dependently typed calculus
  with polymorphic subtyping. The subtyping relation in \name
  generalizes the well-known polymorphic subtyping relation by
  Odersky and L\"aufer (1996). Because \name is dependently typed,
  integrating subtyping to the calculus is non-trivial. To overcome
  many of the issues arizing from integrating subtyping with dependent
  types, the calculus employs \emph{unified subtyping}, which is a technique
  that unifies typing and subtyping into a single relation. Moreover, \name
  employs explicit casts instead of a conversion rule, allowing
  unrestricted recursion to be naturally supported.
  We prove various non-trivial results, including \emph{type soundness} and \emph{transitivity}
  of unified subtyping. The calculus and all corresponding proofs
  are mechanized in the Coq theorem prover and we also have a simple prototype
  implementation of \name.
%A complication that arizes in the type-soundness proof
%is that a direct semantics is non-deterministic due
\end{abstract}


\begin{keyword}
  Type System \sep Dependent Type \sep Subtyping \sep Polymorphism
\end{keyword}
% \keywords{Type System, Dependent Type, Subtyping, Polymorphism}

\maketitle

\section{Overview}

In this section, we introduce \name by going through
some interesting examples to show the expressiveness and major features of the calculus.
Then we discuss the motivation, rationale of our design, and challenges.
The formal system of \name will be
discussed in Sections \ref{sec:system} and \ref{sec:metatheory}.

\subsection{A Tour of \name}
\label{sec:examples}

The \name calculus features a form of \emph{implicit
  higher-ranked polymorphism} with the power of dependent types. Thus the main feature of \name
is the ability to implicitly infer universally quantified arguments.

\paragraph{A First Example of Implicit Polymorphism}
Like most of functional languages, \name supports (implicit) parametric polymorphism.
A first simple example is the polymorphic identity
function:
\begin{flalign*}
&\mathrm{id} : \forall (A : \star).\, A \rightarrow A &&\\
&\mathrm{id} = \lambda (x : A).\, x &&\\
&\mathrm{answer} : \mathbb{N} &&\\
&\mathrm{answer} = \mathrm{id} ~ 42 \qquad \text{-- No type argument needed at the call of $\mathrm{id}$} &&
\end{flalign*}
\noindent The polymorphic parameter \verb|A| is annotated with its type,
which is $\star$. The type $\star$ is the type of types (also known as
\emph{kind}). In \name, the parameters of lambda abstractions are annotated
with their types, and the \verb|A| in the definition refers back to the
polymorphic parameter. In the examples below, we drop the parentheses around
variables and their type annotations such as $[[lambda x : A. x]]$ for conciseness.

Similar to implicit polymorphism in other languages,
the polymorphic parameters of the $\forall$ types are implicitly instantiated
during applications. Thus, in the call of the identity function, we
do not need to specify the argument used for instantiation. In contrast,
in an explicitly polymorphic language (such as System F) we would need
to call $id$ with an extra argument that specifies the instantiation of $A$:
$\mathrm{id}~\mathbb{N}~ 42$.

\paragraph{Recursion and Dependent Types}

\name is dependently typed, and universal quantifications are not limited to work
on arguments of type $\star$, but it allows arguments of other types. This is
a key difference compared to much of the work on type-inference for higher-ranked
polymorphism~\cite{dunfield2013complete,le2003ml,leijen2008hmf,vytiniotis2008fph,jones2007practical}
which has been focusing on System F-like
languages where universal quantification can only have arguments of type $\star$.
Furthermore, \name supports general recursion at both the term and the type-level.

Using these features we can encode an indexed list, a \verb|map| operation over it
and illustrate how the implicit instantiation allows us to use the \verb|map|
function conveniently.
However, because \name is just a core calculus there is no built-in support
yet for algebraic datatypes and pattern matching.
We expect that a source language would provide a more convenient
way to define the \verb|map| function using pattern matching and other useful source-level
constructs. To model algebraic datatypes and pattern matching in \name, we
use an encoding by Yang and Oliveira~\cite{yang2019pure},
which is based on Scott encodings~\cite{mogensen1992efficient}.
Since the details of that encoding are not relevant for this paper,
here we omit the code for most definitions and show only their types.

In a dependently typed language a programmer could write the following definition
for our formulation of indexed lists:
\newcommand{\Nat}[0]{\mathbb{N}}
\newcommand{\Succ}[0]{\mathrm{S}}
\newcommand{\Zero}[0]{\mathrm{Z}}
\newcommand{\List}[0]{\mathrm{List}}
\newcommand{\Nil}[0]{\mathrm{Nil}}
\newcommand{\Cons}[0]{\mathrm{Cons}}
\newcommand{\map}[0]{\mathrm{map}}
\begin{flalign*}
  & \mathbf{data} ~ \Nat = \Zero ~|~ \Succ~\Nat &&\\
  & \mathbf{data} ~ \List~(a : \star)~(n : \Nat) = \Nil ~ | ~ \Cons~a~(\List~a~(\Succ~n)) &&
\end{flalign*}
In \name we can encode the $\mathrm{List}$ and the constructors as conventional terms. We
show the definition for \verb|List|, and the types for the constructors next
(implementation omitted). In later subsequent examples we will just assume some
Haskell-style syntactic sugar for datatype definitions and constructors.
\begin{flalign*}
&\List : \star \rightarrow \Nat \rightarrow \star &&\\
&\List = \mu L : \star \rightarrow \Nat \rightarrow \star.\, \lambda a:\star.\, \lambda n :\Nat.\, \forall r:\star.\, r \rightarrow (a \rightarrow L~a~(\Succ ~ n) \rightarrow r) \rightarrow r &&\\
&\Nil : \forall a : \star.\, \forall n : \Nat.\, \List ~ a ~ n &&\\
&\Cons : \forall a : \star.\, \forall n : \Nat.\, a \rightarrow \List ~ a ~ (\Succ ~ n) \rightarrow \List ~ a ~n &&
\end{flalign*}
\noindent Then we can define a \verb|map| function over \verb|List| with the type:
\begin{flalign*}
  & \map : \forall a : \star.\, \forall b : \star.\, \forall n : \Nat .\, (a \rightarrow b) \rightarrow \List~a~n \rightarrow \List~b~n &&
\end{flalign*}
An example of application of \verb|map| is:
\begin{flalign*}
&\map~\Succ~(\Cons~1~(\Cons~2~\Nil)) &&
\end{flalign*}

\noindent which increases every natural in the list by one.
Note that since the type parameters for \verb|map|, \verb|Cons| and \verb|Nil|
are all implicit, they can be all omitted
and the arguments are instantiated implicitly. Thus the \verb|map| function
only requires two explicit arguments, making it as convenient to use
as in most functional language implementations.

There are a few final points worth mentioning about the example.
Firstly, \verb|List| is an example of a dependently typed function, since it is parametrized
by a natural value. Secondly, in \name (following the design of PITS~\cite{yang2019pure}),
fixpoint operators ($\mu$) serve a dual purpose of defining recursive types and recursive
functions. In \verb|List| the fixpoint operator is used to define a recursive type, whereas
in the definition of \verb|map| the fixpoint operator is used to define term-level recursion.
Moreover, recursion is unrestricted and there is no termination checking, much like approaches
such as Dependently Typed Haskell~\cite{dh}, and unlike various other dependently typed languages
such as Agda~\cite{2007_norell_agda} or Idris~\cite{brady2013idris}.


\paragraph{Implicit Higher-Kinded Types}

The implicit capabilities also extend to the realm of higher-kinded types~\cite{tapl}.
For example we can define a \verb|Functor| datatype as:
\newcommand{\Functor}[0]{\mathrm{Functor}}
\newcommand{\MkFunctor}[0]{\mathrm{MkF}}
\newcommand{\Id}[0]{\mathrm{Id}}
\newcommand{\MkId}[0]{\mathrm{MkId}}
\newcommand{\fmap}[0]{fmap}
\begin{flalign*}
  &\mathbf{data}~\Functor~(F : \star \rightarrow \star) = \MkFunctor~\{\fmap : \forall a : \star.\, \forall b : \star.\, (a \rightarrow b) \rightarrow F~a \rightarrow F~b\} &&
\end{flalign*}
In this case the type of $fmap$ is:
\begin{flalign*}
  & \fmap : \forall F : \star \to \star.~\Functor~F \to \forall a : \star.\, \forall b : \star.\, (a \rightarrow b) \rightarrow F~a \rightarrow F~b &&
\end{flalign*}
Importantly this example illustrates that universal variables can quantify over higher-kinds (i.e.
$F : \star \to \star$).
We can define instances of functor in a standard way:
\begin{flalign*}
  & \mathbf{data}~\mathrm{Id}~a=\MkId~\{runId : a\} && \\
  & \mathrm{idFunctor} : \Functor~\Id && \\
  & \mathrm{idFunctor} = \MkFunctor~\{\fmap = \lambda f : a \rightarrow b.\, \lambda ~x : \Id~a.\, \MkId~(f~(runId~x))\} &&
\end{flalign*}
and then use \verb|fmap| with three arguments:
\begin{flalign*}
& \fmap~\mathrm{idFunctor}~\Succ~(\MkId~0) &&
\end{flalign*}

\noindent Note that, because our calculus has no mechanism like type classes we pass the ``instance'' explicitly.
Nonetheless, three other arguments (the $F$, $a$, and $b$) are implicitly instantiated.

\paragraph{Higher-Ranked Polymorphic Subtyping}
\label{sec:higher-ranked-poly}

In calculi such as the ICC~\cite{miquel2001implicit}, a form of implicit instantiation also exists.
However, such calculi do not employ subtyping, instead, they only apply instantiation
to top-level universal quantifiers. Our next example illustrates how subtyping enables
instantiation to be applied also in nested universal quantifiers, thus enabling
more types to be related.

When programming with continuations~\cite{sussman1998scheme} one of the
functions that are typically needed is call-with-current-continutation
(\verb|callcc|). In a polymorphic language, there are several types that can be
assigned to \verb|callcc|. One of these types is a rank-3 type,
while another one is a rank-1 type.
Using polymorphic subtyping we can show that the rank-3
type is more general than the rank-1 type. Thus the following program type-checks:
\begin{flalign*}
& \mathrm{callcc}' : \forall a : \star.\, ((\forall b : \star.\, a \rightarrow b) \rightarrow a) \rightarrow a && \\
& \mathrm{callcc} : \forall a : \star.\, \forall b : \star.\, ((a \rightarrow b) \rightarrow a) \rightarrow a && \\
& \mathrm{callcc} = \mathrm{callcc}' &&
\end{flalign*}
\noindent The type $\forall b : \star.\, a \rightarrow b$ appears in a positive position
of the whole signature, and it is a more general signature than $a \rightarrow b$
for an arbitrary choice of $b$. Our language captures this subtyping relation so that
we can assign $\mathrm{callcc}'$ to $\mathrm{callcc}$ (but not the other way around).
In contrast, in approaches like the ICC, the types of \verb|callcc| and \verb|callcc'|
are not compatible and the example above would be rejected.

\subsection{Key Features}

We briefly discuss the major features of \name itself and
its formalization. A more formal and technical discussion will be left to
Sections \ref{sec:system} and \ref{sec:metatheory}.

\paragraph{Polymorphic Subtyping Relation}
Figure \ref{fig:polymorphic-subtyping} shows Odersky and L\"aufer polymorphic
subtyping relation~\cite{odersky1996putting}.
This relation captures a \emph{more-general-than} relation between
types as a subtyping relation. The key rules in their
subtyping relation are rules $\forallL$ and $\forallR$:

\begin{itemize}
  \item In rule $\forallL$, a polytype ($\forall x.\, A$) is considered \emph{more-general}
        than another type ($B$), when we can find an arbitrary monotype ($\tau$)
        so that the instantiation is more general than $B$.
        Importantly note that this relation does not guess arbitrary (poly)types,
        but just monotypes. In other words, the relation is \emph{predicative}~\cite{Martin-Lof-1972}.
        This restriction ensures that the relation is \emph{decidable}.

  \item In rule  $\forallR$ a type ($A$) is considered more general than a polytype ($\forall x. B$)
        when it is still more general than the head of the polytype, with the type
        parameter instantiated by an abstract variable $x$.
\end{itemize}

This subtyping relation sets a scene for our work, which
generalizes this relation to a dependently typed setting.

\begin{figure}
\centering

\begin{drulepar}{$\Gamma \vdash A \le B$}{Polymorphic Subtyping}
  \inferrule*[right=$\tau$]
    { }
    {\Gamma \vdash \tau \le \tau}
  \and
  \inferrule*[right=$\rightarrow$]
    {\Gamma \vdash B_1 \le A_1 \\ \Gamma \vdash A_2 \le B_2}
    {\Gamma \vdash A_1 \rightarrow A_2 \le B_1 \rightarrow B_2}
  \\
  \inferrule*[right=$\forallL$]
    {\Gamma \vdash \tau \\ \Gamma \vdash [\tau / x]\, A \le B}
    {\Gamma \vdash \forall x.\, A \le B}
  \and
  \inferrule*[right=$\forallR$]
    {\Gamma ,\, x \vdash A \le B}
    {\Gamma \vdash A \le \forall x.\, B}
\end{drulepar}

\caption{The polymorphic subtyping relation by Odersky and L\"aufer~\cite{odersky1996putting}.}
\label{fig:polymorphic-subtyping}
\end{figure}

\paragraph{Generalizing Polymorphic Subtyping}
\label{sec:polymorphic-subtyping}

The parameters of universal types can only be types in the polymorphic
subtyping relation by Odersky and L\"aufer.
In \name, we generalize the polymorphic parameters so that they can
be values or other kinds of types as well.
The idea of a direct generalization is:

\begin{mathpar}
  \inferrule*[right=$\forallL'$]
    {\Gamma \vdash \tau \rulehl{: A} \\ \Gamma \vdash [\tau / x]\, B \le C}
    {\Gamma \vdash \forall x \rulehl{: A}.\, B \le C}
  \and
  \inferrule*[right=$\forallR'$]
    {\Gamma ,\, x \rulehl{: B} \vdash A \le C}
    {\Gamma \vdash A \le \forall x \rulehl{: B}.\,C}
\end{mathpar}

\noindent The parameters for universal types can have any type (and not just $\star$).
Hence, instead of requiring the monotype $\tau$ to be a well-formed type in rule
$\forallL$, in rule $\forallL'$ it is
required that $\tau$ is well-typed regarding the type of the parameter
in the universal quantifier.
Similarly, for rule $\forallR'$ the context for the subtyping rule should include typing information
for the universally quantified variable.
However, this idea introduces the issue of potential mutual dependency between
subtyping and typing judgements, so further adjustments have to be made to formalize
this idea, which is discussed later in this section and section
\ref{sec:type-system} and \ref{sec:adaptation}.

\paragraph{Higher-Ranked Polymorphic Subtyping}

As the \verb|callcc| example in Section \ref{sec:higher-ranked-poly} shows, these subtyping
rules based on polymorphic subtyping, combined with other subtyping rules,
are able to handle the subtyping relations that occur at not only top-level,
but also at a higher-ranked level. This feature distinguishes our \name from the
Implicit Calculus of Constructions (ICC) \cite{miquel2001implicit} which also talks about
the implicit polymorphism of dependent type languages. The ICC features the two related rules
in the \emph{typing relation}:

\begin{mathpar}
  \inferrule*[lab=inst]
    {[[G |- e : forall x : A. B]] \\ [[G |- e1 : A]]}
    {[[G |- e : [e1 / x] B]]}
  \and
  \inferrule*[lab=gen]
    {[[G, x : A |- e : B]] \\ [[G |- forall x : A. B : k]]}
    {[[G |- e : forall x : A. B]]}
\end{mathpar}

\noindent Without an explicit subtyping relation, the ICC is not always able to handle subtyping
at higher-ranked positions. The approach taken by the ICC is similar to
Hindley-Milner type system~\cite{hindley1969principal,milner1978theory},
which also features similar rules in typing. But Hindley-Milner is designed for
dealing only with rank-1 polymorphism.
In generalizations of higher-ranked polymorphic type-inference~\cite{dunfield2013complete,le2003ml,leijen2008hmf,vytiniotis2008fph,jones2007practical},
it has been shown that rules like $\forallL$ and $\forallR$ generalize rules like
\textsc{GEN} and \textsc{INST}. Since we aim at higher-ranked polymorphic generalization,
we follow a similar, more general, approach in \name.

\paragraph{Unified Subtyping}
However, the revised subtyping relation with $\forallL'$ and $\forallR'$ rules suffers from an
important complication compared to the Odersky and L\"aufer formulation: there is now
a notorious mutual dependency between typing and subtyping.
In Odersky and L\"aufer's rules, the subtyping rules
do not depend on typing. In particular
the rule $\forallL$ depends only on well-formedness ($\Gamma \vdash \tau$).
In contrast, note that rule $\forallL'$ now mentions the typing relation
in its premise ($\Gamma \vdash \tau : A$). Moreover, as usual,
the subsumption rule of
the typing relation (shown below) depends on the subtyping relation.
\begin{mathpar}
  \inferrule*[right=t-sub]
    {\Gamma \vdash e : A \\ \Gamma \vdash A \le B}
    {\Gamma \vdash e : B}
\end{mathpar}
This mutual dependency problem has been a significant
problem when combining subtyping and dependent types~\cite{subdep, hutchins},
and presents itself on our way to the direct generalization of polymorphic subtyping.

To tackle this issue, we adopt a technique called
\emph{unified subtyping}~\cite{full}. Unified subtyping merges the typing relation and
subtyping relation into a single relation to avoid the mutual dependency:
\begin{mathpar}
  \Gamma \vdash e_1 \le e_2 : A
\end{mathpar}
The interpretation of this judgement is: under context $\Gamma$, $e_1$ is a subtype
of $e_2$ and they both are of type $A$. The judgements for subtyping and typing
are both special form of unified subtyping: % with the involvement of kind $[[*]]$:
\begin{mathpar}
  \Gamma \vdash A \le B \triangleq \Gamma \vdash A \le B : [[*]]
  \and
  \Gamma \vdash e : A \triangleq \Gamma \vdash e \le e : A
\end{mathpar}
The technique simplifies the formalization of dependently typed calculi with subtyping,
and especially the proof of transitivity in the original work. Ideally after applying the technique,
the generalization of the polymorphic subtyping should be:

\begin{mathpar}
  \inferrule*[right=$\le\forall'' L$]
    {\Gamma \vdash \tau : A \\ \Gamma \vdash [\tau / x]\, B \le C \rulehl{: [[*]]}}
    {\Gamma \vdash \forall x : A.\, B \le C \rulehl{: [[*]]}}
  \and
  \inferrule*[right=$\le\forall'' R$]
    {\Gamma ,\, x : B \vdash A \le C \rulehl{: [[*]]}}
    {\Gamma \vdash A \le \forall x : B.\,C \rulehl{: [[*]]}}
\end{mathpar}

\noindent The basic idea of our own formalization essentially follows a similar design,
although the actual rules in \name are slightly more sophisticated.
The details will be discussed in Section \ref{sec:type-system}.

\paragraph{``Explicit'' Implicit Instantiation}

With polymorphic subtyping the instantiation of universally quantified type
parameters is done implicitly instead of being manually applied. In non-dependently
typed systems, \emph{implicit} parameters are types (i.e. terms are not involved in
implicit instantiation). For example:
\begin{mathpar}
  (\lambda x.\, x)~42 \longrightarrow 42
\end{mathpar}
\noindent Here $\lambda x.\, x$ has type $\forall A.\, A \rightarrow A$. The implicit
instantiation is not reflected anywhere at term level. In a
language with type annotations on parameters of binders
(for example lambda abstractions), in order
for the polymorphic variables to be well-scoped at type annotations of terms,
another binder $\Lambda$ is added for terms. Nonetheless, instantiations are still
implicit as shown in the following example:
\begin{mathpar}
  (\Lambda A : [[*]].\, \lambda x : A.\, x) ~ 42 \longrightarrow 42
\end{mathpar}
Here $\Lambda A : [[*]].\, \lambda x : A.\, x$ has type $\forall A : \star. \, A \rightarrow A$,
and the polymorphic parameter $A$ is explicitly stated in the polymorphic
term. However as the reduction shows, the instantiations are still implicit.
We purposely omitted the explicit binders for implicit parameters for all the examples
in Section \ref{sec:examples} for conciseness. Such explicit binders can
be recovered with a simple form of syntactic sugar:

\[e : \forall x : A.\, B \triangleq \Lambda x : A.\, e : \forall x : A.\, B\]

\noindent When polymorphic parameters are used, \name provides a binder ($\Lambda x : A.\, e$)
to ensure that the parameters are well-scoped at the term-level.

\paragraph{Computational Irrelevance}
\label{sec:computational-irrelevance-overview}

Implicit parameters in traditional languages with polymorphic subtyping,
the ICC~\cite{miquel2001implicit,barras2008implicit} and \name are computationaly irrelevant.
In traditional (non-dependently) typed languages, types cannot affect computation,
thus computational irrelevance is quite natural and widely adopted.
In dependently typed systems, however, there can be some programs where
it is useful to have computationaly relevant implicit parameters.
For example, accessing the length of a length indexed vector in constant time:
\begin{flalign*}
  &\mathrm{length} : \forall n : [[int]].\, \mathrm{Vector}~n \rightarrow [[int]] &&\\
  &\mathrm{length} = \Lambda n : [[int]].\, \lambda v : \mathrm{Vector}~n.\, n
\end{flalign*}
\noindent Here the implicit parameter $n$ is computationally relevant as it is used as
the return value of the function which is likely to be executed at runtime.
Languages like Agda, Coq or Idris support such programs. However,
computationaly relevant implicit parameters are challenging for proofs of
type soundness. Due to such challenges (see also the discussion in
Section~\ref{subsec:semantics}),
the ICC has a restriction that parameters for implicit function types
must be computationally irrelevant. Since we adopt a similar technique,
we also have a similar restriction and thus cannot encode programs such
as the above.

\paragraph{Type-level Computation and Casts}
\name features a fixpoint that supports general recursion at both
type and term level. In order to avoid diverging computations at type checking,
we do not provide the congruence rule like other dependently
typed systems such as the Calculus of Constructions~\cite{coc}
to support implicit type-level reduction.
\begin{mathpar}
  \inferrule*[lab=Cong]
    {[[G |- e : A]] \\ A =_\beta B}
    {[[G |- e : B]]}
\end{mathpar}
Instead, we partially adopt the
call-by-name design of \emph{Pure Iso-Type Systems} (PITS)~\cite{isotype,yang2019pure},
and provide operators $\castdn$ and $\castup$ to explicitly trigger one-step
type reductions or expansions as shown in the typing rules below.
\begin{mathpar}
  \inferrule*[lab=Castup]
    {[[G |- e : B]] \\ [[A --> B]] \\ [[G |- A : k]]}
    {[[G |- castup [A] e : A]]}
  \and
  \inferrule*[lab=Castdn]
    {[[G |- e : A]] \\ [[A --> B]] \\ [[G |- B : k]]}
    {[[G |- castdn e : B]]}
\end{mathpar}

\section{\System}

This section introduces the static and dynamic semantics of
\name: a dependently typed calculus with type casts
and implicit polymorphism. The calculus employs
\emph{unified subtyping}~\cite{CoquandThierry1988Tcoc}
and has a single relation that generalizes both typing and subtyping.
The calculus can be seen as a variant of the \emph{calculus of constructions}~\cite{},
but it uses a simple form of casts~\cite{} instead of the conversion rule
and features unrestricted recursion. We present syntax, unified subtyping
and reduction for \name.

\begin{figure}[t]
\centering
\begin{equation*}
\begin{array}{llcl}
    \text{Kinds} & k & ~\Coloneqq ~ & [[*]] \mid [[box]] \\
    \text{Expressions} & e, A, B & ~ \Coloneqq ~ & [[x]] \mid [[n]] \mid [[k]] \mid [[int]] \mid [[e1 e2]] \mid [[lambda x : A. e]]
        \mid [[pi x : A. B]] \mid [[bind x : A. e]] \\
        & & \mid & [[forall x : A. B]] \mid [[mu x : A. e]] \mid [[castup [A] e]] \mid [[castdn e]]   \\
    \text{Mono-Expressions} ~ & \tau, \sigma & ~ \Coloneqq ~ & [[x]] \mid [[n]] \mid [[k]] \mid [[int]] \mid \tau_1 ~ \tau_2 \mid \lambda \, x : \tau. ~ \sigma \mid \Pi \, x : \tau. ~ \sigma \mid \Lambda \, x : \tau. ~ \sigma \\
        & & \mid & \mu \, x : \tau. ~ \sigma \mid \castup \, [\tau]~ \sigma \mid \castdn \, \tau \\
    \text{Values} & v & ~ \Coloneqq ~ & [[k]] \mid [[n]] \mid [[int]] \mid [[lambda x : A. e]] \mid [[pi x : A. B]] \mid [[bind x : A. e]] \\
        & & \mid & [[forall x : A. B]] \mid [[castup [A] e]] \\
    \text{Contexts} & \Gamma & ~ \Coloneqq ~ & [[nil]] \mid [[G , x : A]] \\
    \text{Syntactic Sugar} ~ & A \rightarrow B & \triangleq & [[pi x : A. B]] \qquad \text{where} ~ x \notin \mathrm{FV}(B)
\end{array}
\end{equation*}
\caption{Syntax of \name.}
\label{fig:syntax}
\end{figure}
\bruno{Aren't we missing some monotype restrictions? Forall types should have monotype
  arguments, right? Please double check the restrictions, change the syntax if necessary
  and double-check the text.}
\bruno{Be careful with the margins: we are overflowing.}

\subsection{Syntax}

Figure \ref{fig:syntax} presents the syntax of \name. The syntax is similar
syntax to the Calculus of Constructions, featuring
$[[*]]$ and $[[box]]$ in the kind hierarchy, and unifying the concepts of terms
and types as expressions. Due to the unified syntax, types and
expressions ($e$, $A$ and $B$) are used
interchangeably, although we mostly adopt the convention of using $A$ and $B$
for contexts where the expressions are used as types and $e$ for contexts
where the expressions represent terms.
The syntax includes all the constructs of the calculus of constructions:
variables ($[[x]]$), kinds ($[[k]]$), function applications  ($[[e1 e2]]$),
lambda expressions ($[[lambda x : A. e]]$), dependent function types ($[[pi x : A. B]]$)
as well as integer types ($[[int]]$) and integers ($[[n]]$).
Moreover, there are a number of additional language constructs for
supporting implicit polymorphism, recursion and explicit type-level computation
via casts. These constructs are discussed next.

\subsubsection{Implicit Polymorphism}

Forall types $[[forall x : A. B]]$ are used to generalize implicit
polymorphism in non-dependent languages. In contrast to non-dependent
universal quantification in conventional functional languages, the
arguments range over all well-typed expressions besides well-formed
types (i.e. $x$ can have any type $A$ instead of just $\star$).
In other words, ``polymorphic types'' are naturally dependent, so $\forall$
types can be viewed as the implicit counterpart of $\Pi$ types. We also have
implicit lambda expressions ($[[bind x : A. e]]$), which are the implicit counterpart of
  $\lambda$ abstractions. The design used in \name
  is similar to the design of $\text{ICC}^*$~\cite{barras2008implicit}, which
  employs similar constructs for implicit dependent products.
Like conventional universal quantification, the arguments of $\forall$ types are
deduced during applications rather than being explicitly passed.
In addition, following designs for predicative higher-ranked polymorphism~\cite{oderskylufer,DK,PJ}, we have also generalized the concept of ``monotypes'' to
``mono-expressions'', essentially excluding $\forall$ types from expressions.

\subsubsection{Explicit Type-level Computation}
\label{sec:cast}
\System adopts the iso-types~\cite{yang2016unified}, featuring explicit type-level
computations with cast operators $\castdn$ and $\castup$. These operators
respectively perform one-step
type reduction and type expansion based on the operational semantics.
The reduction in cast operators is deterministic so that the type annotation is
only needed during type expansions ($\castup$).

\subsubsection{Fixpoint Operator}

Finally, we add fixpoints ($[[mu x : A. e]]$)
to support general recursion for both term-level
and type-level. Iso-recursive types are supported due to the presence of $\castup$
and $\castdn$, which correspond to the ``fold'' and ``unfold'' operations when
working on recursive types.
%However, we do not support subtyping for recursive
%types in the current system due to the complexity.

\subsection{Operational Semantics}

\bruno{If I understand correctly we need 2 different reductions, one
  (the non-deterministic one) that is used in the type system; and another
  one (which erases types) that is deterministic and would be the basis
  for an actual implementation of reduction at run-time. After reading
  this subsection, I think we want
to tell that story here and present the 2 variants of reduction here.}

Figure \ref{fig:semantics} presents the small-step operational semantics of our system,
which mostly follows the call-by-name variant of Pure Iso-Type Systems (PITS)
\cite{yang2019pure} corresponding to the calculus of constructions.
Note that the arguments of $\beta$-reduction (\rref{r-beta}) and expressions in
the \rref{r-cast-elim} are not required to be values.
Meanwhile we consider $\castup$ terms to be a value,
and only perform reduction inside $\castdn$ terms (\rref{r-castdn}). Also, the unroll
operation of fixpoint operator is supported by \rref{r-mu}.

\begin{figure}[t]
    \centering

    \drules[r]{$[[e1 --> e2]]$}{Operational Semantics}{
        app,beta,inst,mu,castdn,castXXinst,castXXelim}

    \caption{Operational semantics of \name.}
    \label{fig:semantics}
\end{figure}
\bruno{Do not use $mono~e$ in the figure.
  You have the syntax $\tau$ and $\sigma$ to represent
monotypes, so just use that instead.}

\subsubsection{Nondeterministic Implicit Instantiations}
The most interesting rules are those involving instantiation of implicit parameters:
\rref{R-Inst} and \rref{R-Cast-Inst}. These rules are not deterministic, or even
(in the general case) type-preserving. The main reason for this is due to the
guess of $e$, which can be an arbitrary monoexpression.
These issues are resolved by imposing restrictions on the type system such that
\textbf{a)} non-deterministic reductions are not well-typed inside cast operations
\textbf{b)} the instantiation choices of implicit parameters do not affect the
overall runtime type-safety. Details will be explained in the later sections.

\begin{figure}
    \centering
    \drules[wf]{$[[|- G]]$}{Well-formed Context}{nil,cons}

    \drules[s]{$[[G |- e1 <: e2 : A]]$}{Unified Subtyping}{
        var,lit,int,star,abs,pi,app,bind,mu,castup,castdn,forallXXl,forallXXr,forall,sub}
    \begin{equation*}
       \text{Syntactic Sugar} \qquad [[G |- e : A]] \triangleq [[G |- e <: e : A]]
    \end{equation*}
    \caption{(Sub)Typing Rules of \name.}
    \label{fig:typing}
\end{figure}
\bruno{The figure and rules will need a little cosmetic work. Firstly, there are many rules
  stacking up premises vertically. I think it is better to have multiple rules
  horizontally (and only upto 2 or 3 vertical stacks of premises).
  Secondly, we need to look at the layout carefully to use space efficiently.
  At the moment there are rules like S-ABS that could be paired up with some other
  rules side-by-side and use less space (at the same time it is nicer
  if adjancent rules are somehow related: two cast rules; application/abstraction, etc).
  We must organize the rules in a nicer way. Thirdly, I think that, for binders with
  annotations, like $\lambda x : A. e$, we may want to use brackets on the arguments
  to improve readability, as in $\lambda (x : A). e$. It is a bit hard to ``parse''
  the syntax without a little bit of effort. If this change is implemented it would
  affect the whole section, starting from syntax.
}

\subsection{Unified Subtyping System}
\label{sec:type-system}

Figure \ref{fig:typing} shows the (sub)typing rules of the system. We adopt the design
<<<<<<< Updated upstream
of unified subtyping~\cite{yang2017unifying}, where the subtyping rules and
typing rules are merged into a single typing judgment $[[G |- e1 <: e2 : A]]$. 
=======
of Unified Subtyping~\cite{yang2017unifying}, where the subtyping rules and
typing rules are merged into a single typing judgment $[[G |- e1 <: e2 : A]]$.
>>>>>>> Stashed changes
Unified subtyping solves the traditional issue of mutual dependency between typing
and subtyping in a dependently type system.
The interpretation of this judgment is ``under context $[[G]]$, $[[e1]]$ is a
subtype of $[[e2]]$ and they are both of type $[[A]]$''.
In this form of formalization, the typing judgment $[[G |- e : A]]$ is a
special case of unified subtyping judgment $[[G |- e <: e : A]]$,
and the idea of well-formed types $\Gamma \vdash A$ is expressed by
$[[G |- A : k]]$ where $k$ is either $[[*]]$ or $[[box]]$.
\bruno{We should say something about the ``more standard'' rules here.
  I think we can briefly explain them and point the
  reader to Linus work for further details, while observing that the rules
  here are essentially simplified version (due to the absence of bounded quantification)
of his rules.}

\subsubsection{Rules for Universal Quantification}
The subtyping rules for universal quantification (\rref{s-forall-l,s-forall-r}) follow
the spirit of the Odersky and L\"aufer declarative subtyping rules~\cite{oderskylaufer,DunfieldJoshua2013Caeb},
where the subtyping relation is interpreted as a ``more general than'' relation.
A polymorphic type $\forall (x:A). B$
is more general than another type $C$ when its well-typed
instantiation is more general than $C$ (\rref{s-forall-l}). A polymorphic
type $\forall (x:B). C$ is less general than a type $A$,
if $C$ is is less general than $A$ when the argument with the polytype ($x:B$)
is abstracted out (\rref{s-forall-r}). Additionaly, \rref{s-forall}
axiomatizes the subtyping relation between two polymorphic types.
The motivation for \rref{s-forall} is discussed in more detail in later sections.

\subsubsection{Mono-expression Restrictions}
As in other predicative relations (such as the one by Odersky and L\"aufer),
the type arguments for instantiation in \rref{s-forall-l} are
required to be mono-expressions, which has cascading effects on typing rules of
other expressions. The arguments for applications are required to be
mono-expressions, and the whole fixpoint expression is required to be a
mono-expressions. We shall
discuss the reason behind the restrictions in later sections.
\bruno{I feel that we may be deferring a bit too much explanation
to later sections, but lets come back to this after you write later sections.}

\subsubsection{Kind Restriction for Universal Types}
\label{sec:kind-restriction}

For the kinding of types, we mainly follow the design of the Calculus of
Constructions~\cite{CoquandThierry1988Tcoc}. However, we specifically restrict
the $\forall (x : A). B$
expressions to only have the kind $[[*]]$. This prevents other types of kind
$[[*]]$ from having kinds such as $[[forall x : int. *]]$,
which significantly complicates the metatheory when reasoning about the kind of types.
This restriction propagates to the introduction rule of $\forall$ types (\rref{s-bind}),
where $[[B]]$ is required to only have kind $[[*]]$.
This way well-typed implicit abstractions ($\Lambda$ expressions) are kept away
from type computations. Therefore, in cast operators,
the possibility of non-deterministic implicit instantiations is eliminated.

\begin{figure}[t]
    \centering
    \begin{equation*}
        \begin{array}{llcl}
            \text{Extracted Expressions} & e, A, B & ~ \Coloneqq ~ & [[x]] \mid [[n]] \mid [[k]] \mid [[int]] \mid [[ee1 ee2]] \mid [[elambda x. ee]] \mid [[epi x : eA. eB]] \mid [[ebind x. ee]] \\
            & & \mid & [[eforall x : eA. eB]] \mid [[emu x. ee]] \mid [[ecastup ee]] \mid [[ecastdn ee]]
        \end{array}
    \end{equation*}

    \begin{equation*}
    {\setstretch{1.5}
    \begin{array}{cc}
        \extract{[[x]]} = [[x]] \qquad
        \extract{[[n]]} = [[n]] &
        \extract{[[k]]} = [[k]] \qquad
        \extract{[[int]]} = [[int]] \\
        \extract{[[e1 e2]]} = \extract{[[e1]]} ~ \extract{[[e2]]} &
        \extract{[[mu x : A. e]]} = \mu \, x. ~ \extract{[[e]]} \\
        \extract{[[lambda x : A. e]]} = \lambda \, x. ~ \extract{[[e]]} &
        \extract{[[pi x : A. B]]} = \Pi \, x : \extract{[[A]]}. ~ \extract{[[B]]} \\
        \extract{[[bind x : A. e]]} = \Lambda \, x. ~ \extract{[[e]]} &
        \extract{[[forall x : A. B]]} = \forall \, x : \extract{[[A]]}. ~ \extract{[[B]]} \\
        \extract{[[castup [A] e]]} = \castup \, \extract{[[e]]} &
        \extract{[[castdn e]]} = \castdn \, \extract{[[e]]}
    \end{array}
    }
    \end{equation*}

    \caption{Extraction of Expressions}
    \label{fig:extraction}
\end{figure}

\subsubsection{Runtime Irrelevance of Implicit Arguments}

Our direct operational semantics choose random mono-expressions to instantiate
the implicit arguments which potentially breaks type safety. So we adopt a
restriction that is similar to the Implicit Calculus of Constructions (ICC) \cite{miquel2001implicit}
to make the choice of implicit argument irrelevant to the computation, by only
allowing the implicit parameters to occur in type annotations in the body of
implicit abstraction. Figure \ref{fig:extraction} shows the extraction function
that gets rid of all the type annotations in expressions, and enforcing the
freshness of implicit parameter in the extracted expression. The type safety of
the direct operational semantics is proved indirectly with the help of
the extraction of expressions, which we will come back to in later sections.

\subsubsection{Redundant Premises}

All the premises highlight in gray are redundant in a way that
the system without them is proved equivalent to the system in figure \ref{fig:typing}.
They are there to simplify the mechanized proofs of certain lemmas.


\section{The Metatheory of \name}
\label{sec:metatheory}

This section presents the metatheory of \name, and discusses several challenges
that arose during the design of the typing rules to ensure desired subtyping and
typing properties in our system.
The three main results of the metatheory are:
\emph{transitivity of unified subtyping}, \emph{type-soundness} and
\emph{completeness with respect to Odersky and L\"aufer's} polymorphic
subtyping. Transitivity of subtyping is a general challenge for dependent type systems due
to the mutual dependency of typing and subtyping, and the Odersky and L\"aufer style
subtyping brings new issues to the table. For type-soundness, the key challenge
is the non-deterministic and non-type-preserving nature of the reduction relation.
To address this issue, we employ a type soundness proof technique
that makes use of the erased reduction relation shown in Figure~\ref{fig:extraction}.

\subsection{Polymorphic Subtyping in a Dependently Typed Setting}
\label{sec:adaptation}

The polymorphic subtyping relation by Odersky and L\"aufer features the following two rules:
\begin{mathpar}
    \inferrule*[right=$\forallL$]
      {\Gamma \dkvdash \tau \\ \Gamma \dkvdash [\tau / x]\, A \le B}
      {\Gamma \dkvdash \forall x.\, A \le B}
    \and
    \inferrule*[right=$\forallR$]
      {\Gamma ,\, x \dkvdash A \le B}
      {\Gamma \dkvdash A \le \forall x.\, B}
\end{mathpar}
In order for the \emph{well-formedness}~\citep{dunfield2013lemmas} property
(\emph{If $\Gamma \dkvdash A \le B$, then $\Gamma \dkvdash A$ and $\Gamma \dkvdash B$})
to hold in L\"aufer and Odersky's system, these two rules rely on certain properties
that do not hold in our dependently typed generalization. So we make several adjustments
in our adaptation to address these issues, which result in the difference between
our current system and a direct generalization mentioned in Section~\ref{sec:feature-overview}.

\paragraph*{Reverse Substitution of Well-Formedness}
\label{sec:reverse-subst}
Rule $\forallL$ relies on the \emph{reverse substitution} property, but this property
does not hold in a dependently typed setting. Thus we need an alternative design that
still ensures well-formedness, but without relying on the reverse substitution property.

The reverse substitution property is:
\emph{If $\Gamma \dkvdash [B / x] \, A$ and $\Gamma \dkvdash B$, then $\Gamma,\, x \dkvdash A$}.
That is if we have a type $A$ with all occurrences of $x$ substituted by $B$ and $B$
is well-formed, we can conclude that $A$ is well-formed under $\Gamma,\, x$.
In a dependently typed setting, a possible form of generalization of this property
would be:
\emph{If $[[G |- [B / x] A : *]]$ and $[[G |- B : C]]$, then $[[G , x : C |- A : *]]$},
which unfortunately does not hold.
In a dependent type system, the values of expressions also matter during type checking
besides the types of expressions, a counter-example of the property is:
\begin{align*}
& F : [[int]] \rightarrow [[*]],\, a : F~42  \vdash (\lambda y : F~\rulehl{42}.\, [[int]]) ~ a : [[*]] \\
& F : [[int]] \rightarrow [[*]],\, a : F~42  ,\, \rulehl{x : [[int]]} \vdash (\lambda y : F~\rulehl{x}.\, [[int]]) ~ a : [[*]]
\end{align*}
We cannot ``reverse substitute'' the $42$ in the type annotation to a variable
of the same type: the application expression depends specifically on the value
$42$ in order for the type of argument $a$ to match the parameter type.
So we add a premise $[[G , x : A |- B : *]]$ in \rref{s-forall-l} to directly
ensure the well-formedness of types in the conclusion.

\paragraph*{Strengthening of Contexts}
\label{sec:strengthening}

Rule $\forallR$ relies on a strenghtening lemma:
\emph{if $\Gamma,\, x \vdash A$ and $x ~\text{does not occur in}~ A$, then $\Gamma \vdash A$}, which
is trivial to prove in their system.
However the admissibility of its generalization:
\emph{if $[[G, x : B |- A : *]]$ and $x ~\text{does not occur in}~ A$, then $[[G |- A : *]]$},
% does not hold in our system.
is much more complicated to reason about. We can construct the following example:
\begin{equation*}
    F : [[int]] \rightarrow [[*]],\, A : [[*]],\, \rulehl{a : A} \vdash F ~ ([[(bind x : A. lambda y : int. y)]]~ 42) : [[*]]
\end{equation*}
The variable $a$ does not appear in any expression, but plays a
crucial role when considering the subtyping statement
$\Gamma \vdash \forall x : A.\, [[int]] \rightarrow [[int]] \le [[int]] \rightarrow [[int]]$,
which arises when type-checking the application $[[(bind x : A. lambda y : int. y)]]~42$.
In this case, we cannot apply \rref{s-forall-l} unless we find a
well-typed instance for the polymorphic parameter. So the variable in the context
is needed even though it does not occur anywhere in the final judgement.
Note that, since our system has a fixpoint operator, theoretically we could construct
a diverging program $[[mu x : A. x]]$ to instantiate the implicit parameter, but
this possibility leads to several other issues which we are going to discuss in
Section~\ref{sec:habitability}. Furthermore such an approach would not work for calculi
without fixpoints.

Due to these complications, we assume that strengthening does not hold in our system for now.
We add a premise $[[G |- A : *]]$ to \rref{s-forall-r} to workaround this issue,
requiring $A$ to be a well-kinded type without the help of the fresh variable.
A consequence of adding this premise is that we will encounter a circular proof
while trying to prove $[[G |- forall x : A. B <: forall x : A . B : *]]$, for
arbitrary $A$ and $B$ by first applying \rref{s-forall-r}. We resolve this issue by
adding \rref{s-forall}.

\subsection{Typing Properties of \name}
\label{sec:typing-properties}

With our rules properly set up, we can prove most of the basic properties
using techniques borrowed from the \emph{unifed subtyping}~\citep{full} approach.
We introduce \emph{reflexivity}, \emph{weakening},
\emph{context narrowing}, \emph{substitution} and \emph{type correctness}
in this section.

\begin{theorem}[Reflexivity]
   If $[[G |- e1 <: e2 : A]]$,
   then $[[G |- e1 : A]]$ and $[[G |- e2 : A]]$.
\end{theorem}

Usually, a subtyping relation is reflexive when any well-formed type is a subtype
of itself. With unified subtyping, the well-formedness of types is expressed by
subtyping relation as well, so the reflexivity looks more like the generalization
of \emph{well-formedness} mentioned in the previous section. Reflexivity
breaks down into two parts, \emph{left reflexivity} and \emph{right reflexivity}.

\begin{lemma}[Left Reflexivity]
   If $[[G |- e1 <: e2 : A]]$,
   then $[[G |- e1 : A]]$.
\end{lemma}

\begin{lemma}[Right Reflexivity]
   If $[[G |- e1 <: e2 : A]]$,
   then $[[G |- e2 : A]]$.
\end{lemma}

\noindent Both of the branches are proved by induction on the derivation of
$[[G |- e1 <: e2 : A]]$.
Left reflexivity and right reflexivity when derivations end with \rref{s-forall-l}
and \rref{s-forall-r} respectively are directly solved by \rref{s-forall}.

\begin{theorem}[Weakening]
    If $[[G1 ,, G3 |- e1 <: e2 : A]]$ and $[[|- G1 ,, G2 ,, G3]]$,
    then $[[G1 ,, G2 ,, G3 |- e1 <: e2 : A]]$.
\end{theorem}

\noindent \emph{Weakening} is proved by induction on the derivation of
$[[G1 ,, G3 |- e1 <: e2 : A]]$. The redundant premises discussed in Section
\ref{sec:type-system} help to simplify the proof, by creating the induction
hypotheses about the type annotation of various expressions. Otherwise, we are
not able to prove $[[|- G1 ,, G2 ,, G3 , x : A]]$ given only
$[[|- G1 ,, G3 , x : A]]$ and no help from induction hypotheses.

\begin{theorem}[Context Narrowing]
\label{thm:narrowing}
    If $[[G1 , x : B ,, G2 |- e1 <: e2 : C]]$ and $[[G1 |- A <: B : k]]$,
    then $[[G1 , x : A ,, G2 |- e1 <: e2 : C]]$.
\end{theorem}

\begin{lemma}[Well-formedness of Narrowing Context]
\label{thm:wf-narrowing}
   If $[[|- G1 , x : B ,, G2]]$ and $[[G1 |- A <: B : k]]$,
   then $[[|- G1 , x : A ,, G2]]$.
\end{lemma}

\noindent Theorem \ref{thm:narrowing} and Lemma \ref{thm:wf-narrowing} are proved by
mutual induction on the derivations of $[[G1 , x : B ,, G2 |- e1 <: e2 : C]]$
and $[[|- G1 , x : B ,, G2]]$. \Rref{s-var} is the only non-trivial case to
solve: it relies on \emph{weakening} to conclude
$[[G1 , x : A ,, G2 |- A <: B : k]]$ from $[[G1 |- A <: B : k]]$, in order to
derive $[[G1 , x : A ,, G2 |- x : B]]$ through \rref{s-sub}.

\begin{theorem}[Substitution]
    If $[[G1 , x : A ,, G2 |- e1 <: e2 : B]]$ and $[[G1 |- t : A]]$,
    then $[[G1 ,, [t / x] G2 |- [t / x] e1 <: [t / x] e2 : [t / x] B ]]$.
\end{theorem}

\noindent Notably \emph{substitution} has a mono-expression restriction on the
expression being substituted with. This is due to the mono-expression restriction on
the instantiation of polymorphic parameters in \rref{s-forall-l}.

Take the following derivation as an example:

\begin{mathpar}
    \inferrule*[Right=s-forall-l]
      {A : [[*]],\, F: A \rightarrow [[*]],\, \rulehl{a : A} \vdash [\rulehl{a} / x]\, F ~ x \le F~\rulehl{a} : [[*]]}
      {A : [[*]],\, F: A \rightarrow [[*]],\, \rulehl{a : A} \vdash \forall x : A.\, F ~ x \le F~\rulehl{a} : [[*]]}
\end{mathpar}

\noindent Assuming that we have no mono-expression restrictions on \emph{substitution}
and \rref{s-app}. If we substitute $a$ with an arbitrary poly-expression, the
derivation stops working because \rref{s-forall-l} requires a mono-expression
instantiation and \rref{s-app} requires the argument of both sides to be
syntactically the same.

Worth mentioning is that while substitution of poly-expressions breaks the
subtyping aspect of the language, a special case of the substitution theorem
that discusses the typing of one expression
(If $[[G1 , x : A ,, G2 |- e : B]]$ and $[[G1 |- e1 : A]]$,
then $[[G1 ,, [e1 / x] G2 |- [e1 / x] e : [e1 / x] B]]$)
does not hold for similar reasons.
Because, in dependently-typed languages, substitutions are also involved in the types of
expressions as well. Due to the presence of \rref{s-sub}, we still have to
maintain the potential subtyping relation of the types of expressions after substitution,
for example:

\begin{mathpar}
    \inferrule*[Right=s-sub]
      {A : [[*]],\, F: A \rightarrow [[*]],\, a : A,\, b : \forall x : A.\, F~x \vdash b : \rulehl{\forall x : A.\, F~x}}
      {A : [[*]],\, F: A \rightarrow [[*]],\, a : A,\, b : \forall x : A.\, F~x \vdash b : \rulehl{F ~ a}}
\end{mathpar}

% \noindent As a result, we put mono-expression restriction on the \emph{substitution} theorem,
% as well as the typing rules involving expressions that potentially ``trigger''
% substitutions during reductions, including the arguments of applications and the
% fixpoint expressions.
\noindent As a result, \emph{substitution} only holds with the mono-expression
restriction. This has a cascading effect on typing rules like \rref{s-app, s-mu}
whose expressions trigger substitutions during reduction. So the mono-expression
restriction has to be added for those rules for the system to be type-safe.

\begin{lemma}[Context Well-formedness of Substitution]
\label{thm:wf-substitution}
   If $[[|- G1 , x : A  ,, G2]]$ and $[[G1 |- t : A]]$,
   then $[[|- G1 ,, [t / x] G2]]$.
\end{lemma}

\noindent After understanding the mono-expression restriction on \emph{substitution}, the actual
proof is not complicated: it proceeds by mutual induction with
Lemma \ref{thm:wf-substitution} on the derivations of
$[[G1 , x : A ,, G2 |- e1 <: e2 : B]]$ and $[[|- G1 , x : A ,, G2]]$. When the
derivation ends with \rref{s-castup,s-castdn}, the proof
requires the reduction relation to preserve after the substitution.
This property should usually hold, but it puts an interesting constraint which we
have to consider when designing the reduction rules (see Section \ref{sec:cast-design}).

\begin{lemma}[Reduction Substitution]
   If $[[A --> B]]$, then $[[ [t / x] A --> [t / x] B ]]$
\end{lemma}

\begin{theorem}[Type Correctness]
    If $[[G |- e1 <: e2 : A]]$,
    then $\exists k.\, [[G |- A : k]]$ or $A = [[box]]$.
\end{theorem}

\noindent \emph{Type correctness} is a nice property that ensures that
what appears in the position of a type is actually a type.
The theorem is proved by induction on the derivation of $[[G |- e1 <: e2 : A]]$.
The only non-trivial case is when the derivation ends with \rref{s-app}. We make
use of the substitution lemma and the inductive hypothesis to demonstrate the head
of a $\Pi$ type preserves its kind after the argument is applied.

\subsection{Transitivity}
\label{sec:transitivity}

Transitivity is typically one of the most challenging properties to prove in
calculi with subtyping and it was also one of the harder proofs in \name.
The proof of transitivity requires a generalization of the usual transitivity
property:

\begin{theorem}[Generalized Transitivity]
    If $[[G |- e1 <: e2 : A]]$ and $[[G |- e2 <: e3 : B]]$,
    then $[[G |- e1 <: e3 : A]]$.
\end{theorem}

\noindent where the types of the premises are potentially different.
To prove this property we employ sizes for the inductive argument. Moreover we rely on
a subtle property of uniqueness of kinds.
\paragraph{Uniqueness of Kinds} Assuming that the derivation of the second
premise of generalized transitivity ends with \rref{s-forall-r}, then we face the following problem:
\begin{mathpar}
    \inferrule*[]
      {[[G |- e1 <: e2 : A]] \\ [[G, x : B |- e2 <: C : *]]}
      {[[G |- e1 <: forall x : B. C : A]]}
\end{mathpar}
\noindent Before applying \rref{s-forall-r} to the conclusion,
we have to establish the relationship between $A$ and $[[*]]$.
Were there no restrictions on the kinding of $\forall$ types,
this would be a much more complicated situation, where the inversion lemmas of
about kinds and transitivity depend on each other.
This is one of the main reasons why we forbid $\forall$ types having kind $[[box]]$.
Then we can have the following theorem:

\begin{theorem}[Kind Uniqueness]
    If $[[G |- e : k]]$ and $[[G |- e : A]]$,
    then $A = k$.
\end{theorem}

\noindent The proof is achieved by generalizing the shape of $k$ to be
$\Pi x : A.\, \dots \Pi x : B.\, \dots k$ for obtaining useful inductive hypotheses
when the derivation of $[[G |- e : k]]$ ends with \rref{s-app}. Then the proof
proceeds with induction on the derivation of the generalized $[[G |- e : k]]$ and
assembling various lemmas of kinding to solve different cases.

With the help of \emph{kind uniqueness},
we ensure the equivalence of $A$ and $[[*]]$ on this and other similar situations.

\paragraph{The Induction}

We prove \emph{Generalized Transitivity} by performing a strong induction on
the ordered 3-tuple of measures:

$$
\langle \#\forall([[e1]]) + \#\forall([[e2]]) + \#\forall([[e3]]), ~
\mathbf{size}([[e1]]) + \mathbf{size}([[e3]]), ~
\mathcal{D}_1 + \mathcal{D}_2 \rangle
$$

\noindent where

\begin{itemize}
    \item $\#\forall(e)$ counts the number of $\forall$ quantifiers
    in expression $e$, which solves cases when either side of the premise is
    derived by \rref{s-forall, s-forall-l, s-forall-r}.
    \item $\mathbf{size}(e)$ measures the size of the syntax tree of
    expression $e$. The sum of expression sizes solves most of the other
    standard recursive cases.
    \item $\mathcal{D}_1$ and $\mathcal{D}_2$ denote the sizes of the derivation
    trees of the first and the second premise respectively. The sum of sizes
    of derivation tree solve the case involving \rref{s-sub} where the sizes of
    expressions does not decrease.
\end{itemize}

% $\#\forall$ counts the number of $\forall$ quantifiers in the expression,
% $\mathbf{size}$ measures the size of the syntax tree of the expressions,
% $\mathcal{D}_1$ and $\mathcal{D}_2$ denote the sizes of the derivation trees of
% the first and the second premises respectively.
The proof is mainly inspired by DK's transitivity proof of their declarative subtyping
system of induction on the pair of
$\langle \#\forall([[e2]]) ,~ \mathcal{D}_1 + \mathcal{D}_2 \rangle$~\citep{dunfield2013lemmas},
with some adjustments to fit in our system.

The most problematic case to solve is when the first premise ends with \rref{s-forall-r},
and the second ends with \rref{s-forall-l}. Essentially we have to show the following:

\begin{mathpar}
    \inferrule*[]
      {[[G , x : A |- e1 <: B : *]] \\ [[G |- [t / x] B <: e3 : *]] \\ [[G |- t : A]]}
      {[[G |- e1 <: e3 : *]]}
\end{mathpar}

Here the only decreasing measure we can rely on is that
$\#\forall([[ [t / x] B ]])$ is one less than $\#\forall([[forall x : A. B]])$
(since $\tau$ is a monotype which does not contain $\forall$ quantifiers).
To solve this case, we first perform a substitution on the premise
$[[G , x : A |- e1 <: B : *]]$ with the help of the fact that
$x$ does not occur in $e_1$, obtaining $[[G |- e1 <: [t / x] B : *]]$, then
we use the inductive hypothesis provided by $\#\forall([[e2]])$.

The reason why we cannot directly copy DK's proof measure is because of the case
when both premises end with \rref{s-pi}, where we encounter the following problem:
\begin{mathpar}
    \inferrule*[]
      {[[G , x : A2 |- B1 <: B2 : k]] \\ [[G , x : A3 |- B2 <: B3 : k]] \\ [[G |- A3 <: A2 : k2]]}
      {[[G , x : A3 |- B1 <: B3 : k]]}
\end{mathpar}
The first and the second premise above do not share the same context, which
must be unified with the context narrowing theorem to be able to use the
inductive hypothesis.
However context narrowing potentially increases the size of the
derivation tree, so we are not able to use the inductive hypothesis of
$\mathcal{D}_1 + \mathcal{D}_2$, and resort to the sizes of expressions
($\mathbf{size}(e_1) + \mathbf{size}(e_3)$)

We have to make adjustments to solve the cases which preserve the
size of derivation tree, but not the sizes of the expressions, which is
when the first premise ends with \rref{s-forall-l}:

\begin{mathpar}
    \inferrule*[]
      {[[G |- [t / x] B <: e2 : *]] \\ [[G |- e2 <: e3 : C]]}
      {[[G |- forall x : A. B <: e3 : *]]}
\end{mathpar}

\noindent In this case, $\#\forall({[[ [t / x] B]]})$ is one less than $\#\forall([[forall x : A. B]])$,
so it can be solved by applying \rref{s-forall-l} and the inductive hypothesis of $\#\forall(e_1)$.
Additionally, $\#\forall(e_3)$ is used to make the measure ``symmetric''
to deal with the contravariance case of \rref{s-pi}.

Then, most of the cases that do not involve $\forall$ can be
solved by applying the inductive hypothesis corresponding to
$\mathbf{size}(e_1) + \mathbf{size}(e_3)$.
Finally, $\mathcal{D}_1 + \mathcal{D}_2$ solves
the cases where either premise ends with \rref{r-sub}, where the only decreasing
measure is the size of the derivation trees when the sizes of expressions remain
the same.

\begin{corollary}[Transitivity]
    If $[[G |- e1 <: e2 : A]]$ and $[[G |- e2 <: e3 : A]]$,
    then $[[G |- e1 <: e3 : A]]$.
\end{corollary}

\emph{Transitivity} is a special case of
\emph{Generalized Transitivity} where $A = B$.

\subsection{Type Safety}
\label{sec:type-safety}

Since the reduction rules of \name do not have access to typing information, they
cannot perform valid instantiation checks of the implicit arguments during applications.
Thus, the runtime semantics is non-deterministic and potentially non-type-preserving.
We tackle this issue by employing designs that make the choice of implicit instantiation
irrelevant at runtime with the restrictions in \rref{s-bind}.
We define an erasure function (shown in Figure \ref{fig:extraction})
that eliminates the type annotations
in some of the expressions ($\lambda$, $\Lambda$, $\mu$ and $\castup$),
and keep implicit parameters from occurring in the erased expressions.
This way the choices of implicit instantiation only affect type annotations,
which are not relevant for runtime computation.

We show that \name is type-safe in the sense that,
if an expression is well-typed, then the reduction of its erased version
does not ``go wrong''. Figure \ref{fig:extraction} shows the semantics of
erased expressions. The erasure semantics mostly mirrors the semantics
shown in Figure \ref{fig:semantics}, except for \rref{er-elim,er-cast-elim}, which
conveys the idea of the irrelevance of implicit instantiation by eliminating the
parameter directly.

\paragraph{Progress} We show the \emph{progress} property for both
the original expressions and the erased expressions.

\begin{theorem}[Generalized progress]
    If $[[nil |- e1 <: e2 : A]]$,
    then $\exists \, e_1'. \, e_1 \longrightarrow e_1'$ or $e_1$ is a value,
    and $\exists \, e_2'. \, e_2 \longrightarrow e_2'$ or $e_2$ is a value.
\end{theorem}

\begin{theorem}[Generalized progress on erased expressions]
    If $[[nil |- e1 <: e2 : A]]$,
    then $\exists \, E_1'. \, \extract{e_1} \Longrightarrow E_1'$ or $\extract{e_1}$ is an erased value,
    and $\exists \, E_2'. \, \extract{e_2} \Longrightarrow E_2'$ or $\extract{e_2}$ is an erased value.
\end{theorem}

\noindent We prove a generalized version of \emph{progress} that talks about both sides
of the expressions with unified subtyping. Note that they are not necessarily
simultaneously reducible or not reducible due to the presence of \rref{s-forall-l, s-forall-r}.
The left-hand-side may be reducible with the right-hand-side being a value or vice versa.

Both theorems are proved by induction on the derivation of $[[nil |- e1 <: e2 : A]]$.
The proof is mostly straightforward except when the derivation ends with $\castdn$,
where we have to show that the inner expressions $e$ of $\castdn~e$ either reduces, or
is a $\castup$ or a $\Lambda$-expression. We prove another fact to solve the situation:
for a well-typed expression whose type reduces, that expression
cannot be a value unless it is a $\castup$ or $\Lambda$-expression.

\begin{lemma}[Reducible Type]
    If $[[G |- e : A]]$, $[[A --> B]]$ and $e$ is not $\castup$ or a $\Lambda$-expression,
    then $e$ is not a value.
\end{lemma}

\begin{lemma}[Erased Value to Value]
    If $\extract{e}$ is an erased value, then $e$ is a value.
\end{lemma}

This lemma is also useful for the proof of progress for erased expressions.
Since the value definitions are very similar, we can use the property of values
on erased values.

\begin{lemma}[Value to Erased Value]
    If $e$ is a value, then $\extract{e}$ is an erased value.
\end{lemma}

\begin{corollary}[Progress]
    If $[[nil |- e : A]]$,
    then $\exists \, e'. \, e \longrightarrow e'$ or $e$ is a value.
\end{corollary}

\begin{corollary}[Progress on erased expressions]
    If $[[nil |- e : A]]$,
    then $\exists \, E'. \, \extract{e} \Longrightarrow E'$ or $\extract{e}$ is an erased value.
\end{corollary}

Both corollaries directly follow from their generalized versions.

\paragraph{Preservation}
The direct operational semantics is not generally type-preserving and
deterministic because of the implicit instantiations. Thus, we show
preservation with the help of the erased expressions (where implicit parameters
do not matter to the computation). For other reduction rules that do not involve
such issues, we discuss them as though we are proving a normal preservation
for brevity.

\begin{theorem}[Subtype Preservation]
    If $[[G |- e1 <: e2 : A]]$, $\extract{e_1} \Longrightarrow E_1'$ and $\extract{e_2} \Longrightarrow E_2'$,
    then $\exists\,e_1' \, e_2'.$ $\extract{e_1'} = E_1'$, $\extract{e_2'} = E_2'$,
    $e_1 \longrightarrow e_1'$, $e_2 \longrightarrow e_2'$ and $\Gamma \vdash e_1' \le e_2' : A$.
\end{theorem}

% \begin{figure}
%     \centering
%     \begin{tikzcd}[row sep=large, column sep=large]
%         e : A \arrow[r] \arrow[dashed]{d}[swap]{\text{Erasure}} & e' : A \\
%         E \arrow[shorten=1mm]{r}[swap, pos=1]{E} & E' \arrow[dashed]{u}[swap]{\text{Annotation}}
%     \end{tikzcd}
%     \caption{Diagram for Erased Preservation without Subtyping}
%     \label{fig:preservation}
% \end{figure}
\begin{figure}
    \centering
    \begin{tikzcd}[row sep=large, column sep=large]
        e : A \arrow[r] \arrow[dashed]{d}[swap]{\text{Erasure}} & e' : A \\
        E \arrow[Rightarrow]{r} & E' \arrow[dashed]{u}[swap]{\text{Annotation}}
    \end{tikzcd}
    \caption{Diagram for Erased Preservation without Subtyping}
    \label{fig:preservation}
\end{figure}

\noindent The theorem might look a little complicated at first glance. It breaks
down into two aspects: the erasure-annotation process and the subtype preservation.

Figure \ref{fig:preservation} shows the idea of our preservation lemma without
considering the subtyping aspect (assuming $[[G |- e : A]]$ instead of $[[G |- e1 <: e2 : A]]$).
Here we use \emph{annotation} as the reverse process of erasure.
If an expression ($e$) is well-typed, and its erasure ($E$)
reduces to another erased expression ($E'$), we can find a ``annotated''
expression of $E'$ ($e'$) that is reduced by $e$ and also preserves the type $A$.
When no implicit instantiation happens in the reduction, then $e \longrightarrow e'$
is deterministic: i.e. it is just normal type preservation. When there are implicit
instantiations, if the erased expression can reduce, we show that there
exists a valid instantiation for $e$ that preserves its type after the reduction, and
this instantiation only affects type annotations.
In other words, the runtime semantics of \name can be implemented only with
erased expressions.

Aside from the erasure aspect of our preservation lemma, we also consider the
generalized version of preservation in our unified subtyping system, the
\emph{subtype preservation}, where reductions not only preserve the type of expressions,
they also preserve the subtyping relation between expressions as well.

The theorem is proved by induction on the derivation of $[[G |- e1 <: e2 : A]]$,
cases for \rref{r-beta,r-mu} are solved with the substitution theorem,
cases \rref{r-app,r-castdn} are solved by inductive hypotheses. The interesting
cases to prove are \rref{r-cast-elim} and cases involving implicit instantiation
(\rref{r-inst,r-cast-inst}).

\paragraph{Cast Elimination}
The main issue of the cast elimination case can be demonstrated by the following derivation:

\begin{mathpar}
    \hspace{-1.5cm}
    \inferrule*[Right=s-castdn]
      {\rulehl{[[B1 --> B2]]} \\ \inferrule*[Right=s-sub]
        {\rulehl{[[G |- A1 <: B1 : k]]} \\ \inferrule*[Right=s-castup]
          {\rulehl{[[A1 --> A2]]} \\ [[G]] \vdash [[e]] : \rulehl{A_2}}
          {[[G |- castup [A1] e : A1]]}}
        {[[G |- castup [A1] e : B1]]}}
      {[[G]] \vdash [[castdn (castup [A1] e)]] : \rulehl{B_2}}
\end{mathpar}

Here the typing of the inner $\castup$ is not directly derived by
\rref{r-castup}, but by the subsumption rule instead. We want to show that after
the cast elimination (following \rref{r-cast-elim}), expression $e$ has type $B_2$,
while in reality it has type $A_2$ (as highlighted).
Therefore want to show $\Gamma \vdash A_2 \le B_2$ with the information that
$\Gamma \vdash A_1 \le B_1$, $[[A1 --> A2]]$
and $[[B1 --> B2]]$, which depends on the property we want to prove initially,
subtype preservation.
Since subtype preservation needs to solve the cast elimination case, here we have a
circular dependency of properties.
This problem was also observed by \cite{full}. They solved this situation
by a delicate approach with the help of an essential lemma
\emph{Reduction Exists in the Middle} (If $[[G |- e1 <: e2 : A]]$, $[[G |- e2 <: e3 : A]]$
and $e_1 \longrightarrow e_1'$, $e_3 \longrightarrow e_3'$, there exists $e_2'$
such that $e_2 \longrightarrow e_2'$). Unfortunately this lemma does not hold
in our system since universal types, which are not reducible, can appear in the
middle of two reducible types, so we cannot adopt their proof on this case.

We tackle this problem from another direction, with the observation that the
demand for subtype preservation property shifts from the term-level to the type-level.
With the calculus-of-constructions-like kind hierarchy, our system only has
limited layers in types (type $[[int]]$ has kind $[[*]]$, kind $[[*]]$ has kind $[[box]]$).
In fact, we only need to go up one layer in the type hierarchy to be able to
obtain subtype preservation directly,
since there is no subtyping involved at the kind level,
hence no problem for the cast elimination there.
Even better, we show that by going up one level in the type hierarchy
(only discussing the types of terms), the options
for the reduction that can be performed by a well-typed term are very limited.
Implicit abstractions cannot occur in type computation due to the kind
restriction of universal types as explained in Section \ref{sec:kind-restriction}.
Furthermore, we also prove that well-typed reductions never occur
for kinds, so cast operators also do not occur in type-level computation.

\begin{figure}
    \drules[dr]{$[[A1 ==> A2]]$}{Deterministic Reduction}
      {app,beta,mu}
    \caption{Deterministic Reduction.}
    \label{fig:deterministic-reduction}
\end{figure}

Figure \ref{fig:deterministic-reduction} shows the effective reduction rules
inside cast operators.

\begin{lemma}[Deterministic Reduction]
    If $[[A ==> A1]]$ and $[[A ==> A2]]$,
    then $A_1 = A_2$.
\end{lemma}

\begin{lemma}[Deterministic Type Reduction]
    If $[[G |- A1 : k]]$ and $[[A1 --> A2]]$,
    then $[[A1 ==> A2]]$.
\end{lemma}

The cases for implicit abstractions are easy to prove. For the cast operators
we have the following lemma.

\begin{lemma}[Expressions of kind $[[box]]$ are never reduced]
    If $[[A --> B]]$ and $[[G |- e : A]]$,
    then $B$ does not have kind $[[box]]$.
\end{lemma}

Note that the result of this lemma may, at first sight, be surprising
because it means that we cannot construct
a reducible expression like: $([[lambda x : int. *]])~42$ which has kind $[[box]]$.
In reality, the lambda expression must be of type $[[int]] \rightarrow [[box]]$
for the application to be well-typed. However, as we employ the conventional
typing rule for lambda abstractions of Calculus of Constructions~\citep{coc},
the function types of the lambda abstractions themselves
must be well-kinded (see \rref{s-abs}). Since $[[box]]$ itself does not have a kind,
$[[int]] \rightarrow [[box]]$ is not well-kinded, therefore the whole application
is not well-typed. For a similar reason, the position of occurence of $[[*]]$
in a well-typed expression is very restricted, hence the lemma is provable.

With the previous lemmas, the subtype preservation lemma for type computation is easily shown.

\begin{lemma}[Subtype Preservation for Types]
    If $[[G |- A1 <: B1 : k]]$, $[[A1 ==> A2]]$ and $[[B1 ==> B2]]$,
    then $[[G |- A2 <: B2 : k]]$.
\end{lemma}

\paragraph{Implicit Instantiations}
The proof of two cases for implicit instantiations (\rref{r-inst,r-cast-inst})
are quite similar. In our language, implicit instantiations of implicit functions
are only triggered by \rref{s-forall-l}, which is exactly where polymorphic types are instantiated.
The implicit argument is the same mono-expression ($\tau$) that
instantiates the polymorphic type in \rref{s-forall-l}.
The type of the instantiation result of $\Lambda$ expression is the same as the
instantiation of the polymorphic types, with the same argument.

With the observations above, the remaining proofs are finished by standard
inversion lemmas, with the help of the \emph{substitution} theorem to handle type
preservation after the instantiations.

\subsection{Equivalence to a Simplified System}
We mentioned in Section \ref{sec:type-system} that the premises boxed by dashed
lines in the unified subtyping rules are redundant.
They help in the formalization, but the calculus
is equivalent to a variant of the calculus without them.
We define unified subtyping relation
$[[G |= e1 <: e2 : A]]$, whose rules are the same as the unified subtyping rules of \name,
but with all redundant premises eliminated. Also, \rref{s-castdn,s-castup}
are simplified to use deterministic reduction ($A \longrightarrow_D B$)
instead of the reduction rule $A \longrightarrow B$ as shown below
(other rules are omitted):

\begin{drulepar}[ss]{$[[G |= e1 <: e2 : A]]$}{Simplified Unified Subtyping}
    \ottaltinferrule{ss-castdn}{width=20em}
      {[[G |= e1 <: e2 : A]] \\ \rulehl{[[A ==> B]]} \\ [[G |= B : k]]}
      {[[G |= castdn e1 <: castdn e2 : B]]}
    \and
    \ottaltinferrule{ss-castup}{width=20em}
      {[[G |= e1 <: e2 : B]] \\ \rulehl{[[A ==> B]]} \\ [[G |= A : k]]}
      {[[G |= castup [A] e1 <: castup [A] e2 : B]]}
\end{drulepar}

We prove that the two system are equivalent in terms of expressiveness.

\begin{theorem}[Equivalence of \name and the Simplification]
  If $[[G |- e1 <: e2 : A]]$ then $[[G |= e1 <: e2 : A]]$.
  And if $[[G |= e1 <: e2 : A]]$ then $[[G |- e1 <: e2 : A]]$.
\end{theorem}

\subsection{Subsumption of Polymorphic Subtyping}

Finally we show that the subtyping aspect of \name subsumes
Odersky and L\"aufer's polymorphic
subtyping~\citep{odersky1996putting}. To be more precise
we show that our unified subtyping relation subsumes DK's declarative
subtyping relation \citep{dunfield2013complete}, whose syntax and subtyping relation
are shown in figure \ref{fig:polymorphic-subtyping}.

\begin{figure}
    \begin{mathpar}
        \lift{x} = x \and \lift{[[int]]} = [[int]] \and
        \lift{A \rightarrow B} = \lift{A} \rightarrow \lift{B} \and
        \lift{\forall x. \, A} = \forall x : [[*]]. \, \lift{A} \and
    \end{mathpar}
    \begin{mathpar}
        \lift{[[nil]]} = [[nil]] \and
        \lift{\Gamma, \, x} = \lift{\Gamma},\, x : [[*]]
    \end{mathpar}
    \caption{Lifting Types and Contexts in Polymorphic Sutyping to \name}
    \label{fig:lift}
\end{figure}
Figure \ref{fig:lift} shows the transformation from
% Odersky and L\"aufer's
DK's
types to \name's types.
Then we prove the subsumption in terms of type well-formedness and subtyping
by following the interpretation of unified subtyping.

\begin{theorem}[Subsumption of Type Well-formedness]
\label{thm:type-wellformedness}
    If $\Gamma \dkvdash A$, then $\lift{\Gamma} \vdash \lift{A} : [[*]]$
\end{theorem}

Straightforward. For the case where $A = \forall x. B$, \rref{s-forall} can be
used directly bypassing the complications of \rref{s-forall-l} and
\rref{s-forall-r}.

\begin{theorem}[Subsumption of Polymorphic Subtyping]
    If $\Gamma \dkvdash A \le B$, then $\lift{\Gamma} \vdash \lift{A} \le \lift{B} : [[*]]$
\end{theorem}

The interesting cases are when the premise is derived by $\forallL$ or $\forallR$,
because of the addition of premises in our generalized system
($[[G, x : A |- B : *]]$ in \rref{s-forall-l}, $[[G |- A : *]]$ in \rref{s-forall-r}).
Both cases can be solved with the help of the \emph{well-formedness} lemma in
DK's system. We can conclude $\Gamma, x \vdash_{\text{DK}} A$ from
$\Gamma \vdash_{\text{DK}} \forall x.\, A \le B$ for the $\forallL$ case,
and conclude $\Gamma \vdash_{\text{DK}} A$ from $\Gamma \vdash_{\text{DK}} A \le \forall x.\, B$ for the $\forallR$ case.
Then Theorem~\ref{thm:type-wellformedness} can be used to solve the additional
premises in \name.

% The proof is straightforward by making use of the \emph{well-formedness} lemma
% in Odersky and L\"aufer system, mentioned in Section \ref{sec:adaptation}, to conclude that
% $A$ itself is a well-formed type in $\Gamma \vdash A \le \forall x.\, B$, and
% $\forall x. A$ is well-formed in $\Gamma \vdash \forall x. \, A \le B$.

\section{Discussions}

\subsection{The Trouble with Instantiation in Subtyping}
\label{sec:instantiation}

One of the features of DK's subtyping is that the instantiation of implicit type
parameter happen during subtyping of polymorphic types. It makes sense when
viewing the subtyping relation as a more-general-than relation. A polymorphic
type is a subtype of another when we can find specific instantiation of the
type parameter. This idea works well in DK's system, but brings troubles
in the realm of dependent types, consider the following subtyping relation:

\begin{equation*}
    A : [[*]] \vdash [[forall x : A. int]] <: [[int]] : [[*]]
\end{equation*}

The relation above is not derivable in \name, because the type of implicit type
parameter is an abstract type, for which we are not able to find an well-typed
instantiation except for the infinite loop $\mu x : A.\, x$, this is the also the
case when $A$ in an arbitrary uninhabited types (without the help of a fixpoint).

This situation also assign a special role to variables in the context,
for example:

\begin{equation*}
    A : [[*]],\, F : A \rightarrow [[*]],\, \rulehl{a : A} \vdash
    \forall x : A.\, (F~x \rightarrow [[int]]) \le (\forall x : A.\, F~x) \rightarrow [[int]] : [[*]]
\end{equation*}

Without the help of the fixpoint, this subtyping relation is only derivable
with the presence of the highlighted variable in the context. The relation has to
be derived from \rref{s-forall-l}, which requires a well-typed instantiation for
the parameter, in this case, only $x$ is eligible even though it does not occur
anywhere in the expression except for the context.

The help of fixpoint does not solve the general problem, because
in \name the fixpoint expressions are only well-typed when it is not a polymorphic
type. So the general ``Strengthening'' lemma is not admissible in \name. Even
a restricted case where we only consider typing complicated examples can still
be construct to stop us from eliminating the a variable in the context even
when it is fresh everywhere else:

\begin{equation*}
    F : [[int]] \rightarrow [[*]],\, A : [[*]],\, \rulehl{a : A} \vdash F ~ ([[(bind x : A. lambda y : int. y)]]~ 42) : [[*]]
\end{equation*}

For the time being, we think the addition of the premise in \rref{s-forall-r} and
the addition of \rref{s-forall} do not complicate the metatheory as much, so
we leave the further exploration of the issue above in a future work.

\subsection{Design Choices of the Semantics around Cast Operators}
\label{sec:cast-design}

The type reduction in cast operators is potentially under a context that
is not empty, so it is likely that we are performing reduction to a open term.
Intuitively we should generalize the definition of value by introducing inert
terms\cite{yang2017unifying} to handle open term reduction.

However since we adopts the Call-by-Name semantics of Pure Iso-type System\cite{yang2019pure},
there is no value check during the reduction, and whether the result of reduction
inside cast operator does not matter during the reasoning of type safety. It is
not necessary to complicate the metatheory by introducing inert terms.

An alternate design around cast operator is the Call-by-Value (CBV) style\cite{yang2019pure},
by not considering all $\castup$ terms as value, and performing cast elimination only
when the expression inside two casts is a value. Such design requires us to
have a more general definition for value, and thus there is a need for inert terms.

However, a simple design with CBV-style cast semantics and inert terms
potentially lead to a system where \emph{Reduction Substitution} does not hold,
for example:

\begin{gather*}
    \castdn \, \castup \, [A] \, f ~ x \longrightarrow f ~ x \\
    [\lambda x : B. \, x / f] \, \castdn \, \castup \, [A] \, f ~ x \longrightarrow \castdn \, \castup \, [A] \, x
\end{gather*}

So we stick with the Call-by-Name style semantics around cast operators and
leave the discussion of other possibilities of design in a future work.


\bibliographystyle{plainurl}
\bibliography{reference}

\end{document}
