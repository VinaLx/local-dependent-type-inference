\documentclass{elsarticle}

\usepackage[T1]{fontenc}
\usepackage{ottalt}
\usepackage{xcolor}
\usepackage{mathtools}
\usepackage{amsmath}
\usepackage{amsfonts}
\usepackage{amssymb}
\usepackage{setspace}
\usepackage{xspace}
\usepackage{mathpartir}
\usepackage{tikz-cd}
\usepackage{comment}
\usepackage{listings}
%\usepackage[english]{babel}

\inputott{dependent-subtyping.ott}

\newcommand{\rulehl}[2][gray!40]{
  \colorbox{#1}{$\displaystyle#2$}}

\newcommand{\extract}[1]{\lvert #1 \rvert}
\newcommand{\lift}[1]{\lceil #1 \rceil}
\newcommand{\castup}[0]{\mathrm{cast}_\Uparrow}
\newcommand{\castdn}[0]{\mathrm{cast}_\Downarrow}
\newcommand{\system}[0]{the dependent implicitly polymorphic calculus\xspace}
\newcommand{\System}[0]{The dependent implicitly polymorphic calculus\xspace}

\newcommand{\name}[0]{$\lambda_{I}^{\forall}$\xspace}

\newcommand{\an}[3]{{\color{#2} \textsc{#1}:#3}}
\newcommand{\bruno}[1]{\an{bruno}{blue}{#1}}
\newcommand{\alvin}[1]{\an{alvin}{red}{#1}}

\newtheorem{theorem}{Theorem}
\newtheorem{lemma}[theorem]{Lemma}
\newtheorem{corollary}[theorem]{Corollary}

\title{A Dependently Typed Calculus with Polymorphic Subtyping}

% \titlerunning{Dependently Typed Polymorphic Subtyping}

\author{Mingqi Xue}
\ead{mqxue@cs.hku.hk}
\address{The University of Hong Kong}

\author{Bruno C. d. S. Oliveira}
\ead{bruno@cs.hku.hk}
\address{The University of Hong Kong}

% \authorrunning{M. Xue and B. C. d. S. Oliveira}

% \author{Mingqi Xue \and Bruno C. d. S. Oliveira}
% \institute{
%     The University of Hong Kong \\
%     \email{\{mqxue,bruno\}@cs.hku.hk}
% }

% \Copyright{Mingqi Xue and Bruno C. d. S. Oliveira}

% \ccsdesc{Theory of computation~Type theory}
% \ccsdesc{Software and its engineering~Functional languages}



\begin{document}

\begin{abstract}
  A \emph{polymorphic subtyping} relation, which relates more general types
  to more specific ones, is at the core of many modern functional languages.
  As those languages start moving towards dependently typed programming a
  natural question is how can polymorphic subtyping be adapted to such settings.

  This paper presents \system (\name): a simple dependently typed calculus
  with polymorphic subtyping. The subtyping relation in \name
  generalizes the well-known polymorphic subtyping relation by
  Odersky and L\"aufer (1996). Because \name is dependently typed,
  integrating subtyping to the calculus is non-trivial. To overcome
  many of the issues arizing from integrating subtyping with dependent
  types, the calculus employs \emph{unified subtyping}, which is a technique
  that unifies typing and subtyping into a single relation. Moreover, \name
  employs explicit casts instead of a conversion rule, allowing
  unrestricted recursion to be naturally supported.
  We prove various non-trivial results, including \emph{type soundness} and \emph{transitivity}
  of unified subtyping. The calculus and all corresponding proofs
  are mechanized in the Coq theorem prover and we also have a simple prototype
  implementation of \name.
%A complication that arizes in the type-soundness proof
%is that a direct semantics is non-deterministic due
\end{abstract}


\begin{keyword}
  Type System \sep Dependent Type \sep Subtyping \sep Polymorphism
\end{keyword}
% \keywords{Type System, Dependent Type, Subtyping, Polymorphism}

\maketitle

\section{Overview}

In this section, we introduce \name by going through
some interesting examples to show the expressiveness and major features of the calculus.
Then we discuss the motivation, rationale of our design, and challenges.
The formal system of \name will be
discussed in Sections \ref{sec:system} and \ref{sec:metatheory}.

\subsection{A Tour of \name}
\label{sec:examples}

The \name calculus features a form of \emph{implicit
  higher-ranked polymorphism} with the power of dependent types. Thus the main feature of \name
is the ability to implicitly infer universally quantified arguments.

\paragraph{A First Example of Implicit Polymorphism}
Like most of functional languages, \name supports (implicit) parametric polymorphism.
A first simple example is the polymorphic identity
function:
\begin{flalign*}
&\mathrm{id} : \forall (A : \star).\, A \rightarrow A &&\\
&\mathrm{id} = \lambda (x : A).\, x &&\\
&\mathrm{answer} : \mathbb{N} &&\\
&\mathrm{answer} = \mathrm{id} ~ 42 \qquad \text{-- No type argument needed at the call of $\mathrm{id}$} &&
\end{flalign*}
\noindent The polymorphic parameter \verb|A| is annotated with its type,
which is $\star$. The type $\star$ is the type of types (also known as
\emph{kind}). In \name, the parameters of lambda abstractions are annotated
with their types, and the \verb|A| in the definition refers back to the
polymorphic parameter. In the examples below, we drop the parentheses around
variables and their type annotations such as $[[lambda x : A. x]]$ for conciseness.

Similarly to implicit polymorphism in other languages,
the polymorphic parameters of the $\forall$ types are implicitly instantiated
during applications. Thus, in the call of the identity function ($id~42$), we
do not need to specify the argument used for instantiation. In contrast,
in an explicitly polymorphic language (such as System F) we would need
to call $id$ with an extra argument that specifies the instantiation of $A$:
$\mathrm{id}~\mathbb{N}~ 42$.

\paragraph{Recursion and Dependent Types}

\name is dependently typed, and universal quantifications are not limited to work
on arguments of type $\star$, but it allows arguments of other types. This is
a key difference compared to much of the work on type-inference for higher-ranked
polymorphism~\cite{dunfield2013complete,le2003ml,leijen2008hmf,vytiniotis2008fph,jones2007practical}
which has been focusing on System F-like
languages where universal quantification can only have arguments of type $\star$.
Furthermore, \name supports general recursion at both the term and the type-level.

Using these features we can encode an indexed list, a \verb|map| operation over it
and illustrate how the implicit instantiation allows us to use the \verb|map|
function conveniently.
However, because \name is just a core calculus there is no built-in support
yet for algebraic datatypes and pattern matching.
We expect that a source language would provide a more convenient
way to define the \verb|map| function using pattern matching and other useful source-level
constructs. To model algebraic datatypes and pattern matching in \name, we
use an encoding by Yang and Oliveira~\cite{yang2019pure},
which is based on Scott encodings~\cite{mogensen1992efficient}.
The Scott Encoding encodes datatypes with different cases to
Continuation-Passing-Style (CPS) function types. The return branches of these
function types correspond to each case of the datatypes.
Case analyses of terms are encoded to the application of the CPS functions.
Since the details of that encoding are not relevant for this paper,
here we omit the code for most definitions and show only their types.

In a dependently typed language a programmer could write the following definition
for our formulation of indexed lists:
\newcommand{\Nat}[0]{\mathbb{N}}
\newcommand{\Succ}[0]{\mathrm{S}}
\newcommand{\Zero}[0]{\mathrm{Z}}
\newcommand{\List}[0]{\mathrm{List}}
\newcommand{\Nil}[0]{\mathrm{Nil}}
\newcommand{\Cons}[0]{\mathrm{Cons}}
\newcommand{\map}[0]{\mathrm{map}}
\begin{flalign*}
  & \mathbf{data} ~ \Nat = \Zero ~|~ \Succ~\Nat &&\\
  & \mathbf{data} ~ \List~(a : \star)~(n : \Nat) = \Nil ~ | ~ \Cons~a~(\List~a~(\Succ~n)) &&
\end{flalign*}
\noindent In this definition, the index grows towards the tail of the list,
which is admittedly not the most useful definition.
But the reason we did not choose the more practical
example, where the index represents the length of the list, is that it requires
language facilities like the GADT\cite{gadt1,gadt2} which \name does not support.
Here we encode $\mathrm{List}$ and its constructors as conventional terms. We
show the definition for \verb|List|, and the types for the constructors next
(implementation omitted):

\begin{flalign*}
&\List : \star \rightarrow \Nat \rightarrow \star &&\\
&\List = \mu L : \star \rightarrow \Nat \rightarrow \star.\, \lambda a:\star.\, \lambda n :\Nat.\, \Pi r:\star.\, r \rightarrow (a \rightarrow L~a~(\Succ ~ n) \rightarrow r) \rightarrow r &&\\
&\Nil : \forall a : \star.\, \forall n : \Nat.\, \List ~ a ~ n &&\\
&\Cons : \forall a : \star.\, \forall n : \Nat.\, a \rightarrow \List ~ a ~ (\Succ ~ n) \rightarrow \List ~ a ~n &&
\end{flalign*}
\noindent
In later subsequent examples we will just assume some
Haskell-style syntactic sugar for datatype definitions and constructors.
Using the above definitions, we can define a \verb|map| function over \verb|List| with the type:
\begin{flalign*}
  & \map : \forall a : \star.\, \forall b : \star.\, \forall n : \Nat .\, (a \rightarrow b) \rightarrow \List~a~n \rightarrow \List~b~n &&
\end{flalign*}
An example of application of \verb|map| is:
\begin{flalign*}
&\map~\Succ~(\Cons~1~(\Cons~2~\Nil)) &&
\end{flalign*}

\noindent which increases every natural in the list by one.
Note that since the type parameters for \verb|map|, \verb|Cons| and \verb|Nil|
are all implicit, they can be all omitted
and the arguments are instantiated implicitly. Thus the \verb|map| function
only requires two explicit arguments, making it as convenient to use
as in most functional language implementations.

There are a few final points worth mentioning about the example.
Firstly, \verb|List| is an example of a dependently typed function, since it is parameterized
by a natural value. Secondly, in \name (following the design of PITS~\cite{yang2019pure}),
fixpoint operators ($\mu$) serve a dual purpose of defining recursive types and recursive
functions. Besides its usual use of defining term-level general recursive functions,
it can be used to define recursive types, as shown in the encoding of \verb|List|
above.
% In \verb|List| the fixpoint operator is used to define a recursive type, whereas
% in the definition of \verb|map| the fixpoint operator is used to define term-level recursion.
Moreover, recursion is unrestricted and there is no termination checking, much like approaches
such as Dependently Typed Haskell~\cite{dh}, and unlike various other dependently typed languages
such as Agda~\cite{2007_norell_agda} or Idris~\cite{brady2013idris}.


\paragraph{Implicit Higher-Kinded Types}

The implicit capabilities also extend to the realm of higher-kinded types~\cite{tapl}.
For example we can define a record type \verb|Functor|,
to mimic the typeclass\cite{typeclasseswadler,typeclasseskaes} \verb|Functor| in Haskell:
\newcommand{\Functor}[0]{\mathrm{Functor}}
\newcommand{\MkFunctor}[0]{\mathrm{MkF}}
\newcommand{\Id}[0]{\mathrm{Id}}
\newcommand{\MkId}[0]{\mathrm{MkId}}
\newcommand{\fmap}[0]{fmap}
\begin{flalign*}
  &\mathbf{data}~\Functor~(F : \star \rightarrow \star) = \MkFunctor~\{\fmap : \forall a : \star.\, \forall b : \star.\, (a \rightarrow b) \rightarrow F~a \rightarrow F~b\} &&
\end{flalign*}
\noindent Since its a record type, $\MkFunctor$ is the data constructor,
and $\fmap$ is the field accessor.
The type of $fmap$ is:
\begin{flalign*}
  & \fmap : \forall F : \star \to \star.~\Functor~F \to \forall a : \star.\, \forall b : \star.\, (a \rightarrow b) \rightarrow F~a \rightarrow F~b &&
\end{flalign*}
Importantly this example illustrates that universal variables can quantify over higher-kinds (i.e.
$F : \star \to \star$).
We can define instances of functor in a standard way:
\begin{flalign*}
  & \mathbf{data}~\mathrm{Id}~a=\MkId~\{runId : a\} && \\
  & \mathrm{idFunctor} : \Functor~\Id && \\
  & \mathrm{idFunctor} = \MkFunctor~\{\fmap = \lambda f : a \rightarrow b.\, \lambda ~x : \Id~a.\, \MkId~(f~(runId~x))\} &&
\end{flalign*}
and then use \verb|fmap| with three arguments:
\begin{flalign*}
& \fmap~\mathrm{idFunctor}~\Succ~(\MkId~0) &&
\end{flalign*}
\noindent Note that, because our calculus has no mechanism like typeclasses we pass the ``instance'' explicitly.
Nonetheless, three other arguments (the $F$, $a$, and $b$) are implicitly instantiated.

\paragraph{Higher-Ranked Polymorphic Subtyping}
\label{sec:higher-ranked-poly}

In calculi such as the ICC~\cite{miquel2001implicit}, a form of implicit instantiation also exists.
However, such calculi do not employ subtyping, instead, they only apply instantiation
to top-level universal quantifiers. Our next example illustrates how subtyping enables
instantiation to be applied also in nested universal quantifiers, thus enabling
more types to be related.

When programming with continuations~\cite{sussman1998scheme} one of the
functions that are typically needed is call-with-current-continutation
(\verb|callcc|). In a polymorphic language, there are several types that can be
assigned to \verb|callcc|. One of these types is a rank-3 type,
while another one is a rank-1 type.
Using polymorphic subtyping we can show that the rank-3
type is more general than the rank-1 type. Thus the following program type-checks:
\begin{flalign*}
& \mathrm{callcc}' : \forall a : \star.\, ((\forall b : \star.\, a \rightarrow b) \rightarrow a) \rightarrow a && \\
& \mathrm{callcc} : \forall a : \star.\, \forall b : \star.\, ((a \rightarrow b) \rightarrow a) \rightarrow a && \\
& \mathrm{callcc} = \mathrm{callcc}' &&
\end{flalign*}
\noindent The type $\forall b : \star.\, a \rightarrow b$ appears in a positive position
of the whole signature, and it is a more general signature than $a \rightarrow b$
for an arbitrary choice of $b$. Our language captures this subtyping relation so that
we can assign $\mathrm{callcc}'$ to $\mathrm{callcc}$ (but not the other way around).
In contrast, in approaches like the ICC, the types of \verb|callcc| and \verb|callcc'|
are not compatible and the example above would be rejected.

\subsection{Key Features}

We briefly discuss the major features of \name itself and
its formalization. A more formal and technical discussion will be left to
Sections \ref{sec:system} and \ref{sec:metatheory}.

\paragraph{Polymorphic Subtyping Relation}
Figure \ref{fig:polymorphic-subtyping} shows the syntax of types, monomorphic types (or monotypes),
and the polymorphic subtyping relation in
Odersky and L\"aufer's declarative type system \cite{odersky1996putting}.
Here the syntax includes polymorphic types (or polytypes), which are universally quantified over type parameters
($\forall$ types). The definition of monotypes
is standard, and includes all types without occurences of universal quantifiers.
Context $\Gamma$ is a list of variables that are allowed to occur free in types
$A$ and $B$ in the subtyping relation.
The polymorphic subtyping relation captures a \emph{more-general-than} relation
between types. The key rules in their subtyping relation are rules $\forallL$
and $\forallR$:

\begin{itemize}
  \item In rule $\forallL$, a polytype ($\forall x.\, A$) is considered \emph{more-general}
        than another type ($B$), when we can find an arbitrary monotype ($\tau$)
        so that the instantiation is more general than $B$.
        Importantly note that this relation does not guess arbitrary (poly)types,
        but just monotypes. In other words, the relation is \emph{predicative}~\cite{Martin-Lof-1972}.
        This restriction ensures that the relation is \emph{decidable}.

  \item In rule $\forallR$ a type ($A$) is considered more general than a polytype ($\forall x. B$)
        when it is still more general than the head of the polytype, with the type
        parameter instantiated by an abstract variable $x$.
\end{itemize}

This subtyping relation sets a scene for our work, which
generalizes this relation to a dependently typed setting.

\begin{figure}
\centering
\begin{equation*}
\begin{array}{llcl}
  \text{Types} & A, B & ~\Coloneqq~ & [[x]] \mid [[int]] \mid A \rightarrow B \mid \forall x.\, A \\
  \text{Monotypes} & \tau, \sigma & ~\Coloneqq~ & [[x]] \mid [[int]] \mid \tau \rightarrow \sigma
\end{array}
\end{equation*}
\begin{drulepar}{$\Gamma \vdash A \le B$}{Polymorphic Subtyping}
  \inferrule*[lab=$\tau$]
    { }
    {\Gamma \vdash \tau \le \tau}
  \and
  \inferrule*[lab=$\rightarrow$]
    {\Gamma \vdash B_1 \le A_1 \\ \Gamma \vdash A_2 \le B_2}
    {\Gamma \vdash A_1 \rightarrow A_2 \le B_1 \rightarrow B_2}
  \\
  \inferrule*[lab=$\forallL$]
    {\Gamma \vdash \tau \\ \Gamma \vdash [\tau / x]\, A \le B}
    {\Gamma \vdash \forall x.\, A \le B}
  \and
  \inferrule*[lab=$\forallR$]
    {\Gamma ,\, x \vdash A \le B}
    {\Gamma \vdash A \le \forall x.\, B}
\end{drulepar}

\caption{The polymorphic subtyping relation by Odersky and L\"aufer~\cite{odersky1996putting}.}
\label{fig:polymorphic-subtyping}
\end{figure}

\paragraph{Generalizing Polymorphic Subtyping}
\label{sec:polymorphic-subtyping}

The parameters of universal types can only be types in the polymorphic
subtyping relation by Odersky and L\"aufer.
In \name, we generalize the polymorphic parameters so that they can
be values or other kinds of types as well.
A first idea of a direct generalization is:

\begin{mathpar}
  \inferrule*[lab=$\forallL'$]
    {\Gamma \vdash \tau \rulehl{: A} \\ \Gamma \vdash [\tau / x]\, B \le C}
    {\Gamma \vdash \forall x \rulehl{: A}.\, B \le C}
  \and
  \inferrule*[lab=$\forallR'$]
    {\Gamma ,\, x \rulehl{: B} \vdash A \le C}
    {\Gamma \vdash A \le \forall x \rulehl{: B}.\,C}
\end{mathpar}

\noindent The parameters for universal types can have any type (and not just $\star$).
Hence, instead of requiring the monotype $\tau$ to be a well-formed type in rule
$\forallL$, in rule $\forallL'$ it is
required that $\tau$ is well-typed regarding the type of the parameter
in the universal quantifier.
Similarly, for rule $\forallR'$ the context for the subtyping rule should include typing information
for the universally quantified variable.
However, this idea introduces the issue of potential mutual dependency between
subtyping and typing judgements, so further adjustments have to be made to formalize
this idea, which is discussed later in this section and Sections
\ref{sec:type-system} and \ref{sec:adaptation}.

\paragraph{Higher-Ranked Polymorphic Subtyping}

As the \verb|callcc| example in Section \ref{sec:higher-ranked-poly} shows, these subtyping
rules based on polymorphic subtyping, combined with other subtyping rules,
are able to handle the subtyping relations that occur at not only top-level,
but also at a higher-ranked level. This feature distinguishes our \name from the
Implicit Calculus of Constructions (ICC) \cite{miquel2001implicit} which also talks about
the implicit polymorphism of dependent type languages. The ICC features the two related rules
in the \emph{typing relation}:

\begin{mathpar}
  \inferrule*[lab=inst]
    {[[G |- e : forall x : A. B]] \\ [[G |- e1 : A]]}
    {[[G |- e : [e1 / x] B]]}
  \and
  \inferrule*[lab=gen]
    {[[G, x : A |- e : B]] \\ [[G |- forall x : A. B : k]]}
    {[[G |- e : forall x : A. B]]}
\end{mathpar}

\noindent Without an explicit subtyping relation, the ICC is not always able
to handle subtyping at higher-ranked positions. The approach taken by the ICC
is similar to that of the Hindley-Milner type system~\cite{hindley1969principal,damas1982principal},
which is also designed for dealing only with rank-1 polymorphism.
Hindley-Milner's declarative system also has a \textsc{GEN} rule to
convert expressions to polymorphic types, and a
\textsc{INST} rule to instantiate polymorphic parameters.
Both rules work only
for polymorphic types at top-level positions. In Hindley-Milner
the universal quantifier can only quantify over types, whereas in the ICC
it can quantify over terms of an arbitrary type (including types themselves).
% which also features similar rules in typing.
% But Hindley-Milner is designed for dealing only with rank-1 polymorphism.
In generalizations of higher-ranked polymorphic
type-inference~\cite{dunfield2013complete,le2003ml,leijen2008hmf,vytiniotis2008fph,jones2007practical},
it has been shown that rules like $\forallL$ and $\forallR$ generalize rules like
\textsc{GEN} and \textsc{INST}. Since we aim at higher-ranked polymorphic generalization,
we follow a similar, more general, approach in \name.

\paragraph{Unified Subtyping}
The revised subtyping relation with $\forallL'$ and $\forallR'$ rules suffers from an
important complication compared to the Odersky and L\"aufer formulation: there is now
a notorious mutual dependency between typing and subtyping.
In Odersky and L\"aufer's rules, the subtyping rules
do not depend on typing. In particular
the rule $\forallL$ depends only on well-formedness ($\Gamma \vdash \tau$).
In contrast, note that rule $\forallL'$ now mentions the typing relation
in its premise ($\Gamma \vdash \tau : A$). Moreover, as usual,
the subsumption rule of
the typing relation (shown below) depends on the subtyping relation.
\begin{mathpar}
  \inferrule*[lab=t-sub]
    {\Gamma \vdash e : A \\ \Gamma \vdash A \le B}
    {\Gamma \vdash e : B}
\end{mathpar}
This mutual dependency problem has been a significant
problem when combining subtyping and dependent types~\cite{subdep, hutchins},
and presents itself on our way to the direct generalization of polymorphic subtyping.

To tackle this issue, we adopt a technique called
\emph{unified subtyping}~\cite{full}. Unified subtyping merges the typing relation and
subtyping relation into a single relation to avoid the mutual dependency:
\begin{mathpar}
  \Gamma \vdash e_1 \le e_2 : A
\end{mathpar}
The interpretation of this judgement is: under context $\Gamma$, $e_1$ is a subtype
of $e_2$ and they both are of type $A$. The judgements for subtyping and typing
are both special forms of unified subtyping: % with the involvement of kind $[[*]]$:
\begin{mathpar}
  \Gamma \vdash A \le B \triangleq \Gamma \vdash A \le B : [[*]]
  \and
  \Gamma \vdash e : A \triangleq \Gamma \vdash e \le e : A
\end{mathpar}
The technique simplifies the formalization of dependently typed calculi with subtyping,
and especially the proof of transitivity in the original work. Ideally after applying the technique,
the generalization of the polymorphic subtyping should be:

\begin{mathpar}
  \inferrule*[lab=$\le\forall'' L$]
    {\Gamma \vdash \tau : A \\ \Gamma \vdash [\tau / x]\, B \le C \rulehl{: [[*]]}}
    {\Gamma \vdash \forall x : A.\, B \le C \rulehl{: [[*]]}}
  \and
  \inferrule*[lab=$\le\forall'' R$]
    {\Gamma ,\, x : B \vdash A \le C \rulehl{: [[*]]}}
    {\Gamma \vdash A \le \forall x : B.\,C \rulehl{: [[*]]}}
\end{mathpar}

\noindent The basic idea of our own formalization essentially follows a similar design,
although the actual rules in \name are slightly more sophisticated.
The details will be discussed in Section \ref{sec:type-system}.

\paragraph{``Explicit'' Implicit Instantiation}

With polymorphic subtyping the instantiation of universally quantified type
parameters is done implicitly instead of being manually applied. In non-dependently
typed systems, \emph{implicit} parameters are types (i.e. terms are not involved in
implicit instantiation). For example:
\begin{mathpar}
  (\lambda x.\, x)~42 \longrightarrow 42
\end{mathpar}
\noindent Here $\lambda x.\, x$ has type $\forall A.\, A \rightarrow A$, and
instantiation implicitly discovers that $A = Int$.
Notably, and in contrast with explicitly polymorphic languages like System F, implicit
instantiation is not reflected anywhere at term level.
The design that we adopt still provides implicit instantiation, but
it is more explicit regarding the binding of implicit parameters.
We adopt this design to ensure that polymorphic variables are well-scoped in
type annotations of terms. Thus we use another binder, of the form $\Lambda(x : A). e$, for terms.
Nonetheless, instantiations are still
implicit as shown in the following example:
\begin{mathpar}
  (\Lambda A : [[*]].\, \lambda x : A.\, x) ~ 42 \longrightarrow 42
\end{mathpar}
Here $\Lambda A : [[*]].\, \lambda x : A.\, x$ has type $\forall A : \star. \, A \rightarrow A$,
and the polymorphic parameter $A$ is explicitly stated in the polymorphic
term. However as the reduction shows, the instantiations are still implicit.
We purposely omitted the explicit binders for implicit parameters for all the examples
in Section \ref{sec:examples} for conciseness. Such explicit binders can
be recovered with a simple form of syntactic sugar:

\[e : \forall x : A.\, B \triangleq \Lambda x : A.\, e : \forall x : A.\, B\]

%\noindent When polymorphic parameters are used, \name provides a binder ($\Lambda x : A.\, e$)
%to ensure that the parameters are well-scoped at the term-level.

\paragraph{Computational Irrelevance}
\label{sec:computational-irrelevance-overview}

Implicit parameters in traditional languages with polymorphic subtyping,
the ICC~\cite{miquel2001implicit,barras2008implicit} and \name are computationaly irrelevant.
In traditional (non-dependently) typed languages, types cannot affect computation,
thus computational irrelevance is quite natural and widely adopted.
Furthermore, computational irrelevance can benefit performance, since
irrelevant arguments can simply be erased at runtime.
In dependently typed systems, however, there can be some programs where
it is useful to have computationaly relevant implicit parameters.
For example, accessing the length of a length indexed vector in constant time:
\begin{flalign*}
  &\mathrm{length} : \forall n : [[int]].\, \mathrm{Vector}~n \rightarrow [[int]] &&\\
  &\mathrm{length} = \Lambda n : [[int]].\, \lambda v : \mathrm{Vector}~n.\, n
\end{flalign*}
\noindent Here the implicit parameter $n$ is computationally relevant as it is used as
the return value of the function which is likely to be executed at runtime.
Languages like Agda, Coq or Idris support such programs. However,
computationaly relevant implicit parameters are challenging for proofs of
type soundness. Due to such challenges (see also the discussion in
Section~\ref{subsec:semantics}),
the ICC has a restriction that parameters for implicit function types
must be computationally irrelevant. Since we adopt a similar technique for the type
soundness proof, we also have a similar restriction and thus cannot encode programs such
as the above.

\paragraph{Type-level Computation and Casts}
\name features a fixpoint operator that supports general recursion at both
type and term level. In order to avoid diverging computations at type checking,
we do not provide the conversion rule (or congruence rule) like other
dependently typed systems such as the Calculus of Constructions~\cite{coc}
to support implicit type-level reduction.
\begin{mathpar}
  \inferrule*[lab=Cong]
    {[[G |- e : A]] \\ \rulehl{A =_\beta B}}
    {[[G |- e : B]]}
\end{mathpar}

\noindent The presence of the conversion rule makes the decidability of
type checking rely on the strong normalization of type-level computation
(to determine whether two types are $\beta$-equivalent).
But the presence of general recursion denies the strong normalization property
of our language.

Instead of using a conversion rule, we adopt the call-by-name design of
\emph{Pure Iso-Type Systems} (PITS)~\cite{isotype,yang2019pure},
and provide $\castdn$ and $\castup$ operators to explicitly trigger one-step
type reductions or expansions as shown in the typing rules below.
\begin{mathpar}
  \inferrule*[lab=Castup]
    {[[G |- e : B]] \\ \rulehl{[[A --> B]]} \\ [[G |- A : k]]}
    {[[G |- castup [A] e : A]]}
  \and
  \inferrule*[lab=Castdn]
    {[[G |- e : A]] \\ \rulehl{[[A --> B]]} \\ [[G |- B : k]]}
    {[[G |- castdn e : B]]}
\end{mathpar}

\noindent Now since reductions only perform one step per use of cast
operators, whether a term is strongly normalizing no longer affects the
decidability of type checking.
Note that there are some other cast designs in the
literature~\cite{guru,sjoberg:msfp12, kimmel:plpv, zombie:popl15}, but
we adopt the PITS design here for simplicity. We believe that other cast
designs could also be adopted instead, but leave this for future work.

\section{\System}
\label{sec:system}

This section introduces the static and dynamic semantics of
\name: a dependently typed calculus with type casts
and implicit polymorphism. The calculus employs
\emph{unified subtyping}~\cite{CoquandThierry1988Tcoc}
and has a single relation that generalizes both typing and subtyping.
The calculus can be seen as a variant of the \emph{calculus of constructions}~\cite{},
but it uses a simple form of casts~\cite{} instead of the conversion rule
and features unrestricted recursion. We present syntax, unified subtyping
and reduction for \name.

\begin{figure}[t]
\centering
\begin{equation*}
\begin{array}{llcl}
    \text{Kinds} & k & ~\Coloneqq ~ & [[*]] \mid [[box]] \\
    \text{Expressions} & e, A, B & ~ \Coloneqq ~ & [[x]] \mid [[n]] \mid [[k]] \mid [[int]] \mid [[e1 e2]] \mid [[lambda x : A. e]] \mid [[pi x : A. B]] \\
        & & \mid & [[bind x : A. e]] \mid [[forall x : A. B]] \mid [[mu x : A. e]] \\
        & & \mid & [[castup [A] e]] \mid [[castdn e]]   \\
    \text{Mono-Expressions} ~ & \tau, \sigma & ~ \Coloneqq ~ & [[x]] \mid [[n]] \mid [[k]] \mid [[int]] \mid \tau_1 ~ \tau_2 \mid \lambda \, x : \tau. ~ \sigma \mid \Pi \, x : \tau. ~ \sigma \\
        & & \mid & \Lambda \, x : \tau. ~ \sigma \mid \mu \, x : \tau. ~ \sigma \mid \castup \, [\tau]~ \sigma \mid \castdn \, \tau \\
    \text{Values} & v & ~ \Coloneqq ~ & [[k]] \mid [[n]] \mid [[int]] \mid [[lambda x : A. e]] \mid [[pi x : A. B]] \mid [[bind x : A. e]] \\
        & & \mid & [[forall x : A. B]] \mid [[castup [A] e]] \\
    \text{Contexts} & \Gamma & ~ \Coloneqq ~ & [[nil]] \mid [[G , x : A]] \\
    \text{Syntactic Sugar} ~ & A \rightarrow B & \triangleq & [[pi x : A. B]] \qquad \text{where} ~ x \notin \mathrm{FV}(B)
\end{array}
\end{equation*}
\caption{Syntax of \name.}
\label{fig:syntax}
\end{figure}
\subsection{Syntax}

Figure \ref{fig:syntax} presents the syntax of \name. The syntax is similar
syntax to the Calculus of Constructions, featuring
$[[*]]$ and $[[box]]$ in the kind hierarchy, and unifying the concepts of terms
and types as expressions. Due to the unified syntax, types and
expressions ($e$, $A$ and $B$) are used
interchangeably, although we mostly adopt the convention of using $A$ and $B$
for contexts where the expressions are used as types and $e$ for contexts
where the expressions represent terms.
The syntax includes all the constructs of the calculus of constructions:
variables ($[[x]]$), kinds ($[[k]]$), function applications  ($[[e1 e2]]$),
lambda expressions ($[[lambda x : A. e]]$), dependent function types ($[[pi x : A. B]]$)
as well as integer types ($[[int]]$) and integers ($[[n]]$).
Moreover, there are a number of additional language constructs for
supporting implicit polymorphism, recursion and explicit type-level computation
via casts. These constructs are discussed next.

\subsubsection{Implicit Polymorphism.}
In \name, universal types $[[forall x : A. B]]$ are used to generalize implicit
polymorphism in non-dependent.
In contrast to universal quantification in conventional functional languages, the
argument $x$ ranges over all well-typed expressions besides well-formed
types (i.e. $x$ can have any type $A$ instead of just kind $\star$).
In other words, ``polymorphic types'' are naturally dependent, so $\forall$
types can be viewed as the implicit counterpart of $\Pi$ types. We also have
implicit lambda expressions ($[[bind x : A. e]]$), which are the implicit counterpart of
  $\lambda$ abstractions. The design used in \name
  is similar to the design of the \emph{Implicit Calculus of Constructions} ($\text{ICC}^*$)~\cite{barras2008implicit}, which
  employs similar constructs for implicit dependent products.
Like conventional universal quantification, the arguments of $\forall$ types are
deduced during applications rather than being explicitly passed.
In addition, following designs for predicative higher-ranked polymorphism~\cite{oderskylufer,DK,PJ}, we have also generalized the concept of \emph{monotypes} to
\emph{mono-expressions} ($\tau$), essentially excluding $\forall$ types from expressions.

\subsubsection{Recursion and Explicit Type-level Computation.}
\label{sec:cast}
The \name calculus adopts \emph{iso-types}~\cite{yang2016unified,yang2019pure},
featuring explicit type-level computation with cast operators
$\castdn$ and $\castup$. These operators respectively perform one-step
type reduction and type expansion based on the operational semantics.
The reduction in cast operators is deterministic, thus type
annotations are only needed during type expansions ($\castup$). We add
fixpoints ($[[mu x : A. e]]$) to support general recursion for both
term-level and type-level. Iso-recursive types are supported due to
the presence of $\castup$ and $\castdn$, which correspond to the
\verb|fold| and \verb|unfold| operations when working on iso-recursive types.

\subsection{Operational Semantics}

% \bruno{If I understand correctly we need 2 different reductions, one
%   (the non-deterministic one) that is used in the type system; and another
%   one (which erases types) that is deterministic and would be the basis
%   for an actual implementation of reduction at run-time. After reading
%   this subsection, I think we want
% to tell that story here and present the 2 variants of reduction here.}

For the operational semantics we employ two different, but closely related,
reduction relations. The first reduction relation is non-deterministic, and
it is used at the type-level to allow type conversions induced
by the cast operators. The second reduction relation is deterministic and
is employed to give the run-time semantics of expressions.

\subsubsection{Nondeterministic Reduction.}
Figure \ref{fig:semantics} presents the small-step operational semantics of our system,
which mostly follows the call-by-name variant of \emph{Pure Iso-Type Systems} (PITS)
\cite{yang2019pure} corresponding to the calculus of constructions.
Note that the arguments of $\beta$-reduction (\rref{r-beta}) and expressions in
the \rref{r-cast-elim} are not required to be values.
Meanwhile we consider $\castup$ terms to be a value,
and only perform reduction inside $\castdn$ terms (\rref{r-castdn}). Also, the unroll
operation of the fixpoint operator is supported by \rref{r-mu}.

\begin{figure}[t]
    \centering

    \begin{drulepar}[r]{$[[e1 --> e2]]$}{Operational Semantics}
      \drule{app}
      \drule{beta}
      \and \ottaltinferrule{r-inst}{}{ }
        {[[(bind x : A. e1) e2 --> ([t / x] e1) e2]]}
      \drule{mu}
      \drule{castdn}
      \and \ottaltinferrule{r-cast-inst}{}{ }
        {[[castdn (bind x : A. e) --> castdn ([t / x] e)]]}
      \drule{castXXelim}
    \end{drulepar}

    \caption{Operational semantics of \name.}
    \label{fig:semantics}
\end{figure}
% \bruno{Do not use $mono~e$ in the figure.
%   You have the syntax $\tau$ and $\sigma$ to represent
% monotypes, so just use that instead.}

%\subsubsection{Nondeterministic Implicit Instantiations}
Due to the presence of instantiation of implicit parameters, the direct operational
semantics is not deterministic, and potentially not type-preserving because of
\rref{r-inst,r-cast-inst}. The indeterminacy is caused by the guess of $\tau$,
which can be an arbitrary monoexpression, since we do not have access to typing
information in the dynamic semantics.

\begin{figure}
  \label{fig:extraction}
  \centering
  \begin{equation*}
  \begin{array}{llcl}
      \text{Erased Expressions} & e, A, B & ~ \Coloneqq ~ & [[x]] \mid [[n]] \mid [[k]] \mid [[int]] \mid [[ee1 ee2]] \mid [[elambda x. ee]] \mid [[epi x : eA. eB]] \\
      & & \mid & [[ebind x. ee]] \mid [[eforall x : eA. eB]] \mid [[emu x. ee]] \mid [[ecastup ee]] \mid [[ecastdn ee]] \\
      \text{Erased Value} & ev & ~ \Coloneqq ~ & [[k]] \mid [[n]] \mid [[int]] \mid [[elambda x. ee]] \mid [[epi x : eA. eB]] \mid [[ebind x. ee]] \\
      & & \mid & [[eforall x : eA. eB]] \mid [[ecastup ee]]
  \end{array}
  \end{equation*}

  % The behavior of gather and align under this lipics template is extremely
  % weird, this is the best I can do :(
  \begin{gather*}
    \begin{align*}
    \extract{[[x]]} &= [[x]] &
    \extract{[[n]]} &= [[n]] &
    \extract{[[k]]} &= [[k]] &
    \extract{[[int]]} &= [[int]]
    \end{align*} \\
    \begin{align*}
     \extract{[[e1 e2]]} &= \extract{[[e1]]} ~ \extract{[[e2]]} & % \\ % \and
      \extract{[[mu x : A. e]]} &= \mu \, x. ~ \extract{[[e]]} \\
     \extract{[[lambda x : A. e]]} &= \lambda \, x. ~ \extract{[[e]]} & % \\ % \and
      \extract{[[pi x : A. B]]} &= \Pi \, x : \extract{[[A]]}. ~ \extract{[[B]]} \\
     \extract{[[bind x : A. e]]} &= \Lambda \, x. ~ \extract{[[e]]} & % \\ % \and
      \extract{[[forall x : A. B]]} &= \forall \, x : \extract{[[A]]}. ~ \extract{[[B]]} \\
     \extract{[[castup [A] e]]} &= \castup \, \extract{[[e]]} & % \\ % \and
      \extract{[[castdn e]]} &= \castdn \, \extract{[[e]]}
    \end{align*}
  \end{gather*}

  \drules[er]{$[[ee1 *--> ee2]]$}{Erased Semantics}
    {app,beta,elim,mu,castdn,castXXinst,castXXelim}
  \caption{Erased Expressions and Operational Semantics}
\end{figure}
% \bruno{You need to use something like a latex table/tabular,
% to neatly align the erasure function.}

\subsubsection{Deterministic Reduction}
We address the issue of determinancy of the dynamic semantics with
a design similar to $\mathrm{ICC}^*$ \cite{barras2008implicit},
employing type-erased expressions. The erased expressions
essentially mirror the syntax and semantics
of normal expressions, except for the elimination of type annotations in $\lambda$,
$\Lambda$, $\mu$ and $\castup$ expressions.
Figure \ref{fig:extraction} shows the syntax of the erased expressions and
the companion operational semantics. Note that restrictions are imposed in the
typing rules to forbid the implicit parameter occuring in runtime-relevant part
of the expression, i.e. the erased expressions (see section \ref{sec:type-system}).
With such restriction, implicit parameters can be directly eliminated in
\rref{er-elim,er-cast-inst}. For a well-typed expression, the reduction of
its erasure is deterministic. The type safety of our system is built around
this idea and is discussed in \ref{sec:type-safety}.

\begin{figure}
    \centering
    \begin{drulepar}[wf]{$[[|- G]]$}{Well-formed Context}
      \mprset{sep=1.2em}
      \drule{nil}
      \drule[width=30em]{cons}
    \end{drulepar}

    \begin{drulepar}[s]{$[[G |- e1 <: e2 : A]]$}{Unified Subtyping}
      \mprset{sep=1.3em}
      \drule{var}
      \drule{lit}
      \drule{int}
      \drule{star}
      % \drule{abs}
      \and \ottaltinferrule{abs}{width=20em}
        {\rulehl{[[G |- A : k1]]} \\ [[G, x : A |- B : k2]] \\ [[G, x : A |- e1 <: e2 : B]]}
        {[[G |- lambda x : A. e1 <: lambda x : A. e2 : pi x : A. B]]}
      % \drule{app}
      \and \ottaltinferrule{s-app}{}
        {[[G |- t : A]] \\ [[G |- e1 <: e2 : pi x : A. B]]}
        {[[G |- e1 t <: e2 t : [t / x] B]]}
      \drule{pi}
      % \drule{bind}
      \and \ottaltinferrule{s-mu}{width=20em}
        {[[G |- t : k]] \\ [[G , x : t |- s : t]]}
        {[[G |- mu x : t. s <: mu x : t. s : t]]}
      \and \ottaltinferrule{s-bind}{width=20em}
        {\rulehl{[[G |- A : k]]} \\ [[G , x : A |- B : *]] \\
         [[G, x : A |- e1 <: e2 : B]] \\
         x \notin \mathrm{FV}(\extract{[[e1]]}) \cup \mathrm{FV}(\extract{[[e2]]})}
        {[[G |- bind x : A. e1 <: bind x : A. e2 : forall x : A. B]]}
      % \drule{mu}
      \drule{castup}
      \drule{castdn}
      % \drule{forallXXl}
      \and \ottaltinferrule{s-forall-l}{width=20em}
        {\rulehl{[[G |- A : k]]} \\ [[G |- t : A]] \\
         [[G , x : A |- B : *]] \\
         [[G |- [t / x] B <: C : *]]}
        {[[G |- forall x : A. B <: C : *]]}
      % \drule{forallXXr}
      \and \ottaltinferrule{s-forall-r}{width=20em}
        {\rulehl{[[G |- B : k]]} \\ [[G |- A : *]] \\
         [[G , x : B |- A <: C : *]]}
        {[[G |- A <: forall x : B. C : *]]}
      % \drule{forall}
      \and \ottaltinferrule{s-forall}{width=20em}
        {\rulehl{[[G |- A : k]]} \\ [[G , x : A |- B <: C : *]]}
        {[[G |- forall x : A. B <: forall x : A. C : *]]}
      \and \ottaltinferrule{s-sub}{width=15em}
        {[[G |- e1 <: e2 : A]] \\ [[G |- A <: B : k]]}
        {[[G |- e1 <: e2 : B]]}
    \end{drulepar}

    % \drules[s]{$[[G |- e1 <: e2 : A]]$}{Unified Subtyping}{
    %     var,lit,int,star,abs,pi,app,bind,mu,castup,castdn,forallXXl,forallXXr,forall,sub}
    \begin{equation*}
       \text{Syntactic Sugar} \qquad [[G |- e : A]] \triangleq [[G |- e <: e : A]]
    \end{equation*}
    \caption{(Sub)Typing Rules of \name.}
    \label{fig:typing}
\end{figure}
% \bruno{The figure and rules will need a little cosmetic work. Firstly, there are many rules
%   stacking up premises vertically. I think it is better to have multiple rules
%   horizontally (and only upto 2 or 3 vertical stacks of premises).
%   Secondly, we need to look at the layout carefully to use space efficiently.
%   At the moment there are rules like S-ABS that could be paired up with some other
%   rules side-by-side and use less space (at the same time it is nicer
%   if adjancent rules are somehow related: two cast rules; application/abstraction, etc).
%   We must organize the rules in a nicer way. Thirdly, I think that, for binders with
%   annotations, like $\lambda x : A. e$, we may want to use brackets on the arguments
%   to improve readability, as in $\lambda (x : A). e$. It is a bit hard to ``parse''
%   the syntax without a little bit of effort. If this change is implemented it would
%   affect the whole section, starting from syntax.
% }

\subsection{Unified Subtyping System}
\label{sec:type-system}

Figure \ref{fig:typing} shows the (sub)typing rules of the system. We adopt a
simplified design of unified subtyping~\cite{yang2017unifying}, where the subtyping rules and
typing rules are merged into a single typing judgment $[[G |- e1 <: e2 : A]]$.
% The originals design lean towards object-oriented features,
% supporting generalized top type and bounded quantification,
% while our work focus on the subtyping relation between polymorphic types.
% \bruno{The previous sentence is something to be mentioned in related work, not here.}

Unified subtyping solves the challenging issue of mutual dependency between typing
and subtyping in a dependently type system.
% \bruno{Make sure that the overview discusses this issue.}
The interpretation of this judgment is ``under context $[[G]]$, $[[e1]]$ is a
subtype of $[[e2]]$ and they are both of type $[[A]]$''.
In this form of formalization, the typing judgment $[[G |- e : A]]$ is a
special case of unified subtyping judgment $[[G |- e <: e : A]]$,
and the well-formedness of types $\Gamma \vdash A$ is expressed by
$[[G |- A : k]]$ where $k \in \{[[*]], [[box]]\}$.

Although unified, the typing rules can still be viewed as two parts, the ``typing'' part
(\rref{s-abs,s-app,s-bind,s-mu,s-castup,s-castdn,s-sub}) and the ``subtyping'' part
(\rref{s-pi,s-forall,s-forall-l,s-forall-r}). We follow a usual design for
typing rules for lambda abstraction and application, and the subtyping rule of
dependent function types ($\Pi$ type).
% \bruno{We should say something about the ``more standard'' rules here.
%   I think we can briefly explain them and point the
%   reader to Linus work for further details, while observing that the rules
%   here are essentially simplified version (due to the absence of bounded quantification)
% of his rules.}

\subsubsection{Rules for Universal Quantification}
The subtyping rules for universal quantification (\rref{s-forall-l,s-forall-r}) follow
the spirit of the Odersky and L\"aufer's polymorphic subtyping~\cite{odersky1996putting,DunfieldJoshua2013Caeb},
where the subtyping relation is interpreted as a ``more general than'' relation.
A polymorphic type $[[forall x : A. B]]$
is more general than another type $C$ when its well-typed
instantiation is more general than $C$ (\rref{s-forall-l}). A polymorphic
type $[[forall x : B. C]]$ is less general than a type $A$,
if $C$ is is less general than $A$ when the argument with the polytype ($x:B$)
is abstracted out (\rref{s-forall-r}).

Notably our formalization is not a direct generalization of the polymorphic subtyping,
\rref{s-forall} axiomatizes the subtyping relation between two polymorphic types.
And additional premises are added to \rref{s-forall-l,s-forall-r} aside from the
ones we previously mentioned in section \label{sec:polymorphic-subtyping}
The motivations for these changes are discussed in more detail in section \ref{sec:adaptation}.

\subsubsection{Mono-expression Restrictions}
As in other predicative relations (such as the one by Odersky and L\"aufer),
the type arguments for instantiation in \rref{s-forall-l} are
required to be mono-expressions, which has cascading effects on typing rules of
other expressions. The arguments for applications are required to be
mono-expressions, and the whole fixpoint expression is required to be a
mono-expressions. We shall
discuss the reason behind the restrictions in later sections.
\bruno{I feel that we may be deferring a bit too much explanation
to later sections, but lets come back to this after you write later sections.}

\subsubsection{Kind Restriction for Universal Types}
\label{sec:kind-restriction}

For the kinding of types, we mainly follow the design of the Calculus of
Constructions~\cite{CoquandThierry1988Tcoc}. However, we specifically restrict
the $[[forall x : A. B]]$
expressions to only have the kind $[[*]]$. This prevents other types of kind
$[[*]]$ from having kinds such as $[[forall x : int. *]]$,
which significantly complicates the metatheory when reasoning about the kind of types.
This restriction propagates to the introduction rule of $\forall$ types (\rref{s-bind}),
where $[[B]]$ is required to only have kind $[[*]]$.
This way well-typed implicit abstractions ($\Lambda$ expressions) are kept away
from type computations. Therefore, in cast operators,
the possibility of non-deterministic implicit instantiations is eliminated.

\subsubsection{Runtime Irrelevance of Implicit Arguments}

Our direct operational semantics choose random mono-expressions to instantiate
the implicit arguments which potentially breaks type safety, so we adopt a
restriction in \rref{s-bind} that is similar to the
Implicit Calculus of Constructions (ICC) \cite{miquel2001implicit}.
We only allowing the implicit parameters to occur in type annotations in the
body of implicit abstraction, so that the choices of implicit parameters is not
relevant at runtime. The type safety of the direct operational semantics is
proved indirectly in section \ref{sec:type-safety} with the help of the
erasure of expressions.

\subsubsection{Redundant Premises}

All the premises highlight in gray are redundant in a way that
the system without them is proved equivalent to the system in figure \ref{fig:typing}.
They are there to simplify the mechanized proofs of certain lemmas.


\section{Metatheory}

\begin{theorem}[Weakening]
    If $[[G1]] , [[G2 |- e1 <: e2 : A]]$, then $[[G1 , x : B]] , [[G2 |- e1 <: e2 : A]]$
\end{theorem}

\section{Discussions}

\subsection{The Trouble with Instantiation in Subtyping}
\label{sec:instantiation}

One of the features of DK's subtyping is that the instantiation of implicit type
parameter happen during subtyping of polymorphic types. It makes sense when
viewing the subtyping relation as a more-general-than relation. A polymorphic
type is a subtype of another when we can find specific instantiation of the
type parameter. This idea works well in DK's system, but brings troubles
in the realm of dependent types, consider the following subtyping relation:

\begin{equation*}
    A : [[*]] \vdash [[forall x : A. int]] <: [[int]] : [[*]]
\end{equation*}

The relation above is not derivable in \name, because the type of implicit type
parameter is an abstract type, for which we are not able to find an well-typed
instantiation except for the infinite loop $\mu x : A.\, x$, this is the also the
case when $A$ in an arbitrary uninhabited types (without the help of a fixpoint).

This situation also assign a special role to variables in the context,
for example:

\begin{multline*}
    A : [[*]],\, F : A \rightarrow [[*]],\, \rulehl{a : A} \vdash \\
    \forall x : A.\, (F~x \rightarrow F~x) \rightarrow [[int]] \le (\forall x : A.\, F~x \rightarrow F~x) \rightarrow [[int]] : [[*]]
\end{multline*}

Without the help of the fixpoint, this subtyping relation is only derivable
with the presence of the highlighted variable in the context. The relation has to
be derived from \rref{s-forall-l}, which requires a well-typed instantiation for
the parameter, in this case, only $x$ is eligible even though it does not occur
anywhere in the expression except for the context.

The help of fixpoint does not solve the general problem, because
in \name the fixpoint expressions are only well-typed when it is not a polymorphic
type. So the general ``Strengthening'' lemma is not admissible in \name. Even
a restricted case where we only consider typing complicated examples can still
be construct to stop us from eliminating the a variable in the context even
when it is fresh everywhere else:

\begin{equation*}
    F : [[int]] \rightarrow [[*]],\, A : [[*]],\, \rulehl{a : A} \vdash F ~ ([[(bind x : A. lambda y : int. y)]]~ 42) : [[*]]
\end{equation*}

For the time being, we think the addition of the premise in \rref{s-forall-r} and
the addition of \rref{s-forall} do not complicate the metatheory as much, so
we leave the further exploration of the issue above in a future work.

\subsection{Design Choices of the Semantics around Cast Operators}
\label{sec:cast-design}

The type reduction in cast operators is potentially under a context that
is not empty, so it is likely that we are performing reduction to a open term.
Intuitively we should generalize the definition of value by introducing inert
terms\cite{yang2017unifying} to handle open term reduction.

However since we adopts the Call-by-Name semantics of Pure Iso-type System\cite{yang2019pure},
there is no value check during the reduction, and whether the result of reduction
inside cast operator does not matter during the reasoning of type safety. It is
not necessary to complicate the metatheory by introducing inert terms.

An alternate design around cast operator is the Call-by-Value (CBV) style\cite{yang2019pure},
by not considering all $\castup$ terms as value, and performing cast elimination only
when the expression inside two casts is a value. Such design requires us to
have a more general definition for value, and thus there is a need for inert terms.

However, a simple design with CBV-style cast semantics and inert terms
potentially lead to a system where \emph{Reduction Substitution} does not hold,
for example:

\begin{gather*}
    \castdn \, \castup \, [A] \, f ~ x \longrightarrow f ~ x \\
    [\lambda x : B. \, x / f] \castdn \, \castup \, [A] \, f ~ x \longrightarrow \castdn \, \castup \, [A] x
\end{gather*}

So we stick with the Call-by-Name style semantics around cast operators and
leave the discussion of other possibilities of design in a future work.


\bibliographystyle{plainurl}
\bibliography{reference}

\end{document}
