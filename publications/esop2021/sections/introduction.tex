\section{Introduction}

A \emph{polymorphic subtyping} relation, which relates more general
types to more specific ones, is at the core of many modern functional
languages.
%some more background text


In recent years dependent
types~\cite{coc,cayenne,dep:pisigma,sjoberg:msfp12,guru,fc:kind,zombie:popl14,zombie:popl15} 
have become a hot topic in programming
language research. Several new
functional programming languages, most notably Agda~\cite{2007_norell_agda} and
Idris~\cite{brady2013idris}, are now dependently typed. 
A key reason why dependent types are interesting is
that they naturally lead to a unification between types and terms, which enables both
additional \emph{expressiveness} and \emph{economy of concepts}. 
The benefit of economy of concepts is the main
motivator for our proposal. In essence, once various different levels of syntax (such 
as terms and types) are unified, then redundancy of language
constructs at the various levels can be avoided. This leads to more
compact calculi compared to more traditional calculi for
programming languages, which have different stratified levels 
of syntax. In turn, such economy of concepts results in a significantly
more compact metatheory, and can also lead to a reduction of 
the necessary implementation effort.
The key enabler for unifying terms and types in dependently typed
calculi is the adoption of a style similar to 
Pure Type Systems (PTSs)~\cite{pts}. In PTSs there is only a single level 
of syntax for terms, and types (or kinds) that are expressed using the 
same syntax. This is in contrast with more traditional calculi, where 
distinct pieces of syntax (terms, types and kinds) are separated.

Given that the complexity of type systems
for OOP languages is so high, techniques for bringing down such
complexity, while retaining or even increasing expressiveness are certainly
welcome. The economy of concepts afforded by unified syntax typical of 
dependently typed languages can help here, since it can significantly
reduce the number of language constructs and relations needed in a
calculus, and simplify the metatheory.
Unfortunately, there has been less work on dependently typed
calculi for object-oriented programming. We believe that there are
essentially two primary reasons for this:

\begin{itemize}

\item  {\bf The interaction between recursion and dependent types.}
 The first reason is that the
interaction between general recursion and dependent types is
challenging. Essentially recursion breaks strong normalization, which 
many common properties in dependently typed calculi depend upon.
However, this area has been actively investigated in the
last few years, and a general approach~\cite{guru,sjoberg:msfp12, kimmel:plpv, zombie:popl15,
  isotype}, based on explicit 
casts for type-level computation, has emerged as an interesting
solution.

\item {\bf The interaction between dependent types and subtyping.}
The second reason is that how to smoothly combine
dependent types and subtyping is difficult. Subtyping is a
substantial difference to traditional PTSs, which do not have such feature.
The issue with subtyping 
is well summarized by Aspinall and Compagnoni~\cite{subdep}:
\emph{
One thing that makes the study of these systems difficult is that {\bf
  with
dependent types, the typing and subtyping relations become intimately
tangled}, which means that tested techniques of examining subtyping in
isolation no longer apply}.

\end{itemize}

 
The OOPSLA 2017 paper introduces \name, which extends 
\lami with support for OOP features such as
\emph{higher-order subtyping}~\cite{fsubo}, \emph{bounded quantification} and
\emph{top types}. 
To address the challenges posed by
the combination of dependent types and subtyping, \name
employs \emph{unified subtyping}: a novel technique that unifies
\emph{typing}, \emph{subtyping} and \emph{well-formedness} into one
relation. Therefore, \name takes a significantly different
approach compared to previous work, which
attempts to fight the entanglement between typing and subtyping. In
contrast, \name embraces such
tangling by collapsing the typing and subtyping
relations into the same relation.  This approach is different from
Hutchins' technique, which eliminates types and typing. \name
retains types.

As those languages start moving towards dependently typed
programming a natural question is how can polymorphic subtyping be
adapted to such settings.
% some text on existing work.

Two main issues: type-safety and transitivity of subtyping.


As a starting point we started investigating the problem of local
type-inference for a language similar to the \emph{calculus of
  constructions}~\cite{Coquand:Huet} (but with the $\star : \star$ axiom and
some restrictions). As part of our preliminary work we designed a
specification for a calculus that supports local synthesis of arguments.
There two main challenges that we faced in the design of this calculus,
which lead to important design decisions.

The first challenge is due to the conversion rule typically present in
dependently typed languages. %This restriction is motivated by a fundamental problem.
In dependently typed languages the conversion rule enables type-level
computation and it is widely used in languages like Agda or
Idris. Unfortunately it also introduces major complications for
type-inference. A well-known result is that type-inference for systems
with a conversion rule requires \emph{higher-order
  unification}, which is known to be \emph{undecidable}~\cite{goldfarb1981undecidability}.
Therefore to avoid higher-order unification we simply drop
the conversion rule in our initial calculus.
Nevertheless there are still practical applications for languages
with such a restriction. For instance, GHC Haskell language of types and kinds
(which is dependently typed) has no type conversion and it is indeed quite
closely related to the language that we plan to study first. Indeed
our recent POPL 2020 paper~\cite{xie20kind} describes algorithms and specifications
for (the very ambitious form) of type-inference currently present in GHC Haskell.
In GHC Haskell there are no type-level lambdas and thus no type conversion
(although type-level computation can be achieved by other means~\cite{Chakravarty05associated}).

The second challenge is that formulating subtyping for dependently typed
languages is tricky due to the mutual dependency between various relations
(typing, subtyping and well-formedness). Subtyping is commonly used to
specify a \emph{most general than} relation between types. It
is for instance used by DK~\cite{dunfield2013complete} and our own ICFP 2019 work. 
However the mutual dependencies that arise in a dependently typed setting create
various technical problems for the metatheory.
For instance, unlike non-dependently typed languages, it is
not possible to develop the metatheory of subtyping separately from typing. 
This problem has been observed by multiple researchers 
in previous work~\cite{subdep,Chen03coc,hutchins,ptssub}, with multiple solutions making different compromises.
Our OOPSLA 2017 work~\cite{full} is relevant here as it proposes a simple
solution for this problem. Following the same spirit as \emph{Pure Type Systems}~\cite{BarendregtHP:intgts},
which attempt to unify syntax and the typing and well-formedness relations,
our OOPSLA 2017 work proposes to go one step further: it shows how to unify typing
and subtyping into a single relation. This solves the problem of dependencies
in that now there is only a single relation that depends on itself. Furthermore
it results on a compact specification compared to a variant with multiple
independent relations.


A first problem that arizes is type-safety. 
In a System F-like setting type safety can be proved 
directly, by have an operational semantics for
a variant of System F with implicit polymorphism.
Type-safety can also be proved indirectly, by elaboration
into conventional System F with explicit polymorphism.
For instance this is the approach used by Odersky and L\"aufer~\cite{},
as well as other authors.

In a dependently typed setting elaboration poses
complications. For instance an issue is that the reduction
relation is typically used in typing. However, in an elaboration
approach there is no reduction relation for the source calculus
to start with. Therefore, approaches, such as the ICC, turn to
formulating a direct operational semantics instead and proving
type-soundness. 

In our case type-safety could, in principle, follow also either an elaboration
approach or a direct operational semantics approach. 

This paper presents \system (\name): a simple dependently typed
calculus with polymorphic subtyping. The subtyping relation in \name
generalizes the well-known polymorphic subtyping relation by Odersky
and L\"aufer (1996). Because \name is dependently typed, integrating
subtyping in the calculus is non-trivial. To overcome many of the
issues arizing from integrating subtyping with dependent types, the
calculus employs \emph{unified subtyping}, which is a technique that
unifies typing and subtyping into a single relation. Moreover, \name
employs explicit casts instead of a conversion rule, allowing
unrestricted recursion to be naturally supported.  We prove various
non-trivial results, including \emph{type soundness} and
\emph{transitivity} of unified subtyping. The calculus and all
corresponding proofs are mechanized in the Coq theorem prover and we
also have a simple prototype implementation of \name.

The contributions of this paper are:

\begin{itemize}

\item contrib 1...

\end{itemize}
