\section{Overview}

In this section we go through the motivation and main features of \name. We
try to express the subtyping of (implicit) polymorphic types in a dependently-typed
system. And we adopt a newly-studied techniques to solve the inherent dependency
between typing and subtyping. And then we focus on a series of examples to show
the power of our language. A formal illustration of \name will be discussed in
Section \ref{sec:matatheory}

\subsection{The Polymorphic Subtyping Relation by Odersky and L\"aufer}
\bruno{Have a figure and introduce the original relation here, before you move to
giving an overview of the dependently typed version.}

\subsection{Lifting the Polymorphic Subtyping}
\label{sec:polymorphic-subtyping}

The subtyping relation for polymorphic types by Odersky and L\"aufer\cite{odersky1996putting}
captures a \emph{more-general-than} relation between polymorphic and monomorphic
types. It features the following two subtyping rules:

\begin{mathpar}
  \inferrule*[right=$\le\forall L$]
    {\Gamma \vdash \tau \\ \Gamma \vdash [\tau / x]\, A \le B}
    {\Gamma \vdash \forall x.\, A \le B}
  \and
  \inferrule*[right=$\le\forall R$]
    {\Gamma ,\, x \vdash A \le B}
    {\Gamma \vdash A \le \forall x.\, B}
\end{mathpar}

And based on this idea, we try to generalize the subtyping relation to handle
dependent types. The parameters of polymorphic types can only be types in
polymorphic subtyping, in \name, the paramters can be can be ``values'' as well.
So ideally a direct generalization would be:

\begin{mathpar}
  \inferrule*[right=$\le\forall'$]
    {\Gamma \vdash \tau \rulehl{: A} \\ \Gamma \vdash [\tau / x]\, B \le C}
    {\Gamma \vdash \forall x \rulehl{: A}.\, B \le C}
  \and
  \inferrule*[right=$\ge\forall'$]
    {\Gamma ,\, x \rulehl{: B} \vdash A \le C}
    {\Gamma \vdash A \le \forall x \rulehl{: B}.\,C}
\end{mathpar}

Here the parameter for polymorphic types can be well-typed values besides types,
hence instead of requiring the instantiation to be only well-formed in rule
$\le\forall$, it should be required to be well-typed. Similarly, the context
for the subtyping rule should include typing information for the abstract variable
in the adaptation of $\ge\forall$.

\subsection{Unified Subtyping}

But the notorious mutual dependency between typing and subtyping
\cite{aspinall1996subtyping, hutchins2010pure} presents itself in the direct
generalization,
due to the usual presence of subsumption rule in the typing rules:

\begin{mathpar}
  \inferrule*[right=t-sub]
    {\Gamma \vdash e : A \\ \Gamma \vdash A \le B}
    {\Gamma \vdash e : B}
\end{mathpar}

To tackle this issue, we adopt the technique developed by Yang called the
Unified Subtyping\cite{yang2017unifying}, which merges the typing relation and
subtyping relation into a single relation to avoid the mutual dependency:

\begin{mathpar}
  \Gamma \vdash e_1 \le e_2 : A
\end{mathpar}

The interpretation of this judgement is: under context $\Gamma$, $e_1$ is a sub-``type''
of $e_2$ and they both are of type $A$. And the judgements of subtyping and typing
are both special form of unified subtyping with the involvement of kinds $[[*]]$:

\begin{mathpar}
  \Gamma \vdash A \le B \triangleq \Gamma \vdash A \le B : [[*]]
  \and
  \Gamma \vdash e : A \triangleq \Gamma \vdash e \le e : A
\end{mathpar}

The technique simplifies the formalization and especially the proof of
transitivity in the original work. Ideally after applying the technique,
the generalization of the polymorphic subtyping should be:

\begin{mathpar}
  \inferrule*[right=$\le\forall''$]
    {\Gamma \vdash \tau : A \\ \Gamma \vdash [\tau / x]\, B \le C \rulehl{: [[*]]}}
    {\Gamma \vdash \forall x : A.\, B \le C \rulehl{: [[*]]}}
  \and
  \inferrule*[right=$\ge\forall''$]
    {\Gamma ,\, x : B \vdash A \le C \rulehl{: [[*]]}}
    {\Gamma \vdash A \le \forall x : B.\,C \rulehl{: [[*]]}}
\end{mathpar}

However the formalization is not exactly like this due to various issues, the
details will be discussed in section \ref{sec:type-system}.

\subsection{``Explicit'' Implicit Instantiation}

Another feature of the polymorphic subtyping is that the instantiation of type
parameter is done implicitly instead of being manually applied. Normally the
type parameters only stay at type-level so terms are not involved in the
implicit instantiation, for example:

\begin{mathpar}
  (\lambda x.\, x)~42 \longrightarrow 42
\end{mathpar}

Here $\lambda x.\, x$ has type $\forall A.\, A \rightarrow A$. The implicit
instantiation is not reflected anywhere at term level. However for
language with type annotations on parameters of binders
(for example lambda abstractions), in order
for the polymorphic variable to be well-scoped at type annotations of terms,
another binder $\Lambda$ is added for terms, however instantiations are still
implicit as shown in the following examples:

\begin{mathpar}
  (\Lambda A : [[*]].\, \lambda x : A.\, x) ~ 42 \longrightarrow 42 \\
  (\Lambda A : [[*]].\, \lambda x : A.\, \lambda y : A.\, x) \longrightarrow \lambda y : [[int]].\, 42
\end{mathpar}

Here $\Lambda A : [[*]].\, \lambda x : A.\, x$ has type $\forall A : \star. \, A \rightarrow A$,
the polymorphic parameter $A$ is explicitly stated in the polymorphic
terms. However as the reduction show, the instantiations are still implicit.

\subsection{A Tour of \name}

In this section we show some relevant examples to illustrate the expressiveness
of \name.

5 (4) examples:

\begin{itemize}
\item polymorphic identity (already explained above)
\item map on lists (recursive types)
\item Functor and fmap (higher kinds)
\item Ptree (dependent types)
\item Higher-ranked subtyping (to illustrate difference to ICC):
  $(Int \rightarrow Int) \rightarrow Int <: (\forall a . a \rightarrow a) \rightarrow Int$
\end{itemize}
