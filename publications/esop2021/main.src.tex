\documentclass{llncs}

\usepackage[T1]{fontenc}
\usepackage{ottalt}
\usepackage{xcolor}
\usepackage{mathtools}
\usepackage{amsmath}
\usepackage{amsfonts}
\usepackage{amssymb}
\usepackage{setspace}
\usepackage{xspace}

\inputott{dependent-subtyping.ott}

\newcommand{\rulehl}[2][gray!40]{
  \colorbox{#1}{$\displaystyle#2$}}

\newcommand{\extract}[1]{\lvert #1 \rvert}
\newcommand{\castup}[0]{\mathrm{cast}_\Uparrow}
\newcommand{\castdn}[0]{\mathrm{cast}_\Downarrow}
\newcommand{\system}[0]{the dependent implicitly polymorphic calculus\xspace}
\newcommand{\System}[0]{The dependent implicitly polymorphic calculus\xspace}

\newcommand{\name}[0]{$\lambda_{I}^{\forall}$\xspace}

\begin{document}

\title{A Dependently Typed Calculus with Polymorphic Subtyping}

\author{Mingqi Xue \and Bruno C. d. S. Oliveira}
\institute{
    The University of Hong Kong \\
    \email{\{mqxue,bruno\}@cs.hku.hk}
}

\maketitle

\begin{abstract}
  A \emph{polymorphic subtyping} relation, which relates more general types
  to more specific ones, is at the core of many modern functional languages.
  As those languages start moving towards dependently typed programming a
  natural question is how can polymorphic subtyping be adapted to such settings.
  
  This paper presents \system (\name): a simple dependently typed calculus
  with polymorphic subtyping. The subtyping relation in \name
  generalizes the well-known polymorphic subtyping relation by
  Odersky and L\"aufer (1996). Because \name is dependently typed,
  integrating subtyping to the calculus is non-trivial. To overcome
  many of the issues arizing from integrating subtyping with dependent
  types, the calculus employs \emph{unified subtyping}, which is a technique
  that unifies typing and subtyping into a single relation. Moreover, \name
  employs explicit casts instead of a conversion rule, allowing
  unrestricted recursion to be naturally supported.
  We prove various non-trivial results, including \emph{type soundness} and \emph{transitivity}
  of unified subtyping. The calculus and all corresponding proofs
  are mechanized in the Coq theorem prover and we also have a simple prototype
  implementation of \name.
%A complication that arizes in the type-soundness proof
%is that a direct semantics is non-deterministic due 
\end{abstract}

\section{\System}

\begin{figure}[t]
\centering
\begin{equation*}
\begin{array}{llcl}
    \text{Kinds} & k & ~\Coloneqq ~ & [[*]] \mid [[box]] \\
    \text{Expressions} & e, A, B & ~ \Coloneqq ~ & [[x]] \mid [[n]] \mid [[k]] \mid [[int]] \mid [[e1 e2]] \mid [[lambda x : A. e]]
        \mid [[pi x : A. B]] \mid [[bind x : A. e]] \\
        & & \mid & [[forall x : A. B]] \mid [[mu x : A. e]] \mid [[castup [A] e]] \mid [[castdn e]]   \\
    \text{Mono-Expressions} ~ & \tau, \sigma & ~ \Coloneqq ~ & [[x]] \mid [[n]] \mid [[k]] \mid [[int]] \mid \tau_1 ~ \tau_2 \mid \lambda \, x : \tau. ~ \sigma \mid \Pi \, x : \tau. ~ \sigma \mid \Lambda \, x : \tau. ~ \sigma \\
        & & \mid & \mu \, x : \tau. ~ \sigma \mid \castup \, [\tau]~ \sigma \mid \castdn \, \tau \\
    \text{Values} & v & ~ \Coloneqq ~ & [[k]] \mid [[n]] \mid [[int]] \mid [[lambda x : A. e]] \mid [[pi x : A. B]] \mid [[bind x : A. e]] \\
        & & \mid & [[forall x : A. B]] \mid [[castup [A] e]] \\
    \text{Contexts} & \Gamma & ~ \Coloneqq ~ & [[nil]] \mid [[G , x : A]] \\
    \text{Syntactic Sugar} ~ & A \rightarrow B & \triangleq & [[pi x : A. B]] \qquad \text{where} ~ x \notin \mathrm{FV}(B)
\end{array}
\end{equation*}
\caption{Syntax of \system}
\label{fig:syntax}
\end{figure}

\subsection{Syntax}

Figure \ref{fig:syntax} presents the syntax of \system, which follows a similar
syntax to the Calculus of Constructions\cite{CoquandThierry1988Tcoc}, featuring
$[[*]]$ and $[[box]]$ in the kind hierarchy, and unifying the concepts of terms
and types as expressions. ``Type'', ``term'' and ``expression'' are used
interchangeably in later sections for whichever makes sense in the contexts.
And we have following additional language constructs.

\subsubsection{Implicit Polymorphism}

The $\forall$ type is the generalized version of implicit polymorphism in
non-dependent languages, whose arguments range over all well-typed expressions
besides well-formed types. It is implicit in the way that the type arguments are
deduced during application rather than being explicitly passed.
We also generalized the concept of ``monotypes'' to
``mono-expressions'' that excludes the $\forall$ types from expressions.

Also, the idea of ``polymorphic types'' is naturally dependent, so $\forall$
types can be viewed as the implicit counterpart of $\Pi$ types. We also have
the $\Lambda$ (pronounced as ``bind'') expression as the implicit counterpart of
$\lambda$ abstractions, which is similar to the design of $\text{ICC}^*$\cite{barras2008implicit}.

\subsubsection{Explicit Type-level Computation}
\label{sec:cast}

\System adopts the iso-types\cite{yang2016unified}, featuring explicit type-level
computations with cast operators $\castdn$ and $\castup$, which respectively perform one-step
type reduction and type expansion based on the operational semantics.
The reduction in cast operators is deterministic so that the type annotation is
only needed during type expansions ($\castup$).

\subsubsection{Fixpoint Operator}

We add the fixpoint operator $\mu$ to support general recursion for both term-level
and type-level. Iso-recursive types are supported due to the presence of $\castup$
and $\castdn$, which correspond to the ``fold'' and ``unfold'' operations when
working on recursive types. However, we do not support subtyping for recursive
types in the current system due to the complexity.

\begin{figure}
    \centering

    \drules[r]{$[[e1 --> e2]]$}{Operational Semantics}{
        app,beta,inst,mu,castdn,castXXinst,castXXelim}

    \caption{Operational Semantics of \system}
    \label{fig:semantics}
\end{figure}

\subsection{Operational Semantics}

Figure \ref{fig:semantics} presents the operational semantics of our system,
which mostly follows the call-by-name variant of the Pure Iso-type System\cite{yang2019pure},
where arguments of $\beta$-reduction (\rref{r-beta}) and expressions inside $\castdn \castup$
are not required to be values (\rref{r-cast-elim}). Meanwhile we consider $\castup$ terms to be a value,
and only performs reduction inside $\castdn$ terms (\rref{r-castdn}). Also, the unroll
operation of fixpoint operator is supported by \rref{r-mu}.

\subsubsection{indeterministic Implicit Instantiations}

Notably rules involving instantiating implicit parameters of $\Lambda$ terms
(\rref{r-inst,r-cast-inst}) are not deterministic and even type-safe, which seems to
contradict our previous claim of deterministic reduction inside cast in section \ref{sec:cast}.
These issues are resolved by imposing restrictions on type systems such that
\textbf{a)} said indeterministic reductions are not well-typed inside cast operations
\textbf{b)} the instantiation choices of implicit parameters do not affect the
overall runtime type-safety.
Details will be explained in the later sections.

\begin{figure}
    \centering
    \drules[wf]{$[[|- G]]$}{Well-formed Context}{nil,cons}

    \drules[s]{$[[G |- e1 <: e2 : A]]$}{Unified Subtyping}{
        var,lit,int,star,abs,pi,app,bind,mu,castup,castdn,forallXXl,forallXXr,forall,sub}
    \begin{equation*}
       \text{Syntactic Sugar} \qquad [[G |- e : A]] \triangleq [[G |- e <: e : A]]
    \end{equation*}
    \caption{(Sub)Typing Rules of \system}
    \label{fig:typing}
\end{figure}

\subsection{Unified Subtyping System}

Figure \ref{fig:typing} shows the typing rules of the system. We adopt the design
of ``Unified Subtyping''\cite{yang2017unifying} where the subtyping rules and
typing rules are merged into a single typing judgment $[[G |- e1 <: e2 : A]]$,
which elegantly solves the traditional issue of mutual dependency between typing
and subtyping in a dependent type system.
The interpretaion of this judgment is ``under context $[[G]]$, $[[e1]]$ is a
subtype of $[[e2]]$ and they are both of type $[[A]]$''.
In this form of formalization, the typing judgment $[[G |- e : A]]$ is a
special case of unified subtyping judgment $[[G |- e <: e : A]]$,
and the idea of well-formed types $\Gamma \vdash A$ is expressed by
$[[G |- A : k]]$ where $k$ is either $[[*]]$ or $[[box]]$.

\subsubsection{DK-Style Subtyping}

The subtyping rules for $\forall$ types (\rref{s-forall-l,s-forall-r}) follows
the spirit of the DK's declarative subtyping rules\cite{DunfieldJoshua2013Caeb},
where the subtyping relation is interpreted as a ``more general than'' relation.
A polymorphic type is more general than another type when its well-typed
instantiation is more general than that (\rref{s-forall-l}). And a polymorphic
type is less general when another type is more general with the type parameter of
the poly-type abstracted out (\rref{s-forall-r}). Also we add \rref{s-forall}
to axiomize the subtyping relation between two polymorphic types due to the
issues presented in the formalization.
We shall discuss it in more detail in later sections.


\subsubsection{Mono-expression Restrictions}

As in DK's system, the type arguments for instantiation in \rref{s-forall-l} are
required to be mono-expressions, which has cascading effects on typing rules of
other expressions. The arguments for applications are required to be
mono-expressions, and the whole fixpoint expression is required to be mono-expressions. We shall
discuss the reason behind the restrictions in later sections.

\subsubsection{Kind Restriction for $\forall$ Types}
\label{sec:kind-restriction}

For the kinding of types, we mainly follow the design of the Calculus of
Constructions\cite{CoquandThierry1988Tcoc}, but specifically restrict the $\forall$
expressions to only have the kind $[[*]]$, which prevents other types of kind
$[[*]]$ from having kind $[[forall x : int. *]]$, which significantly complicates
the metatheory when reasoning about the kind of types.
This restriction propagates to the introduction rule of $\forall$ types (\rref{s-bind}),
where $[[B]]$ is required to only have kind $[[*]]$.
This way well-typed implicit abstractions ($\Lambda$ expressions) are kept away
from type computations. Therefore in cast operators,
the possibility of indeterministic implicit instantiations is eliminated.

\begin{figure}
    \centering
    \begin{equation*}
        \begin{array}{llcl}
            \text{Extracted Expressions} & e, A, B & ~ \Coloneqq ~ & [[x]] \mid [[n]] \mid [[k]] \mid [[int]] \mid [[ee1 ee2]] \mid [[elambda x. ee]] \mid [[epi x : eA. eB]] \mid [[ebind x. ee]] \\
            & & \mid & [[eforall x : eA. eB]] \mid [[emu x. ee]] \mid [[ecastup ee]] \mid [[ecastdn ee]]
        \end{array}
    \end{equation*}

    \begin{equation*}
    {\setstretch{1.5}
    \begin{array}{cc}
        \extract{[[x]]} = [[x]] \qquad
        \extract{[[n]]} = [[n]] &
        \extract{[[k]]} = [[k]] \qquad
        \extract{[[int]]} = [[int]] \\
        \extract{[[e1 e2]]} = \extract{[[e1]]} ~ \extract{[[e2]]} &
        \extract{[[mu x : A. e]]} = \mu \, x. ~ \extract{[[e]]} \\
        \extract{[[lambda x : A. e]]} = \lambda \, x. ~ \extract{[[e]]} &
        \extract{[[pi x : A. B]]} = \Pi \, x : \extract{[[A]]}. ~ \extract{[[B]]} \\
        \extract{[[bind x : A. e]]} = \Lambda \, x. ~ \extract{[[e]]} &
        \extract{[[forall x : A. B]]} = \forall \, x : \extract{[[A]]}. ~ \extract{[[B]]} \\
        \extract{[[castup [A] e]]} = \castup \, \extract{[[e]]} &
        \extract{[[castdn e]]} = \castdn \, \extract{[[e]]}
    \end{array}
    }
    \end{equation*}

    \caption{Extraction of Expressions}
    \label{fig:extraction}
\end{figure}

\subsubsection{Runtime Irrelevance of Implicit Arguments}

Our direct operational semantics choose random mono-expressions to instantiate
the implicit arguments which potentially breaks type safety. So we adopt a
restriction that is similar to the Implicit Calculus of Constructions (ICC) \cite{miquel2001implicit}
to make the choice of implicit argument irrelevant to the computation, by only
allowing the implicit parameters to occur in type annotations in the body of
implicit abstraction. Figure \ref{fig:extraction} shows the extraction function
that gets rid of all the type annotations in expressions, and enforcing the
freshness of implicit parameter in the extracted expression. The type safety of
the direct operational semantics is proved indirectly with the help of
the extraction of expressions, which we will come back to in later sections.

\subsubsection{Redundant Premises}

All the premises highlight in gray are redundant in a way that
the system without them is proved equivalent to the system in figure \ref{fig:typing}.
They are there to simplify the mechanized proofs of certain lemmas.

\section{Metatheory}

\bibliographystyle{splncs04}
\bibliography{reference}

\end{document}
