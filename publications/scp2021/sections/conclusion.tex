\section{Conclusion}

In this article, we presented a design of a dependently-typed calculus called \name.
\name generalizes non-dependent polymorphic subtyping by Odersky and
L\"aufer~\citep{odersky1996putting} and contains other features like general
recursion and explicit casts for type-level computations.
We adopt the techniques of the Unified Subtyping~\citep{full} to
avoid the mutual dependency between typing and subtyping relation to simplify
the formalization. Besides other relevant theorems about typing and subtyping,
\emph{transitivity} and \emph{type safety} are proved mechanically with the Coq
proof assistant.

In the future, we will attempt to lift various restrictions that originally simplify
the metatheory, such as the kind restriction on polymorphic
types and the runtime irrelevance of implicit arguments. We would also like to
study the impact to the metatheory of adding $\top$ types to our language,
which is a common feature of a subtyping relation.
Most importantly, we consider the development
of a well-specified algorithmic system a major challenge in our future
work. The current formulation of \name is declarative due to the mono-expression guesses.

While \name still has some limitations, we believe that it already includes many
of the core features that are important for typed functional languages.
Assumming that we have an implementation of a core language based on \name,
we expect that interesting and expressive functional languages can be built
on top of such core language. For instance, all the features of Haskell 98,
including higher-kinds, algebraic datatypes and type classes~\citep{typeclasseskaes,typeclasseswadler}
should be easily encodable
in \name. Furthermore, some features not in Haskell 98, but available in modern
versions of GHC Haskell, such as higher-ranked polymorphism or certain
kinds of dependent types are also supported in \name. GADTs~\citep{gadt1,gadt2} and
type-families~\citep{typefamilies} are more challenging as they require a more powerful
form of casts and additional support for equality. Previous work on PITS
has shown how some forms of GADTs and equality can be modelled using
a variant of cast operators that employ a more powerful parallel reduction
relation. We believe that \name can also employ such variant of casts,
although this also remains future work.
