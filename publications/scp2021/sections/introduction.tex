\section{Introduction}

A \emph{polymorphic subtyping} relation, which relates more general
types to more specific ones, is at the core of many modern functional
languages. Polymorphic subtyping enables a form of
\emph{(implicit) parametric polymorphism}, where type arguments to polymorphic
functions are automatically instantiated and the programmer does not specify them.
Traditionally, variants of polymorphic subtyping (in the form of a more-general-than relation)
have been used in functional languages based on the
Hindley-Milner~\citep{hindley1969principal,milner1978theory,damas1982principal}
type system, which supports full type-inference without any type annotations.
However, the Hindley-Milner type system only supports \emph{rank-1 (or first-order)
polymorphism}, where all universal quantifiers only occur at the top-level
of a type. Modern functional programming languages, such as Haskell, go beyond
Hindley-Milner and support \emph{higher-ranked polymorphism}~\citep{odersky1996putting,jones2007practical}
with a more expressive
polymorphic subtyping relation. With higher-ranked
polymorphism there is no restriction on where universal quantifiers can occur.

\citet{odersky1996putting} proposed a
simple declarative specification for polymorphic subtyping, which supports higher-ranked polymorphism.
Since then several
algorithms have been proposed that implement variants of this specification. Most
notably, the algorithm proposed by \citet{jones2007practical} forms the basis
for the implementation of type inference in the GHC compiler.
\citet{dunfield2013complete} (DK) provide an elegant
formalization of another sound and complete algorithm, which has
also inspired implementations of type-inference in some polymorphic
programming languages, such as PureScript~\citep{PureScript} or DDC~\citep{Disciple}.
More recently \citet{zhao19mechanical} have mechanized DK's type system in a theorem prover.

%some more background text

In recent years dependent
types~\citep{coc,cayenne,dep:pisigma,sjoberg:msfp12,guru,fc:kind,zombie:popl14,zombie:popl15}
have become a hot topic in programming
language research. Several newer
functional programming languages, most notably Agda~\citep{2007_norell_agda} and
Idris~\citep{brady2013idris}, are now dependently typed. Moreover a number of existing functional
languages, such as Haskell, have started to move towards dependently typed programming~\citep{dependenthaskell}. Dependent types naturally lead to a unification between types and terms, which enables both
additional \emph{expressiveness} and \emph{economy of concepts}.
The key enabler for unifying terms and types in dependently typed
calculi is the adoption of a style similar to
Pure Type Systems (PTSs)~\citep{pts}. In PTSs there is only a single level
of syntax for terms, i.e. the types (or kinds) are expressed using the
same syntax as the terms. This is in contrast with more traditional calculi, where
distinct pieces of syntax (terms, types and kinds) are separated.

Unified syntax, typical of dependently typed languages,
poses some challenges for language design and implementation.
A first challenge arises from the interaction between recursion and dependent types.
Essentially recursion breaks strong normalization, which
many common properties in dependently typed calculi depend upon. One of the most
typical properties among them is the decidability of type checking,
which simply cannot be guaranteed if some
type-level computations are non-terminating.
However, this area has been actively investigated in the
last few years, and a general approach~\citep{guru,sjoberg:msfp12, kimmel:plpv, zombie:popl15,
  isotype} based on explicit casts for type-level computations,
has emerged as an interesting solution for integrating general recursion
in dependently typed calculi. By avoiding the implicit type-level computation
entirely, whether programs strongly normalize or not no longer matters for the
decidability of type checking.
Current proposals for dependently typed versions of Haskell~\citep{dependenthaskell},
for instance, adopt explicit casts for type-level computation.

The second challenge, for calculi that employ subtyping, is that
smoothly integrating
dependent types and subtyping is difficult. Subtyping is a
substantial difference to traditional PTSs, which do not have this feature.
The issue with subtyping
is well summarized by \cite{subdep}:
\emph{``One thing that makes the study of these systems difficult is that
  with
dependent types, the typing and subtyping relations become intimately
tangled, which means that tested techniques of examining subtyping in
isolation no longer apply''}.
Recent work on \emph{unified subtyping}~\citep{full} provides a
simple technique to address this problem.
Following the same spirit as Pure Type Systems,
which attempt to unify syntax and the typing and well-formedness relations,
unified subtyping suggests unifying typing
and subtyping into a single relation. This solves the problem of dependencies
in that now there is only a single relation that depends only on itself.
Furthermore, it results in a compact specification compared to a variant with
multiple independent relations.

In this paper, we investigate how polymorphic subtyping can be
adapted into a dependently typed calculus with general recursion and
explicit casts for type-level computation. We employ unified subtyping
to address the issues of combining dependent types with subtyping.
The use of explicit casts
for type-level computation means that type equality is essentially
syntactic (or rather up-to $\alpha$-equivalence).
This avoids the use of a traditional conversion rule that allows concluding
$\beta$-equivalent types to be equal. In essence, the use of the conversion
rule requires (implicit) type-level computation, since terms have to be normalized
using $\beta$-reduction to conclude whether or not they are equal.
Dependent type systems with a conversion rule have some major complications.
A well-known one is that type-inference for such systems requires \emph{higher-order
  unification}, which is known to be \emph{undecidable}~\citep{goldfarb1981undecidability}.
By employing a system with $\alpha$-equivalence only we stay closer to existing
languages like Haskell, where type equality (at least at the core language level)
is also essentially only up-to $\alpha$-equivalence.

We present a calculus called \name, and show three main results in this paper:
\emph{transitivity of subtyping}, \emph{type soundness}, and \emph{completeness
of \name's polymorphic subtyping with respect to Odersky and L\"aufer's formulation}.
Transitivity is a non-trivial result (like in most calculi combining dependent types
and subtyping) and requires a proof based on sizes and a property that guarantees
the uniqueness of kinds in our language. Type soundness is also non-trivial and we need
to take a different approach than that employed by existing work on polymorphic
subtyping~\citep{odersky1996putting, jones2007practical}, where type-safety is shown by an
elaboration to System F. In essence elaboration into a target language
brings significant complications to the metatheory in a dependently typed setting.
Thus, instead of elaboration, we use a direct operational semantics approach, which
is partly inspired by the approach used in the \emph{Implicit Calculus of Constructions} (ICC)~\citep{miquel2001implicit,barras2008implicit},
to prove type soundness.
Similar to ICC, we adopt the restriction that arguments for implicit function types
are computationally irrelevant (i.e. they cannot be used in runtime computation).
However, our unified subtyping setting is significantly
different from ICC due to the presence of subtyping,
which brings complications the ICC does not have.
We also prove that any valid subtyping statement in the Odersky and L\"aufer relation
is valid in \name. Thus \name's unified subtyping subsumes the polymorphic subtyping
relation by Odersky and L\"aufer.

\name and all the proofs reported in this paper are formalized in the Coq theorem
prover~(\citeauthor{coqsite}).
This paper does not address decidability or soundness and completeness of \name to an
algorithmic formulation, which are outside of the scope of this work.
Nonetheless, these are important and challenging
questions for practical implementations of \name, which are left open for future work.

In summary, the contributions of this paper are:

\begin{itemize}

\item {\bf The \name calculus}, which is a dependently typed calculus with explicit casts,
  general recursion and implicit higher-ranked polymorphism.

\item {\bf Type-soundness and transitivity of subtyping.} We show that \name
  is type-sound and unified subtyping is transitive.

\item {\bf Subsumption of Odersky and L\"aufer's polymorphic subtyping.} We show that \name's
  unified subtyping can encode all valid poymorphic subtyping statements of Odersky and L\"aufer's
  relation.

\item {\bf Mechanical formalization.} All the results have been mechanically
  formalized in the Coq theorem prover. The formalization is available online at:
  \url{https://github.com/VinaLx/dependent-polymorphic-subtyping}.

\end{itemize}
