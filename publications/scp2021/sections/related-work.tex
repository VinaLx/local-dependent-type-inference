\section{Related Work}

\paragraph{Implicit Dependent Type Calculus}
Implicit polymorphism in a dependently-typed
setting is discussed in the implicit calculus of constructions (ICC)~\citep{miquel2001implicit} and
ICC*\citep{barras2008implicit}. The ICC features generalized polymorphic types and typing rules to express
the idea of implicit instantiation. The ICC does not explicitly have a subtyping
relation between polymorphic types. Therefore the expressiveness of reasoning
between polymorphic types is limited to top-level polymorphic types. Like
in \name, implicit parameters does not impact the runtime semantics of the ICC.

Implicit function types in ICC* are not interpreted as polymorphic
function types. The main focus is on the distinction between implicit
functions (universal types and implicit abstraction) and explicit functions
($\Pi$-types and lambda abstraction).
The typing rules about the implicit part and explicit part of the language mirror
each other. The generalization and instantiation aspect of the implicit function
types are not featured. ICC* depends on its transformation to ICC to obtain type safety
of the language, therefore the parameters of implicit functions have no impact
on runtime behavior as well.

\paragraph{Type-inference and unification with dependent types}
There has been little work on formalizing type inference for calculi
with dependent types, although essentially all implementations of
theorem provers or dependently typed languages perform some form of
type-inference.
One important challenge for type inference in
systems with dependent types and a conversion rule
is that they require \emph{higher-order unification},
which is known to be undecidable~\citep{goldfarb1981undecidability}. The \textit{pattern}
fragment~\citep{miller1991unification} is a well-known decidable
fragment. Much literature on unification for dependent
types~\citep{reed2009higher,abel2011higher, gundry2013tutorial, Cockx:2016:UEP:2951913.2951917, ziliani2015unification, coen2004mathematical} is
built upon the pattern fragment. Algorithms for type-inference used in Agda and
(Dependent) Haskell have been described and formalized to some degree
in various theses~\citep{norell,gundry,dh}. However, as far as we know
there is not a clear specification and complete metatheory (let alone
mechanized) for such algorithms.

The current GHC Haskell's language of types and kinds
is already dependently typed, but has no type conversion. Thus
it is able to avoid higher-order unification. Recent work by
\cite{xie20kind} describes algorithms and specifications
for the form of (dependently typed) kind-inference currently present in GHC Haskell.
The dependently typed language of types and kinds is closely related
to \name. In particular in both calculi type equality is based only
on $\alpha$-equivalence. One difference is that in GHC Haskell and, more precisely,
in the core language employed by GHC, there
are no type-level lambdas. The GHC Haskell source language does allow
type families~\citep{assoctypes}, which mimic type-level functions. However,
type families, unlike lambda functions, are not first class, and do not support partial application.
They are encoded
in terms of equality constraints, casts and mechanisms similar to those
employed by type classes. There is some work to make type-level functions
provided type families first-class~\citep{kiss19higher} and also partially applied, but this
still does not enable full type-level lambdas (see also the discussion in Section~8.1
of ~\citet{kiss19higher} for more details).
In our work we do allow type-level lambdas
but lambdas can only be equal up to $\alpha$-equivalence.
Another difference is that the kind-inference system formalized by Xie et al.
is not higher-ranked like ours. In this way Xie et al. manage to avoid
the mutual dependency issue that we have in our polymorphic subtyping relation.

\paragraph{Type-inference for higher-ranked polymorphism}
Type-inference for \emph{higher-ranked polymorphism}
(HRP)~\citep{dunfield2013complete,le2003ml,leijen2008hmf,vytiniotis2008fph,jones2007practical,Serrano2018, odersky1996putting,zhao19mechanical}
extends the classic Hindley-Milner algorithm~\citep{hindley1969principal,milner1978theory,damas1982principal},
removing the restriction of top-level (let) polymorphism only. Type
inference for HRP aims at providing inference for System F-like
languages. In particular existing HRP approaches allow \emph{synthesis of type arguments}
and use type annotations to aid
inference, since type-inference for full System F is
well-known to be undecidable~\citep{wells1999typability}.

The work on HRP is divided into two strands: \emph{predicative} HRP~\citep{dunfield2013complete,jones2007practical,odersky1996putting,dunfield2019sound,zhao19mechanical}
and \emph{impredicative} HRP~\citep{le2003ml,leijen2008hmf,vytiniotis2008fph,Serrano2018}.
In predicative HRP instantiations can
only synthesize monotypes, whereas in impredicative HRP there's no
such restriction. However impredicative HRP is quite complex because
the polymorphic subtyping relation for impredicative HRP is undecidable~\citep{tiuryn1996subtyping}.
Thus reasonable restrictions that work well in practice are still
a focus of active research.
The monotype restriction on predicative instantiation is considered reasonable
and practical for most programs. It is currently in use by languages such as
(GHC) Haskell, Unison~\citep{Unison} and PureScript~\citep{PureScript}.
The original work on polymorphic subtyping by Odersky and L\"aufer also enforces
the monotype restriction in their subtyping rules (rule $\forallL$) to prevent
choosing a polytype in the instantiation. Based on polymorphic subtyping as
their declarative system,
% \bruno{mention Odersky and Laufer here before moving on to DK}
\cite{dunfield2013complete} (DK) develop an
algorithmic system for predicative HRP type inference. DK's algorithm was
manually proved to be sound, complete, and decidable.
With a more complex declarative system~\citep{dunfield2019sound}, DK
extended their original work with new features.
Recently \cite{zhao19mechanical} formalized DK's type system in the Abella
theorem prover.

\paragraph{Dependent Types and Subtyping}
A major difficulty in languages with subtyping is that the introduction of dependent
types makes typing and subtyping depend on each other. This causes
several difficulties in developing the metatheory for calculi that
combine dependent types and subtyping. Almost all previous
work~\citep{subdep,ptssub,chen1,cocsub,Chen03coc} attempts to address such problem by somehow
\emph{untangling} typing and subtyping, which has the benefit that the
metatheory for subtyping can be developed before the metatheory of
typing. Nevertheless, several results and features prove to be
challenging.

Our work builds on the work done on Pure Iso-Type Systems (PITS)~\citep{yang2019pure}, and
\emph{unified subtyping}~\citep{full}. PITS is a variant of pure type systems (PTSs),
which captures a family of calculi with \emph{iso-types}.
Iso-types generalize \emph{iso-recursive
types}~\citep{tapl}, and provide a simple form of
type casts to address the combination of recursion and
dependent types.
\cite{full} introduce a calculus, called $\lambda_{I}$, supporting OOP features such as
\emph{higher-order subtyping}~\citep{fsubo}, \emph{bounded quantification} and
\emph{top types}.
To address the challenges posed by
the combination of dependent types and subtyping, $\lambda_{I}$
employs \emph{unified subtyping}: a novel technique that unifies
\emph{typing}, \emph{subtyping} and \emph{well-formedness} into one
relation. Therefore, $\lambda_{I}$ takes a significantly different
approach compared to previous work, which
attempts to fight the entanglement between typing and subtyping. In
contrast, $\lambda_{I}$ embraces such
tangling by collapsing the typing and subtyping
relations into the same relation. This approach is different from
Hutchins' technique, which eliminates the typing relation and embeds it into
a combination of subtyping, well-formedness and reduction relations.
In contrast, unified subtyping
retains the traditional concepts of typing and subtyping, which are just two
particular cases of the unified subtyping relation.

Although the $\lambda_{I}$ calculus formalized by Yang and Oliveira shares the use
of unified subtyping with \name, there are substantial differences between the two calculi.
Most importantly, $\lambda_{I}$ only has explicit polymorphism via $\Pi$ types. There
are no implicit functions and universal quantification ($\forall$ types) in $\lambda_{I}$,
and also no guessing of monotypes. \name supports implicit polymorphism, and
guessing the monotypes used for instantiation brings significant complications, for instance
for proving type safety (as discussed in Section~\ref{sec:type-safety}).
The subtyping rules for universal quantification (which do not exist in $\lambda_{I}$) also
bring considerable challenges for transitivity, and the proof technique used by
\name differs considerably from the proof technique used in $\lambda{I}$.
Unlike $\lambda_{I}$, \name does not support bounded quantification, which brings
some welcome simplifications to some of the unified subtyping rules.
Besides these differences other differences include the use of a call-by-name
semantics in \name (see also the discussion in Sections~\ref{sec:open-term-reduction} and \ref{sec:cast-design}),
and the use of the $\star : \square$ axiom in \name versus the use of $\star : \star$
in $\lambda_{I}$.

\paragraph{Dependent Types with Explicit Casts} Another problem is the
interaction between dependent types and recursion. For this
problem, a general solution that has recently emerged is the use
of type casts to control type-level computation. In such an approach explicit casts
are used for performing type-level computations. A motivation for
using type casts is to decouple strong normalization from the
proofs of metatheory, which also makes it possible to allow general
recursion. There have been several studies~\citep{guru,sjoberg:msfp12,
  kimmel:plpv, zombie:popl15, fc:kind, Doorn:2013hq,isotype} working
on using explicit casts instead of conversion rule in a dependently
typed system. In \name we adopt a simple formulation of casts based
on iso-types~\citep{isotype}, but we believe that more powerful notions
of casts could work too.

\paragraph{Dependent Object Types}

Dependent Object Types (DOT)~\citep{dot:dot,dot:path,dot:sound} is another
family of systems that discusses subtyping in a dependently typed setting.
Unlike the traditional dependent type systems that based on lambda calculus, DOT
embraces the idea of ``everything is an object'' and features the \emph{path-dependent types}.
A path-dependent type is a restricted form of dependent types.
Path-dependent types
support return types of functions
to mention their parameters, but only
member accessing operations are allowed for the ``depended value'',
and instead of all terms, only variable names can occur in the accessing path.
This restriction rules out traditional problems in dependent type systems,
like handling type-level computation and allows DOT to focus more on the subtyping
aspect like reasoning about type bounds. Also, DOT can still separate the
concept of terms and types due to the restriction. Therefore, the mutual dependency of
typing and subtyping is also not an issue.

DOT with Implicit Functions (DIF)~\citep{dif} is an interesting extension of DOT
that adds implicit functions to DOT.
Since path-dependent types can encode parametric
polymorphism, adding implicit functions implies adding implicit polymorphism.
The treatment of implicit parameters in DIF is quite similar
to ICC~\citep{miquel2001implicit} in terms of the generalization and
instantiation rules shown in section \ref{sec:polymorphic-subtyping}.
So their system shares similar restrictions of being unable to handle implicits
at higher-ranked positions. However in DIF, implicit arguments are runtime relevant,
and can be retrieved by a special variable. This comes with a restriction that
implicit arguments can only be variables in the typing context when inferred.

\paragraph{Refinement Types and Manifest Systems}

\emph{Manifest} systems~\citep{manifestcontracts} is one of the styles of
contract-oriented programming (in contrast to the \emph{latent} systems~\citep{latentcontracts}),
where contracts (the conditions that programmers expect to satisfy) are expressed
in the type system. $\lambda_H$~\citep{hybridtypes,manifestcontracts} is one calculus that
includes dependent types and subtyping simultaneously. The subtyping relation
expresses the implication relation between contract satisfaction conditions. Unfortunatelly,
this brings the difficulty of potential mutual dependency between typing and subtyping.
To overcome this issue, $\lambda_H$ builds another layer of denotational semantics on top of
subtyping rules to avoid that subtyping depends on typing. However this introduces
complications in terms of metatheory.
System $\mathrm{F}_\mathrm{H}$~\citep{fh} and $\mathrm{F}_\mathrm{H}^\sigma$
~\citep{fhsigma} provide another
interesting idea to deal with this mutual dependency. They get rid of the
subtyping aspect entirely in the type system, but recover it after the system is
defined to prove the ideas expressed by subtyping hold for their systems,
which \citet{fhsigma} also called a \emph{subsumption free formulation}.
However such technique is likely to be difficult to apply for systems that
reason about implicit polymorphism,
as systems like ICC~\citep{miquel2001implicit} that mentions subtyping relations \emph{post facto}
often fail to reason about polymorphism at higher ranks.

\paragraph{Feature Comparison}

Lastly we present a summary comparing \name and some of its related work in
table \ref{table:comparison} to wrap up the discussion of related work.

\begin{table}
  \centering
  % \begin{tabular}{l|c|c|c|c|c|c|c|c}
  \begin{tabular}{l|cccccccc}
   & ICC\mpfootnotemark[1] & $\lambda I_\le$\mpfootnotemark[2] & Zhao19\mpfootnotemark[3] & DK19\mpfootnotemark[4] & Zombie\mpfootnotemark[5] & DH\mpfootnotemark[6] & DIF\mpfootnotemark[7] & \name \\
   \hline
   Dep\mpfootnotemark[8] & $\checkmark$ & $\checkmark$ & $\times$ & Equality\mpfootnotemark[9] & $\checkmark$ & $\checkmark$ & Path\mpfootnotemark[10] & $\checkmark$  \\
   \hline
   Sub\mpfootnotemark[11] & $\times$ & $\checkmark$ & $\checkmark$ & $\checkmark$ & $\times$ & $\times$ & $\checkmark$ & $\checkmark$ \\
   \hline
   Imp\mpfootnotemark[12] & FR\mpfootnotemark[13] & $\times$ & $\checkmark$ & $\checkmark$ & $\times$ & $\checkmark$ & FR\mpfootnotemark[14] & $\checkmark$\mpfootnotemark[13] \\
   \hline
   Alg\mpfootnotemark[15] & $\times$ & S\&C & S\&C & S\&C & S\&C\footnotemark[16] & S & $\times$ & $\times$ \\
   \hline
   Mech\mpfootnotemark[17] & $\times$ & $\checkmark$ & $\checkmark$ & $\times$ & $\times$ & $\times$ & $\times$ & $\checkmark$
  \end{tabular}
  \footnotetext[1]{Implicit Calculus of Constructions~\citep{miquel2001implicit,barras2008implicit}}
  \footnotetext[2]{\cite{full}}
  \footnotetext[3]{Novel algorithm by \cite{zhao19mechanical}, declarative system is the same as \cite{dunfield2013complete}}
  \footnotetext[4]{\cite{dunfield2019sound}}
  \footnotetext[5]{\cite{zombie:popl15}}
  \footnotetext[6]{Dependent Haskell \citep{dh}}
  \footnotetext[7]{Dependent-object-type with Implicit Functions~\citep{dif}}
  \footnotetext[8]{Whether the system support dependent types. $\checkmark$, if general dependent function and some form of type-level computation is supported. $\times$, if dependent types are not supported at all. Other entries see footnotes.}
  \footnotetext[9]{Only type-level equality is dependent, other parts of the system are not.}
  \footnotetext[10]{Path-dependent Types~\citep{dot:path}}
  \footnotetext[11]{Whether there is a subtyping relation in the typing rules. The complexity of subtyping relations varies but we do not dive into details here.}
  \footnotetext[12]{Whether some form of implicit polymorphism is supported. Monotype restriction on instantiation is assumed. $\checkmark$, if implicit polymorphism is supported at higher rank. FR, if only first-rank implicit polymorphism are supported.}
  \footnotetext[13]{Implicit instantiations are runtime irrelevant.}
  \footnotetext[14]{Implicit instantiations can only be variables.}
  \footnotetext[15]{Whether there is an algorithmic system. S, if the algorithm is sound (with respect to \emph{some criteria}). C, if the algorithm is complete.}
  \footnotetext[16]{The congruence relation does not have a algorithmic system, its algorithm is stated and proved seperately from the type system.}
  \footnotetext[17]{Whether the correctness of metatheory (and algorithm, if available) is mechanically checked instead of manually.}
  \caption{Feature Comparison of \name}
  \label{table:comparison}
\end{table}
