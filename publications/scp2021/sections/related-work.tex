\section{Related Work}

\paragraph{Implicit Dependent Type Calculus}
The implicit calculus of constructions (ICC~\cite{miquel2001implicit} and
ICC*\cite{barras2008implicit}) discuss implicit polymorphism in a dependently-typed
setting. ICC features generalized polymorphic types and typing rules to express
the idea of implicit instantiation. They do not explicitly have a subtyping
relation between polymorphic types. Therefore the expressiveness of reasoning
between polymorphic types are limited to the top-level polymorphic types. Like
in \name, the implicit parameter does not impact the runtime semantics of ICC.

Implicit function types of ICC* are not interpreted as polymorphic
function types. The main focus is on the distinction between implicit
functions (universal types and implicit abstraction) and explicit functions
($\Pi$-types and lambda abstraction).
The typing rules about the implicit part and explicit part of the language mirror
each other. The generalization and instantiation aspect of the implicit function
types are not featured. ICC* depends on its transformation to ICC to obtain type safety
of the language, therefore the parameters of implicit functions have no impact
on runtime behavior as well.

\paragraph{Type-inference and unification with dependent types}
There has been little work on formalizing type inference for calculi
with dependent types, although essentially all implementations of
theorem provers or dependently typed languages perform some form of
type-inference.
One important challenge for type inference in
systems with dependent types and a conversion rule
is that they require \emph{higher-order unification},
which is known to be undecidable~\cite{goldfarb1981undecidability}. The \textit{pattern}
fragment~\cite{miller1991unification} is a well-known decidable
fragment. Much literature on unification for dependent
types~\cite{reed2009higher,abel2011higher, gundry2013tutorial, Cockx:2016:UEP:2951913.2951917, ziliani2015unification, coen2004mathematical} is
built upon the pattern fragment. Algorithms for type-inference used in Agda and
(Dependent) Haskell have been described and formalized to some degree
in various theses~\cite{norell,gundry,dh}. However, as far as we know
there is not a clear specification and complete metatheory (let alone
mechanized) for such algorithms.

The current GHC Haskell's language of types and kinds
is already dependently typed, but has no type conversion. Thus
it is able to avoid higher-order unification. Recent work by
Xie et al.~\cite{xie20kind} describes algorithms and specifications
for the form of (dependently typed) kind-inference currently present in GHC Haskell.
The dependently typed language of types and kinds is closely related
to \name. In particular in both calculi type equality is based only
on $\alpha$-equivalence. One difference is that in GHC Haskell and, more precisely,
in the core language employed by GHC, there
are no type-level lambdas. The GHC Haskell source language does allow
type families~\cite{assoctypes}, which mimick type-level functions. However,
type families, unlike lambda functions, are not first class, and do not support partial application.
They are encoded
in terms of equality constraints, casts and mechanisms similar to those
employed by type classes. There is some work to make type-level functions
provided type families first-class~\cite{kiss19higher} and also partially applied, but this
still does not enable full type-level lambdas (see also the discussion in Section~8.1
of ~\cite{kiss19higher} for more details).
In our work we do allow type-level lambdas
but lambdas can only be equal up to $\alpha$-equivalence.
Another difference is that the kind-inference system formalized by Xie et al.
is not higher-ranked like ours. In this way Xie et al. manage to avoid
the mutual dependency issue that we have in our polymorphic subtyping relation.

\begin{comment}
Thus our focus is on Haskell-like languages with dependently typed
features and explicit casts, rather than languages like Agda or Idris
which typically have a conversion rule that triggers implicit type level
computation.
\end{comment}


\paragraph{Type-inference for higher-ranked polymorphism}
Type-inference for \emph{higher-ranked polymorphism}
(HRP)~\cite{dunfield2013complete,le2003ml,leijen2008hmf,vytiniotis2008fph,jones2007practical,Serrano2018, odersky1996putting}
extends the classic Hindley-Milner algorithm~\cite{hindley1969principal,milner1978theory,damas1982principal},
removing the restriction of top-level (let) polymorphism only. Type
inference for HRP aims at providing inference for System F-like
languages. In particular existing HRP approaches allow \emph{synthesis of type arguments}
and use type annotations to aid
inference, since type-inference for full System F is
well-known to be undecidable~\cite{wells1999typability}.

The work on HRP is divided into two strands: \emph{predicative} HRP~\cite{dunfield2013complete,jones2007practical,odersky1996putting,dunfield2019sound}
and \emph{impredicative} HRP~\cite{le2003ml,leijen2008hmf,vytiniotis2008fph,Serrano2018}.
In predicative HRP instantiations can
only synthesize monotypes, whereas in impredicative HRP there's no
such restriction. However impredicative HRP is quite complex because
the polymorphic subtyping relation for impredicative HRP is undecidable~\cite{tiuryn1996subtyping}.
Thus reasonable restrictions that work well in practice are still
a research frontier.
The monotype restriction on predicative instantiation is considered reasonable
and practical for most programs. It is currently in use by languages such as
(GHC) Haskell, Unison~\cite{Unison} and PureScript~\cite{PureScript}.
The original work on polymorphic subtyping by Odersky and L\"aufer also enforces
the monotype restriction in their subtyping rules (rule $\forallL$) to prevent
choosing a polytype in the instantiation. Based on polymorphic subtyping as
their declarative system,
% \bruno{mention Odersky and Laufer here before moving on to DK}
Dunfield and Krishnaswami (DK)~\cite{dunfield2013complete} develop an
algorithmic system for predicative HRP type inference. DK's algorithm was
manually proved to be sound, complete, and decidable.
With a more complex declarative system~\cite{dunfield2019sound}, DK
extended their original work with new features.
Recently Zhao et al.~\cite{zhao19mechanical} formalized DK's type system in the Abella
theorem prover.

\begin{comment}
\paragraph{MLSub} A recent breakthrough in the area of (global) type-inference
for type systems with subtyping is MLSub~\cite{dolan17polymorphism}. MLSub extends the Hindley-Milner
type system with support for subtyping. A key innovation
of MLSub is that it has compact principal types, which had been a challenge
in previous research on type-inference in the presence of subtyping~\cite{eifrig95inference,Trifonov96subtyping,pottier1998inference}.
MLSub is significantly more ambitious than local type-inference, and requires
no annotations (in the tradition of Hindley-Milner). However, MLSub does not
account for HRP and its algorithms and metatheory have not been mechanically
formalized.
\end{comment}

\paragraph{Dependent Types and Subtyping}
A major difficulty is that the introduction of dependent
types makes typing and subtyping depend on each other. This causes
several difficulties in developing the metatheory for calculi that
combine dependent types and subtyping. Almost all previous
work~\cite{subdep,ptssub,chen1,cocsub,Chen03coc} attempts to address such problem by somehow
\emph{untangling} typing and subtyping, which has the benefit that the
metatheory for subtyping can be developed before the metatheory of
typing. Nevertheless, several results and features prove to be
challenging.

Our work builds on the work done on Pure Iso-Type Systems (PITS)~\cite{yang2019pure}, and
\emph{unified subtyping}~\cite{full}. PITS is a variant of pure type systems (PTSs),
which captures a family of calculi with \emph{iso-types}.
Iso-types generalize \emph{iso-recursive
types}~\cite{tapl}, and provide a simple form of
type casts to address the combination of recursion and
dependent types.
Yang and Oliveira~\cite{full} introduce a calculus, called $\lambda_{I}$, supporting OOP features such as
\emph{higher-order subtyping}~\cite{fsubo}, \emph{bounded quantification} and
\emph{top types}.
To address the challenges posed by
the combination of dependent types and subtyping, $\lambda_{I}$
employs \emph{unified subtyping}: a novel technique that unifies
\emph{typing}, \emph{subtyping} and \emph{well-formedness} into one
relation. Therefore, $\lambda_{I}$ takes a significantly different
approach compared to previous work, which
attempts to fight the entanglement between typing and subtyping. In
contrast, $\lambda_{I}$ embraces such
tangling by collapsing the typing and subtyping
relations into the same relation. This approach is different from
Hutchins' technique, which eliminates the typing relation and embeds it into
a combination of subtyping, well-formedness and reduction relations.
In contrast, unified subtyping
retains the traditional concepts of typing and subtyping, which are just two
particular cases of the unified subtyping relation.

Although the $\lambda_{I}$ calculus formalized by Yang and Oliveira shares the use
of unified subtyping with \name, there are substantial differences between the two calculi.
Most importantly, $\lambda_{I}$ only has explicit polymorphism via $\Pi$ types. There
are no implicit functions and universal quantification ($\forall$ types) in $\lambda_{I}$,
and also no guessing of monotypes. \name supports implicit polymorphism, and
guessing the monotypes used for instantiation brings significant complications, for instance
for proving type safety (as discussed in Section~\ref{sec:type-safety}).
The subtyping rules for universal quantification (which do not exist in $\lambda_{I}$) also
bring considerable challenges for transitivity, and the proof technique used by
\name differs considerably from the proof technique used in $\lambda{I}$.
Unlike $\lambda_{I}$, \name does not support bounded quantification, which brings
some welcome simplifications to some of the unified subtyping rules. 

\paragraph{Dependent Types with Explicit Casts} Previously
discussed work is about the interaction between dependent types and
subtyping. However, the other problem is the
interaction between dependent types and recursion. For this
problem, a general solution that has recently emerged is the use
of type casts to control type-level computation. In such an approach explicit casts
are used for performing type-level computations. A motivation for
using type casts is to decouple strong normalization from the
proofs of metatheory, which also makes it possible to allow general
recursion. There have been several studies~\cite{guru,sjoberg:msfp12,
  kimmel:plpv, zombie:popl15, fc:kind, Doorn:2013hq,isotype} working
on using explicit casts instead of conversion rule in a dependently
typed system. In \name we adopt a simple formulation of casts based
on iso-types~\cite{isotype}, but we believe that more powerful notions
of casts could work too.
