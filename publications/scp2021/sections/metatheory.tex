\section{The Metatheory of \name}
\label{sec:metatheory}

This section presents the metatheory of \name, and discusses several challenges
that arose in the design of the rules to ensure the desired properties.

The three main results of the metatheory are:
\emph{transitivity of unified subtyping}, \emph{type-soundness} and
\emph{completeness with respect to Odersky and L\"aufer's} polymorphic
subtyping. Transitivity of subtyping is a general challenge for dependent type systems due
to the mutual dependency of typing and subtyping, and the Odersky and L\"aufer style
subtyping brings new issues to the table. For type-soundness, the key challenge
is the non-deterministic and non-type-preserving nature of the reduction relation.
To address this issue, we employ a type soundness proof technique
that makes use of the erased reduction relation shown in Figure~\ref{fig:extraction}.

\subsection{Polymorphic Subtyping in a Dependently Typed Setting}
\label{sec:adaptation}

The polymorphic subtyping relation by Odersky and L\"aufer features the following two rules:
\begin{mathpar}
    \inferrule*[right=$\forallL$]
      {\Gamma \vdash \tau \\ \Gamma \vdash [\tau / x]\, A \le B}
      {\Gamma \vdash \forall x.\, A \le B}
    \and
    \inferrule*[right=$\forallR$]
      {\Gamma ,\, x \vdash A \le B}
      {\Gamma \vdash A \le \forall x.\, B}
\end{mathpar}
In order for the \emph{well-formedness}~\cite{dunfield2013lemmas} property
(\emph{If $\Gamma \vdash A \le B$, then $\Gamma \vdash A$ and $\Gamma \vdash B$})
to hold in L\"aufer and Odersky's system, these two rules rely on certain properties
that do not hold in our dependently typed generalization. So we make several adjustments
in our adaptation to combat these issues, which result in the difference between
our current system and a direct generalization.

\subsubsection{Reverse Substitution of Well-Formedness.}
Rule $\forallL$ relies on the \emph{reverse substitution} property, but this property
does not hold in a dependently typed setting. Thus we need an alternative design that
still ensures well-formedness, but without relying on the reverse substitution property.

The reverse substitution property is:
\emph{If $\Gamma \vdash [B / x] \, A$ and $\Gamma \vdash B$, then $\Gamma,\, x \vdash A$}.
That is if we have a type $A$ with all occurrences of $x$ substituted by $B$ and $B$
is well-formed, we can conclude that $A$ is well-formed under $\Gamma,\, x$.
A possible generalization of this property in a dependently typed setting would be:
\emph{If $[[G |- [B / x] A : *]]$ and $[[G |- B : C]]$, then $[[G , x : C |- A : *]]$}.
In a dependently-typed setting, the values of expressions also matter during typing
besides the types of expressions, a counter-example of the property is:
\begin{align*}
& F : [[int]] \rightarrow [[*]],\, a : F~42  \vdash (\lambda y : F~\rulehl{42}.\, [[int]]) ~ a : [[*]] \\
& F : [[int]] \rightarrow [[*]],\, a : F~42  \rulehl{,\, x : [[int]]} \vdash (\lambda y : F~\rulehl{x}.\, [[int]]) ~ a : [[*]]
\end{align*}
We cannot ``reverse substitute'' the $42$ in the type annotation to a variable
of the same type: the application expression depends specifically on the value
$42$ in order for the type of argument $a$ to match the parameter type.
So we add a premise $[[G , x : A |- B : *]]$ in \rref{s-forall-l} to directly
ensure the well-formedness of types in the conclusion.

\subsubsection{Strengthening of Contexts.}

Rule $\forallR$ relies on a strenghtening lemma:
\emph{if $\Gamma,\, x \vdash A$ and $x ~\text{is fresh in}~ A$, then $\Gamma \vdash A$}, which
is trivial to prove.
However its generalization:
\emph{if $[[G, x : B |- A : *]]$ and $x ~\text{is fresh in}~ A$, then $[[G |- A : *]]$},
does not hold in our system. We can construct the following example:
\begin{equation*}
    F : [[int]] \rightarrow [[*]],\, A : [[*]],\, \rulehl{a : A} \vdash F ~ ([[(bind x : A. lambda y : int. y)]]~ 42) : [[*]]
\end{equation*}
The variable $a$ does not appear in any expression, but plays a
crucial role when considering the subtyping relation $\Gamma \vdash \forall x : A.\, [[int]] \rightarrow [[int]] \le [[int]] \rightarrow [[int]]$.
In this case, we cannot apply \rref{s-forall-l} while we are not able to find a
well-typed instance for the polymorphic parameter, so the variable in the context
is needed even though it does not occur anywhere.

To counter the lack of strenghtening, we add a premise $[[G |- A : *]]$ to \rref{s-forall-r},
requiring $A$ to be a well-typed type without the help of the fresh variable.
A consequence of adding this premise is that we will encounter a circular proof
while trying to prove $[[G |- forall x : A. B <: forall x : A . B : *]]$ for
arbitrary $A$ and $B$ by first applying \rref{s-forall-r}. We resolve this issue by
adding \rref{s-forall}.

\subsection{Typing Properties of \name}

With our rules properly set up, we can prove most of the basic properties
using techniques borrowed from the \emph{unifed subtyping}~\cite{full} approach.
We introduce \emph{reflexivity}, \emph{weakening},
\emph{context narrowing}, \emph{substitution} and \emph{type correctness}
in this section.

\begin{theorem}[Reflexivity]
   If $[[G |- e1 <: e2 : A]]$,
   then $[[G |- e1 : A]]$ and $[[G |- e2 : A]]$.
\end{theorem}

Usually, a subtyping relation is reflexive when any well-formed type is a subtype
of itself. With unified subtyping, the well-formedness of types is expressed by
subtyping relation as well, so the reflexivity looks more like the generalization
of \emph{well-formedness} mentioned in the previous section. Reflexivity
breaks down into two parts, \emph{left reflexivity} and \emph{right reflexivity}.

\begin{lemma}[Left Reflexivity]
   If $[[G |- e1 <: e2 : A]]$,
   then $[[G |- e1 : A]]$.
\end{lemma}

\begin{lemma}[Right Reflexivity]
   If $[[G |- e1 <: e2 : A]]$,
   then $[[G |- e2 : A]]$.
\end{lemma}

\noindent Both of the branches are proved by induction on the derivation of
$[[G |- e1 <: e2 : A]]$.
Left reflexivity and right reflexivity when derivations end with \rref{s-forall-l}
and \rref{s-forall-r} respectively are directly solved by \rref{s-forall}.

\begin{theorem}[Weakening]
    If $[[G1 ,, G3 |- e1 <: e2 : A]]$ and $[[|- G1 ,, G2 ,, G3]]$,
    then $[[G1 ,, G2 ,, G3 |- e1 <: e2 : A]]$.
\end{theorem}

\noindent \emph{Weakening} is proved by induction on the derivation of
$[[G1 ,, G3 |- e1 <: e2 : A]]$. The redundant premises discussed in Section
\ref{sec:type-system} help simplify the proof, by creating the induction
hypotheses about the type annotation of various expressions. Otherwise, we are
not able to prove $[[|- G1 ,, G2 ,, G3 , x : A]]$ given only
$[[|- G1 ,, G3 , x : A]]$ and no help from induction hypotheses.

\begin{theorem}[Context Narrowing]
\label{thm:narrowing}
    If $[[G1 , x : B ,, G2 |- e1 <: e2 : C]]$ and $[[G1 |- A <: B : k]]$,
    then $[[G1 , x : A ,, G2 |- e1 <: e2 : C]]$.
\end{theorem}

\begin{lemma}[Well-formedness of Narrowing Context]
\label{thm:wf-narrowing}
   If $[[|- G1 , x : B ,, G2]]$ and $[[G1 |- A <: B : k]]$,
   then $[[|- G1 , x : A ,, G2]]$.
\end{lemma}

\noindent Theorem \ref{thm:narrowing} and Lemma \ref{thm:wf-narrowing} are proved by
mutual induction on the derivations of $[[G1 , x : B ,, G2 |- e1 <: e2 : C]]$
and $[[|- G1 , x : B ,, G2]]$. \Rref{s-var} is the only non-trivial case to
solve: it relies on \emph{weakening} to conclude
$[[G1 , x : A ,, G2 |- A <: B : k]]$ from $[[G1 |- A <: B : k]]$, in order to
derive $[[G1 , x : A ,, G2 |- x : B]]$ through \rref{s-sub}.

\begin{theorem}[Substitution]
    If $[[G1 , x : A ,, G2 |- e1 <: e2 : B]]$ and $[[G1 |- t : A]]$,
    then $[[G1 ,, [t / x] G2 |- [t / x] e1 <: [t / x] e2 : [t / x] B ]]$.
\end{theorem}

\noindent Notably \emph{substitution} has a mono-expression restriction on the
expression being substituted with. This is result from the mono-expression restriction on
the instantiation of polymorphic parameters in \rref{s-forall-l}.

Take the following derivation as an example:

\begin{mathpar}
    \inferrule*[Right=s-forall-l]
      {A : [[*]],\, F: A \rightarrow [[*]]\rulehl{,\, a : A} \vdash [\rulehl{a} / x]\, F ~ x \le F~\rulehl{a} : [[*]]}
      {A : [[*]],\, F: A \rightarrow [[*]]\rulehl{,\, a : A} \vdash \forall x : A.\, F ~ x \le F~\rulehl{a} : [[*]]}
\end{mathpar}

\noindent Assuming that we have no mono-expression restrictions on \emph{substitution}
and \rref{s-app}. If we substitute $a$ with an arbitrary poly-expression, the
derivation stops working because \rref{s-forall-l} requires a mono-expression
instantiation and \rref{s-app} requires the argument of both sides to be
syntactically the same.

Worth mentioning is that while substitution of poly-expressions breaks the
subtyping aspect of the language, a special case of the substitution theorem
that discusses the typing of one expression
(If $[[G1 , x : A ,, G2 |- e : B]]$ and $[[G1 |- e1 : A]]$,
then $[[G1 ,, [e1 / x] G2 |- [e1 / x] e : [e1 / x] B]]$)
does not hold for similar reasons.
Because, in dependently-typed languages, substitutions are also involved in the types of
expressions as well. Due to the presence of \rref{s-sub}, we still have to
maintain the potential subtyping relation of the types of expressions after substitution,
for example:

\begin{mathpar}
    \inferrule*[Right=s-sub]
      {A : [[*]],\, F: A \rightarrow [[*]],\, a : A,\, b : \forall x : A.\, F~x \vdash b : \rulehl{\forall x : A.\, F~x}}
      {A : [[*]],\, F: A \rightarrow [[*]],\, a : A,\, b : \forall x : A.\, F~x \vdash b : \rulehl{F ~ a}}
\end{mathpar}

% \noindent As a result, we put mono-expression restriction on the \emph{substitution} theorem,
% as well as the typing rules involving expressions that potentially ``trigger''
% substitutions during reductions, including the arguments of applications and the
% fixpoint expressions.
\noindent As a result, \emph{substitution} only holds with the mono-expression
restriction, It has a cascading effect on typing rules like \rref{s-app, s-mu}
whose expressions trigger substitutions during reduction. So the mono-expression
restriction has to be added for those rules for the system to be type-safe.

\begin{lemma}[Context Well-formedness of Substitution]
\label{thm:wf-substitution}
   If $[[|- G1 , x : A  ,, G2]]$ and $[[G1 |- t : A]]$,
   then $[[|- G1 ,, [t / x] G2]]$.
\end{lemma}

\noindent After understanding the mono-expression restriction on \emph{substitution}, the actual
proof is not complicated: it proceeds by mutual induction with
Lemma \ref{thm:wf-substitution} on the derivations of
$[[G1 , x : A ,, G2 |- e1 <: e2 : B]]$ and $[[|- G1 , x : A ,, G2]]$. When the
derivation ends with \rref{castup,castdn}, the proof
requires the reduction relation to preserve after the substitution.
This property should usually hold, but it puts an interesting constraint which we
have to consider when designing the reduction rules (see Section \ref{sec:cast-design}).

\begin{lemma}[Reduction Substitution]
   If $[[A --> B]]$, then $[[ [t / x] A --> [t / x] B ]]$
\end{lemma}

\begin{theorem}[Type Correctness]
    If $[[G |- e1 <: e2 : A]]$,
    then $\exists k.\, [[G |- A : k]]$ or $A = [[box]]$.
\end{theorem}

\noindent \emph{Type correctness} is a nice property that ensures that
what appears in the position of a type are actually types.
The theorem is proved by induction on the derivation of $[[G |- e1 <: e2 : A]]$,
the only non-trivial case is when the derivation ends with \rref{s-app}. We make
use of the substitution lemma and the inductive hypothesis to demonstrate the head
of a $\Pi$ type preserves its kind after the argument is applied.

\subsection{Transitivity}
\label{sec:transitivity}

Transitivity is typically one of the most challenging properties to prove in
calculi with subtyping and it was also one of the harder proofs in \name.
The proof of transitivity requires a generalization of the usual transitivity
property:

\begin{theorem}[Generalized Transitivity]
    If $[[G |- e1 <: e2 : A]]$ and $[[G |- e2 <: e3 : B]]$,
    then $[[G |- e1 <: e3 : A]]$.
\end{theorem}

\noindent where the types of the premises are potentially different.
To prove this property we employ sizes for the inductive argument. Moreover we rely on
a subtle property of uniqueness of kinds.
\paragraph{Uniqueness of Kinds} Assuming that the derivation of the second
premise of generalized transitivity ends with \rref{s-forall-r}, then we face the following problem:
\begin{mathpar}
    \inferrule*[]
      {[[G |- e1 <: e2 : A]] \\ [[G, x : B |- e2 <: C : *]]}
      {[[G |- e1 <: forall x : B. C : A]]}
\end{mathpar}
\noindent Before applying \rref{s-forall-r} to the conclusion,
we have to establish the relationship between $A$ and $[[*]]$.
Were there no restrictions on the kinding of $\forall$ types,
this would be a much more complicated situation, where the inversion lemmas of
about kinds and transitivity depend on each other.
This is one of the main reasons why we forbid $\forall$ types having kind $[[box]]$.
Then we can have the following theorem:

\begin{theorem}[Kind Uniqueness]
    If $[[G |- e : k]]$ and $[[G |- e : A]]$,
    then $A = k$.
\end{theorem}

\noindent The proof is achieved by generalizing the shape of $k$ to be
$\Pi x : A.\, \dots \Pi x : B.\, \dots k$ for useful inductive hypotheses
when the derivation of $[[G |- e : k]]$ ends with \rref{s-app}. Then the proof
proceeds with induction on the derivation of the generalized $[[G |- e : k]]$ and
assembling various lemmas of kinding to solve different cases.

With the help of \emph{kind uniqueness},
we ensure the equivalence of $A$ and $[[*]]$ on this and other similar situations.

\paragraph{The Induction}

We prove \emph{Generalized Transitivity} by performing the strong induction on
the ordered 3-tuple of measures:

$$
\langle \#\forall([[e1]]) + \#\forall([[e2]]) + \#\forall([[e3]]), ~
\mathbf{size}([[e1]]) + \mathbf{size}([[e3]]), ~
\mathcal{D}_1 + \mathcal{D}_2 \rangle
$$

\noindent where

\begin{itemize}
    \item $\#\forall(e)$ counts the number of $\forall$ quantifiers
    in expression $e$, which solves cases when either side of the premise is
    derived by \rref{s-forall, s-forall-l, s-forall-r}.
    \item $\mathbf{size}(e)$ measures the size of the syntax tree of
    expression $e$. The sum of expression sizes solves most of the other
    standard recursive cases.
    \item $\mathcal{D}_1$ and $\mathcal{D}_2$ denote the sizes of the derivation
    trees of the first and the second premise respectively. The sum of sizes
    of derivation tree solve the case involving \rref{s-sub} where the sizes of
    expressions does not decrease.
\end{itemize}

% $\#\forall$ counts the number of $\forall$ quantifiers in the expression,
% $\mathbf{size}$ measures the size of the syntax tree of the expressions,
% $\mathcal{D}_1$ and $\mathcal{D}_2$ denote the sizes of the derivation trees of
% the first and the second premises respectively.
The proof is mainly motivated by DK's transitivity proof of their declarative subtyping
system of induction on the pair of
$\langle \#\forall([[e2]]) ,~ \mathcal{D}_1 + \mathcal{D}_2 \rangle$~\cite{dunfield2013lemmas},
with some adjustments to fit in our system.

The most problematic case to solve is when the first premise ends with \rref{s-forall-r},
and the second ends with \rref{s-forall-l}, essentially we have to show the following:

\begin{mathpar}
    \inferrule*[]
      {[[G , x : A |- e1 <: B : *]] \\ [[G |- [t / x] B <: e3 : *]] \\ [[G |- t : A]]}
      {[[G |- e1 <: e3 : *]]}
\end{mathpar}

Here the only decreasing measure we can rely on is that
$\#\forall([[ [t / x] B ]])$ is one less than $\#\forall([[forall x : A. B]])$
(since $\tau$ is a monotype which does not contain $\forall$ quantifiers).
To solve this case, we first performs a substitution on the premise
$[[G , x : A |- e1 <: B : *]]$ with the help of the fact that
$x$ does not occur in $e_1$, obtaining $[[G |- e1 <: [t / x] B : *]]$, then
we use the inductive hypothesis provided by $\#\forall([[e2]])$.

The reason why we cannot directly copy DK's proof measure is because of the case
when both premises end with \rref{s-pi}, where we encounter the following problem:
\begin{mathpar}
    \inferrule*[]
      {[[G , x : A2 |- B1 <: B2 : k]] \\ [[G , x : A3 |- B2 <: B3 : k]] \\ [[G |- A3 <: A2 : k2]]}
      {[[G , x : A3 |- B1 <: B3 : k]]}
\end{mathpar}
The first and the second premise above do not share the same context, which
must be unified with the context narrowing theorem to be able to use the
inductive hypotheses.
However context narrowing potentially increases the size of the
derivation tree, so we are not able to use the inductive hypothesis of
$\mathcal{D}_1 + \mathcal{D}_2$, and resort to the sizes of expressions
($\mathbf{size}(e_1) + \mathbf{size}(e_3)$)

And then we have to make adjustments to solve the cases which preserve the
size of derivation tree, but not the sizes of the expressions, which is
when the first premise ends with \rref{s-forall-l}:

\begin{mathpar}
    \inferrule*[]
      {[[G |- [t / x] B <: e2 : *]] \\ [[G |- e2 <: e3 : C]]}
      {[[G |- forall x : A. B <: e3 : *]]}
\end{mathpar}

\noindent In this case, $\#\forall({[[ [t / x] B]]})$ is one less than $\#\forall([[forall x : A. B]])$,
so it can be solved by applying \rref{s-forall-l} and the inductive hypothesis of $\#\forall(e_1)$.
And $\#\forall(e_3)$ is added to make the measure ``symmetric''
to deal with the contravariance case of \rref{s-pi}.

Then, most of the cases that do not involve $\forall$ can be
solved by applying the inductive hypothesis corresponding to
$\mathbf{size}(e_1) + \mathbf{size}(e_3)$.
Finally, $\mathcal{D}_1 + \mathcal{D}_2$ solves
the cases where either premise ends with \rref{r-sub}, where the only decreasing
measure is the size of the derivation trees when the sizes of expressions remain
the same.

\begin{corollary}[Transitivity]
    If $[[G |- e1 <: e2 : A]]$ and $[[G |- e2 <: e3 : A]]$,
    then $[[G |- e1 <: e3 : A]]$.
\end{corollary}

\emph{Transitivity} is a special case of
\emph{Generalized Transitivity} where $A = B$.

\subsection{Type Safety}
\label{sec:type-safety}

Since the reduction rules of \name do not have access to typing information, they
cannot perform valid instantiation checks of the implicit arguments during applications.
Thus, the runtime semantics is non-deterministic and potentially non-type-preserving.
We tackle this issue by employing designs that make the choice of implicit instantiation
irrelevant at runtime with the restrictions in \rref{s-bind}.
We define an erasure function (shown in Figure \ref{fig:extraction})
that eliminates the type annotations
in some of the expressions ($\lambda$, $\Lambda$, $\mu$ and $\castup$),
and keep the implicit parameter from occurring in the erased expressions.
This way the choices of implicit instantiation only affect type annotations,
which are not relevant for runtime computation.

We show that \name is type-safe in the sense that,
if an expression is well-typed, then the reduction of its erased version
does not ``go wrong''. Figure \ref{fig:extraction} shows the semantics of
erased expressions. The erasure semantics mostly mirrors the semantics
shown in Figure \ref{fig:semantics}, except for \rref{er-elim,er-cast-elim}, which
conveys the idea of the irrelevance of implicit instantiation by eliminating the
parameter directly.

\paragraph{Progress} We show the \emph{progress} property for both of
the original expressions and erased expressions.

\begin{theorem}[Progress]
    If $[[nil |- e1 <: e2 : A]]$,
    then $\exists \, e_1'. \, e_1 \longrightarrow e_1'$ or $e_1$ is a value.
    And $\exists \, e_2'. \, e_2 \longrightarrow e_2'$ or $e_2$ is a value.
\end{theorem}

\begin{theorem}[Progress on erased expressions]
    If $[[nil |- e1 <: e2 : A]]$,
    then $\exists \, E_1'. \, \extract{e_1} \longrightarrow_E E_1'$ or $\extract{e_1}$ is an erased value.
    And $\exists \, E_2'. \, \extract{e_2} \longrightarrow_E E_2'$ or $\extract{e_2}$ is an erased value.
\end{theorem}

\noindent We prove a generalized version of \emph{progress} that talks about both sides
of the expressions with unified subtyping. Note that they are not necessarily
simultaneously reducible or not reducible due to the presence of \rref{s-forall-l, s-forall-r}.
The left-hand-side may be reducible with the right-hand-side being a value or vice versa.

Both theorems are proved by induction on the derivation of $[[nil |- e1 <: e2 : A]]$.
The proof is mostly straightforward except when the derivation ends with $\castdn$,
where we have to show that the inner expressions $e$ of $\castdn~e$ either reduces, or
is $\castup$ or $\Lambda$ expressions. We prove another fact to solve the situation:
for a well-typed expression whose type reduces, that expression
cannot be a value unless it is a $\castup$ or $\Lambda$ expression.

\begin{lemma}[Reducible Type]
    If $[[G |- e : A]]$, $[[A --> B]]$ and $e$ is not $\castup$ or $\Lambda$ expressions,
    then $e$ is not a value.
\end{lemma}

\begin{lemma}[Erased Value to Value]
    If $\extract{e}$ is an erased value, then $e$ is a value.
\end{lemma}

This lemma is also useful for the proof of progress for erased expressions.
Since the value definitions are very similar, we can use the property of values
on erased values.

\begin{lemma}[Value to Erased Value]
    If $e$ is a value, then $\extract{e}$ is an erased value.
\end{lemma}

\paragraph{Preservation}
The direct operational semantics is not generally type-preserving and
deterministic because of the implicit instantiations. Thus, we show
preservation with the help of the erased expressions (where implicit parameters
do not matter to the computation). For other reduction rules that do not involve
such issues, we discuss them as though we are proving a normal preservation
for brevity.

\begin{theorem}[Subtype Preservation]
    If $[[G |- e1 <: e2 : A]]$, $\extract{e_1} \longrightarrow_E E_1'$ and $\extract{e_2} \longrightarrow_E E_2'$,
    then $\exists\,e_1' \, e_2'.$ $\extract{e_1'} = E_1'$, $\extract{e_2'} = E_2'$,
    $e_1 \longrightarrow e_1'$, $e_2 \longrightarrow e_2'$ and $\Gamma \vdash e_1' \le e_2' : A$.
\end{theorem}

\begin{figure}
    \centering
    \begin{tikzcd}[row sep=large, column sep=large]
        e : A \arrow[r] \arrow[dashed]{d}[swap]{\text{Erasure}} & e' : A \\
        E \arrow[shorten=1mm]{r}[swap, pos=1]{E} & E' \arrow[dashed]{u}[swap]{\text{Annotation}}
    \end{tikzcd}
    \caption{Diagram for Erased Preservation without Subtyping}
    \label{fig:preservation}
\end{figure}

\noindent The theorem might look a little complicated at first glance. It breaks
down into two aspects: the erasure-annotation process and the subtype preservation.

Figure \ref{fig:preservation} shows the idea of our preservation lemma without
considering the subtyping aspect (assuming $[[G |- e : A]]$ instead of $[[G |- e1 <: e2 : A]]$).
Here we use \emph{annotation} as the reverse process of erasure.
If an expression ($e$) is well-typed, and its erasure ($E$)
reduces to another erased expression ($E'$), we can find a ``annotated''
expression of $E'$ ($e'$) that is reduced by $e$ and also preserves the type $A$.
When no implicit instantiation happens in the reduction, then $e \longrightarrow e'$
is deterministic: i.e. it is just normal type preservation. When there is implicit
instantiation, if the erased expression can reduce, we show that there must
exist a valid instantiation for $e$ that preserves its type after the reduction, and
this instantiation only affects type annotations.
In other words, the runtime semantics of \name can be implemented only with
erased expressions.

Aside from the erasure aspect of our preservation lemma, we also consider the
generalized version of preservation in our unified subtyping system, the
\emph{subtype preservation}: reductions not only preserve the type of expression,
the subtyping relation between expressions is preserved as well.

The theorem is proved by induction on the derivation of $[[G |- e1 <: e2 : A]]$,
cases for \rref{r-beta,r-mu} are solved with the substitution theorem,
cases \rref{r-app,r-castdn} are solved by inductive hypotheses. The interesting
cases to prove are \rref{r-cast-elim} and cases involving implicit instantiation
(\rref{r-inst,r-cast-inst}).

\paragraph{Cast Elimination}
The main issue of the cast elimination case can be demonstrated by the following derivation:

\begin{mathpar}
    \hspace{-1.5cm}
    \inferrule*[Right=s-castdn]
      {\rulehl{[[B1 --> B2]]} \\ \inferrule*[Right=s-sub]
        {\rulehl{[[G |- A1 <: B1 : k]]} \\ \inferrule*[Right=s-castup]
          {\rulehl{[[A1 --> A2]]} \\ [[G]] \vdash [[e]] : \rulehl{A_2}}
          {[[G |- castup [A1] e : A1]]}}
        {[[G |- castup [A1] e : B1]]}}
      {[[G]] \vdash [[castdn (castup [A1] e)]] : \rulehl{B_2}}
\end{mathpar}

Here the typing of the inner $\castdn$ does not directly follow
\rref{r-castup}, but the subsumption rule instead. We want to show that after
the cast elimination (following \rref{r-cast-elim}), expression $e$ has type $B_2$,
while in reality it has type $A_2$ (as highlighted).
Therefore want to show $\Gamma \vdash A_2 \le B_2$ with the information that
$\Gamma \vdash A_1 \le B_1$, $[[A1 --> A2]]$
and $[[B1 --> B2]]$, which depends on the property we want to prove initially,
subtype preservation after reduction.
And subtype preservation needs to solve the cast elimination case, causing a
circular dependency of properties.
This problem was also observed by the previous work on unified subtyping
with cast operators~\cite{full}. They solved this situation
by a delicate approach with the help of an essential lemma
\emph{Reduction Exists in the Middle} (If $[[G |- e1 <: e2 : A]]$, $[[G |- e2 <: e3 : A]]$
and $e_1 \longrightarrow e_1'$, $e_3 \longrightarrow e_3'$, there exists $e_2'$
such that $e_2 \longrightarrow e_2'$). Unfortunately this lemma does not hold
in our system since universal types, which are not reducible, can appear in the
middle of two reducible types, so we cannot adopt their proof on this case.

We tackle this problem from another direction, with the observation that the
demand for subtype preservation property shifts from the term-level to the type-level.
With the calculus-of-constructions-like kind hierarchy, our system only have
limited layers in types.
In fact, we only need to go up one layer in the type hierarchy to be able to
obtain subtype preservation directly,
since there is no subtyping involved at the kind level,
hence no problem for the cast elimination there.
Even better, we show that by going up one level in the type hierarchy
(only discussing the types of terms), the options
for the reduction that can be performed by a well-typed term are very limited.
Implicit abstractions do not occur in type computation due to the kind
restriction of universal types. We prove that well-typed reductions never occur
at kind level, so cast operators also do not occur in type-level computation.

\begin{figure}
    \drules[dr]{$[[A1 ==> A2]]$}{Deterministic Reduction}
      {app,beta,mu}
    \caption{Deterministic Reduction.}
    \label{fig:deterministic-reduction}
\end{figure}

Figure \ref{fig:deterministic-reduction} shows the effective reduction rules
inside cast operators.

\begin{lemma}[Deterministic Reduction]
    If $[[A ==> A1]]$ and $[[A ==> A2]]$,
    then $A_1 = A_2$.
\end{lemma}

\begin{lemma}[Deterministic Type Reduction]
    If $[[G |- A1 : k]]$ and $[[A1 --> A2]]$,
    then $[[A1 ==> A2]]$.
\end{lemma}

The cases for implicit abstractions are easy to prove. For the cast operators
we have the following lemma.

\begin{lemma}[Expressions of kind $[[box]]$ are never reduced]
    If $[[A --> B]]$ and $[[G |- e : A]]$,
    then $B$ does not have kind $[[box]]$.
\end{lemma}

The result of this lemma is interesting, because it seems that we can construct
a reducible expression like: $([[lambda x : int. *]])~42$ which has kind $[[box]]$.
But in reality, the lambda expression must be of type $[[int]] \rightarrow [[box]]$
for the application to be well-typed. But as the conventional typing rule for
lambda abstractions in Calculus of Constructions~\cite{coc}.
The function types of the lambda abstractions themselves
must be well-kinded (see \rref{s-abs}), since $[[box]]$ itself does not have a kind,
$[[int]] \rightarrow [[box]]$ is not well-kinded, therefore the whole application
is not well-typed. And for similar reason, the position of occurence of $[[*]]$
in a well-typed expression is very restricted, hence the lemma is possible to prove.

And then the subtype preservation of type computation can be easily shown.

\begin{lemma}[Subtype Preservation for Types]
    If $[[G |- A1 <: B1 : k]]$, $[[A1 ==> A2]]$ and $[[B1 ==> B2]]$,
    then $[[G |- A2 <: B2 : k]]$.
\end{lemma}

\paragraph{Implicit Instantiations}
The proof of two cases for implicit instantiations (\rref{r-inst,r-cast-inst})
are quite similar. In our language, implicit instantiations are only triggered by
\rref{s-forall-l}, which is exactly where polymorphic types are instantiated.
And the implicit argument is the same mono-expression ($\tau$) that
instantiates the polymorphic type in \rref{s-forall-l}.
The type of the instantiation result of $\Lambda$ expression is the same as the
instantiation of the polymorphic types, with the same argument.

With the observations above, the rest of the proofs are finished by standard
inversion lemmas, with the help of the \emph{substitution} theorem to handle type
preservation after the instantiations.

\subsection{Equivalence to a Simplified System}
We mentioned in Section \ref{sec:type-system} that the premises marked in gray in the
unified subtyping rules are redundant. They help in the formalization, but the calculus
is equivalent to a variant of the calculus without them. We define system
$[[G |= e1 <: e2 : A]]$, whose rules are the same as the typing rules of \name,
but with all redundant premises eliminated. Also, \rref{s-castdn,s-castup}
are simplified to use deterministic reduction ($A \longrightarrow_D B$)
instead of the reduction rule $A \longrightarrow B$ as shown below (other rules omitted):

\begin{drulepar}[ss]{$[[G |= e1 <: e2 : A]]$}{Simplified Unified Subtyping}
    \ottaltinferrule{ss-castdn}{width=20em}
      {[[G |= e1 <: e2 : A]] \\ \rulehl{[[A ==> B]]} \\ [[G |= B : k]]}
      {[[G |= castdn e1 <: castdn e2 : B]]}
    \and
    \ottaltinferrule{ss-castup}{width=20em}
      {[[G |= e1 <: e2 : B]] \\ \rulehl{[[A ==> B]]} \\ [[G |= A : k]]}
      {[[G |= castup [A] e1 <: castup [A] e2 : B]]}
\end{drulepar}

We prove that the two system are equivalent in terms of expressiveness.

\begin{theorem}[Equivalence of \name and the Simplification]
  If $[[G |- e1 <: e2 : A]]$ then $[[G |= e1 <: e2 : A]]$.
  And if $[[G |= e1 <: e2 : A]]$ then $[[G |- e1 <: e2 : A]]$.
\end{theorem}

\subsection{Subsumption of Polymorphic Subtyping}

Finally we show that the subtyping aspect of \name subsumes
Odersky and L\"aufer's polymorphic
subtyping~\cite{odersky1996putting}

by showing our system subsumes the Dunfields and Krishnaswami (DK)'s declarative
subtyping system \cite{dunfield2013complete}, which is a implementation of
the polymorphic subtyping.

\begin{figure}
    \begin{mathpar}
        \lift{x} = x \and \lift{[[int]]} = [[int]] \and
        \lift{A \rightarrow B} = \lift{A} \rightarrow \lift{B} \and
        \lift{\forall x. \, A} = \forall x : [[*]]. \, \lift{A} \and
    \end{mathpar}
    \begin{mathpar}
        \lift{[[nil]]} = [[nil]] \and
        \lift{\Gamma, \, x} = \lift{\Gamma},\, x : [[*]]
    \end{mathpar}
    \caption{Lifting Types and Contexts in Polymorphic Sutyping to \name}
    \label{fig:lift}
\end{figure}
Figure \ref{fig:lift} shows the transformation from
% Odersky and L\"aufer's
DK's
types to \name's types.
Then we prove the subsumption in terms of type well-formedness and subtyping
by following the interpretation of unified subtyping.

\begin{theorem}[Subsumption of Type Well-formedness]
    If $\Gamma \vdash_{DK} A$, then $\lift{\Gamma} \vdash \lift{A} : [[*]]$
\end{theorem}

\begin{theorem}[Subsumption of Polymorphic Subtyping]
    If $\Gamma \vdash_{DK} A \le B$, then $\lift{\Gamma} \vdash \lift{A} \le \lift{B} : [[*]]$
\end{theorem}

The proof is straightforward by making use of the \emph{well-formedness} lemma
in Odersky and L\"aufer system, mentioned in Section \ref{sec:adaptation}, to conclude that
$A$ itself is a well-formed type in $\Gamma \vdash A \le \forall x.\, B$, and
$\forall x. A$ is well-formed in $\Gamma \vdash \forall x. \, A \le B$.