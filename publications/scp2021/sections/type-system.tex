\section{The Dependent Implicitly Polymorphic Calculus}
\label{sec:system}

This section introduces the static and dynamic semantics of
\name: a dependently typed calculus with type casts
and implicit polymorphism. The calculus employs
\emph{unified subtyping}~\citep{full}
and has a single relation that generalizes both typing and subtyping.
The calculus can be seen as a variant of the \emph{Calculus of Constructions}~\citep{coc},
but it uses a simple form of casts~\citep{isotype,yang2019pure} with $\castup$ and $\castdn$ operators
instead of the conversion rule and features unrestricted recursion with the fixpoint operator.
We present the syntax, the unified subtyping relation, and operational semantics for \name.

\begin{figure}[t]
\centering
\begin{equation*}
\begin{array}{llcl}
    \text{Kinds} & k & ~\Coloneqq ~ & [[*]] \mid [[box]] \\
    \text{Expressions} & e, A, B & ~ \Coloneqq ~ & [[x]] \mid [[n]] \mid [[k]] \mid [[int]] \mid [[e1 e2]] \mid [[lambda x : A. e]] \mid [[pi x : A. B]] \\
        & & \mid & [[bind x : A. e]] \mid [[forall x : A. B]] \mid [[mu x : A. e]] \\
        & & \mid & [[castup [A] e]] \mid [[castdn e]]   \\
    \text{Mono-Expressions} ~ & \tau, \sigma & ~ \Coloneqq ~ & [[x]] \mid [[n]] \mid [[k]] \mid [[int]] \mid \tau_1 ~ \tau_2 \mid \lambda \, x : \tau. ~ \sigma \mid \Pi \, x : \tau. ~ \sigma \\
        & & \mid & \Lambda \, x : \tau. ~ \sigma \mid \mu \, x : \tau. ~ \sigma \mid \castup \, [\tau]~ \sigma \mid \castdn \, \tau \\
    \text{Values} & v & ~ \Coloneqq ~ & [[k]] \mid [[n]] \mid [[int]] \mid [[lambda x : A. e]] \mid [[pi x : A. B]] \mid [[bind x : A. e]] \\
        & & \mid & [[forall x : A. B]] \mid [[castup [A] e]] \\
    \text{Contexts} & \Gamma & ~ \Coloneqq ~ & [[nil]] \mid [[G , x : A]] \\
    \text{Syntactic Sugar} ~ & A \rightarrow B & \triangleq & [[pi x : A. B]] \qquad \text{where} ~ x \notin \mathrm{FV}(B)
\end{array}
\end{equation*}
\caption{Syntax of \name.}
\label{fig:syntax}
\end{figure}
\subsection{Syntax}

Figure \ref{fig:syntax} shows the syntax of \name. The syntax is similar
to the Calculus of Constructions, featuring unified terms and types,
and a kind hierarchy with $[[*]]$ and $[[box]]$.
The kind $[[*]]$ is the type (or kind) of other types like $[[int]]$ and $\Pi$ types,
the kind $[[box]]$ is the type of $[[*]]$, but $[[box]]$ itself has no type/kind.
Due to the unified syntax, types and
expressions ($e$, $A$ and $B$) are used
interchangeably, but we mostly adopt the convention of using $A$ and $B$
for contexts where the expressions are used as types and $e$ for contexts
where the expressions represent terms.
The syntax includes all the constructs of the calculus of constructions:
variables ($[[x]]$), kinds ($[[k]]$), function applications  ($[[e1 e2]]$),
lambda expressions ($[[lambda x : A. e]]$), dependent function types ($[[pi x : A. B]]$)
as well as integer types ($[[int]]$) and integers ($[[n]]$).
Moreover, there are several additional language constructs to
support implicit polymorphism, recursion, and explicit type-level computation
via casts. These constructs are discussed next.

\paragraph{Implicit Polymorphism}

In \name, universal types $[[forall x : A. B]]$ generalize implicit
polymorphic types ($\forall x.\, A$) in conventional functional languages.
The parameter $x$ in $[[forall x : A. B]]$ ranges over all well-typed expressions besides well-formed
types (i.e. $x$ can have any type $A$ instead of just kind $\star$). The idea of
\emph{monomorphic types} (or monotypes) is also generalized.
\emph{Mono-expressions} $\tau$ exclude polymorphic types ($[[forall x : A. B]]$) from the syntax,
which follows a similar design in various work about predicative
higher-ranked polymorphism~\citep{odersky1996putting,dunfield2013complete,jones2007practical}.
With dependent types, only generalized universal types are excluded, but not
any other related expressions.

Notably,
generalized ``polymorphic types'' are naturally dependent, and $\forall$
types can be viewed as the implicit counterpart of $\Pi$ types.
So we also have implicit lambda expressions ($[[bind x : A. e]]$),
which is different from the ``usual'' $\lambda$ expressions and explicit function
types ($\Pi$ types), for which arguments should be explicitly passed. The arguments
of implicit lambda ($\Lambda$) expressions are deduced during type checking.
This is a similar design to
the \emph{Implicit Calculus of Constructions} ($\text{ICC}^*$)~\citep{barras2008implicit}, which
employs similar constructs for implicit dependent products.

\paragraph{Recursion and Explicit Type-level Computation}
\label{sec:cast}
The \name calculus adopts \emph{iso-types}~\citep{isotype,yang2019pure},
featuring explicit type-level computation with cast operators
$\castdn$ and $\castup$. These operators respectively perform one-step
type \emph{reduction} and \emph{expansion} based on the operational semantics.
The reduction in cast operators is deterministic, thus type
annotations are only needed during type expansions ($\castup$). We add
fixpoints ($[[mu x : A. e]]$) to support general recursion for both
term-level and type-level. Iso-recursive types are supported by $\castup$ and
$\castdn$ operators, which correspond to the
\verb|fold| and \verb|unfold| operations when working on conventional iso-recursive types.

\subsection{Operational Semantics}\label{subsec:semantics}

For the operational semantics we employ two different, but closely related
reduction relations. The first reduction relation is non-deterministic, and
it is used at the type-level to allow type conversions induced
by the cast operators. The second reduction relation is deterministic and
is employed to give the run-time semantics of expressions.

\paragraph{Non-deterministic Implicit Instantiations}
Figure \ref{fig:semantics} presents the small-step operational semantics of \name.
It mostly follows the ``Call-By-Name'' (CBN) variant of \emph{Pure Iso-Type Systems} (PITS)
~\citep{yang2019pure} corresponding to the calculus of constructions. In such variant
the arguments of the $\beta$-reduction (\rref{r-beta}) and expressions in
the \rref{r-cast-elim} are not required to be values.
Reductions can be performed inside $\castdn$ terms (\rref{r-castdn,r-cast-inst}).
Note that here \rref{r-castdn} reflects the term-level reduction of $\castdn$ terms themselves,
the one-step type-level reductions triggered by the $\castdn$ operator are
reflected in the typing rules.
Following the CBN semantics of PITS, $\castup$ terms are considered to be values
to avoid nondeterministic reduction of terms like $\castdn (\castup [A] ~e)$,
where $e$ is reducible (either performing reduction on $e$, or the reduction via
\rref{r-cast-elim} could be the choice, should $\castup$ terms be reducible).
There is an alternative design following the ``call-by-value'' variant of PITS, which
we will discuss in Section~\ref{sec:cast-design}.
Also, the unfold operation of the fixpoint operator is supported by \rref{r-mu}.

\begin{figure}[t]
    \centering

    \begin{drulepar}[r]{$[[e1 --> e2]]$}{Operational Semantics}
      \drule{app}
      \drule{beta}
      \and \ottaltinferrule{r-inst}{}{ }
        {[[(bind x : A. e1) e2 --> ([t / x] e1) e2]]}
      \drule{mu}
      \drule{castdn}
      \and \ottaltinferrule{r-cast-inst}{}{ }
        {[[castdn (bind x : A. e) --> castdn ([t / x] e)]]}
      \drule{castXXelim}
    \end{drulepar}

    \caption{Operational semantics of \name.}
    \label{fig:semantics}
\end{figure}

Due to the presence of instantiation of implicit parameters, the direct operational
semantics is not deterministic, and potentially not type-preserving because of
\rref{r-inst,r-cast-inst}. The indeterminacy is caused by the guess of $\tau$,
which can be an arbitrary mono-expression, since we do not have access to any typing
information in the dynamic semantics.

\begin{figure}
  \centering
  \begin{equation*}
  \begin{array}{llcl}
      \text{Erased Expressions} & E, A, B & ~ \Coloneqq ~ & [[x]] \mid [[n]] \mid [[k]] \mid [[int]] \mid [[ee1 ee2]] \mid [[elambda x. ee]] \mid [[epi x : eA. eB]] \\
      & & \mid & [[ebind x. ee]] \mid [[eforall x : eA. eB]] \mid [[emu x. ee]] \mid [[ecastup ee]] \mid [[ecastdn ee]] \\
      \text{Erased Values} & ev & ~ \Coloneqq ~ & [[k]] \mid [[n]] \mid [[int]] \mid [[elambda x. ee]] \mid [[epi x : eA. eB]] \mid [[ebind x. ee]] \\
      & & \mid & [[eforall x : eA. eB]] \mid [[ecastup ee]]
  \end{array}
  \end{equation*}

  \begin{gather*}
    \begin{align*}
    \extract{[[x]]} &= [[x]] &
    \extract{[[n]]} &= [[n]] &
    \extract{[[k]]} &= [[k]] &
    \extract{[[int]]} &= [[int]]
    \end{align*} \\
    \begin{align*}
     \extract{[[e1 e2]]} &= \extract{[[e1]]} ~ \extract{[[e2]]} & % \\ % \and
      \extract{[[mu x : A. e]]} &= \mu \, x. ~ \extract{[[e]]} \\
     \extract{[[lambda x : A. e]]} &= \lambda \, x. ~ \extract{[[e]]} & % \\ % \and
      \extract{[[pi x : A. B]]} &= \Pi \, x : \extract{[[A]]}. ~ \extract{[[B]]} \\
     \extract{[[bind x : A. e]]} &= \Lambda \, x. ~ \extract{[[e]]} & % \\ % \and
      \extract{[[forall x : A. B]]} &= \forall \, x : \extract{[[A]]}. ~ \extract{[[B]]} \\
     \extract{[[castup [A] e]]} &= \castup \, \extract{[[e]]} & % \\ % \and
      \extract{[[castdn e]]} &= \castdn \, \extract{[[e]]}
    \end{align*}
  \end{gather*}

  \drules[er]{$[[ee1 *--> ee2]]$}{Erased Semantics}
    {app,beta,elim,mu,castdn,castXXinst,castXXelim}
  \caption{Erased Expressions and Operational Semantics}
  \label{fig:extraction}
\end{figure}

\paragraph{Deterministic Erased Reduction}
We address the issue of determinacy of the dynamic semantics with
a design similar to $\mathrm{ICC}^*$~\citep{barras2008implicit},
employing type-erased expressions. The erased expressions
essentially mirror the syntax and semantics
of normal expressions, except for the elimination of type annotations in $\lambda$,
$\Lambda$, $\mu$, and $\castup$ expressions.
Figure \ref{fig:extraction} shows the syntax of the erased expressions and
the companion operational semantics. Restrictions are imposed in the
typing rules to forbid the implicit parameter occurring in runtime-relevant parts
of the expression, i.e. the erased expressions (see Section \ref{sec:type-system}).
With such restriction, implicit parameters can be directly eliminated in
\rref{er-elim,er-cast-inst}.
For a well-typed expression, the reduction of
its erasure is deterministic.
Although the implicit parameter does not matter at runtime, the erasure
function preserves the structure of the original syntax
by not eliminating the implicit binder altogether. This design
has the advantage that the correspondence
between the original and the erased expression can be established more easily,
when the reductions of erased expressions correspond directly to their erased counterpart.
The proof of type safety of our system is built around this idea of correspondence,
which is discussed in Section \ref{sec:type-safety}.

\subsection{Unified Subtyping}
\label{sec:type-system}

Figure \ref{fig:typing} shows the (sub)typing rules of the system. We adopt a
simplified design based on unified subtyping~\citep{full}. The subtyping rules and
typing rules are merged into a single judgment $[[G |- e1 <: e2 : A]]$.

Unified subtyping solves the challenging issue of mutual dependency between typing
and subtyping in a dependent type system.
The interpretation of this judgment is ``under context $[[G]]$, $[[e1]]$ is a
subtype of $[[e2]]$ and they are both of type $[[A]]$''.
In this form of formalization, the typing judgment $[[G |- e : A]]$ is a
special case of the unified subtyping judgment $[[G |- e <: e : A]]$,
and the well-formedness of types $\Gamma \vdash A$ is expressed by
$[[G |- A : *]]$.
% where $k \in \{[[*]], [[box]]\}$.
% Although unified, the typing rules can still be viewed as two parts, the ``typing'' part
% (\rref{s-abs,s-app,s-bind,s-mu,s-castup,s-castdn,s-sub}) and the ``subtyping'' part
% (\rref{s-pi,s-forall,s-forall-l,s-forall-r}). We follow a usual design for
% typing rules for lambda abstraction and application, and the subtyping rule of
% dependent function types ($\Pi$ type).
% \bruno{I don't agree with this last paragraph. If you have a language with recursive types,
%   type applications and type functions then you need subtyping rules similar to s-abs, s-app or s-mu,
% even when you have a design not based on unified subtyping. I suggest omitting this paragraph.}

\begin{figure}
    \centering
\setlength{\abovedisplayskip}{8pt}
    \begin{drulepar}[wf]{$[[|- G]]$}{Well-formed Context}
      \mprset{sep=1.2em}
      \drule{nil}
      \drule[width=30em]{cons}
    \end{drulepar}
\setlength{\belowdisplayskip}{2.5pt}
\setlength{\belowdisplayshortskip}{1.3pt}

    \begin{drulepar}[s]{$[[G |- e1 <: e2 : A]]$}{Unified Subtyping}
      \mprset{sep=1.3em}
      \drule{var}
      \drule{lit}
      \drule{int}
      \drule{star}
      % \drule{abs}
      \and \ottaltinferrule{abs}{width=20em}
        {\redundant{[[G |- A : k1]]} \\ [[G, x : A |- B : k2]] \\ [[G, x : A |- e1 <: e2 : B]]}
        {[[G |- lambda x : A. e1 <: lambda x : A. e2 : pi x : A. B]]}
      % \drule{app}
      \and \ottaltinferrule{s-app}{}
        {[[G |- t : A]] \\ [[G |- e1 <: e2 : pi x : A. B]]}
        {[[G |- e1 t <: e2 t : [t / x] B]]}
      \drule{pi}
      % \drule{bind}
      \and \ottaltinferrule{s-mu}{width=20em}
        {[[G |- t : k]] \\ [[G , x : t |- s : t]]}
        {[[G |- mu x : t. s <: mu x : t. s : t]]}
      \and \ottaltinferrule{s-bind}{width=20em}
        {\redundant{[[G |- A : k]]} \\ [[G , x : A |- B : *]] \\
         [[G, x : A |- e1 <: e2 : B]] \\
         x \notin \mathrm{FV}(\extract{[[e1]]}) \cup \mathrm{FV}(\extract{[[e2]]})}
        {[[G |- bind x : A. e1 <: bind x : A. e2 : forall x : A. B]]}
      % \drule{mu}
      \drule{castup}
      \drule{castdn}
      % \drule{forallXXl}
      \and \ottaltinferrule{s-forall-l}{width=20em}
        {\redundant{[[G |- A : k]]} \\ [[G |- t : A]] \\
         \rulehl{[[G , x : A |- B : *]]} \\
         [[G |- [t / x] B <: C : *]]}
        {[[G |- forall x : A. B <: C : *]]}
      % \drule{forallXXr}
      \and \ottaltinferrule{s-forall-r}{width=20em}
        {\redundant{[[G |- B : k]]} \\ \rulehl{[[G |- A : *]]} \\
         [[G , x : B |- A <: C : *]]}
        {[[G |- A <: forall x : B. C : *]]}
      % \drule{forall}
      \and \rulehl{\ottaltinferrule{s-forall}{width=20em}
        {\redundant{[[G |- A : k]]} \\ [[G , x : A |- B <: C : *]]}
        {[[G |- forall x : A. B <: forall x : A. C : *]]}}
      \and \ottaltinferrule{s-sub}{width=15em}
        {[[G |- e1 <: e2 : A]] \\ [[G |- A <: B : k]]}
        {[[G |- e1 <: e2 : B]]}
    \end{drulepar}
    % \drules[s]{$[[G |- e1 <: e2 : A]]$}{Unified Subtyping}{
    %     var,lit,int,star,abs,pi,app,bind,mu,castup,castdn,forallXXl,forallXXr,forall,sub}
    \setlength{\abovedisplayskip}{0pt}
    \setlength{\abovedisplayshortskip}{0pt}
    \begin{gather*}
       \text{Syntactic Sugars} \\
       [[G |- e : A]] \triangleq [[G |- e <: e : A]]
      \qquad \Gamma \vdash A \triangleq [[G |- A <: A : *]]
      \qquad \Gamma \vdash A \le B \triangleq [[G |- A <: B : *]]
    \end{gather*}
    \caption{(Sub)Typing Rules of \name.}
    \label{fig:typing}
\end{figure}
% \bruno{Add syntactic sugar also for subtyping and well-formedness.}

\paragraph{Subtyping Rules for Universal Quantification}
The subtyping rules for universal quantification (\rref{s-forall-l,s-forall-r}) follow
the spirit of the Odersky and L\"aufer's polymorphic subtyping~\citep{odersky1996putting,dunfield2013complete},
where the subtyping relation is interpreted as a ``more-general-than'' relation.
A polymorphic type $[[forall x : A. B]]$
is more general than another type $C$ when its well-typed
instantiation is more general than $C$ (\rref{s-forall-l}). A polymorphic
type $[[forall x : B. C]]$ is less general than a type $A$,
if $C$ is is less general than $A$ when the argument with the polytype ($x:B$)
is instantiated abstractly (\rref{s-forall-r}).

Notably, our formalization is not the direct generalization of
Odersky and L\"aufer's polymorphic subtyping, as we mentioned in Section \ref{sec:feature-overview}.
As highlighted in Figure~\ref{fig:typing}:
we add \rref{s-forall} that axiomatizes the subtyping relation between two universal types,
and additional premises are added to \rref{s-forall-l,s-forall-r} besides the
ones that appear in rules $\forallL''$ and $\forallR''$ in Section \ref{sec:feature-overview}.
We discuss the motivations for these changes in more detail in Section \ref{sec:adaptation}.

\paragraph{Mono-expression Restrictions}
As in other predicative relations (such as the one by Odersky and L\"aufer),
the type arguments for instantiation in \rref{s-forall-l} are
required to be mono-expressions, which has cascading effects on typing rules of
other expressions. The arguments for applications are required to be
mono-expressions, and the whole fixpoint expression is required to be a
mono-expression. We shall
discuss the reason behind these restrictions in Section \ref{sec:metatheory}.

\paragraph{Kind Restriction for Universal Types}
\label{sec:kind-restriction}

For the kinding of types, we mainly follow the design of the Calculus of
Constructions~\citep{coc}. However, we specifically restrict
the polymorphic type $[[forall x : A. B]]$ to only have kind $[[*]]$, but not $[[box]]$.
Without this restriction, types of kind $[[*]]$ (such as $[[int]]$) are able to
have ``polymorphic kinds'' like $[[forall x : int. *]]$ through \rref{s-forall-l,s-forall-r,s-sub},
which significantly complicates the kinding reasoning in the metatheory.
Practically speaking, ``polymorphic kinds'' are not very common, so this
restriction has little impact on the expressiveness of our language.

Since we expect the types of well-typed expressions to be well-kinded,
the restriction propagates to the introduction rule of $\forall$ types (\rref{s-bind}).
In this rule, $[[B]]$ is required to only have kind $[[*]]$ but not $[[box]]$.
In contrast, for \rref{s-abs}, the type $[[B]]$ at a similar position can have any kind.
As a result, implicit polymorphic functions at type level are \emph{never well-typed}.
For example, $\Lambda a : [[*]].\, \lambda b : n.\, a$ is not
well-typed since it would have been a polymorphic function of type
$\forall a : [[*]].\, a \rightarrow [[*]]$, which is not well-kinded due to the
kinding restriction of universal types.
Moreover, type-level function applications that involve implicit
abstractions ($\Lambda$ expressions) are also never well-typed
because of the restriction. So well-typed non-deterministic implicit instantiation can never occur
at the type-level.

\paragraph{Computational Irrelevance of Implicit Parameters}

As mentioned in Section \ref{sec:computational-irrelevance-overview}, our language
does not handle computationally relevant implicit parameters.
The direct operational semantics shown in Figure \ref{fig:semantics} chooses
arbitrary mono-expressions to instantiate the implicit arguments,
which potentially breaks type safety.
Thus, we adopt a restriction in \rref{s-bind} that is similar to the
Implicit Calculus of Constructions (ICC)~\citep{miquel2001implicit}.
We only allow the implicit parameters to occur in type annotations in the
body of implicit abstraction, so that the choices of implicit parameters are not
relevant at runtime. The type safety of the direct operational semantics is
proved indirectly in Section \ref{sec:type-safety} with the help of the
erasure of expressions.

\paragraph{Redundant Premises}

All the premises boxed by dash lines are redundant in a way that
the system without them is proved equivalent to the system with them.
These redundant premises are there to simplify the mechanized proofs of
certain lemmas, but can be safely dropped in an actual implementation.
