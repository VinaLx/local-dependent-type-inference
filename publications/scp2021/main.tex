\documentclass[authoryear]{elsarticle}

\usepackage[T1]{fontenc}
\usepackage{ottalt}
\usepackage{xcolor}
\usepackage{mathtools}
\usepackage{amsmath}
\usepackage{amsfonts}
\usepackage{amssymb}
\usepackage{setspace}
\usepackage{xspace}
\usepackage{mathpartir}
\usepackage{tikz-cd}
\usepackage{comment}
\usepackage{listings}
\usepackage{MnSymbol}
\usepackage{url}
\usepackage{dashbox}
\usepackage{floatrow}
\usepackage{pifont}

\floatsetup[table]{footnoterule=none}

%\usepackage[english]{babel}

\inputott{dependent-subtyping.ott}

\newcommand{\rulehl}[2][gray!40]{
  \colorbox{#1}{$\displaystyle#2$}}
\newcommand{\redundant}[1]{\dbox{$\displaystyle#1$}}

\newcommand{\extract}[1]{\lvert #1 \rvert}
\newcommand{\lift}[1]{\lceil #1 \rceil}
\newcommand{\castup}[0]{\mathrm{cast}_\Uparrow}
\newcommand{\castdn}[0]{\mathrm{cast}_\Downarrow}
\newcommand{\dkvdash}{\vdash_\text{\tiny DK}}
\newcommand{\system}[0]{the dependent implicitly polymorphic calculus\xspace}
\newcommand{\System}[0]{The dependent implicitly polymorphic calculus\xspace}

\newcommand{\fsub}[0]{$F_{<:}$}
\newcommand{\fsubo}[0]{$???$}
\newcommand{\lami}[0]{$\lambda_{I}$\xspace}
\newcommand{\fw}[0]{$F_{\omega}$\xspace}

\newcommand{\name}[0]{$\lambda_{I}^{\forall}$\xspace}

\newcommand{\forallL}[0]{\mathbin{{\le}{\forall}} L}
\newcommand{\forallR}[0]{\mathbin{{\le}{\forall}}R}
\newcommand{\an}[3]{{\color{#2} \textsc{#1}:#3}}
\newcommand{\bruno}[1]{\an{bruno}{blue}{#1}}
\newcommand{\alvin}[1]{\an{alvin}{red}{#1}}

\newtheorem{theorem}{Theorem}
\newtheorem{lemma}[theorem]{Lemma}
\newtheorem{corollary}[theorem]{Corollary}

\title{A Dependently Typed Calculus with Polymorphic Subtyping}

\author{Mingqi Xue}
  \ead{mqxue@cs.hku.hk}

\author{Bruno C. d. S. Oliveira}
  \ead{bruno@cs.hku.hk}

\address{The University of Hong Kong}

\begin{document}

\begin{abstract}
  A \emph{polymorphic subtyping} relation, which relates more general types
  to more specific ones, is at the core of many modern functional languages.
  As those languages start moving towards dependently typed programming a
  natural question is how can polymorphic subtyping be adapted to such settings.

  This paper presents \system (\name): a simple dependently typed calculus
  with polymorphic subtyping. The subtyping relation in \name
  generalizes the well-known polymorphic subtyping relation by
  Odersky and L\"aufer (1996). Because \name is dependently typed,
  integrating subtyping in the calculus is non-trivial. To overcome
  many of the issues arising from integrating subtyping with dependent
  types, the calculus employs \emph{unified subtyping}, which is a technique
  that unifies typing and subtyping into a single relation. Moreover, \name
  employs explicit casts instead of a conversion rule, allowing
  unrestricted recursion to be naturally supported.
  We prove various non-trivial results, including \emph{type soundness} and \emph{transitivity}
  of unified subtyping. \name and all corresponding proofs
  are mechanized in the Coq theorem prover.
  %and we also have a simple prototype
  %implementation of \name.
%A complication that arizes in the type-soundness proof
%is that a direct semantics is non-deterministic due
\end{abstract}


\begin{keyword}
  Type Systems \sep Dependent Types \sep Subtyping \sep Polymorphism
\end{keyword}
% \keywords{Type System, Dependent Type, Subtyping, Polymorphism}

\maketitle

\section{Introduction}

A \emph{polymorphic subtyping} relation, which relates more general
types to more specific ones, is at the core of many modern functional
languages. Polymorphic subtyping enables a form of
\emph{(implicit) parametric polymorphism}, where type arguments to polymorphic
functions are automatically instantiated and the programmer does not specify them.
Traditionally, variants of polymorphic subtyping (in the form of a more-general-than relation)
have been used in functional languages based on the
Hindley-Milner~\cite{hindley1969principal,milner1978theory,damas1982principal}
type system, which supports full type-inference without any type annotations.
However, the Hindley-Milner type system only supports \emph{rank-1 (or first order)
polymorphism}, where all universal quantifiers only occur at the top-level
of a type.  Modern functional programming languages, such as Haskell, go beyond
Hindley-Milner and support \emph{higher-ranked polymorphism}~\cite{odersky1996putting,jones2007practical}
with a more expressive
polymorphic subtyping relation. With higher-ranked
polymorphism there is no restriction on where universal quantifiers can occur.

Odersky and L\"aufer~\cite{odersky1996putting} proposed a
simple declarative specification for polymorphic subtyping, which supports higher-ranked polymorphism.
Since then several
algorithms have been proposed that implement variants of this specification. Most
notably, the algorithm proposed by Peyton Jones et al.~\cite{jones2007practical} forms the basis
for the implementation of type inference in the GHC compiler.
Dunfield and Krishnaswami (DK)~\cite{dunfield2013complete} provide an elegant
formalization of another sound and complete algorithm, which has
also inspired implementations of type-inference in some polymorphic
programming languages (such as PureScript~\cite{PureScript} or DDC~\cite{Disciple}).
More recently Zhao et al.~\cite{zhao19mechanical} have mechanized DK's type system in a theorem prover.

%some more background text

In recent years dependent
types~\cite{coc,cayenne,dep:pisigma,sjoberg:msfp12,guru,fc:kind,zombie:popl14,zombie:popl15}
have become a hot topic in programming
language research. Several newer
functional programming languages, most notably Agda~\cite{2007_norell_agda} and
Idris~\cite{brady2013idris}, are now dependently typed. Moreover a number of existing functional
languages, such as Haskell, have started to move towards dependently typed programming~\cite{dependenthaskell}. Dependent types naturally lead to a unification between types and terms, which enables both
additional \emph{expressiveness} and \emph{economy of concepts}.
The key enabler for unifying terms and types in dependently typed
calculi is the adoption of a style similar to
Pure Type Systems (PTSs)~\cite{pts}. In PTSs there is only a single level
of syntax for terms, i.e. the types (or kinds) are expressed using the
same syntax as the terms. This is in contrast with more traditional calculi, where
distinct pieces of syntax (terms, types and kinds) are separated.

Unified syntax, typical of dependently typed languages,
poses some challenges for language design and implementation.
A first challenge arises from the interaction between recursion and dependent types.
Essentially recursion breaks strong normalization, which
many common properties in dependently typed calculi depend upon. One of the most
typical properties among them is the decidability of type checking,
which simply cannot be guaranteed if the termination of
type-level computations involving recursions cannot be guaranteed.
However, this area has been actively investigated in the
last few years, and a general approach~\cite{guru,sjoberg:msfp12, kimmel:plpv, zombie:popl15,
  isotype} based on explicit casts for type-level computations,
has emerged as an interesting solution for integrating general recursion
in dependently typed calculi. By avoiding the implicit type-level computation
entirely, whether programs strongly normalize or not no longer matters for the
decidability of type checking.
Current proposals for dependently typed versions of Haskell~\cite{dependenthaskell},
for instance, adopt explicit casts for type-level computation.

A second challenge, for calculi that employ subtyping, is that
smoothly integrating
dependent types and subtyping is difficult. Subtyping is a
substantial difference to traditional PTSs, which do not have such feature.
The issue with subtyping
is well summarized by Aspinall and Compagnoni~\cite{subdep}:
\emph{``One thing that makes the study of these systems difficult is that
  with
dependent types, the typing and subtyping relations become intimately
tangled, which means that tested techniques of examining subtyping in
isolation no longer apply''}.
Recent work on \emph{unified subtyping}~\cite{full} provides a
simple technique to address this problem.
Following the same spirit as Pure Type Systems,
which attempt to unify syntax and the typing and well-formedness relations,
unified subtyping suggests unifying typing
and subtyping into a single relation. This solves the problem of dependencies
in that now there is only a single relation that depends only on itself. Furthermore
it results in a compact specification compared to a variant with multiple
independent relations.

In this paper we investigate how polymorphic subtyping can be
adapted into a dependently typed calculus with general recursion and
explicit casts for type-level computation. We employ unified subtyping
to address the issues of combining dependent types with subtyping.
The use of explicit casts
for type-level computation means that type equality is essentially
syntactic (or rather up-to $\alpha$-equivalence).
This avoids the use of a traditional conversion rule that concludes
$\beta$-equivalent types to be equal,
which essentially allows implicit type-level computation.
Dependent type systems with a conversion rule has some major complications,
% that arise in the presence of a conversion rule
a well-known one is that type-inference for such systems requires \emph{higher-order
  unification}, which is known to be \emph{undecidable}~\cite{goldfarb1981undecidability}.
By employing a system with $\alpha$-equivalence only we stay closer to existing
languages like Haskell, where type equality (at least at the core language level)
is also essentially only up-to $\alpha$-equivalence.

We present a calculus called \name, and show three main results in this paper:
\emph{transitivity of subtyping}, \emph{type soundness}, and \emph{completeness
of \name's polymorphic subtyping with respect to Odersky and L\"aufer's formulation}.
Transitivity is a non-trivial result (like in most calculi combining dependent types
and subtyping) and requires a proof based on sizes and a property that guarantees
the uniqueness of kinds in our language. Type soundness is also non-trivial and we need
to take a different approach than that employed by existing work on polymorphic
subtyping~\cite{odersky1996putting, jones2007practical}, where type-safety is shown by an
elaboration to System F. In essence elaboration into a target language
brings significant complications to the metatheory in a dependently typed setting.
Thus, instead of an elaboration, we use a direct operational semantics approach, which
is partly inspired by the approach used in the \emph{Implicit Calculus of Constructions} (ICC)~\cite{miquel2001implicit,barras2008implicit},
to prove type soundness.
Similarly to ICC we adopt the restriction that arguments for implicit function types
are computationally irrelevant (i.e. they cannot be used in runtime computation).
However, our unified subtyping setting is significantly
different from ICC due to the presence of subtyping,
which brings complications not in the ICC.
We also prove that any valid subtyping statement in the Odersky and L\"aufer relation
is valid in \name. Thus \name's unified subtyping subsumes the polymorphic subtyping
relation by Odersky and L\"aufer.

\name and all the proofs reported in this paper are formalized in the Coq theorem
prover~\cite{coq}.
%Moreover we have a simple prototype implementation of \name by adapting
%the algorithms from Zhao et al.~\cite{zhao19mechanical} for polymorphic subtyping.
%\bruno{Can the implementation run the examples we show?}
This paper
does not address decidability or soundness and completeness of \name to an
algorithmic formulation, which are outside of the scope of this work.
Nonetheless, these are important and challenging
questions for practical implementations of \name, which are left open for future work.

In summary, the contributions of this paper are:

\begin{itemize}

\item {\bf The \name calculus}, which is a dependently typed calculus with explicit casts,
  general recursion and implicit higher-ranked polymorphism.

\item {\bf Type-soundness and transitivity of subtyping.} We show that \name
  is type-sound and unified subtyping is transitive.

\item {\bf Subsumption of Odersky and L\"aufer's polymorphic subtyping.} We show that \name's
  unified subtyping can encode all valid poymorphic subtyping statements of Odersky and L\"aufer's
  relation.

\item {\bf Mechanical formalization.} All the results have been mechanically
  formalized in the Coq theorem prover. The formalization is available online at:
  \url{https://github.com/VinaLx/dependent-polymorphic-subtyping}.

\end{itemize}

\section{Overview}

In this section, we introduce \name by going through
some interesting examples to show the expressiveness and major features of the calculus.
Then we discuss the motivation, rationale of our design, and challenges.
The formal system of \name will be
discussed in Sections \ref{sec:system} and \ref{sec:metatheory}.

\subsection{A Tour of \name}
\label{sec:examples}

The \name calculus features a form of \emph{implicit
  higher-ranked polymorphism} with the power of dependent types. Thus the main feature of \name
is the ability to implicitly infer universally quantified arguments.

\paragraph{A First Example of Implicit Polymorphism}
Like most of functional languages, \name supports (implicit) parametric polymorphism.
A first simple example is the polymorphic identity
function:
\begin{flalign*}
&\mathrm{id} : \forall (A : \star).\, A \rightarrow A &&\\
&\mathrm{id} = \lambda (x : A).\, x &&\\
&\mathrm{answer} : \mathbb{N} &&\\
&\mathrm{answer} = \mathrm{id} ~ 42 \qquad \text{-- No type argument needed at the call of $\mathrm{id}$} &&
\end{flalign*}
\noindent The polymorphic parameter \verb|A| is annotated with its type,
which is $\star$. The type $\star$ is the type of types (also known as
\emph{kind}). In \name, the parameters of lambda abstractions are annotated
with their types, and the \verb|A| in the definition refers back to the
polymorphic parameter. In the examples below, we drop the parentheses around
variables and their type annotations such as $[[lambda x : A. x]]$ for conciseness.

Similarly to implicit polymorphism in other languages,
the polymorphic parameters of the $\forall$ types are implicitly instantiated
during applications. Thus, in the call of the identity function ($id~42$), we
do not need to specify the argument used for instantiation. In contrast,
in an explicitly polymorphic language (such as System F) we would need
to call $id$ with an extra argument that specifies the instantiation of $A$:
$\mathrm{id}~\mathbb{N}~ 42$.

\paragraph{Recursion and Dependent Types}

\name is dependently typed, and universal quantifications are not limited to work
on arguments of type $\star$, but it allows arguments of other types. This is
a key difference compared to much of the work on type-inference for higher-ranked
polymorphism~\cite{dunfield2013complete,le2003ml,leijen2008hmf,vytiniotis2008fph,jones2007practical}
which has been focusing on System F-like
languages where universal quantification can only have arguments of type $\star$.
Furthermore, \name supports general recursion at both the term and the type-level.

Using these features we can encode an indexed list, a \verb|map| operation over it
and illustrate how the implicit instantiation allows us to use the \verb|map|
function conveniently.
However, because \name is just a core calculus there is no built-in support
yet for algebraic datatypes and pattern matching.
We expect that a source language would provide a more convenient
way to define the \verb|map| function using pattern matching and other useful source-level
constructs. To model algebraic datatypes and pattern matching in \name, we
use an encoding by Yang and Oliveira~\cite{yang2019pure},
which is based on Scott encodings~\cite{mogensen1992efficient}.
The Scott Encoding encodes datatypes with different cases to
Continuation-Passing-Style (CPS) function types. The return branches of these
function types correspond to each case of the datatypes.
Case analyses of terms are encoded to the application of the CPS functions.
Since the details of that encoding are not relevant for this paper,
here we omit the code for most definitions and show only their types.

In a dependently typed language a programmer could write the following definition
for our formulation of indexed lists:
\newcommand{\Nat}[0]{\mathbb{N}}
\newcommand{\Succ}[0]{\mathrm{S}}
\newcommand{\Zero}[0]{\mathrm{Z}}
\newcommand{\List}[0]{\mathrm{List}}
\newcommand{\Nil}[0]{\mathrm{Nil}}
\newcommand{\Cons}[0]{\mathrm{Cons}}
\newcommand{\map}[0]{\mathrm{map}}
\begin{flalign*}
  & \mathbf{data} ~ \Nat = \Zero ~|~ \Succ~\Nat &&\\
  & \mathbf{data} ~ \List~(a : \star)~(n : \Nat) = \Nil ~ | ~ \Cons~a~(\List~a~(\Succ~n)) &&
\end{flalign*}
\noindent In this definition, the index grows towards the tail of the list,
which is admittedly not the most useful definition.
But the reason we did not choose the more practical
example, where the index represents the length of the list, is that it requires
language facilities like the GADT\cite{gadt1,gadt2} which \name does not support.
Here we encode $\mathrm{List}$ and its constructors as conventional terms. We
show the definition for \verb|List|, and the types for the constructors next
(implementation omitted):

\begin{flalign*}
&\List : \star \rightarrow \Nat \rightarrow \star &&\\
&\List = \mu L : \star \rightarrow \Nat \rightarrow \star.\, \lambda a:\star.\, \lambda n :\Nat.\, \Pi r:\star.\, r \rightarrow (a \rightarrow L~a~(\Succ ~ n) \rightarrow r) \rightarrow r &&\\
&\Nil : \forall a : \star.\, \forall n : \Nat.\, \List ~ a ~ n &&\\
&\Cons : \forall a : \star.\, \forall n : \Nat.\, a \rightarrow \List ~ a ~ (\Succ ~ n) \rightarrow \List ~ a ~n &&
\end{flalign*}
\noindent
In later subsequent examples we will just assume some
Haskell-style syntactic sugar for datatype definitions and constructors.
Using the above definitions, we can define a \verb|map| function over \verb|List| with the type:
\begin{flalign*}
  & \map : \forall a : \star.\, \forall b : \star.\, \forall n : \Nat .\, (a \rightarrow b) \rightarrow \List~a~n \rightarrow \List~b~n &&
\end{flalign*}
An example of application of \verb|map| is:
\begin{flalign*}
&\map~\Succ~(\Cons~1~(\Cons~2~\Nil)) &&
\end{flalign*}

\noindent which increases every natural in the list by one.
Note that since the type parameters for \verb|map|, \verb|Cons| and \verb|Nil|
are all implicit, they can be all omitted
and the arguments are instantiated implicitly. Thus the \verb|map| function
only requires two explicit arguments, making it as convenient to use
as in most functional language implementations.

There are a few final points worth mentioning about the example.
Firstly, \verb|List| is an example of a dependently typed function, since it is parameterized
by a natural value. Secondly, in \name (following the design of PITS~\cite{yang2019pure}),
fixpoint operators ($\mu$) serve a dual purpose of defining recursive types and recursive
functions. Besides its usual use of defining term-level general recursive functions,
it can be used to define recursive types, as shown in the encoding of \verb|List|
above.
% In \verb|List| the fixpoint operator is used to define a recursive type, whereas
% in the definition of \verb|map| the fixpoint operator is used to define term-level recursion.
Moreover, recursion is unrestricted and there is no termination checking, much like approaches
such as Dependently Typed Haskell~\cite{dh}, and unlike various other dependently typed languages
such as Agda~\cite{2007_norell_agda} or Idris~\cite{brady2013idris}.


\paragraph{Implicit Higher-Kinded Types}

The implicit capabilities also extend to the realm of higher-kinded types~\cite{tapl}.
For example we can define a record type \verb|Functor|,
to mimic the typeclass\cite{typeclasseswadler,typeclasseskaes} \verb|Functor| in Haskell:
\newcommand{\Functor}[0]{\mathrm{Functor}}
\newcommand{\MkFunctor}[0]{\mathrm{MkF}}
\newcommand{\Id}[0]{\mathrm{Id}}
\newcommand{\MkId}[0]{\mathrm{MkId}}
\newcommand{\fmap}[0]{fmap}
\begin{flalign*}
  &\mathbf{data}~\Functor~(F : \star \rightarrow \star) = \MkFunctor~\{\fmap : \forall a : \star.\, \forall b : \star.\, (a \rightarrow b) \rightarrow F~a \rightarrow F~b\} &&
\end{flalign*}
\noindent Since its a record type, $\MkFunctor$ is the data constructor,
and $\fmap$ is the field accessor.
The type of $fmap$ is:
\begin{flalign*}
  & \fmap : \forall F : \star \to \star.~\Functor~F \to \forall a : \star.\, \forall b : \star.\, (a \rightarrow b) \rightarrow F~a \rightarrow F~b &&
\end{flalign*}
Importantly this example illustrates that universal variables can quantify over higher-kinds (i.e.
$F : \star \to \star$).
We can define instances of functor in a standard way:
\begin{flalign*}
  & \mathbf{data}~\mathrm{Id}~a=\MkId~\{runId : a\} && \\
  & \mathrm{idFunctor} : \Functor~\Id && \\
  & \mathrm{idFunctor} = \MkFunctor~\{\fmap = \lambda f : a \rightarrow b.\, \lambda ~x : \Id~a.\, \MkId~(f~(runId~x))\} &&
\end{flalign*}
and then use \verb|fmap| with three arguments:
\begin{flalign*}
& \fmap~\mathrm{idFunctor}~\Succ~(\MkId~0) &&
\end{flalign*}
\noindent Note that, because our calculus has no mechanism like typeclasses we pass the ``instance'' explicitly.
Nonetheless, three other arguments (the $F$, $a$, and $b$) are implicitly instantiated.

\paragraph{Higher-Ranked Polymorphic Subtyping}
\label{sec:higher-ranked-poly}

In calculi such as the ICC~\cite{miquel2001implicit}, a form of implicit instantiation also exists.
However, such calculi do not employ subtyping, instead, they only apply instantiation
to top-level universal quantifiers. Our next example illustrates how subtyping enables
instantiation to be applied also in nested universal quantifiers, thus enabling
more types to be related.

When programming with continuations~\cite{sussman1998scheme} one of the
functions that are typically needed is call-with-current-continutation
(\verb|callcc|). In a polymorphic language, there are several types that can be
assigned to \verb|callcc|. One of these types is a rank-3 type,
while another one is a rank-1 type.
Using polymorphic subtyping we can show that the rank-3
type is more general than the rank-1 type. Thus the following program type-checks:
\begin{flalign*}
& \mathrm{callcc}' : \forall a : \star.\, ((\forall b : \star.\, a \rightarrow b) \rightarrow a) \rightarrow a && \\
& \mathrm{callcc} : \forall a : \star.\, \forall b : \star.\, ((a \rightarrow b) \rightarrow a) \rightarrow a && \\
& \mathrm{callcc} = \mathrm{callcc}' &&
\end{flalign*}
\noindent The type $\forall b : \star.\, a \rightarrow b$ appears in a positive position
of the whole signature, and it is a more general signature than $a \rightarrow b$
for an arbitrary choice of $b$. Our language captures this subtyping relation so that
we can assign $\mathrm{callcc}'$ to $\mathrm{callcc}$ (but not the other way around).
In contrast, in approaches like the ICC, the types of \verb|callcc| and \verb|callcc'|
are not compatible and the example above would be rejected.

\subsection{Key Features}

We briefly discuss the major features of \name itself and
its formalization. A more formal and technical discussion will be left to
Sections \ref{sec:system} and \ref{sec:metatheory}.

\paragraph{Polymorphic Subtyping Relation}
Figure \ref{fig:polymorphic-subtyping} shows the syntax of types, monomorphic types (or monotypes),
and the polymorphic subtyping relation in
Odersky and L\"aufer's declarative type system \cite{odersky1996putting}.
Here the syntax includes polymorphic types (or polytypes), which are universally quantified over type parameters
($\forall$ types). The definition of monotypes
is standard, and includes all types without occurences of universal quantifiers.
Context $\Gamma$ is a list of variables that are allowed to occur free in types
$A$ and $B$ in the subtyping relation.
The polymorphic subtyping relation captures a \emph{more-general-than} relation
between types. The key rules in their subtyping relation are rules $\forallL$
and $\forallR$:

\begin{itemize}
  \item In rule $\forallL$, a polytype ($\forall x.\, A$) is considered \emph{more-general}
        than another type ($B$), when we can find an arbitrary monotype ($\tau$)
        so that the instantiation is more general than $B$.
        Importantly note that this relation does not guess arbitrary (poly)types,
        but just monotypes. In other words, the relation is \emph{predicative}~\cite{Martin-Lof-1972}.
        This restriction ensures that the relation is \emph{decidable}.

  \item In rule $\forallR$ a type ($A$) is considered more general than a polytype ($\forall x. B$)
        when it is still more general than the head of the polytype, with the type
        parameter instantiated by an abstract variable $x$.
\end{itemize}

This subtyping relation sets a scene for our work, which
generalizes this relation to a dependently typed setting.

\begin{figure}
\centering
\begin{equation*}
\begin{array}{llcl}
  \text{Types} & A, B & ~\Coloneqq~ & [[x]] \mid [[int]] \mid A \rightarrow B \mid \forall x.\, A \\
  \text{Monotypes} & \tau, \sigma & ~\Coloneqq~ & [[x]] \mid [[int]] \mid \tau \rightarrow \sigma
\end{array}
\end{equation*}
\begin{drulepar}{$\Gamma \vdash A \le B$}{Polymorphic Subtyping}
  \inferrule*[lab=$\tau$]
    { }
    {\Gamma \vdash \tau \le \tau}
  \and
  \inferrule*[lab=$\rightarrow$]
    {\Gamma \vdash B_1 \le A_1 \\ \Gamma \vdash A_2 \le B_2}
    {\Gamma \vdash A_1 \rightarrow A_2 \le B_1 \rightarrow B_2}
  \\
  \inferrule*[lab=$\forallL$]
    {\Gamma \vdash \tau \\ \Gamma \vdash [\tau / x]\, A \le B}
    {\Gamma \vdash \forall x.\, A \le B}
  \and
  \inferrule*[lab=$\forallR$]
    {\Gamma ,\, x \vdash A \le B}
    {\Gamma \vdash A \le \forall x.\, B}
\end{drulepar}

\caption{The polymorphic subtyping relation by Odersky and L\"aufer~\cite{odersky1996putting}.}
\label{fig:polymorphic-subtyping}
\end{figure}

\paragraph{Generalizing Polymorphic Subtyping}
\label{sec:polymorphic-subtyping}

The parameters of universal types can only be types in the polymorphic
subtyping relation by Odersky and L\"aufer.
In \name, we generalize the polymorphic parameters so that they can
be values or other kinds of types as well.
A first idea of a direct generalization is:

\begin{mathpar}
  \inferrule*[lab=$\forallL'$]
    {\Gamma \vdash \tau \rulehl{: A} \\ \Gamma \vdash [\tau / x]\, B \le C}
    {\Gamma \vdash \forall x \rulehl{: A}.\, B \le C}
  \and
  \inferrule*[lab=$\forallR'$]
    {\Gamma ,\, x \rulehl{: B} \vdash A \le C}
    {\Gamma \vdash A \le \forall x \rulehl{: B}.\,C}
\end{mathpar}

\noindent The parameters for universal types can have any type (and not just $\star$).
Hence, instead of requiring the monotype $\tau$ to be a well-formed type in rule
$\forallL$, in rule $\forallL'$ it is
required that $\tau$ is well-typed regarding the type of the parameter
in the universal quantifier.
Similarly, for rule $\forallR'$ the context for the subtyping rule should include typing information
for the universally quantified variable.
However, this idea introduces the issue of potential mutual dependency between
subtyping and typing judgements, so further adjustments have to be made to formalize
this idea, which is discussed later in this section and Sections
\ref{sec:type-system} and \ref{sec:adaptation}.

\paragraph{Higher-Ranked Polymorphic Subtyping}

As the \verb|callcc| example in Section \ref{sec:higher-ranked-poly} shows, these subtyping
rules based on polymorphic subtyping, combined with other subtyping rules,
are able to handle the subtyping relations that occur at not only top-level,
but also at a higher-ranked level. This feature distinguishes our \name from the
Implicit Calculus of Constructions (ICC) \cite{miquel2001implicit} which also talks about
the implicit polymorphism of dependent type languages. The ICC features the two related rules
in the \emph{typing relation}:

\begin{mathpar}
  \inferrule*[lab=inst]
    {[[G |- e : forall x : A. B]] \\ [[G |- e1 : A]]}
    {[[G |- e : [e1 / x] B]]}
  \and
  \inferrule*[lab=gen]
    {[[G, x : A |- e : B]] \\ [[G |- forall x : A. B : k]]}
    {[[G |- e : forall x : A. B]]}
\end{mathpar}

\noindent Without an explicit subtyping relation, the ICC is not always able
to handle subtyping at higher-ranked positions. The approach taken by the ICC
is similar to that of the Hindley-Milner type system~\cite{hindley1969principal,damas1982principal},
which is also designed for dealing only with rank-1 polymorphism.
Hindley-Milner's declarative system also has a \textsc{GEN} rule to
convert expressions to polymorphic types, and a
\textsc{INST} rule to instantiate polymorphic parameters.
Both rules work only
for polymorphic types at top-level positions. In Hindley-Milner
the universal quantifier can only quantify over types, whereas in the ICC
it can quantify over terms of an arbitrary type (including types themselves).
% which also features similar rules in typing.
% But Hindley-Milner is designed for dealing only with rank-1 polymorphism.
In generalizations of higher-ranked polymorphic
type-inference~\cite{dunfield2013complete,le2003ml,leijen2008hmf,vytiniotis2008fph,jones2007practical},
it has been shown that rules like $\forallL$ and $\forallR$ generalize rules like
\textsc{GEN} and \textsc{INST}. Since we aim at higher-ranked polymorphic generalization,
we follow a similar, more general, approach in \name.

\paragraph{Unified Subtyping}
The revised subtyping relation with $\forallL'$ and $\forallR'$ rules suffers from an
important complication compared to the Odersky and L\"aufer formulation: there is now
a notorious mutual dependency between typing and subtyping.
In Odersky and L\"aufer's rules, the subtyping rules
do not depend on typing. In particular
the rule $\forallL$ depends only on well-formedness ($\Gamma \vdash \tau$).
In contrast, note that rule $\forallL'$ now mentions the typing relation
in its premise ($\Gamma \vdash \tau : A$). Moreover, as usual,
the subsumption rule of
the typing relation (shown below) depends on the subtyping relation.
\begin{mathpar}
  \inferrule*[lab=t-sub]
    {\Gamma \vdash e : A \\ \Gamma \vdash A \le B}
    {\Gamma \vdash e : B}
\end{mathpar}
This mutual dependency problem has been a significant
problem when combining subtyping and dependent types~\cite{subdep, hutchins},
and presents itself on our way to the direct generalization of polymorphic subtyping.

To tackle this issue, we adopt a technique called
\emph{unified subtyping}~\cite{full}. Unified subtyping merges the typing relation and
subtyping relation into a single relation to avoid the mutual dependency:
\begin{mathpar}
  \Gamma \vdash e_1 \le e_2 : A
\end{mathpar}
The interpretation of this judgement is: under context $\Gamma$, $e_1$ is a subtype
of $e_2$ and they both are of type $A$. The judgements for subtyping and typing
are both special forms of unified subtyping: % with the involvement of kind $[[*]]$:
\begin{mathpar}
  \Gamma \vdash A \le B \triangleq \Gamma \vdash A \le B : [[*]]
  \and
  \Gamma \vdash e : A \triangleq \Gamma \vdash e \le e : A
\end{mathpar}
The technique simplifies the formalization of dependently typed calculi with subtyping,
and especially the proof of transitivity in the original work. Ideally after applying the technique,
the generalization of the polymorphic subtyping should be:

\begin{mathpar}
  \inferrule*[lab=$\le\forall'' L$]
    {\Gamma \vdash \tau : A \\ \Gamma \vdash [\tau / x]\, B \le C \rulehl{: [[*]]}}
    {\Gamma \vdash \forall x : A.\, B \le C \rulehl{: [[*]]}}
  \and
  \inferrule*[lab=$\le\forall'' R$]
    {\Gamma ,\, x : B \vdash A \le C \rulehl{: [[*]]}}
    {\Gamma \vdash A \le \forall x : B.\,C \rulehl{: [[*]]}}
\end{mathpar}

\noindent The basic idea of our own formalization essentially follows a similar design,
although the actual rules in \name are slightly more sophisticated.
The details will be discussed in Section \ref{sec:type-system}.

\paragraph{``Explicit'' Implicit Instantiation}

With polymorphic subtyping the instantiation of universally quantified type
parameters is done implicitly instead of being manually applied. In non-dependently
typed systems, \emph{implicit} parameters are types (i.e. terms are not involved in
implicit instantiation). For example:
\begin{mathpar}
  (\lambda x.\, x)~42 \longrightarrow 42
\end{mathpar}
\noindent Here $\lambda x.\, x$ has type $\forall A.\, A \rightarrow A$, and
instantiation implicitly discovers that $A = Int$.
Notably, and in contrast with explicitly polymorphic languages like System F, implicit
instantiation is not reflected anywhere at term level.
The design that we adopt still provides implicit instantiation, but
it is more explicit regarding the binding of implicit parameters.
We adopt this design to ensure that polymorphic variables are well-scoped in
type annotations of terms. Thus we use another binder, of the form $\Lambda(x : A). e$, for terms.
Nonetheless, instantiations are still
implicit as shown in the following example:
\begin{mathpar}
  (\Lambda A : [[*]].\, \lambda x : A.\, x) ~ 42 \longrightarrow 42
\end{mathpar}
Here $\Lambda A : [[*]].\, \lambda x : A.\, x$ has type $\forall A : \star. \, A \rightarrow A$,
and the polymorphic parameter $A$ is explicitly stated in the polymorphic
term. However as the reduction shows, the instantiations are still implicit.
We purposely omitted the explicit binders for implicit parameters for all the examples
in Section \ref{sec:examples} for conciseness. Such explicit binders can
be recovered with a simple form of syntactic sugar:

\[e : \forall x : A.\, B \triangleq \Lambda x : A.\, e : \forall x : A.\, B\]

%\noindent When polymorphic parameters are used, \name provides a binder ($\Lambda x : A.\, e$)
%to ensure that the parameters are well-scoped at the term-level.

\paragraph{Computational Irrelevance}
\label{sec:computational-irrelevance-overview}

Implicit parameters in traditional languages with polymorphic subtyping,
the ICC~\cite{miquel2001implicit,barras2008implicit} and \name are computationaly irrelevant.
In traditional (non-dependently) typed languages, types cannot affect computation,
thus computational irrelevance is quite natural and widely adopted.
Furthermore, computational irrelevance can benefit performance, since
irrelevant arguments can simply be erased at runtime.
In dependently typed systems, however, there can be some programs where
it is useful to have computationaly relevant implicit parameters.
For example, accessing the length of a length indexed vector in constant time:
\begin{flalign*}
  &\mathrm{length} : \forall n : [[int]].\, \mathrm{Vector}~n \rightarrow [[int]] &&\\
  &\mathrm{length} = \Lambda n : [[int]].\, \lambda v : \mathrm{Vector}~n.\, n
\end{flalign*}
\noindent Here the implicit parameter $n$ is computationally relevant as it is used as
the return value of the function which is likely to be executed at runtime.
Languages like Agda, Coq or Idris support such programs. However,
computationaly relevant implicit parameters are challenging for proofs of
type soundness. Due to such challenges (see also the discussion in
Section~\ref{subsec:semantics}),
the ICC has a restriction that parameters for implicit function types
must be computationally irrelevant. Since we adopt a similar technique for the type
soundness proof, we also have a similar restriction and thus cannot encode programs such
as the above.

\paragraph{Type-level Computation and Casts}
\name features a fixpoint operator that supports general recursion at both
type and term level. In order to avoid diverging computations at type checking,
we do not provide the conversion rule (or congruence rule) like other
dependently typed systems such as the Calculus of Constructions~\cite{coc}
to support implicit type-level reduction.
\begin{mathpar}
  \inferrule*[lab=Cong]
    {[[G |- e : A]] \\ \rulehl{A =_\beta B}}
    {[[G |- e : B]]}
\end{mathpar}

\noindent The presence of the conversion rule makes the decidability of
type checking rely on the strong normalization of type-level computation
(to determine whether two types are $\beta$-equivalent).
But the presence of general recursion denies the strong normalization property
of our language.

Instead of using a conversion rule, we adopt the call-by-name design of
\emph{Pure Iso-Type Systems} (PITS)~\cite{isotype,yang2019pure},
and provide $\castdn$ and $\castup$ operators to explicitly trigger one-step
type reductions or expansions as shown in the typing rules below.
\begin{mathpar}
  \inferrule*[lab=Castup]
    {[[G |- e : B]] \\ \rulehl{[[A --> B]]} \\ [[G |- A : k]]}
    {[[G |- castup [A] e : A]]}
  \and
  \inferrule*[lab=Castdn]
    {[[G |- e : A]] \\ \rulehl{[[A --> B]]} \\ [[G |- B : k]]}
    {[[G |- castdn e : B]]}
\end{mathpar}

\noindent Now since reductions only perform one step per use of cast
operators, whether a term is strongly normalizing no longer affects the
decidability of type checking.
Note that there are some other cast designs in the
literature~\cite{guru,sjoberg:msfp12, kimmel:plpv, zombie:popl15}, but
we adopt the PITS design here for simplicity. We believe that other cast
designs could also be adopted instead, but leave this for future work.

\section{\System}
\label{sec:system}

This section introduces the static and dynamic semantics of
\name: a dependently typed calculus with type casts
and implicit polymorphism. The calculus employs
\emph{unified subtyping}~\cite{CoquandThierry1988Tcoc}
and has a single relation that generalizes both typing and subtyping.
The calculus can be seen as a variant of the \emph{calculus of constructions}~\cite{},
but it uses a simple form of casts~\cite{} instead of the conversion rule
and features unrestricted recursion. We present syntax, unified subtyping
and reduction for \name.

\begin{figure}[t]
\centering
\begin{equation*}
\begin{array}{llcl}
    \text{Kinds} & k & ~\Coloneqq ~ & [[*]] \mid [[box]] \\
    \text{Expressions} & e, A, B & ~ \Coloneqq ~ & [[x]] \mid [[n]] \mid [[k]] \mid [[int]] \mid [[e1 e2]] \mid [[lambda x : A. e]] \mid [[pi x : A. B]] \\
        & & \mid & [[bind x : A. e]] \mid [[forall x : A. B]] \mid [[mu x : A. e]] \\
        & & \mid & [[castup [A] e]] \mid [[castdn e]]   \\
    \text{Mono-Expressions} ~ & \tau, \sigma & ~ \Coloneqq ~ & [[x]] \mid [[n]] \mid [[k]] \mid [[int]] \mid \tau_1 ~ \tau_2 \mid \lambda \, x : \tau. ~ \sigma \mid \Pi \, x : \tau. ~ \sigma \\
        & & \mid & \Lambda \, x : \tau. ~ \sigma \mid \mu \, x : \tau. ~ \sigma \mid \castup \, [\tau]~ \sigma \mid \castdn \, \tau \\
    \text{Values} & v & ~ \Coloneqq ~ & [[k]] \mid [[n]] \mid [[int]] \mid [[lambda x : A. e]] \mid [[pi x : A. B]] \mid [[bind x : A. e]] \\
        & & \mid & [[forall x : A. B]] \mid [[castup [A] e]] \\
    \text{Contexts} & \Gamma & ~ \Coloneqq ~ & [[nil]] \mid [[G , x : A]] \\
    \text{Syntactic Sugar} ~ & A \rightarrow B & \triangleq & [[pi x : A. B]] \qquad \text{where} ~ x \notin \mathrm{FV}(B)
\end{array}
\end{equation*}
\caption{Syntax of \name.}
\label{fig:syntax}
\end{figure}
\subsection{Syntax}

Figure \ref{fig:syntax} presents the syntax of \name. The syntax is similar
syntax to the Calculus of Constructions, featuring
$[[*]]$ and $[[box]]$ in the kind hierarchy, and unifying the concepts of terms
and types as expressions. Due to the unified syntax, types and
expressions ($e$, $A$ and $B$) are used
interchangeably, although we mostly adopt the convention of using $A$ and $B$
for contexts where the expressions are used as types and $e$ for contexts
where the expressions represent terms.
The syntax includes all the constructs of the calculus of constructions:
variables ($[[x]]$), kinds ($[[k]]$), function applications  ($[[e1 e2]]$),
lambda expressions ($[[lambda x : A. e]]$), dependent function types ($[[pi x : A. B]]$)
as well as integer types ($[[int]]$) and integers ($[[n]]$).
Moreover, there are a number of additional language constructs for
supporting implicit polymorphism, recursion and explicit type-level computation
via casts. These constructs are discussed next.

\subsubsection{Implicit Polymorphism.}
In \name, universal types $[[forall x : A. B]]$ are used to generalize implicit
polymorphism in non-dependent.
In contrast to universal quantification in conventional functional languages, the
argument $x$ ranges over all well-typed expressions besides well-formed
types (i.e. $x$ can have any type $A$ instead of just kind $\star$).
In other words, ``polymorphic types'' are naturally dependent, so $\forall$
types can be viewed as the implicit counterpart of $\Pi$ types. We also have
implicit lambda expressions ($[[bind x : A. e]]$), which are the implicit counterpart of
  $\lambda$ abstractions. The design used in \name
  is similar to the design of the \emph{Implicit Calculus of Constructions} ($\text{ICC}^*$)~\cite{barras2008implicit}, which
  employs similar constructs for implicit dependent products.
Like conventional universal quantification, the arguments of $\forall$ types are
deduced during applications rather than being explicitly passed.
In addition, following designs for predicative higher-ranked polymorphism~\cite{oderskylufer,DK,PJ}, we have also generalized the concept of \emph{monotypes} to
\emph{mono-expressions} ($\tau$), essentially excluding $\forall$ types from expressions.

\subsubsection{Recursion and Explicit Type-level Computation.}
\label{sec:cast}
The \name calculus adopts \emph{iso-types}~\cite{yang2016unified,yang2019pure},
featuring explicit type-level computation with cast operators
$\castdn$ and $\castup$. These operators respectively perform one-step
type reduction and type expansion based on the operational semantics.
The reduction in cast operators is deterministic, thus type
annotations are only needed during type expansions ($\castup$). We add
fixpoints ($[[mu x : A. e]]$) to support general recursion for both
term-level and type-level. Iso-recursive types are supported due to
the presence of $\castup$ and $\castdn$, which correspond to the
\verb|fold| and \verb|unfold| operations when working on iso-recursive types.

\subsection{Operational Semantics}

% \bruno{If I understand correctly we need 2 different reductions, one
%   (the non-deterministic one) that is used in the type system; and another
%   one (which erases types) that is deterministic and would be the basis
%   for an actual implementation of reduction at run-time. After reading
%   this subsection, I think we want
% to tell that story here and present the 2 variants of reduction here.}

For the operational semantics we employ two different, but closely related,
reduction relations. The first reduction relation is non-deterministic, and
it is used at the type-level to allow type conversions induced
by the cast operators. The second reduction relation is deterministic and
is employed to give the run-time semantics of expressions.

\subsubsection{Nondeterministic Reduction.}
Figure \ref{fig:semantics} presents the small-step operational semantics of our system,
which mostly follows the call-by-name variant of \emph{Pure Iso-Type Systems} (PITS)
\cite{yang2019pure} corresponding to the calculus of constructions.
Note that the arguments of $\beta$-reduction (\rref{r-beta}) and expressions in
the \rref{r-cast-elim} are not required to be values.
Meanwhile we consider $\castup$ terms to be a value,
and only perform reduction inside $\castdn$ terms (\rref{r-castdn}). Also, the unroll
operation of the fixpoint operator is supported by \rref{r-mu}.

\begin{figure}[t]
    \centering

    \begin{drulepar}[r]{$[[e1 --> e2]]$}{Operational Semantics}
      \drule{app}
      \drule{beta}
      \and \ottaltinferrule{r-inst}{}{ }
        {[[(bind x : A. e1) e2 --> ([t / x] e1) e2]]}
      \drule{mu}
      \drule{castdn}
      \and \ottaltinferrule{r-cast-inst}{}{ }
        {[[castdn (bind x : A. e) --> castdn ([t / x] e)]]}
      \drule{castXXelim}
    \end{drulepar}

    \caption{Operational semantics of \name.}
    \label{fig:semantics}
\end{figure}
% \bruno{Do not use $mono~e$ in the figure.
%   You have the syntax $\tau$ and $\sigma$ to represent
% monotypes, so just use that instead.}

%\subsubsection{Nondeterministic Implicit Instantiations}
Due to the presence of instantiation of implicit parameters, the direct operational
semantics is not deterministic, and potentially not type-preserving because of
\rref{r-inst,r-cast-inst}. The indeterminacy is caused by the guess of $\tau$,
which can be an arbitrary monoexpression, since we do not have access to typing
information in the dynamic semantics.

\begin{figure}
  \label{fig:extraction}
  \centering
  \begin{equation*}
  \begin{array}{llcl}
      \text{Erased Expressions} & e, A, B & ~ \Coloneqq ~ & [[x]] \mid [[n]] \mid [[k]] \mid [[int]] \mid [[ee1 ee2]] \mid [[elambda x. ee]] \mid [[epi x : eA. eB]] \\
      & & \mid & [[ebind x. ee]] \mid [[eforall x : eA. eB]] \mid [[emu x. ee]] \mid [[ecastup ee]] \mid [[ecastdn ee]] \\
      \text{Erased Value} & ev & ~ \Coloneqq ~ & [[k]] \mid [[n]] \mid [[int]] \mid [[elambda x. ee]] \mid [[epi x : eA. eB]] \mid [[ebind x. ee]] \\
      & & \mid & [[eforall x : eA. eB]] \mid [[ecastup ee]]
  \end{array}
  \end{equation*}

  % The behavior of gather and align under this lipics template is extremely
  % weird, this is the best I can do :(
  \begin{gather*}
    \begin{align*}
    \extract{[[x]]} &= [[x]] &
    \extract{[[n]]} &= [[n]] &
    \extract{[[k]]} &= [[k]] &
    \extract{[[int]]} &= [[int]]
    \end{align*} \\
    \begin{align*}
     \extract{[[e1 e2]]} &= \extract{[[e1]]} ~ \extract{[[e2]]} & % \\ % \and
      \extract{[[mu x : A. e]]} &= \mu \, x. ~ \extract{[[e]]} \\
     \extract{[[lambda x : A. e]]} &= \lambda \, x. ~ \extract{[[e]]} & % \\ % \and
      \extract{[[pi x : A. B]]} &= \Pi \, x : \extract{[[A]]}. ~ \extract{[[B]]} \\
     \extract{[[bind x : A. e]]} &= \Lambda \, x. ~ \extract{[[e]]} & % \\ % \and
      \extract{[[forall x : A. B]]} &= \forall \, x : \extract{[[A]]}. ~ \extract{[[B]]} \\
     \extract{[[castup [A] e]]} &= \castup \, \extract{[[e]]} & % \\ % \and
      \extract{[[castdn e]]} &= \castdn \, \extract{[[e]]}
    \end{align*}
  \end{gather*}

  \drules[er]{$[[ee1 *--> ee2]]$}{Erased Semantics}
    {app,beta,elim,mu,castdn,castXXinst,castXXelim}
  \caption{Erased Expressions and Operational Semantics}
\end{figure}
% \bruno{You need to use something like a latex table/tabular,
% to neatly align the erasure function.}

\subsubsection{Deterministic Reduction}
We address the issue of determinancy of the dynamic semantics with
a design similar to $\mathrm{ICC}^*$ \cite{barras2008implicit},
employing type-erased expressions. The erased expressions
essentially mirror the syntax and semantics
of normal expressions, except for the elimination of type annotations in $\lambda$,
$\Lambda$, $\mu$ and $\castup$ expressions.
Figure \ref{fig:extraction} shows the syntax of the erased expressions and
the companion operational semantics. Note that restrictions are imposed in the
typing rules to forbid the implicit parameter occuring in runtime-relevant part
of the expression, i.e. the erased expressions (see section \ref{sec:type-system}).
With such restriction, implicit parameters can be directly eliminated in
\rref{er-elim,er-cast-inst}. For a well-typed expression, the reduction of
its erasure is deterministic. The type safety of our system is built around
this idea and is discussed in \ref{sec:type-safety}.

\begin{figure}
    \centering
    \begin{drulepar}[wf]{$[[|- G]]$}{Well-formed Context}
      \mprset{sep=1.2em}
      \drule{nil}
      \drule[width=30em]{cons}
    \end{drulepar}

    \begin{drulepar}[s]{$[[G |- e1 <: e2 : A]]$}{Unified Subtyping}
      \mprset{sep=1.3em}
      \drule{var}
      \drule{lit}
      \drule{int}
      \drule{star}
      % \drule{abs}
      \and \ottaltinferrule{abs}{width=20em}
        {\rulehl{[[G |- A : k1]]} \\ [[G, x : A |- B : k2]] \\ [[G, x : A |- e1 <: e2 : B]]}
        {[[G |- lambda x : A. e1 <: lambda x : A. e2 : pi x : A. B]]}
      % \drule{app}
      \and \ottaltinferrule{s-app}{}
        {[[G |- t : A]] \\ [[G |- e1 <: e2 : pi x : A. B]]}
        {[[G |- e1 t <: e2 t : [t / x] B]]}
      \drule{pi}
      % \drule{bind}
      \and \ottaltinferrule{s-mu}{width=20em}
        {[[G |- t : k]] \\ [[G , x : t |- s : t]]}
        {[[G |- mu x : t. s <: mu x : t. s : t]]}
      \and \ottaltinferrule{s-bind}{width=20em}
        {\rulehl{[[G |- A : k]]} \\ [[G , x : A |- B : *]] \\
         [[G, x : A |- e1 <: e2 : B]] \\
         x \notin \mathrm{FV}(\extract{[[e1]]}) \cup \mathrm{FV}(\extract{[[e2]]})}
        {[[G |- bind x : A. e1 <: bind x : A. e2 : forall x : A. B]]}
      % \drule{mu}
      \drule{castup}
      \drule{castdn}
      % \drule{forallXXl}
      \and \ottaltinferrule{s-forall-l}{width=20em}
        {\rulehl{[[G |- A : k]]} \\ [[G |- t : A]] \\
         [[G , x : A |- B : *]] \\
         [[G |- [t / x] B <: C : *]]}
        {[[G |- forall x : A. B <: C : *]]}
      % \drule{forallXXr}
      \and \ottaltinferrule{s-forall-r}{width=20em}
        {\rulehl{[[G |- B : k]]} \\ [[G |- A : *]] \\
         [[G , x : B |- A <: C : *]]}
        {[[G |- A <: forall x : B. C : *]]}
      % \drule{forall}
      \and \ottaltinferrule{s-forall}{width=20em}
        {\rulehl{[[G |- A : k]]} \\ [[G , x : A |- B <: C : *]]}
        {[[G |- forall x : A. B <: forall x : A. C : *]]}
      \and \ottaltinferrule{s-sub}{width=15em}
        {[[G |- e1 <: e2 : A]] \\ [[G |- A <: B : k]]}
        {[[G |- e1 <: e2 : B]]}
    \end{drulepar}

    % \drules[s]{$[[G |- e1 <: e2 : A]]$}{Unified Subtyping}{
    %     var,lit,int,star,abs,pi,app,bind,mu,castup,castdn,forallXXl,forallXXr,forall,sub}
    \begin{equation*}
       \text{Syntactic Sugar} \qquad [[G |- e : A]] \triangleq [[G |- e <: e : A]]
    \end{equation*}
    \caption{(Sub)Typing Rules of \name.}
    \label{fig:typing}
\end{figure}
% \bruno{The figure and rules will need a little cosmetic work. Firstly, there are many rules
%   stacking up premises vertically. I think it is better to have multiple rules
%   horizontally (and only upto 2 or 3 vertical stacks of premises).
%   Secondly, we need to look at the layout carefully to use space efficiently.
%   At the moment there are rules like S-ABS that could be paired up with some other
%   rules side-by-side and use less space (at the same time it is nicer
%   if adjancent rules are somehow related: two cast rules; application/abstraction, etc).
%   We must organize the rules in a nicer way. Thirdly, I think that, for binders with
%   annotations, like $\lambda x : A. e$, we may want to use brackets on the arguments
%   to improve readability, as in $\lambda (x : A). e$. It is a bit hard to ``parse''
%   the syntax without a little bit of effort. If this change is implemented it would
%   affect the whole section, starting from syntax.
% }

\subsection{Unified Subtyping System}
\label{sec:type-system}

Figure \ref{fig:typing} shows the (sub)typing rules of the system. We adopt a
simplified design of unified subtyping~\cite{yang2017unifying}, where the subtyping rules and
typing rules are merged into a single typing judgment $[[G |- e1 <: e2 : A]]$.
% The originals design lean towards object-oriented features,
% supporting generalized top type and bounded quantification,
% while our work focus on the subtyping relation between polymorphic types.
% \bruno{The previous sentence is something to be mentioned in related work, not here.}

Unified subtyping solves the challenging issue of mutual dependency between typing
and subtyping in a dependently type system.
% \bruno{Make sure that the overview discusses this issue.}
The interpretation of this judgment is ``under context $[[G]]$, $[[e1]]$ is a
subtype of $[[e2]]$ and they are both of type $[[A]]$''.
In this form of formalization, the typing judgment $[[G |- e : A]]$ is a
special case of unified subtyping judgment $[[G |- e <: e : A]]$,
and the well-formedness of types $\Gamma \vdash A$ is expressed by
$[[G |- A : k]]$ where $k \in \{[[*]], [[box]]\}$.

Although unified, the typing rules can still be viewed as two parts, the ``typing'' part
(\rref{s-abs,s-app,s-bind,s-mu,s-castup,s-castdn,s-sub}) and the ``subtyping'' part
(\rref{s-pi,s-forall,s-forall-l,s-forall-r}). We follow a usual design for
typing rules for lambda abstraction and application, and the subtyping rule of
dependent function types ($\Pi$ type).
% \bruno{We should say something about the ``more standard'' rules here.
%   I think we can briefly explain them and point the
%   reader to Linus work for further details, while observing that the rules
%   here are essentially simplified version (due to the absence of bounded quantification)
% of his rules.}

\subsubsection{Rules for Universal Quantification}
The subtyping rules for universal quantification (\rref{s-forall-l,s-forall-r}) follow
the spirit of the Odersky and L\"aufer's polymorphic subtyping~\cite{odersky1996putting,DunfieldJoshua2013Caeb},
where the subtyping relation is interpreted as a ``more general than'' relation.
A polymorphic type $[[forall x : A. B]]$
is more general than another type $C$ when its well-typed
instantiation is more general than $C$ (\rref{s-forall-l}). A polymorphic
type $[[forall x : B. C]]$ is less general than a type $A$,
if $C$ is is less general than $A$ when the argument with the polytype ($x:B$)
is abstracted out (\rref{s-forall-r}).

Notably our formalization is not a direct generalization of the polymorphic subtyping,
\rref{s-forall} axiomatizes the subtyping relation between two polymorphic types.
And additional premises are added to \rref{s-forall-l,s-forall-r} aside from the
ones we previously mentioned in section \label{sec:polymorphic-subtyping}
The motivations for these changes are discussed in more detail in section \ref{sec:adaptation}.

\subsubsection{Mono-expression Restrictions}
As in other predicative relations (such as the one by Odersky and L\"aufer),
the type arguments for instantiation in \rref{s-forall-l} are
required to be mono-expressions, which has cascading effects on typing rules of
other expressions. The arguments for applications are required to be
mono-expressions, and the whole fixpoint expression is required to be a
mono-expressions. We shall
discuss the reason behind the restrictions in later sections.
\bruno{I feel that we may be deferring a bit too much explanation
to later sections, but lets come back to this after you write later sections.}

\subsubsection{Kind Restriction for Universal Types}
\label{sec:kind-restriction}

For the kinding of types, we mainly follow the design of the Calculus of
Constructions~\cite{CoquandThierry1988Tcoc}. However, we specifically restrict
the $[[forall x : A. B]]$
expressions to only have the kind $[[*]]$. This prevents other types of kind
$[[*]]$ from having kinds such as $[[forall x : int. *]]$,
which significantly complicates the metatheory when reasoning about the kind of types.
This restriction propagates to the introduction rule of $\forall$ types (\rref{s-bind}),
where $[[B]]$ is required to only have kind $[[*]]$.
This way well-typed implicit abstractions ($\Lambda$ expressions) are kept away
from type computations. Therefore, in cast operators,
the possibility of non-deterministic implicit instantiations is eliminated.

\subsubsection{Runtime Irrelevance of Implicit Arguments}

Our direct operational semantics choose random mono-expressions to instantiate
the implicit arguments which potentially breaks type safety, so we adopt a
restriction in \rref{s-bind} that is similar to the
Implicit Calculus of Constructions (ICC) \cite{miquel2001implicit}.
We only allowing the implicit parameters to occur in type annotations in the
body of implicit abstraction, so that the choices of implicit parameters is not
relevant at runtime. The type safety of the direct operational semantics is
proved indirectly in section \ref{sec:type-safety} with the help of the
erasure of expressions.

\subsubsection{Redundant Premises}

All the premises highlight in gray are redundant in a way that
the system without them is proved equivalent to the system in figure \ref{fig:typing}.
They are there to simplify the mechanized proofs of certain lemmas.


\section{Metatheory}

\begin{theorem}[Weakening]
    If $[[G1]] , [[G2 |- e1 <: e2 : A]]$, then $[[G1 , x : B]] , [[G2 |- e1 <: e2 : A]]$
\end{theorem}

\section{Discussions}

\subsection{The Trouble with Instantiation in Subtyping}
\label{sec:instantiation}

One of the features of DK's subtyping is that the instantiation of implicit type
parameter happen during subtyping of polymorphic types. It makes sense when
viewing the subtyping relation as a more-general-than relation. A polymorphic
type is a subtype of another when we can find specific instantiation of the
type parameter. This idea works well in DK's system, but brings troubles
in the realm of dependent types, consider the following subtyping relation:

\begin{equation*}
    A : [[*]] \vdash [[forall x : A. int]] <: [[int]] : [[*]]
\end{equation*}

The relation above is not derivable in \name, because the type of implicit type
parameter is an abstract type, for which we are not able to find an well-typed
instantiation except for the infinite loop $\mu x : A.\, x$, this is the also the
case when $A$ in an arbitrary uninhabited types (without the help of a fixpoint).

This situation also assign a special role to variables in the context,
for example:

\begin{multline*}
    A : [[*]],\, F : A \rightarrow [[*]],\, \rulehl{a : A} \vdash \\
    \forall x : A.\, (F~x \rightarrow F~x) \rightarrow [[int]] \le (\forall x : A.\, F~x \rightarrow F~x) \rightarrow [[int]] : [[*]]
\end{multline*}

Without the help of the fixpoint, this subtyping relation is only derivable
with the presence of the highlighted variable in the context. The relation has to
be derived from \rref{s-forall-l}, which requires a well-typed instantiation for
the parameter, in this case, only $x$ is eligible even though it does not occur
anywhere in the expression except for the context.

The help of fixpoint does not solve the general problem, because
in \name the fixpoint expressions are only well-typed when it is not a polymorphic
type. So the general ``Strengthening'' lemma is not admissible in \name. Even
a restricted case where we only consider typing complicated examples can still
be construct to stop us from eliminating the a variable in the context even
when it is fresh everywhere else:

\begin{equation*}
    F : [[int]] \rightarrow [[*]],\, A : [[*]],\, \rulehl{a : A} \vdash F ~ ([[(bind x : A. lambda y : int. y)]]~ 42) : [[*]]
\end{equation*}

For the time being, we think the addition of the premise in \rref{s-forall-r} and
the addition of \rref{s-forall} do not complicate the metatheory as much, so
we leave the further exploration of the issue above in a future work.

\subsection{Design Choices of the Semantics around Cast Operators}
\label{sec:cast-design}

The type reduction in cast operators is potentially under a context that
is not empty, so it is likely that we are performing reduction to a open term.
Intuitively we should generalize the definition of value by introducing inert
terms\cite{yang2017unifying} to handle open term reduction.

However since we adopts the Call-by-Name semantics of Pure Iso-type System\cite{yang2019pure},
there is no value check during the reduction, and whether the result of reduction
inside cast operator does not matter during the reasoning of type safety. It is
not necessary to complicate the metatheory by introducing inert terms.

An alternate design around cast operator is the Call-by-Value (CBV) style\cite{yang2019pure},
by not considering all $\castup$ terms as value, and performing cast elimination only
when the expression inside two casts is a value. Such design requires us to
have a more general definition for value, and thus there is a need for inert terms.

However, a simple design with CBV-style cast semantics and inert terms
potentially lead to a system where \emph{Reduction Substitution} does not hold,
for example:

\begin{gather*}
    \castdn \, \castup \, [A] \, f ~ x \longrightarrow f ~ x \\
    [\lambda x : B. \, x / f] \castdn \, \castup \, [A] \, f ~ x \longrightarrow \castdn \, \castup \, [A] x
\end{gather*}

So we stick with the Call-by-Name style semantics around cast operators and
leave the discussion of other possibilities of design in a future work.

\section{Related Work}

\paragraph{Implicit Dependent Type Calculus}
The implicit calculus of constructions (ICC~\cite{miquel2001implicit} and
ICC*\cite{barras2008implicit}) discuss implicit polymorphism in a dependently-typed
setting. ICC features generalized polymorphic types and typing rules to express
the idea of implicit instantiation. They do not explicitly have a subtyping
relation between polymorphic types. Therefore the expressiveness of reasoning
between polymorphic types are limited to the top-level polymorphic types. Like
in \name, the implicit parameter does not impact the runtime semantics of ICC.

Implicit function types of ICC* are not interpreted as polymorphic
function types. The main focus is on the distinction between implicit
functions (universal types and implicit abstraction) and explicit functions
($\Pi$-types and lambda abstraction).
The typing rules about the implicit part and explicit part of the language mirror
each other. The generalization and instantiation aspect of the implicit function
types are not featured. ICC* depends on its transformation to ICC to obtain type safety
of the language, therefore the parameters of implicit functions have no impact
on runtime behavior as well.

\paragraph{Type-inference and unification with dependent types}
There has been little work on formalizing type inference for calculi
with dependent types, although essentially all implementations of
theorem provers or dependently typed languages perform some form of
type-inference.
One important challenge for type inference in
systems with dependent types and a conversion rule
is that they require \emph{higher-order unification},
which is known to be undecidable~\cite{goldfarb1981undecidability}. The \textit{pattern}
fragment~\cite{miller1991unification} is a well-known decidable
fragment. Much literature on unification for dependent
types~\cite{reed2009higher,abel2011higher, gundry2013tutorial, Cockx:2016:UEP:2951913.2951917, ziliani2015unification, coen2004mathematical} is
built upon the pattern fragment. Algorithms for type-inference used in Agda and
(Dependent) Haskell have been described and formalized to some degree
in various theses~\cite{norell,gundry,dh}. However, as far as we know
there is not a clear specification and complete metatheory (let alone
mechanized) for such algorithms.

The current GHC Haskell's language of types and kinds
is already dependently typed, but has no type conversion. Thus
it is able to avoid higher-order unification. Recent work by
Xie et al.~\cite{xie20kind} describes algorithms and specifications
for the form of (dependently typed) kind-inference currently present in GHC Haskell.
The dependently typed language of types and kinds is closely related
to \name. In particular in both calculi type equality is based only
on $\alpha$-equivalence. One difference is that in GHC Haskell and, more precisely,
in the core language employed by GHC, there
are no type-level lambdas. The GHC Haskell source language does allow
type families~\cite{assoctypes}, which mimick type-level functions. However,
type families, unlike lambda functions, are not first class, and do not support partial application.
They are encoded
in terms of equality constraints, casts and mechanisms similar to those
employed by type classes. There is some work to make type-level functions
provided type families first-class~\cite{kiss19higher} and also partially applied, but this
still does not enable full type-level lambdas (see also the discussion in Section~8.1
of ~\cite{kiss19higher} for more details).
In our work we do allow type-level lambdas
but lambdas can only be equal up to $\alpha$-equivalence.
Another difference is that the kind-inference system formalized by Xie et al.
is not higher-ranked like ours. In this way Xie et al. manage to avoid
the mutual dependency issue that we have in our polymorphic subtyping relation.

\begin{comment}
Thus our focus is on Haskell-like languages with dependently typed
features and explicit casts, rather than languages like Agda or Idris
which typically have a conversion rule that triggers implicit type level
computation.
\end{comment}


\paragraph{Type-inference for higher-ranked polymorphism}
Type-inference for \emph{higher-ranked polymorphism}
(HRP)~\cite{dunfield2013complete,le2003ml,leijen2008hmf,vytiniotis2008fph,jones2007practical,Serrano2018, odersky1996putting}
extends the classic Hindley-Milner algorithm~\cite{hindley1969principal,milner1978theory,damas1982principal},
removing the restriction of top-level (let) polymorphism only. Type
inference for HRP aims at providing inference for System F-like
languages. In particular existing HRP approaches allow \emph{synthesis of type arguments}
and use type annotations to aid
inference, since type-inference for full System F is
well-known to be undecidable~\cite{wells1999typability}.

The work on HRP is divided into two strands: \emph{predicative} HRP~\cite{dunfield2013complete,jones2007practical,odersky1996putting,dunfield2019sound}
and \emph{impredicative} HRP~\cite{le2003ml,leijen2008hmf,vytiniotis2008fph,Serrano2018}.
In predicative HRP instantiations can
only synthesize monotypes, whereas in impredicative HRP there's no
such restriction. However impredicative HRP is quite complex because
the polymorphic subtyping relation for impredicative HRP is undecidable~\cite{tiuryn1996subtyping}.
Thus reasonable restrictions that work well in practice are still
a research frontier.
The monotype restriction on predicative instantiation is considered reasonable
and practical for most programs. It is currently in use by languages such as
(GHC) Haskell, Unison~\cite{Unison} and PureScript~\cite{PureScript}.
The original work on polymorphic subtyping by Odersky and L\"aufer also enforces
the monotype restriction in their subtyping rules (rule $\forallL$) to prevent
choosing a polytype in the instantiation. Based on polymorphic subtyping as
their declarative system,
% \bruno{mention Odersky and Laufer here before moving on to DK}
Dunfield and Krishnaswami (DK)~\cite{dunfield2013complete} develop an
algorithmic system for predicative HRP type inference. DK's algorithm was
manually proved to be sound, complete, and decidable.
With a more complex declarative system~\cite{dunfield2019sound}, DK
extended their original work with new features.
Recently Zhao et al.~\cite{zhao19mechanical} formalized DK's type system in the Abella
theorem prover.

\begin{comment}
\paragraph{MLSub} A recent breakthrough in the area of (global) type-inference
for type systems with subtyping is MLSub~\cite{dolan17polymorphism}. MLSub extends the Hindley-Milner
type system with support for subtyping. A key innovation
of MLSub is that it has compact principal types, which had been a challenge
in previous research on type-inference in the presence of subtyping~\cite{eifrig95inference,Trifonov96subtyping,pottier1998inference}.
MLSub is significantly more ambitious than local type-inference, and requires
no annotations (in the tradition of Hindley-Milner). However, MLSub does not
account for HRP and its algorithms and metatheory have not been mechanically
formalized.
\end{comment}

\paragraph{Dependent Types and Subtyping}
A major difficulty is that the introduction of dependent
types makes typing and subtyping depend on each other. This causes
several difficulties in developing the metatheory for calculi that
combine dependent types and subtyping. Almost all previous
work~\cite{subdep,ptssub,chen1,cocsub,Chen03coc} attempts to address such problem by somehow
\emph{untangling} typing and subtyping, which has the benefit that the
metatheory for subtyping can be developed before the metatheory of
typing. Nevertheless, several results and features prove to be
challenging.

Our work builds on the work done on Pure Iso-Type Systems (PITS)~\cite{yang2019pure}, and
\emph{unified subtyping}~\cite{full}. PITS is a variant of pure type systems (PTSs),
which captures a family of calculi with \emph{iso-types}.
Iso-types generalize \emph{iso-recursive
types}~\cite{tapl}, and provide a simple form of
type casts to address the combination of recursion and
dependent types.
Yang and Oliveira~\cite{full} introduce a calculus, called $\lambda_{I}$, supporting OOP features such as
\emph{higher-order subtyping}~\cite{fsubo}, \emph{bounded quantification} and
\emph{top types}.
To address the challenges posed by
the combination of dependent types and subtyping, $\lambda_{I}$
employs \emph{unified subtyping}: a novel technique that unifies
\emph{typing}, \emph{subtyping} and \emph{well-formedness} into one
relation. Therefore, $\lambda_{I}$ takes a significantly different
approach compared to previous work, which
attempts to fight the entanglement between typing and subtyping. In
contrast, $\lambda_{I}$ embraces such
tangling by collapsing the typing and subtyping
relations into the same relation. This approach is different from
Hutchins' technique, which eliminates the typing relation and embeds it into
a combination of subtyping, well-formedness and reduction relations.
In contrast, unified subtyping
retains the traditional concepts of typing and subtyping, which are just two
particular cases of the unified subtyping relation.

Although the $\lambda_{I}$ calculus formalized by Yang and Oliveira shares the use
of unified subtyping with \name, there are substantial differences between the two calculi.
Most importantly, $\lambda_{I}$ only has explicit polymorphism via $\Pi$ types. There
are no implicit functions and universal quantification ($\forall$ types) in $\lambda_{I}$,
and also no guessing of monotypes. \name supports implicit polymorphism, and
guessing the monotypes used for instantiation brings significant complications, for instance
for proving type safety (as discussed in Section~\ref{sec:type-safety}).
The subtyping rules for universal quantification (which do not exist in $\lambda_{I}$) also
bring considerable challenges for transitivity, and the proof technique used by
\name differs considerably from the proof technique used in $\lambda{I}$.
Unlike $\lambda_{I}$, \name does not support bounded quantification, which brings
some welcome simplifications to some of the unified subtyping rules.
Besides these differences other differences include the use of a call-by-name
semantics in \name (see also the discussion in Sections~\ref{sec:open-term-reduction} and \ref{sec:cast-design}),
and the use of the $\star : \square$ axiom in \name versus the use of $\star : \star$
in $\lambda_{I}$.

\paragraph{Dependent Types with Explicit Casts} Previously
discussed work is about the interaction between dependent types and
subtyping. However, the other problem is the
interaction between dependent types and recursion. For this
problem, a general solution that has recently emerged is the use
of type casts to control type-level computation. In such an approach explicit casts
are used for performing type-level computations. A motivation for
using type casts is to decouple strong normalization from the
proofs of metatheory, which also makes it possible to allow general
recursion. There have been several studies~\cite{guru,sjoberg:msfp12,
  kimmel:plpv, zombie:popl15, fc:kind, Doorn:2013hq,isotype} working
on using explicit casts instead of conversion rule in a dependently
typed system. In \name we adopt a simple formulation of casts based
on iso-types~\cite{isotype}, but we believe that more powerful notions
of casts could work too.

\paragraph{Dependent Object Types}

Dependent Object Types (DOT)~\cite{dot:dot,dot:path,dot:sound} is another
family of systems that discusses subtyping in a dependently typed setting.
Unlike the traditional dependent type systems that based on lambda calculus, DOT
embraces the idea of ``everything is an object'' and features the \emph{path-dependent types}.
The path-dependent type is a restricted form of general dependent type,
although allowing return types of functions
to mention their parameters, only member accessing operations are allowed for the ``depended value'',
and instead of all terms, only variable names can occur in the accessing path.
This restriction rules out the traditional problems in dependent type system
like handling type-level computation and allows DOT to focus more on the subtyping
aspect like reasoning about type bounds. Also, since they can still seperate the
concept of terms and types due to the restriction, the mutual dependency of
typing and subtyping is also not an issue.

DOT with Implicit Functions (DIF)~\cite{dif} is an interesting extension of DOT
that adds implicit functions to the DOT.
Since path-dependent types can encode parametric
polymorphism, adding implicit functions implies adding implicit polymorphism.
The treatment of implicit parameters in DIF is quite similar
to ICC~\cite{miquel2001implicit} in terms of the generalization and
instantiation rules shown in section \ref{sec:polymorphic-subtyping}.
So their system share similar restrictions of being unable to handle implicits
at higher-ranked position. However in DIF, implicit arguments are runtime relevant,
and can be retrieved by a specicial variable. This comes with a restriction of
only implicit arguments can only be variables in the typing context when inferred.

\paragraph{Refinement Types and Manifest Systems}

The \emph{manifest} systems~\cite{manifestcontracts} is one of the styles of
contract-oriented programming (in contrast to the \emph{latent} systems~\cite{latentcontracts}),
where contracts (the conditions that programmers expect to satisfy) are expressed
in the type system. $\lambda_H$~\cite{hybridtypes,manifestcontracts} is one of such systems that
includes dependent types and subtyping simultaneously. The subtyping relation
expresses the implication relation between contract satisfaction conditions. They
also observed the difficulty of potential mutual dependency between typing and subtyping,
they build another layer of denotational semantics on top of
subtyping rules to avoid that subtyping depends on typing. However it introduces
massive complications in terms of metatheory. System $\mathrm{F}_\mathrm{H}$~\cite{fh} and $\mathrm{F}_\mathrm{H}^\sigma$
~\cite{fhsigma} provide another
interesting idea to deal with this mutual dependency. They get rid of the
subtyping aspect entirely in the type system, but recover it after the system is
defined to prove the ideas expressed by subtyping hold for their systems,
which \citet{fhsigma} also called a \emph{subsumption free formulation}.
However such technique is likely to be difficult to apply for systems that
reason about implicit polymorphism,
as systems like ICC\cite{miquel2001implicit} that mentions subtyping relations \emph{post facto}
often fail to reason about polymorphism at a higher rank.

\section{Conclusion}

In this article, we presented a design of a dependently typed calculus called \name.
\name generalizes non-dependent polymorphic subtyping by Odersky and
L\"aufer~\citep{odersky1996putting} and contains other features like general
recursion and explicit casts for type-level computations.
We adopt the techniques of the Unified Subtyping~\citep{full} to
avoid the mutual dependency between typing and subtyping relation to simplify
the formalization. Besides other relevant theorems about typing and subtyping,
\emph{transitivity} and \emph{type safety} are proved mechanically with the Coq
proof assistant.

In the future, we will attempt to lift various restrictions that originally simplify
the metatheory, such as the kind restriction on polymorphic
types and the runtime irrelevance of implicit arguments. We would also like to
study the impact to the metatheory of adding $\top$ types to our language,
which is a common feature of a subtyping relation.
Most importantly, we consider the development
of a well-specified algorithmic system a major challenge in our future
work. The current formulation of \name is declarative due to the mono-expression guesses.

While \name still has some limitations, we believe that it already includes many
of the core features that are important for typed functional languages.
Assuming that we have an implementation of a core language based on \name,
we expect that interesting and expressive functional languages can be built
on top of such core language. For instance, all the features of Haskell 98,
including higher-kinds, algebraic datatypes and type classes~\citep{typeclasseskaes,typeclasseswadler}
should be easily encodable
in \name. Furthermore, some features not in Haskell 98, but available in modern
versions of GHC Haskell, such as higher-ranked polymorphism or certain
kinds of dependent types are also supported in \name. GADTs~\citep{gadt1,gadt2} and
type-families~\citep{typefamilies} are more challenging as they require a more powerful
form of casts and additional support for equality. Previous work on PITS
has shown how some forms of GADTs and equality can be modelled using
a variant of cast operators that employ a more powerful parallel reduction
relation. We believe that \name can also employ such variant of casts,
although this also remains future work.


% \bibliographystyle{plainurl}
\bibliographystyle{elsarticle-num-names}
\bibliography{reference,RW,RW2}

\end{document}
