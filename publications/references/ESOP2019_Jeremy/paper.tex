% This is samplepaper.tex, a sample chapter demonstrating the
% LLNCS macro package for Springer Computer Science proceedings;
% Version 2.20 of 2017/10/04
%
\documentclass[runningheads]{llncs}
\usepackage{llncsdoc}
%
\usepackage{graphicx}
% Used for displaying a sample figure. If possible, figure files should
% be included in EPS format.
%
% If you use the hyperref package, please uncomment the following line
% to display URLs in blue roman font according to Springer's eBook style:
% \renewcommand\UrlFont{\color{blue}\rmfamily}

%%%%%%%%%%%%%%%%%%%%%%%%%%%%%%%%%%%%%%%%%%%%%%%%%%%%%%%%%%%%%%%%%%%%%%%%
% Load up my personal packages
%%%%%%%%%%%%%%%%%%%%%%%%%%%%%%%%%%%%%%%%%%%%%%%%%%%%%%%%%%%%%%%%%%%%%%%%

\usepackage[numbers,sort]{natbib}
% Basics
\usepackage{fixltx2e}
\usepackage{url}
\usepackage{fancyvrb}
\usepackage{mdwlist}  % Miscellaneous list-related commands
\usepackage{bookmark}
\usepackage{xspace}
\usepackage{comment}
\usepackage{booktabs} % For formal tables

\usepackage[euler]{textgreek}

%% Math
\usepackage{amsmath}
% \usepackage{amsthm}
\usepackage{amssymb}
\usepackage{bm}       % Bold symbols in maths mode
% http://tex.stackexchange.com/questions/114151/how-do-i-reference-in-appendix-a-theorem-given-in-the-body
\usepackage{thmtools, thm-restate}

\usepackage{tikz}
\usetikzlibrary{matrix}
\usetikzlibrary{arrows,automata}
\usetikzlibrary{positioning}
\usetikzlibrary{shapes.geometric}


%% Theoretical computer science
\usepackage{stmaryrd}
\usepackage{mathtools}  % For "::=" ( \Coloneqq )

\usepackage{savesym}
\savesymbol{checkmark}
\usepackage{dingbat}

\usepackage{longtable}
\usepackage{subcaption}
\usepackage{paralist}
\usepackage{xcolor}
\usepackage{array}


% Code highlighting
\usepackage{listings}

\lstset{%
  % backgroundcolor=\color{white},
  basicstyle=\small\ttfamily,
  keywordstyle=\sffamily\bfseries,
  captionpos=none,
  columns=flexible,
  keepspaces=true,
  showspaces=false,               % show spaces adding particular underscores
  showstringspaces=false,         % underline spaces within strings
  showtabs=false,                 % show tabs within strings adding particular underscores
  breaklines=true,                % sets automatic line breaking
  breakatwhitespace=true,         % sets if automatic breaks should only happen at whitespace
  escapeinside={(*}{*)},
  literate={lam}{{$\lambda$}}1  {->}{{$\rightarrow$}}1 {Top}{{$\top$}}1 {o+o}{{$\oplus$}}1 {=>}{{$\Rightarrow$}}1 {/\\}{{$\Lambda$}}1,
  tabsize=2,
  commentstyle=\color{purple}\ttfamily,
  stringstyle=\color{red}\ttfamily,
  sensitive=false
}

\lstdefinelanguage{sedel}{
  keywords={self, Void, Class, All, extends, this, trait, inherits, super, type, Trait, override, new, if, then, else, let, in, letrec},
  identifierstyle=\color{black},
  morecomment=[l]{--},
  morecomment=[l]{//},
  morestring=[b]",
  xleftmargin  = 2mm,
  morestring=[b]'
}


\lstset{language=sedel}

% \theoremstyle{plain}
% \newtheorem{theorem}{Theorem}
% \newtheorem{lemma}{Lemma}
% \newtheorem{corollary}{Corollary}
% \newtheorem{proposition}{Proposition}
% \newtheorem{conjecture}{Conjecture}
% \theoremstyle{definition}
% \newtheorem{definition}{Definition}
% \newtheorem{example}{Example}
% \newtheorem{observation}{Observation}
% \theoremstyle{remark}
% \newtheorem*{remark}{Remark}

% General
\newcommand{\code}[1]{\texttt {#1}}
\newcommand{\highlight}[1]{\colorbox{yellow}{#1}}

% Logic
\newcommand{\turns}{\vdash}

% PL
\newcommand{\subst}[2]{\lbrack #1 / #2 \rbrack}
\newcommand{\concatOp}{+\kern-1.3ex+\kern0.8ex}  % http://tex.stackexchange.com/a/4195/73122

% Constructors
% \newcommand{\for}[2]{\forall #1. \, #2}
\newcommand{\lam}[2]{\lambda #1. \, #2}
\newcommand{\app}[2]{#1 \; #2}
\newcommand{\blam}[2]{\Lambda #1. #2}
\newcommand{\tapp}[2]{#1 \; #2}

\newcommand{\pair}[2]{\langle #1, #2 \rangle}
\newcommand{\inter}[2]{#1 \,\&\, #2}
\newcommand{\mer}[2]{#1 \, ,, \, #2}
\newcommand{\proj}[2]{{\code{proj}}_{#1} #2}
\newcommand{\ctx}[2]{#1\left\{#2\right\}}
\newcommand{\bra}[1]{\llbracket #1 \rrbracket}


\newcommand{\recordType}[2]{\{ #1 : #2 \}}
\newcommand{\recordCon}[2]{\{ #1 = #2 \}}

\newcommand{\ifThenElse}[3]{\code{if} \; #1 \; \code{then} \; #2 \; \code{else} \; #3}

\newcommand{\defeq}{\triangleq}

\newcommand{\logeq}[2]{#1 \backsimeq_{log} #2}
\newcommand{\kleq}[2]{#1 \backsimeq #2}
\newcommand{\ctxeq}[3]{#1 \backsimeq_{ctx} #2 : #3}

\newcommand{\stepn}{\longmapsto^*}
\newcommand{\step}{\longmapsto}


\newcommand{\name}{$\mathsf{NeColus}$\xspace}
\newcommand{\namee}{$\lambda_{i}^{+}$\xspace}
\newcommand\oname{$\lambda_{i}$\xspace}
\newcommand\fname{$\mathsf{F}_{i}$\xspace}
\newcommand\fnamee{$\mathsf{F}_{i}^{+}$\xspace}
\newcommand\tname{$\lambda_{co}$\xspace}
\newcommand\tnamee{$\mathsf{F}_{co}$\xspace}
\newcommand\sedel{$\mathsf{SEDEL}$\xspace}
\newcommand\visitor{\textsc{Visitor}s\xspace}
\newcommand\rname{$\lambda^{||}$\xspace}
\newcommand\fsub{$\mathsf{F}_{<:}$\xspace}
\newcommand\dname{$\lambda_{,,}$\xspace}
\newcommand\lname{$\lambda_{\land}^{\lor}$\xspace}

\newcommand{\cmark}{\ding{51}}%
\newcommand{\xmark}{\ding{55}}%



% Logical equivalence related macros
\newcommand{\valR}[2]{\mathcal{V}\bra{#1 ; #2}}
\newcommand{\valRR}[1]{\mathcal{V}\bra{#1}}
\newcommand{\eeR}[2]{\mathcal{E}\bra{#1 ; #2}}
\newcommand{\eeRR}[1]{\mathcal{E}\bra{#1}}
\newcommand{\ggR}[1]{\mathcal{G}\bra{#1}}
\newcommand{\ddR}[1]{\mathcal{D}\bra{#1}}

\newcommand{\hll}[2][gray!40]{\colorbox{#1}{#2}}
\newcommand{\hlmath}[2][gray!40]{%
  \colorbox{#1}{$\displaystyle#2$}}

\newcommand{\rulehl}[2][gray!40]{%
  \colorbox{#1}{$\displaystyle#2$}}



%%%%%%%%%%%%%%%%%%%%%%%%%%%%%%%%%%%%%%%%%%%%%%%%%%%%%%%%%%%%%%%%%%%%%%%%
% Hyperlinks
%%%%%%%%%%%%%%%%%%%%%%%%%%%%%%%%%%%%%%%%%%%%%%%%%%%%%%%%%%%%%%%%%%%%%%%%

\usepackage[capitalise]{cleveref}


%%%%%%%%%%%%%%%%%%%%%%%%%%%%%%%%%%%%%%%%%%%%%%%%%%%%%%%%%%%%%%%%%%%%%%%%
% Ott includes
%%%%%%%%%%%%%%%%%%%%%%%%%%%%%%%%%%%%%%%%%%%%%%%%%%%%%%%%%%%%%%%%%%%%%%%%

\usepackage{ottalt}
\inputott{ott-rules}
\renewcommand\ottaltinferrule[4]{
  \inferrule*[narrower=0.6,lab=#1,#2]
    {#3}
    {#4}
}


%%%%%%%%%%%%%%%%%%%%%%%%%%%%%%%%%%%%%%%%%%%%%%%%%%%%%%%%%%%%%%%%%%%%%%%%
% Revision tool
%%%%%%%%%%%%%%%%%%%%%%%%%%%%%%%%%%%%%%%%%%%%%%%%%%%%%%%%%%%%%%%%%%%%%%%%
% \newcommand\mynote[3]{\textcolor{#2}{#1: #3}}
% \newcommand\bruno[1]{\mynote{Bruno}{red}{#1}}
% \newcommand\tom[1]{\mynote{Tom}{blue}{#1}}
% \newcommand\ningning[1]{\mynote{Ningning}{orange}{#1}}
% \newcommand\jeremy[1]{\mynote{Jeremy}{purple}{#1}}



\begin{document}
%
\title{Distributive Disjoint Polymorphism for Compositional Programming}
%
\titlerunning{Distributive Disjoint Polymorphism}
% If the paper title is too long for the running head, you can set
% an abbreviated paper title here
%
\author{Xuan Bi\inst{1} \and
Ningning Xie\inst{1} \and
Bruno C. d. S. Oliveira\inst{1} \and
Tom Schrijvers\inst{2}}
%
\authorrunning{X.\,Bi et al.}
% First names are abbreviated in the running head.
% If there are more than two authors, 'et al.' is used.
%
\institute{The University of Hong Kong, Hong Kong, China \\
\email{\{xbi,nnxie,bruno\}@cs.hku.hk} \and
KU Leuven, Leuven, Belgium \\
\email{tom.schrijvers@cs.kuleuven.be}}
%
\maketitle              % typeset the header of the contribution
%
\begin{abstract}

\begin{comment}
Compositionality is a desirable property in programming
designs. Broadly defined, compositionality is the principle that a
system should be built by composing smaller subsystems.
Programming techniques such as \emph{shallow embeddings} of
Domain Specific Languages (DSLs),  \emph{finally tagless} or \emph{object algebras}
are built on the principle of compositionality.
However, programming languages often only support well simple
compositional designs, but language support for more sophisticated
compositional designs is lacking.

In this paper we present a calculus and polymorphic type system with \emph{(disjoint)
intersection types}, called \fnamee,
that supports a broader notion of compositional designs, and enables the
development of highly modular and reusable programs. \fnamee is a
generalization and extension of Alpuim et al. \fname calculus,
which proposed the idea of \emph{disjoint polymorphism}.
The main novelty of \fnamee is a novel subtyping algorithm with
distributivity laws on types. Distributivity plays a fundamental role
in improving compositional designs, by enabling the automatic
composition of multiple operations/interpretations. The main technical
challenge is the proof of coherence for \fnamee as impredicativity makes it
hard to develop a well-founded logical relation for coherence.
However, by restricting the system to predicative instantiations only
we are able to develop a suitable logical relation and show
coherence.
We illustrate the use of the calculus to model highly modular
interpretations of DSLs, that can be smoothly composed with
the merge operator of \fnamee. Furthermore, we provide a detailed
comparison between \emph{distributive disjoint polymorphism} and
\emph{row types}.
\end{comment}

Popular programming techniques such as \emph{shallow embeddings} of Domain
Specific Languages (DSLs), \emph{finally tagless} or \emph{object
  algebras} are built on the principle of \emph{compositionality}.  However,
existing programming languages only support simple compositional designs well,
and have limited support for more sophisticated ones.

This paper presents the \fnamee calculus, which supports
highly modular and compositional designs that
improve on existing techniques. These improvements are due to the
combination of three features: \emph{disjoint intersection
  types} with a \emph{merge operator}; \emph{parametric (disjoint)
  polymorphism}; and \emph{BCD-style distributive subtyping}.  The
main technical challenge is  \fnamee's proof of coherence. A
naive adaptation of ideas used in System F's \emph{parametricity} to
\emph{canonicity} (the logical relation used by \fnamee to prove coherence) results in an ill-founded logical relation. To solve the
problem our canonicity relation employs a different technique based on
immediate substitutions and a restriction to predicative
instantiations. Besides coherence, we show
several other important meta-theoretical results, such as type-safety, 
sound and complete algorithmic subtyping, and
decidability of the type system. Remarkably, unlike 
\fsub's \emph{bounded polymorphism}, disjoint polymorphism
in \fnamee supports decidable type-checking.


\begin{comment}
\fnamee  is a
generalization and extension of Alpuim et al. \fname calculus,
which proposed the idea of \emph{disjoint polymorphism}.
The main novelty of \fnamee is a novel polymorphic subtyping algorithm with
distributivity laws on types. Distributivity plays a fundamental role
in improving compositional designs, by enabling the automatic
composition of multiple operations/interpretations.
The main technical
challenge is the proof of coherence for \fnamee as impredicativity makes it
hard to develop a well-founded logical relation for coherence.
However, by restricting the system to predicative instantiations only
we are able to develop a suitable logical relation and show
coherence.
We illustrate the use of the calculus to model highly modular
interpretations of DSLs, that can be smoothly composed with
the merge operator of \fnamee. Furthermore, we provide a detailed
comparison between \emph{distributive disjoint polymorphism} and \emph{row types}.
\end{comment}

% \keywords{First keyword  \and Second keyword \and Another keyword.}
\end{abstract}
%
%
%
\section{Introduction}

A \emph{polymorphic subtyping} relation, which relates more general
types to more specific ones, is at the core of many modern functional
languages. Polymorphic subtyping enables a form of
\emph{(implicit) parametric polymorphism}, where type arguments to polymorphic
functions are automatically instantiated and the programmer does not specify them.
Traditionally, variants of polymorphic subtyping (in the form of a more-general-than relation)
have been used in functional languages based on the
Hindley-Milner~\citep{hindley1969principal,milner1978theory,damas1982principal}
type system, which supports full type-inference without any type annotations.
However, the Hindley-Milner type system only supports \emph{rank-1 (or first order)
polymorphism}, where all universal quantifiers only occur at the top-level
of a type.  Modern functional programming languages, such as Haskell, go beyond
Hindley-Milner and support \emph{higher-ranked polymorphism}~\citep{odersky1996putting,jones2007practical}
with a more expressive
polymorphic subtyping relation. With higher-ranked
polymorphism there is no restriction on where universal quantifiers can occur.

\citet{odersky1996putting} proposed a
simple declarative specification for polymorphic subtyping, which supports higher-ranked polymorphism.
Since then several
algorithms have been proposed that implement variants of this specification. Most
notably, the algorithm proposed by \citet{jones2007practical} forms the basis
for the implementation of type inference in the GHC compiler.
\citet{dunfield2013complete} (DK) provide an elegant
formalization of another sound and complete algorithm, which has
also inspired implementations of type-inference in some polymorphic
programming languages, such as PureScript~\citep{PureScript} or DDC~\citep{Disciple}.
More recently \citet{zhao19mechanical} have mechanized DK's type system in a theorem prover.

%some more background text

In recent years dependent
types~\citep{coc,cayenne,dep:pisigma,sjoberg:msfp12,guru,fc:kind,zombie:popl14,zombie:popl15}
have become a hot topic in programming
language research. Several newer
functional programming languages, most notably Agda~\citep{2007_norell_agda} and
Idris~\citep{brady2013idris}, are now dependently typed. Moreover a number of existing functional
languages, such as Haskell, have started to move towards dependently typed programming~\citep{dependenthaskell}. Dependent types naturally lead to a unification between types and terms, which enables both
additional \emph{expressiveness} and \emph{economy of concepts}.
The key enabler for unifying terms and types in dependently typed
calculi is the adoption of a style similar to
Pure Type Systems (PTSs)~\citep{pts}. In PTSs there is only a single level
of syntax for terms, i.e. the types (or kinds) are expressed using the
same syntax as the terms. This is in contrast with more traditional calculi, where
distinct pieces of syntax (terms, types and kinds) are separated.

Unified syntax, typical of dependently typed languages,
poses some challenges for language design and implementation.
A first challenge arises from the interaction between recursion and dependent types.
Essentially recursion breaks strong normalization, which
many common properties in dependently typed calculi depend upon. One of the most
typical properties among them is the decidability of type checking,
which simply cannot be guaranteed if some
type-level computations are non-terminating.
However, this area has been actively investigated in the
last few years, and a general approach~\citep{guru,sjoberg:msfp12, kimmel:plpv, zombie:popl15,
  isotype} based on explicit casts for type-level computations,
has emerged as an interesting solution for integrating general recursion
in dependently typed calculi. By avoiding the implicit type-level computation
entirely, whether programs strongly normalize or not no longer matters for the
decidability of type checking.
Current proposals for dependently typed versions of Haskell~\citep{dependenthaskell},
for instance, adopt explicit casts for type-level computation.

A second challenge, for calculi that employ subtyping, is that
smoothly integrating
dependent types and subtyping is difficult. Subtyping is a
substantial difference to traditional PTSs, which do not have such feature.
The issue with subtyping
is well summarized by \cite{subdep}:
\emph{``One thing that makes the study of these systems difficult is that
  with
dependent types, the typing and subtyping relations become intimately
tangled, which means that tested techniques of examining subtyping in
isolation no longer apply''}.
Recent work on \emph{unified subtyping}~\citep{full} provides a
simple technique to address this problem.
Following the same spirit as Pure Type Systems,
which attempt to unify syntax and the typing and well-formedness relations,
unified subtyping suggests unifying typing
and subtyping into a single relation. This solves the problem of dependencies
in that now there is only a single relation that depends only on itself. Furthermore
it results in a compact specification compared to a variant with multiple
independent relations.

In this paper we investigate how polymorphic subtyping can be
adapted into a dependently typed calculus with general recursion and
explicit casts for type-level computation. We employ unified subtyping
to address the issues of combining dependent types with subtyping.
The use of explicit casts
for type-level computation means that type equality is essentially
syntactic (or rather up-to $\alpha$-equivalence).
This avoids the use of a traditional conversion rule that allows concluding
$\beta$-equivalent types to be equal. In essence, the use of the conversion
rule requires (implicit) type-level computation, since terms have to be normalized
using $\beta$-reduction to conclude whether or not they are equal.
Dependent type systems with a conversion rule have some major complications.
% that arise in the presence of a conversion rule
A well-known one is that type-inference for such systems requires \emph{higher-order
  unification}, which is known to be \emph{undecidable}~\citep{goldfarb1981undecidability}.
By employing a system with $\alpha$-equivalence only we stay closer to existing
languages like Haskell, where type equality (at least at the core language level)
is also essentially only up-to $\alpha$-equivalence.

We present a calculus called \name, and show three main results in this paper:
\emph{transitivity of subtyping}, \emph{type soundness}, and \emph{completeness
of \name's polymorphic subtyping with respect to Odersky and L\"aufer's formulation}.
Transitivity is a non-trivial result (like in most calculi combining dependent types
and subtyping) and requires a proof based on sizes and a property that guarantees
the uniqueness of kinds in our language. Type soundness is also non-trivial and we need
to take a different approach than that employed by existing work on polymorphic
subtyping~\citep{odersky1996putting, jones2007practical}, where type-safety is shown by an
elaboration to System F. In essence elaboration into a target language
brings significant complications to the metatheory in a dependently typed setting.
Thus, instead of an elaboration, we use a direct operational semantics approach, which
is partly inspired by the approach used in the \emph{Implicit Calculus of Constructions} (ICC)~\citep{miquel2001implicit,barras2008implicit},
to prove type soundness.
Similarly to ICC we adopt the restriction that arguments for implicit function types
are computationally irrelevant (i.e. they cannot be used in runtime computation).
However, our unified subtyping setting is significantly
different from ICC due to the presence of subtyping,
which brings complications not in the ICC.
We also prove that any valid subtyping statement in the Odersky and L\"aufer relation
is valid in \name. Thus \name's unified subtyping subsumes the polymorphic subtyping
relation by Odersky and L\"aufer.

\name and all the proofs reported in this paper are formalized in the Coq theorem
prover~(\citeauthor{coqsite}).
%Moreover we have a simple prototype implementation of \name by adapting
%the algorithms from Zhao et al.~\citep{zhao19mechanical} for polymorphic subtyping.
%\bruno{Can the implementation run the examples we show?}
This paper
does not address decidability or soundness and completeness of \name to an
algorithmic formulation, which are outside of the scope of this work.
Nonetheless, these are important and challenging
questions for practical implementations of \name, which are left open for future work.

In summary, the contributions of this paper are:

\begin{itemize}

\item {\bf The \name calculus}, which is a dependently typed calculus with explicit casts,
  general recursion and implicit higher-ranked polymorphism.

\item {\bf Type-soundness and transitivity of subtyping.} We show that \name
  is type-sound and unified subtyping is transitive.

\item {\bf Subsumption of Odersky and L\"aufer's polymorphic subtyping.} We show that \name's
  unified subtyping can encode all valid poymorphic subtyping statements of Odersky and L\"aufer's
  relation.

\item {\bf Mechanical formalization.} All the results have been mechanically
  formalized in the Coq theorem prover. The formalization is available online at:
  \url{https://github.com/VinaLx/dependent-polymorphic-subtyping}.

\end{itemize}

\section{Overview}

In this section, we introduce \name by going through
some interesting examples to show the expressiveness and major features of the calculus.
Then we discuss the motivation, rationale of our design, and challenges.
The formal system of \name will be
discussed in Sections \ref{sec:system} and \ref{sec:metatheory}.

\subsection{A Tour of \name}
\label{sec:examples}

The \name calculus features a form of \emph{implicit
  higher-ranked polymorphism} with the power of dependent types. Thus the main feature of \name
is the ability to implicitly infer universally quantified arguments.

\paragraph{A First Example of Implicit Polymorphism}
Like most of functional languages, \name supports (implicit) parametric polymorphism.
A first simple example is the polymorphic identity
function:
\begin{flalign*}
&\mathrm{id} : \forall (A : \star).\, A \rightarrow A &&\\
&\mathrm{id} = \lambda (x : A).\, x &&\\
&\mathrm{answer} : \mathbb{N} &&\\
&\mathrm{answer} = \mathrm{id} ~ 42 \qquad \text{-- No type argument needed at the call of $\mathrm{id}$} &&
\end{flalign*}
\noindent The polymorphic parameter \verb|A| is annotated with its type,
which is $\star$. The type $\star$ is the type of types (also known as
\emph{kind}). In \name, the parameters of lambda abstractions are annotated
with their types, and the \verb|A| in the definition refers back to the
polymorphic parameter. In the examples below, we drop the parentheses around
variables and their type annotations such as $[[lambda x : A. x]]$ for conciseness.

Similar to implicit polymorphism in other languages,
the polymorphic parameters of the $\forall$ types are implicitly instantiated
during applications. Thus, in the call of the identity function, we
do not need to specify the argument used for instantiation. In contrast,
in an explicitly polymorphic language (such as System F) we would need
to call $id$ with an extra argument that specifies the instantiation of $A$:
$\mathrm{id}~\mathbb{N}~ 42$.

\paragraph{Recursion and Dependent Types}

\name is dependently typed, and universal quantifications are not limited to work
on arguments of type $\star$, but it allows arguments of other types. This is
a key difference compared to much of the work on type-inference for higher-ranked
polymorphism~\cite{dunfield2013complete,le2003ml,leijen2008hmf,vytiniotis2008fph,jones2007practical}
which has been focusing on System F-like
languages where universal quantification can only have arguments of type $\star$.
Furthermore, \name supports general recursion at both the term and the type-level.

Using these features we can encode an indexed list, a \verb|map| operation over it
and illustrate how the implicit instantiation allows us to use the \verb|map|
function conveniently.
However, because \name is just a core calculus there is no built-in support
yet for algebraic datatypes and pattern matching.
We expect that a source language would provide a more convenient
way to define the \verb|map| function using pattern matching and other useful source-level
constructs. To model algebraic datatypes and pattern matching in \name, we
use an encoding by Yang and Oliveira~\cite{yang2019pure},
which is based on Scott encodings~\cite{mogensen1992efficient}.
Since the details of that encoding are not relevant for this paper,
here we omit the code for most definitions and show only their types.

In a dependently typed language a programmer could write the following definition
for our formulation of indexed lists:
\newcommand{\Nat}[0]{\mathbb{N}}
\newcommand{\Succ}[0]{\mathrm{S}}
\newcommand{\Zero}[0]{\mathrm{Z}}
\newcommand{\List}[0]{\mathrm{List}}
\newcommand{\Nil}[0]{\mathrm{Nil}}
\newcommand{\Cons}[0]{\mathrm{Cons}}
\newcommand{\map}[0]{\mathrm{map}}
\begin{flalign*}
  & \mathbf{data} ~ \Nat = \Zero ~|~ \Succ~\Nat &&\\
  & \mathbf{data} ~ \List~(a : \star)~(n : \Nat) = \Nil ~ | ~ \Cons~a~(\List~a~(\Succ~n)) &&
\end{flalign*}
In \name we can encode the $\mathrm{List}$ and the constructors as conventional terms. We
show the definition for \verb|List|, and the types for the constructors next
(implementation omitted). In later subsequent examples we will just assume some
Haskell-style syntactic sugar for datatype definitions and constructors.
\begin{flalign*}
&\List : \star \rightarrow \Nat \rightarrow \star &&\\
&\List = \mu L : \star \rightarrow \Nat \rightarrow \star.\, \lambda a:\star.\, \lambda n :\Nat.\, \forall r:\star.\, r \rightarrow (a \rightarrow L~a~(\Succ ~ n) \rightarrow r) \rightarrow r &&\\
&\Nil : \forall a : \star.\, \forall n : \Nat.\, \List ~ a ~ n &&\\
&\Cons : \forall a : \star.\, \forall n : \Nat.\, a \rightarrow \List ~ a ~ (\Succ ~ n) \rightarrow \List ~ a ~n &&
\end{flalign*}
\noindent Then we can define a \verb|map| function over \verb|List| with the type:
\begin{flalign*}
  & \map : \forall a : \star.\, \forall b : \star.\, \forall n : \Nat .\, (a \rightarrow b) \rightarrow \List~a~n \rightarrow \List~b~n &&
\end{flalign*}
An example of application of \verb|map| is:
\begin{flalign*}
&\map~\Succ~(\Cons~1~(\Cons~2~\Nil)) &&
\end{flalign*}

\noindent which increases every natural in the list by one.
Note that since the type parameters for \verb|map|, \verb|Cons| and \verb|Nil|
are all implicit, they can be all omitted
and the arguments are instantiated implicitly. Thus the \verb|map| function
only requires two explicit arguments, making it as convenient to use
as in most functional language implementations.

There are a few final points worth mentioning about the example.
Firstly, \verb|List| is an example of a dependently typed function, since it is parametrized
by a natural value. Secondly, in \name (following the design of PITS~\cite{yang2019pure}),
fixpoint operators ($\mu$) serve a dual purpose of defining recursive types and recursive
functions. In \verb|List| the fixpoint operator is used to define a recursive type, whereas
in the definition of \verb|map| the fixpoint operator is used to define term-level recursion.
Moreover, recursion is unrestricted and there is no termination checking, much like approaches
such as Dependently Typed Haskell~\cite{dh}, and unlike various other dependently typed languages
such as Agda~\cite{2007_norell_agda} or Idris~\cite{brady2013idris}.


\paragraph{Implicit Higher-Kinded Types}

The implicit capabilities also extend to the realm of higher-kinded types~\cite{tapl}.
For example we can define a \verb|Functor| datatype as:
\newcommand{\Functor}[0]{\mathrm{Functor}}
\newcommand{\MkFunctor}[0]{\mathrm{MkF}}
\newcommand{\Id}[0]{\mathrm{Id}}
\newcommand{\MkId}[0]{\mathrm{MkId}}
\newcommand{\fmap}[0]{fmap}
\begin{flalign*}
  &\mathbf{data}~\Functor~(F : \star \rightarrow \star) = \MkFunctor~\{\fmap : \forall a : \star.\, \forall b : \star.\, (a \rightarrow b) \rightarrow F~a \rightarrow F~b\} &&
\end{flalign*}
In this case the type of $fmap$ is:
\begin{flalign*}
  & \fmap : \forall F : \star \to \star.~\Functor~F \to \forall a : \star.\, \forall b : \star.\, (a \rightarrow b) \rightarrow F~a \rightarrow F~b &&
\end{flalign*}
Importantly this example illustrates that universal variables can quantify over higher-kinds (i.e.
$F : \star \to \star$).
We can define instances of functor in a standard way:
\begin{flalign*}
  & \mathbf{data}~\mathrm{Id}~a=\MkId~\{runId : a\} && \\
  & \mathrm{idFunctor} : \Functor~\Id && \\
  & \mathrm{idFunctor} = \MkFunctor~\{\fmap = \lambda f : a \rightarrow b.\, \lambda ~x : \Id~a.\, \MkId~(f~(runId~x))\} &&
\end{flalign*}
and then use \verb|fmap| with three arguments:
\begin{flalign*}
& \fmap~\mathrm{idFunctor}~\Succ~(\MkId~0) &&
\end{flalign*}

\noindent Note that, because our calculus has no mechanism like type classes we pass the ``instance'' explicitly.
Nonetheless, three other arguments (the $F$, $a$, and $b$) are implicitly instantiated.

\paragraph{Higher-Ranked Polymorphic Subtyping}
\label{sec:higher-ranked-poly}

In calculi such as the ICC~\cite{miquel2001implicit}, a form of implicit instantiation also exists.
However, such calculi do not employ subtyping, instead, they only apply instantiation
to top-level universal quantifiers. Our next example illustrates how subtyping enables
instantiation to be applied also in nested universal quantifiers, thus enabling
more types to be related.

When programming with continuations~\cite{sussman1998scheme} one of the
functions that are typically needed is call-with-current-continutation
(\verb|callcc|). In a polymorphic language, there are several types that can be
assigned to \verb|callcc|. One of these types is a rank-3 type,
while another one is a rank-1 type.
Using polymorphic subtyping we can show that the rank-3
type is more general than the rank-1 type. Thus the following program type-checks:
\begin{flalign*}
& \mathrm{callcc}' : \forall a : \star.\, ((\forall b : \star.\, a \rightarrow b) \rightarrow a) \rightarrow a && \\
& \mathrm{callcc} : \forall a : \star.\, \forall b : \star.\, ((a \rightarrow b) \rightarrow a) \rightarrow a && \\
& \mathrm{callcc} = \mathrm{callcc}' &&
\end{flalign*}
\noindent The type $\forall b : \star.\, a \rightarrow b$ appears in a positive position
of the whole signature, and it is a more general signature than $a \rightarrow b$
for an arbitrary choice of $b$. Our language captures this subtyping relation so that
we can assign $\mathrm{callcc}'$ to $\mathrm{callcc}$ (but not the other way around).
In contrast, in approaches like the ICC, the types of \verb|callcc| and \verb|callcc'|
are not compatible and the example above would be rejected.

\subsection{Key Features}

We briefly discuss the major features of \name itself and
its formalization. A more formal and technical discussion will be left to
Sections \ref{sec:system} and \ref{sec:metatheory}.

\paragraph{Polymorphic Subtyping Relation}
Figure \ref{fig:polymorphic-subtyping} shows Odersky and L\"aufer polymorphic
subtyping relation~\cite{odersky1996putting}.
This relation captures a \emph{more-general-than} relation between
types as a subtyping relation. The key rules in their
subtyping relation are rules $\forallL$ and $\forallR$:

\begin{itemize}
  \item In rule $\forallL$, a polytype ($\forall x.\, A$) is considered \emph{more-general}
        than another type ($B$), when we can find an arbitrary monotype ($\tau$)
        so that the instantiation is more general than $B$.
        Importantly note that this relation does not guess arbitrary (poly)types,
        but just monotypes. In other words, the relation is \emph{predicative}~\cite{Martin-Lof-1972}.
        This restriction ensures that the relation is \emph{decidable}.

  \item In rule  $\forallR$ a type ($A$) is considered more general than a polytype ($\forall x. B$)
        when it is still more general than the head of the polytype, with the type
        parameter instantiated by an abstract variable $x$.
\end{itemize}

This subtyping relation sets a scene for our work, which
generalizes this relation to a dependently typed setting.

\begin{figure}
\centering

\begin{drulepar}{$\Gamma \vdash A \le B$}{Polymorphic Subtyping}
  \inferrule*[right=$\tau$]
    { }
    {\Gamma \vdash \tau \le \tau}
  \and
  \inferrule*[right=$\rightarrow$]
    {\Gamma \vdash B_1 \le A_1 \\ \Gamma \vdash A_2 \le B_2}
    {\Gamma \vdash A_1 \rightarrow A_2 \le B_1 \rightarrow B_2}
  \\
  \inferrule*[right=$\forallL$]
    {\Gamma \vdash \tau \\ \Gamma \vdash [\tau / x]\, A \le B}
    {\Gamma \vdash \forall x.\, A \le B}
  \and
  \inferrule*[right=$\forallR$]
    {\Gamma ,\, x \vdash A \le B}
    {\Gamma \vdash A \le \forall x.\, B}
\end{drulepar}

\caption{The polymorphic subtyping relation by Odersky and L\"aufer~\cite{odersky1996putting}.}
\label{fig:polymorphic-subtyping}
\end{figure}

\paragraph{Generalizing Polymorphic Subtyping}
\label{sec:polymorphic-subtyping}

The parameters of universal types can only be types in the polymorphic
subtyping relation by Odersky and L\"aufer.
In \name, we generalize the polymorphic parameters so that they can
be values or other kinds of types as well.
The idea of a direct generalization is:

\begin{mathpar}
  \inferrule*[right=$\forallL'$]
    {\Gamma \vdash \tau \rulehl{: A} \\ \Gamma \vdash [\tau / x]\, B \le C}
    {\Gamma \vdash \forall x \rulehl{: A}.\, B \le C}
  \and
  \inferrule*[right=$\forallR'$]
    {\Gamma ,\, x \rulehl{: B} \vdash A \le C}
    {\Gamma \vdash A \le \forall x \rulehl{: B}.\,C}
\end{mathpar}

\noindent The parameters for universal types can have any type (and not just $\star$).
Hence, instead of requiring the monotype $\tau$ to be a well-formed type in rule
$\forallL$, in rule $\forallL'$ it is
required that $\tau$ is well-typed regarding the type of the parameter
in the universal quantifier.
Similarly, for rule $\forallR'$ the context for the subtyping rule should include typing information
for the universally quantified variable.
However, this idea introduces the issue of potential mutual dependency between
subtyping and typing judgements, so further adjustments have to be made to formalize
this idea, which is discussed later in this section and section
\ref{sec:type-system} and \ref{sec:adaptation}.

\paragraph{Higher-Ranked Polymorphic Subtyping}

As the \verb|callcc| example in Section \ref{sec:higher-ranked-poly} shows, these subtyping
rules based on polymorphic subtyping, combined with other subtyping rules,
are able to handle the subtyping relations that occur at not only top-level,
but also at a higher-ranked level. This feature distinguishes our \name from the
Implicit Calculus of Constructions (ICC) \cite{miquel2001implicit} which also talks about
the implicit polymorphism of dependent type languages. The ICC features the two related rules
in the \emph{typing relation}:

\begin{mathpar}
  \inferrule*[lab=inst]
    {[[G |- e : forall x : A. B]] \\ [[G |- e1 : A]]}
    {[[G |- e : [e1 / x] B]]}
  \and
  \inferrule*[lab=gen]
    {[[G, x : A |- e : B]] \\ [[G |- forall x : A. B : k]]}
    {[[G |- e : forall x : A. B]]}
\end{mathpar}

\noindent Without an explicit subtyping relation, the ICC is not always able to handle subtyping
at higher-ranked positions. The approach taken by the ICC is similar to
Hindley-Milner type system~\cite{hindley1969principal,milner1978theory},
which also features similar rules in typing. But Hindley-Milner is designed for
dealing only with rank-1 polymorphism.
In generalizations of higher-ranked polymorphic type-inference~\cite{dunfield2013complete,le2003ml,leijen2008hmf,vytiniotis2008fph,jones2007practical},
it has been shown that rules like $\forallL$ and $\forallR$ generalize rules like
\textsc{GEN} and \textsc{INST}. Since we aim at higher-ranked polymorphic generalization,
we follow a similar, more general, approach in \name.

\paragraph{Unified Subtyping}
However, the revised subtyping relation with $\forallL'$ and $\forallR'$ rules suffers from an
important complication compared to the Odersky and L\"aufer formulation: there is now
a notorious mutual dependency between typing and subtyping.
In Odersky and L\"aufer's rules, the subtyping rules
do not depend on typing. In particular
the rule $\forallL$ depends only on well-formedness ($\Gamma \vdash \tau$).
In contrast, note that rule $\forallL'$ now mentions the typing relation
in its premise ($\Gamma \vdash \tau : A$). Moreover, as usual,
the subsumption rule of
the typing relation (shown below) depends on the subtyping relation.
\begin{mathpar}
  \inferrule*[right=t-sub]
    {\Gamma \vdash e : A \\ \Gamma \vdash A \le B}
    {\Gamma \vdash e : B}
\end{mathpar}
This mutual dependency problem has been a significant
problem when combining subtyping and dependent types~\cite{subdep, hutchins},
and presents itself on our way to the direct generalization of polymorphic subtyping.

To tackle this issue, we adopt a technique called
\emph{unified subtyping}~\cite{full}. Unified subtyping merges the typing relation and
subtyping relation into a single relation to avoid the mutual dependency:
\begin{mathpar}
  \Gamma \vdash e_1 \le e_2 : A
\end{mathpar}
The interpretation of this judgement is: under context $\Gamma$, $e_1$ is a subtype
of $e_2$ and they both are of type $A$. The judgements for subtyping and typing
are both special form of unified subtyping: % with the involvement of kind $[[*]]$:
\begin{mathpar}
  \Gamma \vdash A \le B \triangleq \Gamma \vdash A \le B : [[*]]
  \and
  \Gamma \vdash e : A \triangleq \Gamma \vdash e \le e : A
\end{mathpar}
The technique simplifies the formalization of dependently typed calculi with subtyping,
and especially the proof of transitivity in the original work. Ideally after applying the technique,
the generalization of the polymorphic subtyping should be:

\begin{mathpar}
  \inferrule*[right=$\le\forall'' L$]
    {\Gamma \vdash \tau : A \\ \Gamma \vdash [\tau / x]\, B \le C \rulehl{: [[*]]}}
    {\Gamma \vdash \forall x : A.\, B \le C \rulehl{: [[*]]}}
  \and
  \inferrule*[right=$\le\forall'' R$]
    {\Gamma ,\, x : B \vdash A \le C \rulehl{: [[*]]}}
    {\Gamma \vdash A \le \forall x : B.\,C \rulehl{: [[*]]}}
\end{mathpar}

\noindent The basic idea of our own formalization essentially follows a similar design,
although the actual rules in \name are slightly more sophisticated.
The details will be discussed in Section \ref{sec:type-system}.

\paragraph{``Explicit'' Implicit Instantiation}

With polymorphic subtyping the instantiation of universally quantified type
parameters is done implicitly instead of being manually applied. In non-dependently
typed systems, \emph{implicit} parameters are types (i.e. terms are not involved in
implicit instantiation). For example:
\begin{mathpar}
  (\lambda x.\, x)~42 \longrightarrow 42
\end{mathpar}
\noindent Here $\lambda x.\, x$ has type $\forall A.\, A \rightarrow A$. The implicit
instantiation is not reflected anywhere at term level. In a
language with type annotations on parameters of binders
(for example lambda abstractions), in order
for the polymorphic variables to be well-scoped at type annotations of terms,
another binder $\Lambda$ is added for terms. Nonetheless, instantiations are still
implicit as shown in the following example:
\begin{mathpar}
  (\Lambda A : [[*]].\, \lambda x : A.\, x) ~ 42 \longrightarrow 42
\end{mathpar}
Here $\Lambda A : [[*]].\, \lambda x : A.\, x$ has type $\forall A : \star. \, A \rightarrow A$,
and the polymorphic parameter $A$ is explicitly stated in the polymorphic
term. However as the reduction shows, the instantiations are still implicit.
We purposely omitted the explicit binders for implicit parameters for all the examples
in Section \ref{sec:examples} for conciseness. Such explicit binders can
be recovered with a simple form of syntactic sugar:

\[e : \forall x : A.\, B \triangleq \Lambda x : A.\, e : \forall x : A.\, B\]

\noindent When polymorphic parameters are used, \name provides a binder ($\Lambda x : A.\, e$)
to ensure that the parameters are well-scoped at the term-level.

\paragraph{Computational Irrelevance}
\label{sec:computational-irrelevance-overview}

Implicit parameters in traditional languages with polymorphic subtyping,
the ICC~\cite{miquel2001implicit,barras2008implicit} and \name are computationaly irrelevant.
In traditional (non-dependently) typed languages, types cannot affect computation,
thus computational irrelevance is quite natural and widely adopted.
In dependently typed systems, however, there can be some programs where
it is useful to have computationaly relevant implicit parameters.
For example, accessing the length of a length indexed vector in constant time:
\begin{flalign*}
  &\mathrm{length} : \forall n : [[int]].\, \mathrm{Vector}~n \rightarrow [[int]] &&\\
  &\mathrm{length} = \Lambda n : [[int]].\, \lambda v : \mathrm{Vector}~n.\, n
\end{flalign*}
\noindent Here the implicit parameter $n$ is computationally relevant as it is used as
the return value of the function which is likely to be executed at runtime.
Languages like Agda, Coq or Idris support such programs. However,
computationaly relevant implicit parameters are challenging for proofs of
type soundness. Due to such challenges (see also the discussion in
Section~\ref{subsec:semantics}),
the ICC has a restriction that parameters for implicit function types
must be computationally irrelevant. Since we adopt a similar technique,
we also have a similar restriction and thus cannot encode programs such
as the above.

\paragraph{Type-level Computation and Casts}
\name features a fixpoint that supports general recursion at both
type and term level. In order to avoid diverging computations at type checking,
we do not provide the congruence rule like other dependently
typed systems such as the Calculus of Constructions~\cite{coc}
to support implicit type-level reduction.
\begin{mathpar}
  \inferrule*[lab=Cong]
    {[[G |- e : A]] \\ A =_\beta B}
    {[[G |- e : B]]}
\end{mathpar}
Instead, we partially adopt the
call-by-name design of \emph{Pure Iso-Type Systems} (PITS)~\cite{isotype,yang2019pure},
and provide operators $\castdn$ and $\castup$ to explicitly trigger one-step
type reductions or expansions as shown in the typing rules below.
\begin{mathpar}
  \inferrule*[lab=Castup]
    {[[G |- e : B]] \\ [[A --> B]] \\ [[G |- A : k]]}
    {[[G |- castup [A] e : A]]}
  \and
  \inferrule*[lab=Castdn]
    {[[G |- e : A]] \\ [[A --> B]] \\ [[G |- B : k]]}
    {[[G |- castdn e : B]]}
\end{mathpar}


%%%%%%%%%%%%%%%%%%%%%%%%%%%%%%%%%%%%%%%%%%%%%%%%%%%%%%%%%%%%%%%%%%%%%%%%
\section{Semantics of the \fnamee Calculus}
\label{sec:typesystem}
%%%%%%%%%%%%%%%%%%%%%%%%%%%%%%%%%%%%%%%%%%%%%%%%%%%%%%%%%%%%%%%%%%%%%%%%

% \bruno{Regarding text about differences to \fname. I think that maybe
%   we can make a comparision in Related Work, rather than the technical
% section. That will not disrupt the flow of the technical section, and
% will be useful to add weight to the related work section.}

This section gives a formal account of \fnamee, the first typed calculus combining
disjoint polymorphism~\cite{alpuimdisjoint} (and disjoint intersection
types) with BCD
subtyping~\cite{Barendregt_1983}. The main differences to \fname are in the
subtyping, well-formedness and disjointness relations. \fnamee adds
BCD subtyping and unrestricted intersections, and also closes an open problem of
\fname by including the bottom type.
% through appropriate adjustments to the disjointness relation.
% \fnamee is a variant of Levant's
% predicative System F~\cite{leivant1991finitely}. \jeremy{Do we really need to
%   cite this? Leivant's System F has finitely stratified levels} The choice of
% predicativity is due to the coherence proof, which is
% discussed in \cref{sec:coherence:poly}.
The dynamic semantics
of \fnamee is given by elaboration to the target calculus \tnamee---a variant of
System F extended with products and explicit coercions.
% \jeremy{Should we summaries the differences from \fname here? }

% \subsection{Motivation}
% \label{sec:poly:motivation}

% Parametric polymorphism~\cite{reynolds1983types} is a well-beloved (and
% well-studied) programming feature. It enables a single piece of code to be
% reused on data of different types. So it is quite natural and theoretically
% interesting to study combining parametric polymorphism with disjoint
% intersection types, especially how it affects disjointness and coherence. On a
% more pragmatic note, the combination of parametric polymorphism and disjoint
% intersection types also reveals new insights into practical applications.
% Dynamically typed languages (such as JavaScript) usually embrace quite flexible
% mechanisms for class/object composition, e.g., mixin composition where objects
% can be composed at run time, and their structures are not necessarily statically
% known. The use of intersection types in TypeScript is inspired by such flexible
% programming patterns. For example, an important use of intersection types in
% TypeScript is the following function for mixin composition:
% \begin{lstlisting}[language=JavaScript]
% function extend<T, U>(first: T, second : U) : T & U {...}
% \end{lstlisting}
% which is analogous to our merge operator in that it takes two objects and
% produces an object with the intersection of the types of the argument objects.
% However, the types of the two objects are not known, i.e., they are generic. An
% important point is that, while it is possible to define such function in
% TypeScript (albeit using some low-level (and type-unsafe) features of
% JavaScript), it can also cause, as pointed out by \cite{alpuimdisjoint},
% run-time type errors! Clearly a well-defined meaning for intersection types with
% type variables is needed.


% \paragraph{Disjoint polymorphism.}

% Motivated by the above two points, \cite{alpuimdisjoint} proposed disjoint
% polymorphism, a variant of parametric polymorphism. The main novelty is
% \emph{disjoint (universal) quantification} of the form $[[ \ X ** A . B ]]$.
% Inspired by bounded quantification~\cite{cardelli1994extension} where a type
% variable is constrained by a type bound, disjoint quantification allows type
% variables to be associated with \emph{disjointness constraints}.
% Correspondingly, a term construct of the form $[[ \ X ** A. ee ]]$ is used to
% create values of disjoint quantification. We have seen some examples of disjoint
% polymorphism at work in \cref{bg:sec:disjoint_poly}. With disjointness constraints
% and a built-in merge operator, a type-safe and conflict-free \lstinline{extend}
% function can be naturally defined as follows:
% \begin{lstlisting}
% extend T [U * T] (first : T) (second : U) : T & U = first ,, second;
% \end{lstlisting}
% The disjointness constraint on the type variable \lstinline{U} ensures that no
% conflicts can occur when composing two objects, which is quite similar to
% the trait model~\cite{scharli2003traits} in object-orientated programming.
% We shall devote a whole chapter (\cref{chap:traits}) to further this point.


% \paragraph{Adding BCD subtyping.}

% While \cite{alpuimdisjoint} studied the combination of disjoint intersection
% types and parametric polymorphism, they follow the then-standard approach
% of \cite{oliveira2016disjoint} to ensure coherence, thus excluding the
% possibility of adding BCD subtyping. The combination of BCD subtyping and
% disjoint polymorphism opens doors for more expressiveness. For example, we can
% define the following function
% \begin{lstlisting}
% combine A [B * A] (f : {foo : Int -> A})
%                   (g : {foo : Int -> B}) : {foo : Int -> A & B} = f ,, g;
% \end{lstlisting}
% which ``combines'' two singleton records with parts of types unknown and returns
% another singleton record containing an intersection type. A variant of this
% function plays a fundamental role in defining Object Algebra combinators (cf.
% \cref{chap:case_study}).



\subsection{Syntax and Semantics}

\begin{figure}[t]
  \centering
\begin{tabular}{llll} \toprule
  Types & $[[A]], [[B]], [[C]]$ & $\Coloneqq$ & $[[nat]] \mid [[Top]] \mid [[Bot]] \mid [[A -> B]]  \mid [[A & B]] \mid [[{l : A}]] \mid [[X]] \mid [[\ X ** A . B]] $\\
  % Monotypes & $[[t]]$ & $\Coloneqq$ & $[[nat]] \mid [[Top]] \mid [[Bot]] \mid [[t1 -> t2]]  \mid [[t1 & t2]] \mid [[X]] \mid [[{l : t}]]$\\
  Expressions & $[[ee]]$ & $\Coloneqq$ & $[[x]] \mid [[ii]] \mid [[Top]] \mid [[\x . ee]] \mid [[ee1 ee2]] \mid [[ ee1 ,, ee2 ]] \mid [[ ee : A ]] \mid [[{l = ee}]] \mid [[ ee.l  ]] $ \\
        & & $\mid$ & $  [[\X ** A . ee]] \mid [[ ee A ]] $ \\
  Term contexts & $[[GG]]$ & $\Coloneqq$ &  $[[empty]] \mid [[GG , x : A]] $  \\
  Type contexts & $[[DD]] $ & $\Coloneqq$ & $ [[empty]] \mid [[DD , X ** A]] $  \\ \bottomrule
\end{tabular}
  \caption{Syntax of \fnamee}
  \label{fig:syntax:fi}
\end{figure}


\Cref{fig:syntax:fi} shows the syntax of \fnamee.
% For brevity of the
% meta-theoretic study, we do not consider primitive operations on primitive
% types. They can be easily added to the language, and our prototype
% implementation is indeed equipped with common primitive types and their
% operations.
Metavariables $[[A]], [[B]], [[C]]$ range over types. Types include
standard constructs from prior work~\cite{oliveira2016disjoint, alpuimdisjoint}:
integers $[[nat]]$, the top type $[[Top]]$, arrows $[[A -> B]]$, intersections $[[A & B]]$, single-field record types
$[[{l : A}]]$ and disjoint quantification $[[ \X ** A . B ]]$. One
novelty in \fnamee is the addition of the uninhabited bottom type $[[Bot]]$.
% Monotypes $[[t]]$ are the same, minus the disjoint quantification.
Metavariable $[[ee]]$ ranges over expressions. Expressions are
integer literals $[[ii]]$, the top value $[[Top]]$,
lambda abstractions $[[ \x . ee]]$, applications $[[ee1 ee2]]$, merges
$[[ee1 ,, ee2]]$, annotated terms $[[ ee : A ]]$, single-field records
$[[ {l = ee} ]]$, record projections $[[ ee . l ]]$, type abstractions
$[[ \X ** A . ee ]]$ and type applications $[[ee A]]$.
% We sometimes (ab)use $[[\X . A]]$ to abbreviate $[[ \X ** Top. A  ]]$, similarly for $[[ \ X . ee ]]$.

%To support
%polymorphism, we inherit from \fname two constructs: type abstractions $[[ \X ** A . ee ]]$
%of type $[[ \X ** A . B ]]$, and type applications $[[ee A]]$.

% We use $[[ A [X ~> B]  ]]$ to denote capture-avoiding substitution.

% \paragraph{Contexts.}

% In the traditional formulation of System F, there is a single context for
% tracking both type and term variables. Here we use another style of
% presentation~\cite[chap.~16]{Harper_2016} where contexts are split into
% \emph{term contexts} $[[GG]]$ and \emph{type contexts} $[[DD]]$. The former
% track term variables $[[x]]$ with their types $[[A]]$; and the latter track type
% variables $[[X]]$ with their disjointness constraints $[[A]]$. This formulation
% also turns out to be convenient for the presentation of logical relations in
% \cref{sec:coherence:poly}.

% \begin{figure}[t]
%   \begin{small}
%   \drules[swfte]{$[[ ||- DD]]$}{Well-formedness of type context}{empty, var}
%   \drules[swfe]{$[[DD ||- GG]]$}{Well-formedness of value context}{empty, var}
%   \drules[swft]{$[[DD |- A]]$}{Well-formedness of type}{top, bot, nat, var, arrow, all, and, rcd}
%   \end{small}
%   \centering
%   \caption{Well-formedness of types}
%   \label{fig:well-formedness:fi}
% \end{figure}


% \paragraph{Disjoint polymorphism.}
% \bruno{Will this be redundant with the text about disjoint
%   polymorphism in Section 2? Try to unify the the 2 texts perhaps and
%   write everything in Section 2?}
% Disjoint polymorphism~\cite{alpuimdisjoint} was recently proposed to combine
% (disjoint) intersection types with parametric polymorphism. The main novelty is
% \emph{disjoint quantification} $[[ \ X ** A . B ]]$.
% Inspired by bounded quantification~\cite{cardelli1994extension} where a type
% variable is constrained by a type bound, disjoint quantification associates type
% variables with \emph{disjointness constraints}.
% A term construct $[[ \ X ** A. ee ]]$ is used to
% create values. With disjointness quantification,
% it is possible to write a generic and conflict-free \lstinline{merge} function
% as follows:
% \begin{lstlisting}
% merge T [U * T] (first : T) (second : U) : T & U = first ,, second;
% \end{lstlisting}
% which allows calling \lstinline{merge Int Char 1 'a'} for example, but rejects
% \lstinline{merge Int Int 1 2} due to ambiguity. The type system ensures that no
% conflicts can occur when composing two objects. This is quite similar to the
% trait model~\cite{scharli2003traits} in object-oriented programming.


\paragraph{Well-formedness and unrestricted intersections.}

\renewcommand\ottaltinferrule[4]{
  \inferrule*[narrower=0.8,right=#1,#2]
    {#3}
    {#4}
}


\fnamee's well-formedness judgment of types $[[ DD |- A ]]$ is
standard, and only enforces well-scoping. This
is one of the key differences from \fname, which uses
well-formedness to also ensure that all intersection types are disjoint. 
In other words, while in \fname all valid intersection types must be
disjoint, in \fnamee unrestricted intersection types such as
$[[nat & nat]]$ are allowed.
More specifically,
the well-formedness of intersection types in \fnamee and \fname is:
{\small
\begin{mathpar}
  \ottaltinferrule{wf-\fnamee}{}{ [[DD |- A]] \\ [[DD |- B]]  }{ [[ DD |- A & B ]] } \and
  \ottaltinferrule{wf-\fname}{}{ [[DD |- A]] \\ [[DD |- B]] \\ \hlmath{[[DD |- A ** B]]}  }{ [[ DD |- A & B ]] }
\end{mathpar}
}%
Notice that \fname has an extra disjointness condition $[[ DD |- A ** B ]]$ in the premise.
This is crucial for \fname's syntactic method for proving coherence,
but also burdens the calculus with various syntactic restrictions and
complicates its metatheory. For example, it requires extra effort to
show that \fname only produces disjoint intersection types. As a consequence,
\fname features a \emph{weaker} substitution lemma (note the gray
part in \cref{lemma:weaker_lemma}) than \fnamee (\cref{lemma:general_lemma}).

\renewcommand\ottaltinferrule[4]{
  \inferrule*[narrower=0.6,lab=#1,#2]
    {#3}
    {#4}
}


\begin{proposition}[Type substitution in \fname] \label{lemma:weaker_lemma}
  If $[[DD |- A]]$, $[[DD |- B]]$, $[[ (X ** C) in DD  ]]$, $\hlmath{[[ DD |- B ** C ]]}$
  and well-formed context $[[  DD [ X ~> B ]   ]]$, then $[[   DD [ X ~> B ] |-  A [ X ~> B  ]     ]]$.
\end{proposition}

\begin{figure}[t]
  \centering
    \drules[S]{$ [[A <: B ~~> c]]  $}{Declarative subtyping}{refl,trans,top,rcd,andl,andr,arr,and,distArr,topArr,distRcd,topRcd,bot,forall,topAll,distAll}
  \caption{Declarative subtyping}
  \label{fig:subtype_decl:fi}
\end{figure}


\begin{lemma}[Type substitution in \fnamee] \label{lemma:general_lemma}
  If $[[DD |- A]]$, $[[DD |- B]]$, $[[ (X ** C) in DD  ]]$
  and well-formed context $[[  DD [ X ~> B ]   ]]$, then $[[   DD [ X ~> B ] |-  A [ X ~> B  ]     ]]$.
\end{lemma}
% \begin{proof}
%   By induction on the derivation of $[[ DD |- A  ]]$.
% \end{proof}


\paragraph{Declarative subtyping.}

\fnamee's subtyping judgment is another major difference to \fname, because it
features BCD-style subtyping and a rule for the bottom type.
The full set of subtyping rules are shown in
\cref{fig:subtype_decl:fi}. The reader is advised to ignore the gray parts for
now. Our subtyping rules extend the BCD-style subtyping rules from
\namee~\cite{bi_et_al:LIPIcs:2018:9227} with a rule for parametric
(disjoint) polymorphism (\rref{S-forall}). Moreover, we have three new rules:
\rref{S-bot} for the bottom type, and \rref{S-distAll,S-topAll} for distributivity of
disjoint quantification. The subtyping relation is a partial order
(\rref{S-refl,S-trans}). Most of the rules are quite standard. $[[Bot]]$ is a subtype of all types (\rref{S-bot}).
Subtyping of disjoint quantification is covariant in its body, and
contravariant in its disjointness constraints (\rref{S-forall}). Of particular interest are those
so-called ``distributivity'' rules: \rref{S-distArr} says intersections
distribute over arrows; \rref{S-distRcd} says intersections distribute over
records. Similarly, \rref{S-distAll} dictates that
intersections may distribute over disjoint quantifiers.
%It should be noted that \fnamee features explicit polymorphism, so type
%variables are neutral to subtyping, i.e., $[[X <: X]]$, which is already
%contained in \rref{S-refl}.

% \begin{remark}
%   In our Coq formalization, we require that the two types $[[A]]$ and $[[B]]$ are
%   well-formed relative to some type context, resulting in the subtyping
%   judgment $[[DD |- A <: B]]$. But this is not very important
%   for the purpose of presentation, thus we omit contexts.
% \end{remark}

\paragraph{Typing rules.}

\begin{figure}[t]
  \centering
    \drules[T]{$[[DD; GG |- ee => A ~~> e]]$}{Inference}{top, nat, var, app, merge, anno, rcd, proj, tabs, tapp}
    \drules[T]{$[[DD ; GG |- ee <= A ~~> e]]$}{Checking}{abs, sub}
  \caption{Bidirectional type system}
  \label{fig:typing:fi}
\end{figure}


\fnamee features a bidirectional type system inherited from \fname.
% \bruno{We can just focus on the
%   T-TApp, T-TAbs and T-Merge. Maybe you can drop subsumption and lambdas,
%   since those are standard.
% Refer to the full rules in the appendix.}
The full set of typing rules are shown in \cref{fig:typing:fi}. Again we ignore
the gray parts and explain them in \cref{sec:elaboration:fi}.
The inference judgment $[[ DD; GG |- ee => A ]]$ says
that we can synthesize the type $[[A]]$ under the contexts $[[DD]]$ and
$[[GG]]$. The checking judgment $[[ DD ; GG |- ee <= A ]]$ asserts that $[[ee]]$
checks against the type $[[A]]$ under the contexts $[[DD]]$ and $[[GG]]$.
Most of the rules are quite standard in the literature.
The merge expression $[[ee1 ,, ee2]]$ is well-typed if both sub-expressions are
well-typed, and their types are \textit{disjoint} (\rref{T-merge}).
The disjointness relation will be explained in \cref{sec:disjoint:fi}.
To infer a type abstraction (\rref{T-tabs}), we add disjointness constraints to the type
context. For a type
application (\rref{T-tapp}), we check that the type argument
satisfies the disjointness constraints.
\Rref{T-merge,T-tapp} are the only rules checking disjointness.


\subsection{Disjointness}
\label{sec:disjoint:fi}

\renewcommand{\rulehl}[1]{#1}

\begin{figure}[t]
  \centering
  \drules[TL]{$[[ A top  ]]$}{Top-like types}{top,and,arr,rcd,all}
  \drules[D]{$[[DD |- A ** B]]$}{Disjointness}{topL, topR, arr, andL, andR, rcdEq, rcdNeq, tvarL, tvarR, forall,ax}
  % \drules[Dax]{$[[A **a B]]$}{Disjointness axioms}{sym, intArr, intRcd,intAll,arrAll,arrRcd,allRcd}
  \caption{Selected rules for disjointness}
  \label{fig:disjoint:fi}
\end{figure}

We now turn to another core judgment of \fnamee---the disjointness relation,
shown in \cref{fig:disjoint:fi}. The disjointness rules are mostly inherited
from \fname~\cite{alpuimdisjoint}, but the new bottom type requires
a notable change regarding disjointness with \emph{top-like types}.

\paragraph{Top-like types.}

Top-like types are all types that are isomorphic to $[[Top]]$ (i.e.,
simultaneously sub- and supertypes of $[[Top]]$). Hence, they are inhabited by a
single value, isomorphic to the $[[Top]]$ value. \cref{fig:disjoint:fi} captures
this notion in a syntax-directed fashion in the $[[A top]]$ predicate.
As a historical note, the concept of top-like types was already known by
Barendregt et al.~\cite{Barendregt_1983}. The \oname calculus~\cite{oliveira2016disjoint}
re-discovered it and coined the term ``top-like types''; the \fname calculus~\cite{alpuimdisjoint} extended it
with universal quantifiers. Note that in both calculi,
top-like types are solely employed for enabling a syntactic method of
proving coherence, and due to the lack of BCD subtyping, they do not have a
type-theoretic interpretation of top-like types.





% \jeremy{mention previous work do not have good explanations of top-like types?}
% \ningning{that's a good idea. I think ``all types that are isomorphic to
%   $[[Top]]$'' is a more accurate definition of Top-like. A comparison with
%   previous work would be great (i.e. why \oname and \fname include top-like,
%   and \namee doesn't, while \fnamee does again?)}

\paragraph{Disjointness rules.}

%The disjointness judgment $[[DD |- A ** B]]$ plays a central role in \fnamee,
%and is used to ensure that the merge operator can only form intersections of
%disjoint types.

The disjointness judgment $[[DD |- A ** B]]$ is helpful to check whether the merge of two
expressions of type $[[A]]$ and $[[B]]$ preserves coherence. Incoherence
arises when both expressions produce distinct values for the same type, either
directly when they are both of that same type, or through implicit upcasting to
a common supertype. Of course we can safely disregard top-like types in this matter
because they do not have two distinct values. In short, it suffices to check that the
two types have only top-like supertypes in common.

% (As a precondition, $[[A]]$ and $[[B]]$ are
% required to be both well-formed under the context $[[DD]]$.)
% A guiding principle
% of checking whether $[[A]]$ and $[[B]]$ are disjoint is to ask if their
% supertypes are top-like types. Let us apply this principle to design
% disjointness rules for bottom types. Given $[[Bot]]$ and $[[A]]$, we know that
Because $[[Bot]]$ and any another type $[[A]]$ always have $[[A]]$ as
a common supertype,
it follows that $[[Bot]]$ is only disjoint to $[[A]]$ when
$[[A]]$ is top-like.
More generally, if $[[A]]$ is a top-like type, then $[[A]]$ is disjoint to any
type. This is the rationale behind the two rules \rref*{D-topL,D-topR}, which
generalize and subsume $[[ DD |- Top ** A ]]$ and $[[ DD |- A ** Top ]]$ from \fname, and
also cater to the bottom type.
Two other interesting rules are
\rref*{D-tvarL,D-tvarR}, which dictate that a type variable $[[X]]$ is disjoint
with some type $[[B]]$ if its disjointness constraints $[[A]]$ is a subtype of
$[[B]]$.
Disjointness axioms $[[ A **a B ]]$ (appearing in \rref{D-ax}) take
care of two types with different type constructors (e.g., $[[nat]]$ and records). Axiom rules can be found in the appendix.
Finally we note that the disjointness relation is symmetric.

% \begin{lemma}[Symmetry of disjointness]
%   \label{lemma:symmetry-disjoint}
%   If $[[ DD |- A ** B  ]]$, then $[[  DD |- B ** A   ]]$.
% \end{lemma}
% \begin{proof}
%   By induction on the disjointness derivation. In the case for \rref{FD-forall},
%   apply \cref{lemma:narrow:disjoint}.
% \end{proof}


\subsection{Elaboration and Type Safety}
\label{sec:elaboration:fi}

\begin{figure}[t]
  \centering
\begin{tabular}{llll} \toprule
  Types & $[[T]]$ & $\Coloneqq$ & $[[nat]] \mid [[Unit]] \mid [[T1 -> T2]]  \mid [[T1 * T2]] \mid [[X]] \mid [[\ X . T]]$\\
  Terms & $[[e]]$ & $\Coloneqq$ & $[[x]] \mid [[ii]] \mid [[unit]] \mid [[\x . e]] \mid [[e1 e2]] \mid [[< e1 , e2>]] \mid [[\X . e]] \mid [[ e T ]] \mid [[c e]]$ \\
  Coercions & $[[c]]$ & $\Coloneqq$ & $[[id]] \mid [[c1 o c2]] \mid [[top]] \mid [[bot]] \mid [[c1 -> c2]] \mid [[< c1 , c2 >]] \mid [[pp1]] \mid [[pp2]] $ \\
  & & $\mid$ & $ [[\ c]] \mid [[distArr]] \mid [[topArr]] \mid [[topAll]] \mid [[distPoly]] $ \\
  Values & $[[v]]$ & $\Coloneqq$ & $[[ii]] \mid [[unit]] \mid [[\x . e]] \mid [[< v1 , v2>]] \mid [[\X . e]] \mid [[ (c1 -> c2) v ]] \mid [[\c v]]  $ \\
  & & $\mid$ & $ [[distArr v]] \mid [[topArr v]] \mid [[topAll v]] \mid [[distPoly v]]  $ \\
  Term contexts & $[[gg]]$ & $\Coloneqq$ &  $[[empty]] \mid [[gg , x : T]] $ \\
  Type contexts & $[[dd]] $ & $\Coloneqq $ & $[[empty]] \mid [[dd , X]] $  \\
  Evaluation contexts & $[[EE]]$ & $\Coloneqq$ &  $  [[__]] \mid [[EE e]] \mid [[v EE]] \mid [[ < EE , e >  ]] \mid [[ < v , EE > ]] \mid [[ c EE  ]] \mid [[ EE T  ]]  $ \\ \bottomrule
\end{tabular}
\caption{Syntax of \tnamee}
\label{fig:syntax:fco}
\end{figure}


The dynamic semantics of \fnamee is given by elaboration into
a target calculus. The target calculus \tnamee is the standard call-by-value
System F extended with products and coercions. The syntax of \tnamee is shown in
\cref{fig:syntax:fco}.

\paragraph{Type translation.}

\cref{def:type:translate:fi} defines the type translation function $| \cdot |$
from \fnamee types $[[A]]$ to \tnamee types $[[T]]$. Most cases are
straightforward. For example, $[[Bot]]$ is mapped to an uninhabited
type $[[\X . X]]$; disjoint quantification is mapped to universal
quantification, dropping the disjointness constraints. $| \cdot |$ is
naturally extended to work on contexts as well.

\begin{definition} Type translation $| \cdot |$ is defined as follows:
  \label{def:type:translate:fi}
  \begin{center}
\begin{tabular}{rlllrlllrll} \toprule
  $| [[nat]] |$ &$=$ & $ [[nat]]$ & $\qquad$ &   $| [[Top]] |$ & $ = $ & $ [[ Unit  ]] $ & $\qquad$ &   $| [[A -> B]] |$ & $ = $ & $ [[ |A| -> |B| ]]$ \\
  $| [[A & B]] |$ &$=$ & $ [[ |A| * |B|    ]] $ & $\qquad$ &   $| [[{l : A}]] |$ & $ = $ & $ [[ | A | ]] $ & $\qquad$ &   $| [[X]] |$ & $ = $ & $ [[ X ]]$ \\
  $| [[ Bot  ]] |$ &$=$& $[[ \X . X   ]]$ & $\qquad$   & $| [[\X ** A . B]] |$ & $ = $ & $[[\X . |B|]]$ \\ \bottomrule
\end{tabular}
  \end{center}
\end{definition}


\paragraph{Coercions and coercive subtyping.}

We follow prior work~\cite{bi_et_al:LIPIcs:2018:9227, biernacki2015logical} by having a syntactic category
for coercions~\cite{Henglein_1994}. In \cref{fig:syntax:fco}, we have several new coercions:
$[[bot]]$, $[[ \ c ]]$, $[[distPoly]]$ and $[[topAll]]$ due to the addition of polymorphism and bottom type. As
seen in
\cref{fig:subtype_decl:fi} the coercive subtyping judgment has the form $[[ A <: B ~~> c ]]$, which
says that the subtyping derivation for $[[A <: B]]$ produces a coercion $[[c]]$ that converts terms of type $[[ |A|  ]]$
to $[[| B |]]$.
% Each subtyping rule has its own specific form of coercion.


% Now we go back
% to the coercion part in \rref{S-forall}. Since the disjointness constraints are
% erased during elaboration, they do not contribute to the overall coercion; we
% only need the coercion generated by the subtyping of the bodies $[[B1]]$ and
% $[[B2]]$.

% As a cognitive aid, it is
% instructive to mentally ``desugar'' the coercion $[[\ c]]$ to the regular term
% $[[ \f . \ X . c (f X)]]$, as shown in \cref{tab:coercion2}, then the expression
% $ [[\c v]] $ is ``equal'' to $[[ \X . c (v X) ]]$, which is why we can treat $[[ \c v]]$ as a value.

% \begin{figure}[t]
%   \centering

% \begin{tabular}{lllll} \toprule
%   \textbf{Coercion} & \textbf{Term} & & \textbf{Coercion} & \textbf{Term} \\ \midrule
%   $[[id]]$         & $[[ \x . x]]$  & & $[[ c1 o c2  ]]$    &  $[[  \x . c1 (c2 x) ]]$ \\
%   $[[top]]$         & $[[ \x . unit ]]$  & & $[[ c1 -> c2  ]]$    &  $[[  \f . \x . c2 (f (c1 x))  ]]$ \\
%   $[[  pp1   ]]$         & $[[ \x . pp1 x    ]]$  & & $[[ pp2  ]]$    &  $[[  \x . pp2 x  ]]$ \\
%   $[[  < c1 , c2 >   ]]$         & $[[ \x . < c1 x, c2 x >    ]]$  & & $[[ distArr  ]]$    &  $[[  \x . \y . < (pp1 x) y , (pp2 x) y > ]]$ \\
%   $[[  topArr   ]]$         & $[[ \x . \ y . unit    ]]$  & & $[[ \ c  ]]$ &  $[[   \f . \ X . c (f X)    ]]$   \\
%   $[[  topAll   ]]$         & $[[ \x . \ X . unit    ]]$  & & $[[ distPoly  ]]$ &  $[[   \f . \ X .< (pp1 f) X , (pp2 f) X >     ]]$   \\ \bottomrule
% \end{tabular}
%   \caption{Correspondence between coercions and terms}
%   \label{tab:coercion2}
% \end{figure}




\paragraph{\tnamee static semantics.}

The typing rules of \tnamee are quite standard. We have one rule \rref*{t-capp} regarding
coercion application, which uses the judgment $[[ c |- T tri T' ]]$ to
type coercions. We show two representative rules \rref*{ct-forall,ct-bot}.
{\small
  \begin{mathpar}
  \drule{t-capp} \and
  \drule{ct-forall}  \and \drule{ct-bot}
  \end{mathpar}
}%


\begin{figure}[t]
  \centering
  \drules[r]{$[[e --> e']]$}{Single-step reduction}{forall,topAll, distAll,tapp,app,ctxt}
  \caption{Selected reduction rules}
  \label{fig:red:fi}
\end{figure}

\paragraph{\tnamee dynamic semantics.}

The dynamic semantics of \tnamee is mostly unremarkable. We write $[[e --> e']]$
to mean one-step reduction. \Cref{fig:red:fi} shows selected reduction
rules. The first line shows three representative rules regarding coercion reductions.
They do not contribute to computation but merely rearrange coercions.
Our coercion reduction rules are quite standard but not efficient in terms of
space. Nevertheless, there is existing work on space-efficient
coercions~\citep{Siek_2015, herman2010space}, which should be applicable to our
work as well. \Rref{r-app} is the usual $\beta$-rule that performs actual
computation, and \rref{r-ctxt} handles reduction under an evaluation context.
As usual, $[[-->>]]$ is the reflexive, transitive closure of $[[-->]]$.
Now we can show that \tnamee is type safe:

\begin{theorem}[Preservation]
  If $[[empty; empty |- e : T]]$ and $[[e --> e']]$, then $[[empty; empty |- e' : T]]$.
\end{theorem}

\begin{theorem}[Progress]
  If $[[empty; empty |- e : T]]$, either $[[e]]$ is a value, or $\exists [[e']].\ [[e --> e']]$.
\end{theorem}


\paragraph{Elaboration.}

Now consider the translation parts in \cref{fig:typing:fi}. The key idea of
the translation follows the prior work~\cite{dunfield2014elaborating, oliveira2016disjoint,
  alpuimdisjoint, bi_et_al:LIPIcs:2018:9227}: merges are elaborated to pairs
(\rref{T-merge}); disjoint quantification and disjoint type applications
(\rref{T-tabs,T-tapp}) are elaborated to regular universal quantification and
type applications, respectively.
% Below we show an example translation:
% \[
%   [[ (\X ** nat . (\x . x) : X -> X)  : \ X ** nat . X & nat -> X ]] \hlmath{\rightsquigarrow [[\ (pp1 -> id)  (\ X . \x . x)]]}
% \]
% \begin{align*}
%   & [[ (\X ** nat . (\x . x) : X -> X)  : \ X ** nat . X & nat -> X ]] \\
%   \rightsquigarrow & \\
%   & [[\ (pp1 -> id)  (\ X . \x . x)]]
% \end{align*}
Finally, the following lemma connects \fnamee to \tnamee:

% \begin{lemma}[Coercions preserve types]
%   \label{lemma:sub-correct:fi}
% \end{lemma}
% \begin{proof}
%   By structural induction on the derivation of subtyping.
% \end{proof}
\begin{lemma}[Elaboration soundness] We have that:
  \begin{itemize}
  \item If $[[A <: B ~~> c]]$, then $[[c |-  |A| tri |B|]]$.
  \item If $[[DD ; GG |- ee => A ~~> e]]$, then $[[ |DD| ; |GG| |- e : |A | ]]$.
  \item If $[[DD ; GG |- ee <= A ~~> e]]$, then $[[ |DD| ; |GG| |- e : |A | ]]$.
  \end{itemize}
\end{lemma}
% \begin{proof}
%   By structural induction on the derivation of typing.
% \end{proof}




\section{Algorithmic System and Decidability}


The subtyping relation in \cref{fig:subtype_decl:fi} is highly non-algorithmic
due to the presence of a transitivity rule.
This section presents an alternative algorithmic formulation.
Our algorithm extends that of \namee, which itself was inspired by
Pierce's decision
procedure~\cite{pierce1989decision}, to handle
disjoint quantifiers and the bottom type. We then prove that the algorithm is sound and
complete with respect to declarative subtyping.

Additionally we prove that the subtyping and disjointness relations are
decidable. Although the proofs of this fact are fairly straightforward, it is
nonetheless remarkable since it contrasts with the subtyping
relation for (full) \fsub~\cite{cardelli1985understanding}, which is 
undecidable~\cite{pierce1994bounded}. Thus while bounded quantification is
infamous for its undecidability, disjoint quantification has the nicer property
of being decidable.

\subsection{Algorithmic Subtyping Rules}


While \cref{fig:subtype_decl:fi} is a fine specification of how subtyping
should behave, it cannot be read directly as a subtyping algorithm for two
reasons: (1) the conclusions of \rref{S-refl,S-trans} overlap with the other
rules, and (2) the premises of \rref{S-trans} mention a type that does not
appear in the conclusion. Simply dropping the two offending rules
from the system is not possible without losing
expressivity~\cite{Laurent12note}. Thus we need a
different approach. Following \namee, we intend the algorithmic judgment $[[ fs |- A <: B ]]$ to be
equivalent to $[[ A <: fs => B ]]$, where $[[fs]]$ is a queue used to track record labels, domain types and disjointness constraints.
The full rules of the algorithmic subtyping of \fnamee are shown \cref{fig:algo:sub:fi}.

\begin{definition}[$[[fs]] \Coloneqq [[ []   ]] \mid [[ l , fs  ]] \mid [[  B , fs ]] \mid [[ X ** B , fs  ]]$]
$[[fs => A]]$ is defined as follows:
  \begin{center}
  \begin{tabular}{rlllrll} \toprule
    $[[ [] => A]]$ &=& $[[A]]$ & $\qquad$ & $[[ (B , fs) => A]]$ &=& $[[B -> (fs => A)]]$  \\
    $[[ (l , fs) => A]]$ &=& $[[{l : fs => A}]]$ & $\qquad$ & $[[ (X ** B , fs) => A]]$ &=& $[[\ X ** B . fs => A]]$ \\  \bottomrule
  \end{tabular}
  \end{center}
\end{definition}


\renewcommand{\rulehl}[2][gray!40]{%
  \colorbox{#1}{$\displaystyle#2$}}

\begin{figure}[t]
  \centering
  \drules[A]{$[[fs |- A <: B ~~> c]]$}{Algorithmic subtyping}{top,and,arr,rcd,forall,const, bot,arrConst,rcdConst,andConst,allConst}
  \caption{Algorithmic subtyping}
  \label{fig:algo:sub:fi}
\end{figure}

\renewcommand{\rulehl}[1]{#1}



For brevity of the algorithm, we use metavariable $[[rho]]$ to mean type constants:
\[
  \text{Type Constants}\quad [[rho]] \Coloneqq  [[nat]] \mid [[Bot]] \mid [[X]]
\]
The basic idea of $[[ fs |- A <: B ]]$ is to perform a case analysis on $[[B]]$
until it reaches type constants. We explain new rules regarding disjoint
quantification and the bottom type.
When a quantifier is encountered in $[[B]]$, \rref{A-forall} pushes
the type variables with its disjointness constraints onto $[[fs]]$ and continue
with the body. Correspondingly, in \rref{A-allConst}, when a quantifier is
encountered in $[[A]]$, and the head of $[[fs]]$ is a type variable, this
variable is popped out and we continue with the body.
\Rref{A-bot} is similar to its declarative counterpart.
Two meta-functions $[[ < fs >1 ]]$
and $[[ < fs >2 ]]$ are meant to generate correct forms of coercions, and their
definitions are shown in the appendix. For other algorithmic rules, we refer to
\namee~\cite{bi_et_al:LIPIcs:2018:9227} for detailed explanations.


\paragraph{Correctness of the algorithm.}

We prove that the algorithm is sound and complete with respect to the
specification. We refer the reader to our Coq formalization for more details.
We only show the two major theorems:

\begin{theorem}[Soundness]
  If $[[ fs |- A <: B ~~> c]]$ then $ [[   A <: fs => B ~~> c  ]]   $.
\end{theorem}

\begin{theorem}[Completeness]
  If $[[A <: B ~~> c]]$, then $\exists [[c']].\ [[ [] |- A <: B ~~> c']]$.
\end{theorem}


\subsection{Decidability}

Moreover, we prove that our algorithmic type system is decidable. To see this,
first notice that the bidirectional type system is syntax-directed, so we only
need to show decidability of algorithmic subtyping and
disjointness. The full (manual) proofs for decidability can be found in
the appendix.

\begin{restatable}[Decidability of algorithmic subtyping]{lemma}{decidesub} \label{lemma:decide-sub}
  Given $[[fs]]$, $[[A]]$ and $[[B]]$, it is decidable whether there exists
  $[[c]]$, such that $[[fs |- A <: B ~~> c]]$.
\end{restatable}

\begin{restatable}[Decidability of disjointness checking]{lemma}{decidedis} \label{lemma:decide-dis}
  Given $[[DD]]$, $[[A]]$ and $[[B]]$, it is decidable whether $[[ DD |- A ** B ]]$.
\end{restatable}

% Given algorithmic subtyping and disjointness are decidable, it follows that our
% bidirectional type checking is decidable ($[[dirflag]]$ is a short-hand for $[[ => ]]$ and $[[<=]]$).

% \begin{restatable}[Decidability of typing]{theorem}{decidetyp} \label{lemma:decide-typing}
%   Given $[[DD]]$, $[[GG]]$, $[[ee]]$ and $[[A]]$, it is decidable whether $[[DD ; GG  |- ee dirflag A]]$.
% \end{restatable}

\renewcommand\ottaltinferrule[4]{
  \inferrule*[narrower=0.8,right=#1,#2]
    {#3}
    {#4}
}


One interesting observation here is that although our disjointness
quantification has a similar shape to bounded quantification $[[\/X<:A. B]]$ in
\fsub~\citep{cardelli1985understanding}, subtyping for \fsub~is
undecidable~\citep{pierce1994bounded}. In \fsub, the subtyping relation between
bounded quantification is:
{\small
\[
  \drule{fsub-forall}
\]
}%
Compared with \rref{S-forall}, both rules are contravariant on
bounded/disjoint types, and covariant on the body. However, with bounded
quantification it is fundamental to track the bounds in the
environment, which complicates the design of the rules and makes
subtyping undecidable with \rref{fsub-forall}.
Decidability can be recovered
by employing an invariant rule for bounded quantification
(that is by forcing $[[A1]]$ and $[[A2]]$ to be identical).
Disjoint quantification does not require such invariant rule for
decidability.

\renewcommand\ottaltinferrule[4]{
  \inferrule*[narrower=0.6,lab=#1,#2]
    {#3}
    {#4}
}


\begin{comment}
For example, in the original
type $[[\/X<:A1. B1]]$, the $[[X]]$ in $[[B1]]$ is thought of being bound to
$[[A1]]$, the premise $[[DD, X <: A2 |- B1 <: B2]]$ forces $[[X]]$ to be bound
to $[[A2]]$ in $[[B1]]$. This destroys the original connection and makes it
impossible to give a decision procedure.
\end{comment}

%%% Local Variables:
%%% mode: latex
%%% TeX-master: "../paper"
%%% org-ref-default-bibliography: "../paper.bib"
%%% End:


%%%%%%%%%%%%%%%%%%%%%%%%%%%%%%%%%%%%%%%%%%%%%%%%%%%%%%%%%%%%%%%%%%%%%%%%
\section{Establishing Coherence for \fnamee}
\label{sec:coherence:poly}
%%%%%%%%%%%%%%%%%%%%%%%%%%%%%%%%%%%%%%%%%%%%%%%%%%%%%%%%%%%%%%%%%%%%%%%%

In this section, we establish the coherence property for \fnamee. The proof
strategy mostly follows that of \namee, but the construction of the
heterogeneous logical relation is significantly more complicated. Firstly in
\cref{sec:para:intuition} we discuss why adding BCD subtyping to disjoint
polymorphism introduces significant complications. In
\cref{sec:failed:lr}, we discuss why a natural extension of
System F's logical relation to deal with disjoint polymorphism fails. The technical
difficulty is \emph{well-foundedness}, stemming from the interaction between
impredicativity and disjointness. Finally in \cref{sec:succeed:lr}, we present
our (predicative) logical relation that is specially crafted to prove coherence
for \fnamee.
% and allude to a potential solution to lift the predicativity restriction.

\subsection{The Challenge}
\label{sec:para:intuition}

Before we tackle the coherence of \fnamee, let us first consider how \fname
(and its predecessor \oname) enforces coherence. Its essentially syntactic
approach is to make sure that there is at most one subtyping derivation for any
two types. As an immediate consequence, the produced coercions are uniquely determined and thus
the calculus is clearly coherent. Key to this approach is the invariant that
the type system only produces \emph{disjoint} intersection types. As we
mentioned in \cref{sec:typesystem}, this invariant complicates the calculus
and its metatheory, and leads to a weaker substitution lemma.
% To see this, consider the judgment $[[ X ** nat |- X & nat ]]$. 
% Clearly $[[X]]$ cannot be instantiated to an arbitrary type. For
% instance, substituting $[[X]]$ with $[[nat]]$ would lead to an ill-formed
% intersection type $[[nat & nat]]$ in \fname. 
% Therefore in the
% substitution lemma, the range of substituted types is narrowed down to those
% that respect the disjointness constraints.
% The motivation of maintaining this invariant was to enable
% Generally speaking, in \fname all meta-theoretic properties are weakened to
% account for disjointness pre-conditions. All of these contribute
Moreover, the syntactic coherence approach is incompatible with BCD subtyping,
which leads to multiple subtyping derivations with different coercions and
requires a more general substitution lemma.
% For example, consider the
% coercions produced by $[[ \X ** nat . X & X <: \X ** nat & nat . X ]]$ (neither
% type is ``well-formed'' in the sense of \fname). Two possible ones are
% $[[ \f . \X . pp1 (f X) ]]$ and $[[ \f . \X . pp2 (f X) ]]$. It is not at all
% obvious that they should be equivalent in an appropriate sense.
To accommodate BCD into \oname, Bi et al.~\cite{bi_et_al:LIPIcs:2018:9227}
have created the \namee calculus and
developed a semantically-founded proof method based on logical relations.
Because \namee does not feature polymorphism, the problem at hand is to
incorporate support for polymorphism in this semantic approach to coherence,
which turns out to be more challenging than is apparent.

% preclude the possibility of adding BCD
% subtyping, which requires a general substitution lemma. This implies that the
% avenue taken by Alpuim et al.~\cite{alpuimdisjoint} to prove coherence does not
% work for \fnamee anymore. In particular, subtyping does not necessarily produces unique
% coercions. For example, consider the possible coercions generated by $[[ \X ** nat . X & X <: \X ** nat & nat . X ]]$ (neither of which is ``well-formed''
% in the sense of \fname). Two possible coercions are $[[ \f . \X . pp1 (f X) ]]$
% and $[[ \f . \X . pp2 (f X) ]]$. It is not at all obvious that these two
% coercions are equivalent in an appropriate sense. Moreover, the addition of BCD subtyping
% aggravates the matter even more---the subtyping relation can produce additional
% syntactically different coercions that are harder to argue to be equivalent.
% Inspired by Bi et al.~\cite{bi_et_al:LIPIcs:2018:9227}, a new semantically-founded
% proof method is called for. Logical relations \`a la System F might shed some
% light, as we will discuss next.

\begin{figure}[t]
  \centering
  \begin{tabular}{rll}
    $[[(v1 , v2) in V ( nat ; nat ) ]]$  & $\defeq$ & $\exists [[i]].\, [[v1]] = [[v2]] = [[i]]$ \\
    $[[(v1, v2)  in V(T1 -> T2; T1' -> T2') ]]$ &$\defeq$ & $\forall [[(v, v') in V (T1; T1')   ]].\, [[  (v1 v , v2 v') in E (T2 ; T2') ]]$ \\
    $[[( < v1 , v2 > , v3  )  in V ( T1 * T2 ;  T3  )  ]]$  &$\defeq$& $[[ (v1, v3)  in V (T1 ; T3)  ]] \land [[ (v2, v3)  in V (T2 ; T3)  ]]$ \\
    $[[( v3 , < v1 , v2 >  )  in V ( T3 ; T1 * T2  )  ]]$  &$\defeq$& $[[ (v3, v1)  in V (T3 ; T1)  ]] \land [[ (v3, v2)  in V (T3 ; T2)  ]]$
  \end{tabular}
  \caption{Selected cases from \namee's canonicity relation}
  \label{fig:logical:necolus}
\end{figure}

\subsection{Impredicativity and Disjointness at Odds}
\label{sec:failed:lr}

\Cref{fig:logical:necolus} shows selected cases of \emph{canonicity},
which is \namee's (heterogeneous) logical relation used
in the coherence proof. The definition captures that two values
$[[v1]]$ and $[[v2]]$ of types $[[ T1 ]]$ and $[[T2]]$ are in $\valR{[[T1]]}{[[T2]]}$ iff
either the types are disjoint or the types are equal and the values are
semantically equivalent. Because both alternatives entail coherence, 
canonicity is key to \namee's coherence proof.

\paragraph{Well-foundedness issues.}
For \fnamee, we need to extend canonicity with additional cases to
account for universally quantified types. For reasons that will become clear in
\cref{sec:succeed:lr}, the type indices become source types (rather than target types as in \cref{fig:logical:necolus}).
A naive formulation of one case rule is:
{\small
\begin{align*}
    &[[(v1, v2)  in V(\X ** A1 . B1; \X ** A2 . B2) ]] \defeq  \\
    &\qquad \forall [[C1 ** A1]], [[C2 ** A2]].\ [[( v1 | C1 | , v2 | C2 | ) in E ( B1 [X ~> C1]; B2 [X ~> C2]) ]]
\end{align*}
}%
This case is problematic because it destroys the well-foundedness of \namee's
logical relation, which is based on structural induction on the type indices.
Indeed, the type $[[ B1 [X ~> C1] ]]$ may well be larger than $[[ \X ** A1 . B1 ]]$.


% \begin{verbatim}
% Further outline
% - show System F-style case with deferred substitions
% - introduce variable case
% - show well-foundedness problem with variable case (also present in System F)
% - show System F solution for the problem by adding a relation parameter R
% - introduce problem with heterogeneous case
% \end{verbatim}

However, System F's well-known parametricity logical
relation~\cite{reynolds1983types} provides us with a means to avoid this
problem.  Rather than performing the type substitution immediately as in the
above rule, we can defer it to a later point by adding it to an extra parameter
$[[pq]]$ of the relation, which accumulates the deferred substitutions. This yields a modified rule where the type indices in the recursive occurrences are indeed smaller:
{\small
\begin{align*}
  &[[(v1, v2)  in V(\X ** A1 . B1; \X ** A2 . B2) with pq ]]  \defeq  \\
  &\qquad \forall [[C1 ** A1]], [[C2 ** A2]]. ([[v1 | C1 | ]] ,  [[v2 | C2 |]]) \in \eeR{[[B1]]}{{[[B2]]}}_{[[pq]] [ [[X]] \mapsto ([[C1]], [[C2]])]}
\end{align*}
}%
Of course, the deferred substitution has to be performed eventually, to be precise when the type indices are type variables.
\[
    [[(v1, v2)  in V(X ; X) with pq ]] \defeq [[ (v1, v2) in V(pq1 (X); pq2 (X)) with emp  ]]
\]
Unfortunately, this way we have not only moved the type substitution to the type variable case, but also the ill-foundedness problem. Indeed, this problem is also
present in System F. The standard solution is to not fix the relation $[[Rel]]$ by which values
at type $[[X]]$ are related to $\valR{[[pq1 (X)]]}{[[pq2 (X)]]}$, but instead to make it a parameter that is tracked by $[[pq]]$.
This yields the following two rules for disjoint quantification and type variables:
{\small
\begin{align*}
  [[(v1, v2)  in V(\X ** A1 . B1; \X ** A2 . B2) with pq ]] &\defeq \forall [[C1 ** A1]], [[C2 ** A2]], [[Rel]] \subseteq [[C1]] \times [[C2]]. \\
                                                            & ([[v1 | C1 | ]] ,  [[v2 | C2 |]]) \in \eeR{[[B1]]}{{[[B2]]}}_{[[pq]] [ [[X]] \mapsto ([[C1]], [[C2]], [[Rel]])]} \\
    [[(v1, v2)  in V(X; X) with pq ]] & \defeq ([[v1]], [[v2]]) \in [[pq]]_{[[Rel]]}([[X]])
\end{align*}
}%
Now we have finally recovered the well-foundedness of the relation. It is again
structurally inductive on the size of the type indexes.


\paragraph{Heterogeneous issues.}

We have not yet accounted for one major difference between the parametricity relation, from which we have borrowed ideas, and the canonicity relation, to which we have been adding. The former is homogeneous (i.e., the types of the two values is the same) and therefore has one type index, while the latter is heterogeneous (i.e., the two values may have different types) and therefore has two type indices. Thus we must also consider cases like
$\valR{[[X]]}{[[nat]]}$. A definition that seems to handle this case
appropriately is:
{\small
  \begin{align} \label{eq:var}
    [[(v1, v2)  in V(X; nat) with pq ]] \defeq [[ (v1, v2) in V(pq1 (X); nat) with emp  ]]
  \end{align}
}%
Here is an example to motivate it.
Let  $  [[ee]] = [[\ X ** Top . ((\x . x) : X & nat -> X & nat)]] $.
We expect that $[[ee nat 1 ]]$ evaluates to $[[ <1 , 1> ]]$. To prove that,
%%\footnote{The reader is advised to try it out in our prototype interpreter.}
we need to show $  (1 , 1)   \in \valR{[[X]]}{[[nat]]}_{[ [[X]] \mapsto ([[nat]], [[nat]], [[Rel]])   ]}  $.
According to \cref{eq:var}, this is indeed the case. However, we run into ill-foundedness issue again, because
$[[pq1 (X)]]$ could be larger than $[[X]]$. Alas, this time the parametricity relation has no solution for us.


\subsection{The Canonicity Relation for \fnamee}
\label{sec:succeed:lr}

% \bruno{Perhaps we are still showing too many auxiliary lemmas here? We
% could cut on some of these if we are looking for space.}

\renewcommand\ottaltinferrule[4]{
  \inferrule*[narrower=0.8,right=#1,#2]
    {#3}
    {#4}
}

In light of the fact that substitution in the logical relation seems unavoidable
in our setting, and that impredicativity is at odds with substitution, we turn
to \emph{predicativity}: we change \rref{T-tapp} to its predicative version:
{\small
\[
  \drule{T-tappMono}
\]
}%
where metavariable $[[t]]$ ranges over monotypes (types minus disjoint quantification).
We do not believe that predicativity is a severe restriction in practice, since many source
languages (e.g., those based on the Hindley-Milner type system~\cite{milner1978theory, hindley1969principal} like Haskell and
OCaml) are themselves predicative and do not require the full generality of an
impredicative core language.

\renewcommand\ottaltinferrule[4]{
  \inferrule*[narrower=0.6,lab=#1,#2]
    {#3}
    {#4}
}


% The restriction to
% predicative polymorphism, though reducing expressiveness in theory, does not seem to cost much
% in practice. Languages based on the Hindley–Milner type
% system~\cite{milner1978theory, hindley1969principal}, such as Haskell and ML,
% have such restriction. We also plan to study a variant of \fnamee with implicit
% polymorphism in the future, where a predicativity restriction is
% likely to be required anyway.

\begin{figure}[t]
  \centering
  \begin{tabular}{rll}
    $[[(v1 , v2) in V ( nat ; nat ) ]]$  & $\defeq$ & $\exists [[i]].\, [[v1]] = [[v2]] = [[i]]$ \\
    $[[(v1, v2) in V ( {l : A}  ; {l : B} ) ]]$ & $\defeq$ & $[[ (v1, v2) in V ( A ; B ) ]]$\\
    $[[(v1 , v2) in V ( A1 -> B1 ; A2 -> B2 ) ]]$  & $\defeq$ & $\forall [[(v2' , v1') in V ( A2 ; A1 ) ]].\, [[ (v1 v1' , v2 v2') in E ( B1 ; B2 ) ]]$ \\
    $[[( < v1 , v2 > , v3  )  in V ( A & B ;  C  ) ]]$  & $\defeq$ & $[[ (v1, v3)  in V (A ; C) ]] \land [[ (v2, v3)  in V (B ; C) ]]$  \\
    $[[( v3 , < v1 , v2 >  )  in V ( C; A & B  ) ]]$  & $\defeq$ & $[[ (v3, v1)  in V (C ; A) ]] \land [[ (v3, v2)  in V (C ; B) ]]$  \\
    $[[(v1, v2)  in V ( \ X ** A1 . B1; \ X ** A2 . B2 ) ]]$  &$\defeq$ & $\forall [[empty |- t ** A1 & A2 ]].\ [[  (v1 |t| , v2 |t|) in E ( B1 [X ~> t] ;  B2 [ X ~> t]) ]]$ \\
  % $[[(v1, v2) in V ( A  ; B ) ]]$ & $\defeq$ & $[[A top]] \, \lor \, [[B top]]    $ \\
    $[[(v1 , v2) in V (A; B)]] $  &$\defeq$ & $\mathsf{true} \quad \text{otherwise} $ \\
    $[[(e1, e2) in E (A; B)]]$ & $\defeq$ & $\exists [[v1]], [[v2]].\, [[e1 -->> v1]] \land [[e2 -->> v2]] \ \land [[(v1, v2) in V (A; B)]]$ \\ \\
  \end{tabular}

  \begin{tabular}{rrll}
    $[[p in  DD]]$ & $\defeq$ &  $\ottaltinferrule{}{}{  }{ [[empp in empty]] }$ &     $\ottaltinferrule{}{}{ [[p in DD]] \\ [[empty |- t ** p(B)]] \\  }{ [[p [ X -> t ] in DD , X ** B]]  }$ \\ \\
    $[[  (g1, g2)  in GG with p ]]$ & $\defeq$ &  $\ottaltinferrule{}{}{  }{ [[(empg, empg) in empty with p ]]  }$ & $\ottaltinferrule{}{}{ [[(g1, g2) in GG with p ]] \\ [[(v1, v2) in V (p(A) ; p(A)) ]] }{ [[(g1 [ x -> v1 ] , g2 [ x -> v2 ]  )  in GG , x : A with p ]] }$
  \end{tabular}
  \caption{The canonicity relation for \fnamee}
  \label{fig:logical:fi}
\end{figure}

Luckily, substitution with monotypes does not prevent well-foundedness.
\Cref{fig:logical:fi} defines the \emph{canonicity} relation for
\fnamee. The canonicity relation is a family of binary relations over \tnamee
values that are \emph{heterogeneous}, i.e., indexed by two \fnamee types. Two
points are worth mentioning. (1) An apparent difference from \namee's logical
relation is that our relation is now indexed by \emph{source types}. The reason is that
the type translation function (\cref{def:type:translate:fi}) discards disjointness
constraints, which are crucial in our setting, whereas \namee's
type translation does not have information loss. (2) Heterogeneity
allows relating values of different types, and in particular values whose types are
disjoint. The rationale behind the canonicity relation is to combine equality
checking from traditional (homogeneous) logical relations with disjointness
checking. It consists of two relations: the value relation $\valR{[[A]]}{[[B]]}$
relates \emph{closed} values; and the expression relation
$\eeR{[[A]]}{[[B]]}$---defined in terms of the value relation---relates closed
expressions.

% \paragraph{Value relation.}

The relation $\valR{[[A]]}{[[B]]}$ is defined by induction on the structures of $[[A]]$ and
$[[B]]$. For integers, it requires the two values to be literally the same. For
two records to behave the same, their fields must behave the same. For two
functions to behave the same, they are required to produce outputs related at
$[[B1]]$ and $[[B2]]$ when given related inputs at $[[A1]]$ and $[[A2]]$. For
the next two cases regarding intersection types, the relation distributes
over intersection constructor $[[&]]$. Of particular interest is the case for
disjoint quantification. Notice that it \emph{does not} quantify over arbitrary
relations, but directly substitutes $[[X]]$ with monotype $[[t]]$ in $[[B1]]$ and
$[[B2]]$. This means that our canonicity relation \emph{does not} entail
parametricity. % , and as such, the free theorem in \cref{sec:failed:lr}
% cannot be proved using the canonicity relation.
However, it suffices for our
purposes to prove coherence. Another noticeable thing is that we keep the
invariant that $[[A]]$ and $[[B]]$ are closed types throughout the relation, so
we no longer need to consider type variables. This simplifies things a lot. % The
% other cases are quite standard.
Note that when one type is $[[Bot]]$, two
values are vacuously related because there simply are no values of type $[[Bot]]$.
% We refer to Bi et al.~\cite{bi_et_al:LIPIcs:2018:9227} for more explanations of
% the canonicity relation.
We need to show that the relation is indeed well-founded:

\begin{restatable}[Well-foundedness]{lemma}{wellfounded}\label{lemma:well-founded}
  The canonicity relation of \fnamee is well-founded.
\end{restatable}
\proof
  Let $| \cdot |_{\forall}$ and $| \cdot |_s$ be the number of
  $\forall$-quantifies and the size of types, respectively. Consider the measure $\langle
  | \cdot |_{\forall} , | \cdot |_s \rangle$,
  where $\langle \dots \rangle$ denotes lexicographic order. For the case of
  disjoint quantification, the number of $\forall$-quantifiers decreases.
  For the other cases, the measure of $| \cdot |_{\forall}$ does not increase, and
  the measure of $| \cdot |_s$ strictly decreases.
\qed

% \begin{lemma}[Symmetry]
%   If $[[ (v1, v2) in V ( A ; B ) ]]$ then $[[ (v2, v1) in V ( B ; A ) ]]$.
% \end{lemma}
% \begin{proof}
%   The proof proceeds by first induction on $ | [[A]] |_{\forall} $, then
%   simultaneous induction on the structures of $[[A]]$ and $[[B]]$.
% \end{proof}

% We give the logical interpretations of type and term contexts ($[[p]]$ is a mapping
% from type variables to monotypes, $[[g]]$ is a mapping from variables to values).

% The canonicity relation is so constructed to contain values of disjoint types:
% We need to first show an auxiliary lemma regarding top-like types:

% \begin{lemma}
%   If $[[  empty ; empty |-  v1 : |A|  ]]$,
%   $[[  empty ; empty |-  v2 : |B|  ]]$ and
%   $[[ A top  ]]$,
%   then $[[   (v1, v2) in V ( A ; B  )  ]]$.
% \end{lemma}
% \begin{proof}
%   By simultaneous induction on $[[t1]]$ and $[[t2]]$.
% \end{proof}

% \begin{lemma}[Disjoint values are related]
%   If $[[DD |- A ** B]]$, $[[ p in DD  ]]$, $[[  empty ; empty |-  v1 : |p (A)|  ]]$ and $[[  empty ; empty |-  v2 : |p (B)|  ]]$
%   then $[[   (v1, v2) in V ( p(A) ; p(B)  )    ]]$.
% \end{lemma}


\subsection{Establishing Coherence}

\paragraph{Logical equivalence.}

The canonicity relation can be lifted to open expressions in the standard way,
i.e., by considering all possible interpretations of free type and term variables.
The logical interpretations of type and term contexts are found in the bottom
half of \cref{fig:logical:fi}.
\begin{definition}[Logical equivalence $\backsimeq_{log}$]
  {\small
  \begin{align*}
    &[[DD ; GG |- e1 == e2 : A ; B]]   \defeq  [[|DD| ; |GG| |- e1 : |A|]] \land [[ |DD | ; |GG| |- e2 : | B | ]] \ \land \\
    &\quad (\forall [[p]], [[g1]], [[g2]]. \ [[p in DD]] \land [[(g1, g2) in GG with p ]] \Longrightarrow [[(g1 (p1 (e1)), g2 (p2 (e2)))  in E (p(A) ; p(B)) ]])
  \end{align*}
  }%
\end{definition}
For conciseness, we write $[[DD ; GG |- e1 == e2 : A]]$ to mean $[[DD ; GG |- e1 == e2 : A ; A]]$.

\paragraph{Contextual equivalence.}

% \begin{figure}[t]
%   \centering
% \begin{tabular}{llll}\toprule
%   \tnamee contexts & $[[cc]]$ & $\Coloneqq$ &  $[[__]] \mid [[\ x . cc]] \mid [[\ X . cc]]  \mid [[ cc T  ]] \mid [[cc e]] \mid [[e cc]] \mid [[< cc , e>]] \mid [[<e , cc>]] \mid [[c cc]] $ \\
%   \fnamee contexts & $[[CC]]$ & $\Coloneqq$ &  $[[__]] \mid [[\ x . CC]] \mid [[\ X ** A. CC]] \mid [[ CC A  ]] \mid [[CC ee]] \mid [[ee CC]] \mid [[ CC ,, ee  ]] \mid [[ ee ,, CC  ]] \mid [[ { l = CC}  ]]  \mid [[ CC . l]] \mid [[ CC : A ]] $ \\ \bottomrule
% \end{tabular}
%   \caption{Expression contexts}
%   \label{fig:contexts:fi}
% \end{figure}

Following \namee, the notion of coherence is based on \emph{contextual
  equivalence}. The intuition is that two programs are equivalent if we
\emph{cannot} tell them apart in any context. As usual, contextual
equivalence is expressed using \emph{expression contexts} ($[[CC]]$ and $[[cc]]$ denote \fnamee and \tnamee expression contexts, respectively),
Due to the bidirectional nature of the type system, the typing judgment of $[[CC]]$
features 4 different forms (full rules are in the appendix),
e.g., $[[CC : (DD; GG => A) ~> (DD'; GG' => A') ~~> cc]]$ reads if $[[DD ; GG |- ee => A]]$
then $[[DD' ; GG' |- CC { ee } => A']]$. The judgment also generates a well-typed \tnamee context $[[cc]]$. The
following two definitions capture the notion of contextual equivalence:

\begin{definition}[Kleene Equality $\backsimeq$]
  Two complete programs (i.e., closed terms of type $[[nat]]$), $[[e]]$ and $[[e']]$, are Kleene equal, written
  $\kleq{[[e]]}{[[e']]}$, iff there exists an integer $[[ii]]$ such that $[[e -->> ii]]$ and
  $[[e' -->> ii]]$.
\end{definition}

\begin{definition}[Contextual Equivalence $\backsimeq_{ctx}$] \label{def:cxtx2}
  {\small
  \begin{align*}
    &[[DD ; GG |- ee1 ~= ee2 : A]]  \defeq \forall [[e1]], [[e2]].\  [[DD ; GG |- ee1 => A ~~> e1]] \land [[DD ; GG |- ee2 => A ~~> e2]] \ \land   \\
    &\qquad (\forall [[C]], [[cc]].\ [[CC : (DD; GG => A) ~> (empty ; empty => nat) ~~> cc]] \Longrightarrow \kleq{[[cc{e1}]]}{[[cc{e2}]]})
  \end{align*}
  }%
\end{definition}

% \noindent In other words, for all possible experiments $[[ cc ]]$, the outcome of an
% experiment on $[[e1]]$ is the same as the outcome on $[[e2]]$
% (i.e., $\kleq{[[cc{e1}]]}{[[cc{e2}]]}$).

% \begin{proof}
%   By induction on the derivation of disjointness. The most interesting case is the variable rule:
%   \[
%     \drule{D-tvarL}
%   \]
%   By the definition of $[[p]]$, we know $[[p(X)]]$ is a monotype. If $[[B]]$ is
%   a polytype, then it follows easily from the definition of logical relation. If
%   $[[B]]$ is also a monotype, we know $[[p(X)]]$ and $[[p(A)]]$ are disjoint by
%   definition. Then by \cref{lemma:covariance:disjoint} and $[[A <: B]]$,
%   we have $[[p(X)]]$ and $[[p(B)]]$ are also disjoint. Finally we apply
%   \cref{lemma:disjoint:mono}.
% \end{proof}

% \paragraph{Compatibility.}

% Firstly we need the compatibility lemmas. Most of them are standard and are thus
% omitted. We show only two compatibility lemmas that are specific to our setting:

% \begin{lemma}[Coercion compatibility] \label{lemma:co-compa} % APPLYCOQ=COERCION_COMPAT
%   Suppose that $[[A1 <: A2 ~~> c]]$,
%   \begin{itemize}
%   \item If $[[DD ; GG |- e1 == e2 : A1 ; A0]]$ then $[[DD ; GG |- c e1 == e2 : A2 ; A0]]$.
%   \item If $[[DD ; GG |- e1 == e2 : A0 ; A1]]$ then $[[DD ; GG |- e1 == c e2 : A0 ; A2]]$.
%   \end{itemize}
% \end{lemma}
% % \begin{proof}
% %   By induction on the subtyping derivation.
% % \end{proof}

% \begin{lemma}[Merge compatibility] % APPLYCOQ=MERGE_COMPAT
%   If $[[ DD ;   GG |- e1 == e1' : A ]]$, $[[  DD ; GG |- e2 == e2' : B ]]$ and $[[ DD |- A ** B ]]$,
%   then $[[ DD ;  GG |- < e1, e2 > == <e1', e2'> : A & B ]]$.
% \end{lemma}
% \begin{proof}
%   By the definition of logical relation and \cref{lemma:disjoint}.
% \end{proof}


% \paragraph{Fundamental property.}

% The ``Fundamental Property'' states that any well-typed expression is related to
% itself by the logical relation. In our elaboration setting, we rephrase it so
% that any two \tnamee terms elaborated from the \emph{same} \fnamee expression
% are related To prove it, we require \cref{thm:uniq}.

% \begin{theorem} \label{thm:uniq}
%   If $[[DD ; GG |- ee => A1]]$ and $[[DD ; GG |- ee => A2]]$, then $[[A1]] \equiv_\alpha [[A2]]$.
% \end{theorem}

% \begin{theorem}[Fundamental property] We have that:
%   \begin{itemize}
%   \item If $[[DD; GG |- ee => A ~~> e]]$ and $[[DD; GG |- ee => A ~~> e']]$, then $[[DD; GG |- e == e' : A ]]$.
%   \item If $[[DD ; GG |- ee <= A ~~> e]]$ and $[[DD ; GG |- ee <= A ~~> e']]$, then $[[DD; GG |- e == e' : A ]]$.
%   \end{itemize}
% \end{theorem}


% We show that logical equivalence is preserved by \fnamee contexts:

% \begin{theorem}[Congruence]
%  If $[[CC : (DD ; GG dirflag A) ~> (DD' ; GG' dirflag' A') ~~> cc]]$, $[[DD ; GG |- ee1 dirflag A ~~> e1]]$, $[[DD ; GG |- ee2 dirflag A ~~> e2]]$
%  and $[[DD ; GG |- e1 == e2 : A ]]$, then $[[DD' ; GG' |- cc{e1} == cc{e2} : A']]$.
% \end{theorem}

\paragraph{Coherence.}

For space reasons, we directly show the coherence statement of \fnamee.
We need several technical lemmas such as compatibility lemmas, fundamental property, etc.
The interested reader can refer to our Coq formalization.

\begin{theorem}[Coherence] \label{thm:coherence:fi}
  We have that
  \begin{itemize}
  \item If $[[DD ; GG |- ee => A ]]$ then $[[DD ; GG |- ee ~= ee : A]]$.
  \item If $[[DD ; GG |- ee <= A ]]$ then $[[DD ; GG |- ee ~= ee : A]]$.
  \end{itemize}
\end{theorem}
\noindent That is, coherence is a special case of \cref{def:cxtx2} where
$[[ee1]]$ and $[[ee2]]$ are the same. At first glance, this
appears underwhelming: of course $[[ee]]$ behaves the same as itself! The tricky
part is that, if we expand it according to \cref{def:cxtx2}, it is not $[[ee]]$
itself but all its translations $[[e1]]$ and $[[e2]]$ that behave the same!




% Local Variables:
% org-ref-default-bibliography: "../paper.bib"
% End:

% \renewcommand{\rulehl}[2][gray!40]{%
  \colorbox{#1}{$\displaystyle#2$}}


\section{Taming Row Polymorphism}

In this section we show how to systematically translate
\rname~\cite{Harper:1991:RCB:99583.99603}---a polymorphic record calculus---into
\fnamee. The translation itself is interesting in two regards: first, it shows
that disjoint polymorphism can simulate row polymorphism;\ningning{won't this be
too strong an argument? there are many row polymorphism systems and we are
only talking about one of them.} second, it also
reveals a significant difference of expressiveness between disjoint polymorphism
and row polymorphism---in particular, we point out that row polymorphism alone
is impossible to encode nested composition, which is crucial for applications of
extensible designs. We first review the syntax and semantics of
\rname. We then discuss a seemingly correct translation that failed to
faithfully convey the essence of row polymorphism. By a careful comparison of
the two calculi, we present a type-directed translation,
and prove that the translation is type safe, i.e., well-typed \rname terms map
to well-typed \fnamee terms.
\bruno{besides nested composition, the other advantage of disjoint
  polymorphism is subtyping. Row polymorphism usually does not support
subtyping, so you cannnot write a function f : \{x : Int\} $\to$ Int and
type-check the following application: f \{x=2,y=3\}. With row
polymorphism you must generalize the type of f, and make the interface
of the function more complicated.}
\jeremy{I don't think this is entirely true for row polymorphism in general. Some row systems do have subtyping (see Harper's paper).
  As argued by Harper, it was their design choice to not have subsumption (or subtyping) in the first place,
  in order to have a simpler system. Also with row type inference, you don't actually need to write complicated types for this example.}

% In the process, we identified one broken lemma of \rname due to the design of type equivalence, which is remedied in our presentation.


\subsection{Syntax of \rname}

\begin{figure}[t]
  \centering
\begin{tabular}{llll@{\hskip 0.6cm}llll} \toprule
  Types & $[[rt]]$ & $\Coloneqq$ & $[[base]] \mid [[rt1 -> rt2]] \mid [[\/ a # R .  rt]] \mid [[ r  ]]$ & Constraint lists & $[[R]]$&  $\Coloneqq$ &$[[ <>  ]] \mid [[ r , R ]] $ \\ 
  Records & $[[r]]$ & $\Coloneqq$ & $[[a]] \mid [[Empty]] \mid [[ {l : rt}  ]]  \mid [[  r1 || r2 ]] $  & Term contexts & $[[Gtx]]$ &  $\Coloneqq$ &  $[[ <> ]] \mid [[Gtx , x : rt ]] $ \\
  Terms & $[[re]]$ & $\Coloneqq$ & $[[x]] \mid [[\x : rt . re]] \mid [[re1 re2]] \mid [[rempty]] $ & Type contexts & $[[Ttx]]$ & $\Coloneqq$ & $[[ <> ]] \mid [[Ttx , a # R ]] $ \\
        &          & $\mid $ & $[[{ l = re }]] \mid [[re1 || re2]] \mid [[ re \ l  ]]  \mid [[ re . l  ]] $ \\
        &          & $ \mid$ & $ [[ /\ a # R . re  ]] \mid [[  re [ r ]  ]]$ \\
    \bottomrule
\end{tabular}
  \caption{Syntax of \rname}
  \label{fig:syntax:record}
\end{figure}

We start by briefly reviewing the syntax of \rname, shown in \cref{fig:syntax:record}.

\paragraph{Types.}

Metavariable $[[rt]]$ ranges over types, which include integer types
$[[base]]$, function types $[[rt1 -> rt2]]$, constrained quantification $[[ \/ a # R . rt ]]$
and record types $[[r]]$. The record types are built from record type
variables $[[a]]$, empty record $[[Empty]]$, single-field record types $[[ { l : rt}]]$
and record merges $[[ r1 || r2 ]]$.\footnote{The original \rname also include record
  type restrictions $[[r \ l]]$, which, as they later proved, can be systematically
  erased using type equivalence, thus we omit type-level restrictions but keep term-level restrictions.}
A constraint list $[[R]]$ is a list of record types, used to constrain instantiations of record type variables.
% , and plays
% an important role in the calculus, as we will explain shortly.

\paragraph{Terms.}

Metavariable $[[re]]$ ranges over terms, which include term
variables $[[x]]$, lambda abstractions $[[ \x : rt . re ]]$, function applications $[[re1 re2]]$, empty records $[[rempty]]$,
single-filed records $[[{l = re}]]$, record merges $[[re1 || re2]]$, record restrictions $[[ re \ l ]]$, record projections $[[ re . l  ]]$,
type abstractions $[[  /\ a # R . re ]]$ and type applications $[[ re [ r ]   ]]$.
As a side note, from the syntax of type applications $[[re [ r ] ]]$, it already can be seen that \rname only supports
quantification over \emph{record types}.
% ---though a separate form of quantifier that quantifies over \emph{all types}.
% can be added, Harper and Pierce decided to have only one form of quantifier for the sake of simplicity.

% \paragraph{An example.}

% Before proceeding to the formal semantics, let us first see some examples that
% can be written in \rname, which may be of help in understanding the overall
% system better.
\paragraph{An example.}

When it comes to extension, every record calculus must decide what to do with
duplicate labels. According to Leijen~\cite{leijen2005extensible}, record calculi can
be divided into those that support \emph{free} extension, and those that support
\emph{strict} extension. The former allows duplicate labels to coexist, whereas
the latter does not. In that sense, \rname belongs to the strict camp. What sets
\rname apart from other strict record calculi is its ability to merge records
with statically unknown fields, and a mechanism to ensure the resulting record
is conflict-free (i.e., no duplicate labels). For example, the following
function merges two records:
\[
  \mathsf{mergeRcd} = [[  /\ a1 # Empty . /\ a2 # a1  . \ x1 : a1 . \ x2 : a2 . x1 || x2  ]]
\]
which takes two type variables: the first one has no constraint
($[[Empty]]$) at all and the second one has only one constraint ($[[ a1 ]]$). It
may come as no surprise that $\mathsf{mergeRcd}$ can take any record type as its
first argument, but the second type must be \emph{compatible} with the first. In
other words, the second record cannot have any labels that already exist in the
first. These constraints are enough to ensure that the resulting record $[[x1 ||
x2]]$ has no duplicate labels. If later we want to say that the first record
$[[x1]]$ has \emph{at least} a field $[[l1]]$ of type $[[nat]]$, we can refine
the constraint list of $[[a1]]$ and the type of $[[x1]]$ accordingly:
\[
  [[  /\ a1 # {l1 : nat} . /\ a2 # a1  . \ x1 : a1 || {l1 : nat} . \ x2 : a2 . x1 || x2  ]]
\]
The above examples suggest an important point: the form of constraint used in
\rname can only express \emph{negative} information about record type variables.
Nonetheless, with the help of the merge operator, positive information can be
encoded as merges of record type variables, e.g., the assigned type of $[[x1]]$
illustrates that the missing field $[[ {l1 : nat} ]]$ is merged back into
$[[a1]]$.

The acute reader may have noticed some correspondences between \rname and
\fnamee: for instance, $[[ /\ a # R . re ]]$ vs. $[[ \ X ** A . ee ]]$,
and $[[x1 || x2]]$ vs.  $[[ x1 ,, x2 ]]$. Indeed, the very
function can be written in \fnamee almost verbatim:
\[
  \mathsf{mergeAny} = [[\ a1 ** Top . \ a2 ** a1 . \x1 : a1 . \x2 : a2 . x1 ,, x2 ]]
\]
However, as the name suggests, $\mathsf{mergeAny}$ works for \emph{any} two types,
not just record types.

\subsection{Typing Rules of \rname}
\label{sec:typing_rname}

% \jeremy{We may only want to show selected rules. }

The type system of \rname consists of several conventional judgments. The
complete set of rules appear in \cref{appendix:rname}.
\Cref{fig:rname_well_formed} presents the well-formedness rules for record
types. % Most cases are quite standard.
A merge $[[r1 || r2]]$ is well-formed in
$[[Ttx]]$ if $[[r1]]$ and $[[r2]]$ are well-formed, and moreover,
$[[r1]]$ and $[[r2]]$ are compatible in $[[Ttx]]$ (\rref{wfr-Merge})---the most
important judgment in \rname, as we will explain next.

\begin{figure}[t]
  \centering
% \drules[wftc]{$[[  Ttx ok ]]$}{Well-formed type contexts}{Empty, Tvar}

% \drules[wfc]{$[[  Ttx |- Gtx ok ]]$}{Well-formed term contexts}{Empty, Var}

% \drules[wfrt]{$[[  Ttx |- rt type ]]$}{Well-formed types}{Prim, Arrow, All, Rec}

\drules[wfr]{$[[  Ttx |- r record ]]$}{Well-formed record types}{Var, Merge}

% \drules[wfcl]{$[[  Ttx |- R ok ]]$}{Well-formed constraint lists}{Nil, Cons}
  \caption{Selected rules for well-formedness of record types}
  \label{fig:rname_well_formed}
\end{figure}



\paragraph{Compatibility.}

The compatibility relation in \cref{fig:compatible} plays a central role in \rname. It is the underlying
mechanism of deciding when merging two records is ``sensible''. Informally,
$[[Ttx |- r1 # r2]]$ holds if $[[r1]]$ and $[[r2]]$ are mergeable, that is,
$[[r1]]$ lacks every field contained in $[[r2]]$ and vice versa.
Compatibility is
symmetric (\rref{cmp-Symm}) and respects type equivalence (\rref{cmp-Eq}).
\Rref{cmp-Base} says that if a record is compatible with a single-field record
$[[{l : t}]]$, it is also compatible with every record $[[{l : t'}]]$. A type variable is compatible
with the records in its constraint list (\rref{cmp-Tvar}). Two single-field
records are compatible if they have different labels (\rref{cmp-BaseBase}). The
rest are self-explanatory.

% and we refer the reader to their paper for further explanations.

% The judgment of constraint list satisfaction $[[Ttx |- r # R]]$
% ensures that $[[r]]$ must be compatible with every record in the constraint list $[[R]]$.
% With the compatibility rules, let us go back to the definition of $\mathsf{mergeRcd}$
% and see why it can type check, i.e.,  why $[[a1]]$ and $[[a2]]$ are compatible---because
% $[[a1]]$ is in the constraint list of $[[a2]]$, and by \rref{cmp-Tvar}, they are compatible.


\begin{figure}[t]
  \centering
\drules[cmp]{$[[  Ttx |- r1 # r2 ]]$}{Compatibility}{Eq, Symm, Base, Tvar, MergeE,Empty,MergeI,BaseBase}
% \drules[cmpList]{$[[  Ttx |- r # R ]]$}{Constraint list satisfaction}{Nil, Cons}
\caption{Compatibility}
\label{fig:compatible}

\end{figure}

\begin{figure}[t]
  \centering
\drules[teq]{$[[  rt1 ~ rt2 ]]$}{Type equivalence}{MergeUnit,MergeAssoc,MergeComm,CongAll}
\drules[ceq]{$[[  R1 ~ R2 ]]$}{Constraint list equivalence}{Swap,Empty,Merge,Dupl,Base}
\caption{Selected type equivalence rules}
\label{fig:type_equivalence}
\end{figure}

\paragraph{Type equivalence.}

Unlike \fnamee, \rname does not have subtyping. Instead, \rname uses type
equivalence to convert terms of one type to another. A selection of the rules
defining equivalence of types and constraint lists appears in
\cref{fig:type_equivalence}. The relation $[[rt1 ~ rt2]]$ is an
equivalence relation, and is a congruence with respect to the type constructors.
Finally merge is associative (\rref{teq-MergeAssoc}), commutative
(\rref{teq-MergeComm}), and has $[[Empty]]$ as its unit (\rref{teq-MergeUnit}).
As a consequence, records are identified up to permutations. Since the
quantifier in \rname is constrained, the equivalence of constrained
quantification (\rref{teq-CongAll}) relies on the equivalence of constraint
lists $[[R1 ~ R2]]$. Again, it is an equivalence relation, and respects
type equivalence. Constraint lists are essentially finite sets, so order and
multiplicity of constraints are irrelevant (\rref{ceq-Swap,ceq-Dupl}). Merges of
constraints can be ``flattened'' (\rref{ceq-Merge}), and occurences of
$[[Empty]]$ may be eliminated (\rref{ceq-Empty}). The last rule \rref*{ceq-Base}
is quite interesting: it says that the types of single-field records are
ignored. The reason is that as far as compatibility is concerned, only labels
matter, thus changing the types of records will not affect their compatibility
relation. We will have more to say about this in \cref{sec:trouble}, in
particular, this is the rule that forbids a simple translation.

% \begin{remark}
% \jeremy{If we have space trouble, we can delete this}
%   The original rules of type equivalence~\cite{Harper:1991:RCB:99583.99603} do
%   not have contexts (i.e.,  judgment of the form $[[rt1 ~ rt2]]$). However this is incorrect, as it invalidates one of the key
%   lemmas (Lemma 2.3.1.7) in their type system, which says that
%     if $[[Ttx |- r1 # r2]]$, then $[[Ttx |- r1 record]]$ and $[[Ttx |- r2 record]]$.
%   Consider two types $[[  {l1 : nat}  ]]$ and $[[ {l2 : \/ a # {l : nat} || {l : bool} . nat  }   ]]$.
%   According to the original rules, they are compatible because
%   \begin{inparaenum}[(1)]
%   \item $[[  {l1 : nat} ]]  $ is compatible with $ [[ {l2 : \/ a # {l : nat} , {l : bool} . nat }  ]]$ by \rref{cmp-BaseBase};
%   \item $ [[ {l2 : \/ a # {l : nat} , {l : bool} . nat }  ~ {l2 : \/ a # {l : nat} || {l : bool} . nat } ]]$.
%   \end{inparaenum}
%   Then it follows that $[[ {l2 : \/ a # {l : nat} || {l : bool} . nat } ]]$ is well-formed.
%   However, this record type is not well-formed in any context because $[[{l : nat} || {l : bool}]]$
%   is not well-formed in any context. To fix this, we add context throughout type equivalence.
%   % The culprit is \rref{ceq-Merge}---the well-formedness of $[[ r1 , (r2, R) ]]$
%   % does not necessarily entail the well-formedness of $[[ (r1 || r2) ,R]]$, as
%   % the latter also requires the compatibility of $[[r1]]$ and $[[r2]]$.
%   % That is why we need to explicitly add contexts to type equivalence
%   % so that $ [[ {l2 : \/ a # {l : nat} , {l : bool} . nat } ]] $ and $[[ {l2 : \/ a # {l : nat} || {l : bool} . nat } ]]$
%   % are not considered equivalent in the first place.
% \end{remark}


\paragraph{Typing rules.}

A selection of typing rules are shown in \cref{fig:typing_rname}. At
first reading, the gray parts can be ignored, which will be covered in
\cref{sec:row_trans}. % Most of the typing rules are quite standard.
% Typing is
% invariant under type equivalence (\rref{wtt-Eq}).
Two terms can be merged if their types are compatible (\rref{wtt-Merge}). Type
application $[[ re [ r ] ]]$ is well-typed if the type argument $[[r]]$
satisfies the constraints $[[R]]$ (\rref{wtt-AllE}).


\begin{remark}
  A few simplifications have been placed compared to the original \rname,
  notably the typing of record selection (\rref{wtt-Select}) and restriction
  (\rref{wtt-Restr}). In the original formulation, both typing rules need a
  partial function $ r \_ l $ which means the type associated with label $[[l]]$
  in $[[r]]$. Instead of using partial functions, here we explicitly expose the
  expected label in a record. It can be shown that if label $[[l]]$ is present
  in record type $[[r]]$, then the fields in $[[r]]$ can be rearranged so that
  $[[l]]$ comes first by type equivalence. This formulation was also adopted by
  Leijen~\cite{leijen2005extensible}.
\end{remark}


\begin{figure}[t]
  \centering
\drules[wtt]{$[[  Ttx ; Gtx |- re : rt ~~> ee  ]]$}{Type-directed translation}{Base,Restr,Select,Empty,Merge,AllE,AllI}
\caption{Selected typing rules with translations}
\label{fig:typing_rname}
\end{figure}



\renewcommand{\rulehl}[1]{#1}

\subsection{A Failed Attempt}
\label{sec:trouble}

In this section, we sketch out an intuitive translation scheme.
On the syntactic level, it is pretty straightforward to see a one-to-one
correspondence between \rname terms and \fnamee expressions. % For example,
% constrained type abstractions $[[/\ a # R . re ]]$ correspond to \fnamee type
% abstractions $[[ \ a ** A . ee]]$; record merges can be simulated by the more
% general merge operator of \fnamee; record restriction can be modeled as annotate terms, and so on.
On the semantic level, all well-formedness judgments of \rname have corresponding well-formedness judgments
of \fnamee, given a ``suitable'' translation function $[[< rt >]]$ from \rname types to \fnamee types
Compatibility relation corresponds to disjointness relation. What might not be
so obvious is that type equivalence can be expressed via subtyping. More
specifically, $[[ rt1 ~ rt2 ]]$ induces mutual subtyping relations
$[[ < rt1 > <: < rt2 > ]]$ and $[[ < rt2 > <: < rt1 > ]]$.
Informally, type-safety of translation is something along the lines of
``if a term has type $[[rt]]$, then its translation has type $[[< rt > ]]$''.
With all these in mind, let us consider two examples:

\begin{example} \label{eg:1} %
  Consider term $[[ /\ a # {l : nat} . \x : a . x ]]$. It could be
  assigned type (among others) $[[ \/ a # {l : nat} . a -> a ]]$, and its ``obvious'' translation
  $[[  \ X ** {l : nat} . \ x : X . x  ]]$ has type $[[ \ X ** { l : nat} . X -> X   ]]$, which corresponds very well to
  $[[ \/ a # {l : nat} . a -> a  ]]$. The same term could also be assigned type $[[  \/ a # {l : bool} . a -> a   ]]$, since
  $[[  \/ a # {l : bool} . a -> a   ]]$ is equivalent to $[[  \/ a # {l : nat} . a -> a   ]]$ by \rref{teq-CongAll,ceq-Base}. However,
  as far as \fnamee is concerned, these two types  have no relationship at all---$[[  \ X ** {l : nat} . \ x : X . x  ]]$
  cannot have type $[[  \ X ** {l : bool} . X -> X   ]]$, and indeed it should not, as these two types have completely different meanings!
\end{example}

\begin{remark}
  Interestingly, the algorithmic system of \rname can only infer
  type $[[ \/ a # {l : nat} . a -> a ]]$ for the aforementioned term.
  To relate to the declarative system (in particular, to prove completeness of the algorithm),
  they show that the type inferred by the algorithm is equivalent
  (in the sense of type equivalence) to the assignable type in the declarative system.
  Proving type-safety of translation is, in a sense, like proving completeness. So
  maybe we should change the type-safety statement to
  ``if a term has type $[[rt]]$, then there exists type $[[rt']]$ such that $[[ rt ~ rt' ]]$ and the
  translation has type $[[ < rt' > ]]$''. As we shall see, this is still incorrect.
\end{remark}

\begin{example} \label{eg:2} %
  Consider term $[[re]] = [[  /\ a # {l : bool} . \ x : a . \ y : {l : nat} . x || y  ]]$.
  It has type $[[ \/ a # {l : bool} . a -> {l : nat} -> a || {l : nat}    ]]$, and
  its ``obvious'' translation is $[[ee]] = [[ \ X ** {l : bool} . \x  : X . \ y : {l : nat} . x ,, y  ]]$.
  However, expression $[[ee]]$ is ill-typed in \fnamee
  for the reasons of coherence: think about
  the result of evaluating $[[ (ee {l : nat} {l = 1} {l = 2}).l ]]$---it could evaluate to $1$ or $2$!
\end{example}

\begin{remark}
  Let us think about why \rname allows type-checking $[[re]]$. Unlike \fnamee,
  the existence of $[[re]]$ in \rname will not cause incoherence because \rname
  would reject type application $[[re [{l : nat}] ]]$ in the first place---more
  generally, $[[re]]$ can only be applied to records that do not contain label
  $[[l]]$ due to the stringency of the compatibility relation. This example
  underlines a crucial difference between the compatibility relation and the
  disjointness relation. The former can only relate records with different
  labels, whereas the latter is more fine-grained in the sense that it can also
  relate records with the same label (\rref{D-rcdEq}). Note that \rref{D-rcdEq}
  is very important for the applications of extensible designs, as we need to
  combine records with the same label, which is impossible to do in \rname.
\end{remark}


\paragraph{Taming \rname.}

It seems to imply that \rname and \fnamee are incompatible in that there are
some \rname programs that are not typable in \fnamee, and vice versa. A careful
comparison between the two calculi reveals that \rref{cmp-Base,ceq-Base} are
``to blame''. For \rname in general, these two rules are reasonable, as ``the
relevant properties of a record, for the purposes of consistency checking, are
its atomic components''~\cite{Harper:1991:RCB:99583.99603}. As far as
compatibility is concerned, a constraint list is just a list of labels and type
variables, whereas in \fnamee, disjointness constraints also care about record
types. This subtle discrepancy tells that we should have a different treatment
for those records that appear in a constraint list from those that appear
elsewhere: we translate a single-field record $[[ {l : rt}
]]$ in a constraint list to $[[ { l : Bot} ]]$. For \cref{eg:1}, both $[[ \/ a #
{l : nat} . a -> a ]]$ and $[[ \/ a # {l : bool} . a -> a ]]$ translate to $[[ \
X ** { l : Bot} . X -> X ]]$. For \cref{eg:2}, $[[re]]$ is translated to
$[[ee']] = [[ \ X ** {l : Bot} . \x : X . \ y : {l : nat} . x ,, y ]]$, which
type checks in \fnamee. Moreover, $[[ee' {l : nat} ]]$ gets rejected because
$[[Bot]]$ is not disjoint with $[[nat]]$.



\subsection{Type-Directed Translation}
\label{sec:row_trans}

\begin{figure}[t]
  \centering
\begin{tabular}{rrlllrlll} \toprule
  $[[< rt >]] \defeq \,$ & $[[ <base> ]]$ & $=$ &  $[[nat]]$ & $,$ & $[[< rt1 -> rt2 >]]$ & $=$ & $[[<rt1> -> <rt2>]]$ & $,$  \\
                       &$[[< \/ a # R . rt>]]$ & $=$ &  $[[\ X ** <R>. \ Xb ** <R>. <rt>]]$ & $,$ & & & & \\
                       &$[[ <a> ]]$ & $=$ &  $[[X]]$ & $,$ & $[[< Empty > ]]$ & $=$ & $[[Top]] $ & $,$ \\
                       &$[[ <{l:rt}> ]]$ & $=$ &  $[[ {l:<rt>} ]]$ & $,$ & $[[<r1 ||  r2> ]]$ & $=$ & $[[<r1> & <r2>]] $ \\
  $[[ <r>_b  ]] \defeq\,$ & $[[ <a>_b ]]$ & $=$ &  $[[Xb]]$ & $,$ & $[[< Empty >_b ]]$ & $=$ & $[[Top]] $ & $,$  \\
                       &$[[ <{l:rt}>_b ]]$ & $=$ &  $[[ {l:Bot} ]]$ & $,$ & $[[<r1 ||  r2>_b ]]$ & $=$ & $[[<r1>_b & <r2>_b]] $ \\
  $[[< R >]] \defeq \,$  &$[[ < <> >  ]]$ & $=$ &  $[[ Top  ]]$ & $,$ & $[[  <r , R>    ]]$ & $=$ & $[[<r>_b & <R>]] $ \\
  $[[< Ttx >]] \defeq \,$ & $[[ < <> > ]]$ & $=$ &  $[[empty]]$ & $,$ & $[[ <Ttx, a # R>  ]]$ & $=$ & $[[ <Ttx>, X ** <R>, Xb ** <R>]] $  \\
  $[[ < Gtx> ]] \defeq \,$ & $[[ < <> > ]]$ & $=$ &  $[[empty]]$ & $,$ & $[[ <Gtx, x : rt>  ]]$ & $=$ & $[[  <Gtx>, x : <rt>   ]] $ \\   \bottomrule
\end{tabular}
\caption{Translation functions}
\label{fig:trans_func}
\end{figure}




Now we can give a formal account of the translation. But there is still a twist.
Having two ways of translating records does not work out of the box. To see
this, consider $[[ \/ a # b . b ]]$, and note that a reasonably defined translation function
should commute with substitution, i.e., $[[ < [r / a] rt > ]] = [[ <rt> [X ~> <r>] ]] $.
We have LHS:
$$[[ < [ {l : nat} / b  ] (\/ a # b . b ) >  ]] =  [[  < \/ a # {l : nat} . { l : nat} > ]] = [[  \ X ** { l : Bot} . { l : nat}   ]]  $$
which is not the same as RHS:
$$[[ <\/ a # b . b>  [Y ~> < {l : nat} > ]       ]] = [[ ( \ X ** Y . Y) [Y ~> < {l : nat} > ]   ]] = [[   \ X ** {l : nat} . {l : nat}    ]]  $$
The tricky part is that we should also distinguish those record type variables
that appear in a constraint list from those that appear elsewhere. To do so, we
map record type variable $[[a]]$ to a pair of type variables $[[ X ]]$ and
$[[Xb]]$, where $[[Xb]]$ is supposed to be substituted by records with bottom
types. More specifically, we define the translation functions as in
\cref{fig:trans_func}. There are two ways of translating records: $[[<r>]]$ for
regular translation and $[[ < r >_b ]]$ for bottom translation; the latter is
used by $[[< R >]]$ for translating constraint lists. The most interesting one
is translating quantifiers: each quantifier $[[\/ a # R . rt]]$ in \rname is
split into two quantifiers $[[ \ X ** <R>. \ Xb ** <R>. <rt> ]]$ in \fnamee.
Correspondingly, each record type variable $[[a]]$ is translated to either
$[[X]]$ or $[[Xb]]$, depending on whether it appears in a constraint list or
not. The relative order of $[[X]]$ and $[[Xb]]$ is not so much relevant, as long
as we respect the order when translating type applications. Now let us go back
to the gray parts in \cref{fig:typing_rname}. In the type application $[[ re [ r
] ]]$ (\rref{wtt-AllE}), we first translate $[[e]]$ to $[[ee]]$. The translation
$[[ee]]$ is then applied to two types $[[ <r> ]]$ and $[[ <r >_b ]]$, because as
we mentioned earlier, $[[ee]]$ has two quantifiers resulting from the
translation. It is of great importance that the relative order of $[[<r>]]$ and
$[[< r >_b]]$ should match the order of $[[ X ]]$ and $[[Xb]]$ in translating
quantifiers. There is a ``protocol'' that we must keep during translation: if
$[[X]]$ is substituted by $[[ <r> ]]$, then $[[ Xb ]]$ is substitute by $[[ < r
>_b ]]$. In other words, we can safely assume $[[ Xb <: X ]]$ because $[[ < r>_b <: <r> ]]$ always holds.
Similarly, in \rref{wtt-AllI} we translate constrained type abstractions to disjointness type abstractions
with two quantifiers, matching the translation of constrained quantification.
The other rules are mostly straightforward translations. Finally we show that our translation function does commute with
substitution, but in a slightly involved form:

\begin{restatable}{lemma}{substrt} \label{lemma:subst_rt}
  $[[ <[r / a] rt> ]]$ = $ [[ <rt> [X ~> <r>] [Xb ~> <r>_b] ]] $.
\end{restatable}

% With the modified \rname, we are now ready to explain the gray parts in \cref{fig:typing_rname}. First we
% show how to translate \rname types to \fnamee types in
% \cref{fig:type_trans_rname}. Most of them are straightforward. Record merges are
% translated into intersection types, so are the constraint lists. Next we look at the
% translations of terms. Most of the them are quite intuitive. In \rref{wtt-eq},
% we put annotation $[[ | rt' | ]]$ around the translation of $[[re]]$. Record
% restrictions are translated to annotated terms (\rref{wtt-Restr}) since we
% already know the type without label $[[l]]$. Record merges are translated to
% general merges (\rref{wtt-Merge}). The translation of record selections (\rref{wtt-Select}) is  a bit
% complicated. Note that we cannot simply translate to $[[ ee . l ]]$ because our
% typing rule for record selections (\rref{T-proj}) only applies when $[[ee]]$ is a
% single-field record. Instead, we need to first transform $[[ee]]$ to a
% single-field record by annotation, and then project.

% \begin{remark}
%   The acute reader may have noticed that in \rref{wtt-AllE}, the translation type
%   $| [[r]] |$ could be a quantifier, but our rule of type applications
%   (\rref{T-tapp}) only applies to monotypes. The reason is that, for the
%   purposes of translation, we lift the monotype restrictions, which does not
%   compromise type-safety of \fnamee.
% \end{remark}

\paragraph{Type-safety of translation.}

With everything in place, we prove that our translation in
\cref{fig:typing_rname} is type-safe. The main idea is to map each judgment in
\rname to a corresponding judgment in \fnamee: well-formedness to
well-formedness, compatibility to disjointness, type-equivalence to subtyping.
The reader can refer to \cref{appendix:proofs} for detailed proofs. We
show a key lemma that bridges the ``gap'' (i.e., \rref{cmp-Base}) between row and disjoint polymorphism.

\begin{restatable}{lemma}{cmprcd} \label{lemma:cmp-rcd}
  If $[[ Ttx |- r # {l:rt} ]]$ then $[[ < Ttx > |-  < r > ** {l:A}    ]] $ and $[[ < Ttx > |-  < r >_b ** {l:A}    ]]$
  for all $A$.
\end{restatable}

Finally here is the central type-safety theorem:

\begin{restatable}{theorem}{typesafe}
  If $[[ Ttx ; Gtx |- re : rt ~~> ee ]]$ then $[[ < Ttx > ; < Gtx > |-  ee => < rt >  ]]$.
\end{restatable}
% \begin{proof}
%   By induction on the typing derivation.
% \end{proof}




% Local Variables:
% TeX-master: "../paper"
% org-ref-default-bibliography: "../paper.bib"
% End:


\section{Related Work}
\label{sec:related}

\paragraph{Coherence.}

In calculi featuring coercive subtyping, a semantics that interprets the
subtyping judgment by introducing explicit coercions is typically defined on
typing derivations rather than on typing judgments. A natural question that
arises for such systems is whether the semantics is \emph{coherent}, i.e.,
distinct typing derivations of the same typing judgment possess the same
meaning. Since Reynolds~\cite{Reynolds_1991} proved the coherence of a calculus with
intersection types, % based on the denotational semantics for intersection types,
many researchers have studied the problem of coherence in a variety of typed
calculi. Two approaches are commonly found in the literature.
The first approach is to find a normal form for a representation of
the derivation and show that normal forms are unique for a given typing
judgment~\cite{Breazu_Tannen_1991,Curien_1992,SCHWINGHAMMER_2008}.
However, this approach cannot be directly applied to Curry-style
calculi (where the lambda abstractions are not type annotated).
% Also this line of reasoning cannot be used when the calculus has general recursion.
Biernacki and Polesiuk~\cite{biernacki2015logical} considered the coherence
problem of coercion semantics. Their criterion for coherence of the translation
is \emph{contextual equivalence} in the target calculus. % They presented a
% construction of logical relations for establishing so constructed coherence for
% coercion semantics, showing that this approach is applicable in a variety of
% calculi, including delimited continuations and control-effect subtyping.
Inspired by this approach, Bi et al.~\cite{bi_et_al:LIPIcs:2018:9227} proposed the canonicity relation
to prove coherence for a calculus with disjoint intersection types and BCD
subtyping. As we have shown in \cref{sec:coherence:poly}, constructing a
suitable logical relation for \fnamee is challenging. On the one hand, the
original approach by Alpuim et al.~\cite{alpuimdisjoint} in \fname does not work
any more due to the addition of BCD subtyping. On the other hand, simply
combining System F's logical relation with \namee's canonicity relation does not
work as expected, due to the issue of well-foundedness. To solve the problem, we
employ immediate substitutions and a restriction to predicative instantiations.
%Our work is the first to use logical relations to show coherence for a
%rich calculus with
%disjoint intersection types, a merge operator, BCD subtyping and
%parametric polymorphism.



% \bruno{Drop most of what follows and focus
% on summarizing the challenges of adding disjoint quantification.}

% BCD subtyping in our setting poses a non-trivial complication over
% their simple structural subtyping. Indeed, because any two
% coercions between given types are behaviorally equivalent in the target
% language, their coherence reasoning can all take place in the target language.
% This is not true in our setting, where some coercions can be distinguished by
% arbitrary target programs, but not those that are elaborations of source
% programs. % (e.g., $ [[\x . pp1 x]] $ and $ [[ \x . pp2 x]] $ should be equated in
% % our setting.)
% Hence, we have to restrict our reasoning to the latter class,
% which is reflected in a more complicated notion of contextual equivalence and
% our logical relation's non-trivial treatment of pairs.

\paragraph{BCD subtyping and decidability.}

The BCD type system was first introduced by Barendregt et
al.~\cite{Barendregt_1983} to characterize exactly the strongly normalizing
terms. The BCD type system features a powerful subtyping relation, which serves
as a base for our subtyping relation.
% Bessai el at.\cite{DBLP:journals/corr/BessaiDDCd15} show how to type classes and mixins in a
% BCD-style record calculus with a merge-like operator~\cite{bracha1990mixin}
% that only operates on records, and they only study a type assignment system.
The decidability of BCD subtyping has been shown in several
works~\cite{pierce1989decision, Kurata_1995, Rehof_2011, Statman_2015}.
Laurent~\cite{laurent2012intersection} formalized the relation in Coq in order
to eliminate transitivity cuts from it, but his formalization does not deliver
an algorithm. Only recently, Laurent~\cite{Laurent18b} presented a general way of
defining a BCD-like subtyping relation extended with generic
contravariant/covariant type constructors that enjoys the ``sub-formula
property''. Our Coq formalization extends the approach used in \namee,
which follows a different idea based on Pierce's
decision procedure~\cite{pierce1989decision},
with parametric (disjoint) polymorphism and corresponding
distributivity rules. More recently,
Muehlboeck and Tate~\cite{muehlboeck2018empowering} presented a
decidable algorithmic system (proved in Coq)
with union and intersection types. Similar to \fnamee,
their system also has distributive subtyping rules. They also 
discussed the addition of polymorphism, but left a Coq formalization
for future work. In their work they regard intersections of disjoint
types (e.g., $[[str & nat]]$) as uninhabitable, which is different
from our interpretation. 
As a consequence, coherence is a non-issue for them.




\paragraph{Intersection types, the merge operator and polymorphism.}

Forsythe~\cite{reynolds1988preliminary} has intersection types and a merge-like
operator. However to ensure coherence, various restrictions were added to limit
the use of merges. In Forsythe merges cannot contain more than one function.
% whereas our merges do not have this restriction.
Castagna et al.~\cite{Castagna_1992} proposed a coherent calculus $\lambda \&$ to study
overloaded functions. $\lambda \&$ has a special merge operator
that works on functions only. % Like ours, they also impose well-formedness
% conditions on the formation of a (functional) merge. However, those conditions
% operate on function types only, and it is not clear how to generalize them to
% arbitrary types.
Dunfield proposed a calculus~\cite{dunfield2014elaborating}
(which we call \dname) that shows significant expressiveness of type systems
with unrestricted intersection types and an (unrestricted) merge operator.
However, because of his unrestricted merge operator (allowing $[[1,,2]]$), his calculus lacks coherence.
Blaauwbroek's \lname~\cite{lasselambda} enriched \dname with BCD subtyping and
computational effects, but he did not address coherence.
The coherence issue for a calculus similar to \dname was first addressed in
\oname~\cite{oliveira2016disjoint} with the notion of disjointness,
but at the cost of dropping unrestricted intersections, and a strict
notion of coherence (based on $\alpha$-equivalence). Later Bi et
al.~\cite{bi_et_al:LIPIcs:2018:9227} improved calculi with disjoint intersection
types by removing several restrictions, adopted BCD subtyping
and a semantic notion of coherence (based on contextual equivalence)
proved using canonicity. The combination of intersection types, a merge operator and
parametric polymorphism, while achieving coherence was first studied in
\fname~\cite{alpuimdisjoint}, which serves as a foundation for \fnamee. However,
\fname suffered the same problems as \oname. 
Additionally in \fname a bottom type is problematic due
to interactions with disjoint polymorphism and the lack of
unrestricted intersections. The issues can be illustrated with the 
well-typed \fnamee expression $[[ \ X ** Bot . \ x : X . x ,, x     ]]$.
In this expression the type of $[[x ,, x]]$ is
$[[ X & X ]]$. Such a merge does not violate disjointness because
the only types that $[[X]]$ can be instantiated with are top-like, and
top-like types do not introduce incoherence.
In \fname a type variable $[[X]]$ can never be disjoint to another
type that contains $[[X]]$, but (as the previous expression shows)
the addition of a bottom type allows
expressions where such (strict) condition does not hold.
In this work, we removed those
restrictions, extended BCD subtyping with polymorphism, and proposed a more
powerful logical relation for proving coherence. \Cref{fig:compare} summarizes
the main differences between the aforementioned calculi.


\newcommand{\tikzcircle}[2][black,fill=black]{\tikz[baseline=-0.5ex]\draw[#1,radius=#2] (0,0) circle ;}%
\newcommand{\halfcircle}{\tikz[baseline=-0.5ex]{\draw[black, fill=white] (0, 0.07) arc (90:270:0.07); \draw[black, fill=black](0, -0.07) arc (-90:90:0.07);}}
\newcommand{\emptycircle}{\tikzcircle[black,fill=white]{2pt}\xspace}
\newcommand{\fullcircle}{\tikzcircle{2pt}\xspace}

\begin{figure}[t]
  \centering
  \newcolumntype{L}{>{\raggedright\arraybackslash}p{4cm}}
  \newcolumntype{C}{>{\centering\arraybackslash}p{1.2cm}}
\begin{tabular}{LCCCCCC} \toprule
  & \dname~\cite{dunfield2014elaborating} & \oname~\cite{oliveira2016disjoint} & \lname~\cite{lasselambda}  & \namee~\cite{bi_et_al:LIPIcs:2018:9227} & \fname~\cite{alpuimdisjoint}  & \fnamee \\ \midrule
Disjointness &  \emptycircle &  \fullcircle &  \emptycircle  &    \fullcircle   & \fullcircle   &   \fullcircle   \\
Unrestricted intersections &  \fullcircle &  \emptycircle &  \fullcircle  &  \fullcircle &   \emptycircle   &   \fullcircle   \\
BCD subtyping &  \emptycircle & \emptycircle &  \fullcircle  &  \fullcircle             &  \emptycircle         &  \fullcircle  \\
Polymorphism  &  \emptycircle &  \emptycircle  &  \emptycircle  &  \emptycircle &  \fullcircle  &  \fullcircle \\
Coherence &  \emptycircle &  \halfcircle &  \emptycircle  &  \fullcircle &   \halfcircle   &   \fullcircle   \\
Bottom type &  \emptycircle & \emptycircle &  \fullcircle  &  \emptycircle &  \emptycircle & \fullcircle   \\
  \bottomrule
\end{tabular}
\caption{Summary of intersection calculi (\fullcircle $=$ yes, \emptycircle $=$ no, \protect\halfcircle\ $=$ syntactic coherence)}
  \label{fig:compare}
\end{figure}



%%\paragraph{Other forms of intersection types.}

There are also several other calculi with intersections and polymorphism. Pierce
proposed $\mathsf{F}_{\land}$~\cite{pierce1991programming}, a calculus combining
intersection types and bounded quantification. Pierce translates
$\mathsf{F}_{\land}$ to System F extended with products, but he left coherence
as a conjecture. More recently, Castagna et al.~\cite{castagna2014polymorphic}
proposed a polymorphic calculus with set-theoretic type connectives
(intersections, unions, negations). But their calculus does not include a merge
operator. Castagna and Lanvin also proposed a gradual type
system~\cite{Castagna_2017} with intersection and union types, but also without
a merge operator.
% Moreover, their intersections are only used between function types,
% allowing overloading of types. Also they adopted the \emph{semantic} approach
% for defining the subtyping relation. The benefit of this approach, compared with
% the more used \emph{syntactic} approach, is that the subtyping relation is by
% definition \emph{complete}. In that regard, their subtyping relation thus
% completely subsumes BCD subtyping.
%Bi and Oliveira recently proposed
%a typed model~\cite{bi_et_al:LIPIcs:2018:9214} for first-class traits
%using an elaboration into \fname. Our implementation is built on \sedel extended
%with BCD-style subtyping.


% They also
% showed that dynamic inheritance, self-references and abstract methods can all be
% encoded by employing ideas from the denotational model of
% inheritance~\cite{cook1989denotational}.


% \subsection{Intersection Types and Multiple Inheritance}

% Compagnoni and Pierce~\cite{compagnoni1996higher} proposed a lambda calculus
% $\mathsf{F}_{\land}^{\omega}$, an extension of System $\mathsf{F}^{\omega}$ with
% intersection types to model multiple inheritance. $\mathsf{F}_{\land}^{\omega}$
% allows arbitrary finite intersections, where all the type members must have the
% same kind. On the language side, modern object-oriented languages such as Scala, TypeScript,
% Flow, Ceylon, and Grace have adopted some form of intersection types. Notably,
% the DOT calculus~\cite{amin2012dependent,Rompf_2016}---a new type-theoretic
% foundation for Scala---has a native support for intersection types. Generally
% speaking, the most significant difference between our calculi and those
% languages/calculi is that they do not have an explicit introduction form of
% intersection types, like our merge operator. The lack of a native merge operator
% leads to some awkward and type-unsafe solutions for defining a merge operator in
% those languages. As noted by \cite{alpuimdisjoint}, one important use of
% intersection types in TypeScript is the following function:
% \begin{lstlisting}[language=JavaScript]
% function extend<T, U>(first: T, second : U) : T & U {...}
% \end{lstlisting}
% which is analogous to our merge operator in that it takes two objects and
% produces an object with the intersection of the types of the argument objects.
% The implementation of \lstinline{extend} relies on low-level (and type-unsafe)
% features of JavaScript. Similar encodings have also been proposed for Scala to
% enable applications where the merge operator plays a fundamental
% role~\cite{oliveira2013feature, rendel14attributes}. As we have shown in
% \cref{sec:poly:motivation}, with disjointness constraints and a built-in merge
% operator, a type-safe and conflict-free \lstinline{extend} function can be
% naturally defined.



\paragraph{Row polymorphism and bounded polymorphism.}

Row polymorphism was originally proposed by Wand~\cite{wand1987complete}
as a mechanism to enable type inference for a simple object-oriented language
based on recursive records. These ideas were later adopted into type systems for
extensible records~\cite{Harper:1991:RCB:99583.99603, leijen2005extensible, gaster1996polymorphic}.
% Cardelli and
% Mitchell~\cite{cardelli1989operations} define three primitive operations on
% records: \emph{selection}, \emph{restriction} and \emph{extension}.
Our merge operator can be seen as a generalization of record
extension/concatenation, and selection is also built-in. In contrast
to most record calculi, restriction is not a primitive operation in \fnamee,
but can be simulated via subtyping. Disjoint quantification can 
simulate the \emph{lacks} predicate often present in systems
with row polymorphism.
Recently Morris and McKinna presented a
typed language~\cite{morrisrow}, generalizing and abstracting existing systems
of row types and row polymorphism.
Alpuim et al.~\cite{alpuimdisjoint}
informally studied the relationship between row polymorphism and disjoint
polymorphism, but it would be interesting to study such relationship more
formally. The work of  Morris and McKinna may be interesting for such
study in that it gives a general framework for row type systems.

Bounded quantification is currently the dominant mechanism in major mainstream object-oriented languages 
supporting both subtyping and polymorphism.
% , and it features in essentially all
% major mainstream OO languages.
\fsub~\cite{cardelli1985understanding} provides a
simple model for bounded quantification, but type-checking in full \fsub is
proved to be undecidable~\cite{pierce1994bounded}.
Pierce's thesis~\cite{pierce1991programming} discussed the relationship between calculi
with simple polymorphism and intersection types and bounded quantification. He
observed that there is a way to ``encode'' many forms of bounded quantification
in a system with intersections and pure (unbounded) second-order
polymorphism. That encoding can be easily adapted to \fnamee:
{\small
\[
[[ \/ X <: A . B ]] \defeq [[ \ X ** Top . (B [ X ~> A & X]) ]]
\]
}%
The idea is to replace bounded quantification by (unrestricted)
universal quantification and all occurrences of $[[X]]$
by $[[A & X]]$ in the body. Such an encoding seems to indicate
that \fnamee could be used as a decidable alternative to (full) \fsub. 
It is worthwhile to note that this encoding does not
work in \fname because $[[A & X]]$ is not well-formed ($[[X]]$ is not
disjoint to $[[A]]$). In other words, the encoding requires
unrestricted intersections.


% Local Variables:
% org-ref-default-bibliography: "../paper.bib"
% TeX-master: "../paper"
% End:

\section{Conclusion}

In this article, we come up with a design of a dependently-typed calculus called \name.
\name generalizes non-dependent polymorphic subtyping by Odersky and
L\"aufer~\cite{odersky1996putting} and contains other features like general
recursion and explicit casts for type-level computations.
We adopt the techniques of the Unified Subtyping~\cite{yang2017unifying} to
avoid the mutual dependency between typing and subtyping relation to simplify
the formalization. Besides other relevant theorems about typing and subtyping,
\emph{transitivity} and \emph{type safety} are proved mechanically with a proof assistant.

In the future, we will attempt to lift various restrictions that originally simplify
the metatheory, such as the kind restriction of polymorphic
types and the runtime irrelevance of implicit arguments. Also we would like to
study the impact to the metatheory of adding $\top$ types to our language,
which is a common feature of a subtyping relation.
Most importantly, we consider the development
of a well-specified algorithmic system a major challenge in our future
work. The current formulation of \name is declarative due to the mono-expression guesses.


\bibliographystyle{splncs04}
\bibliography{paper}


% \newpage
% \appendix
% 
\section{Full Typing Rules of \fnamee}
\label{appendix:fnamee}

\drules[swfte]{$[[||- DD]]$}{Well-formedness}{empty, var}

\drules[swfe]{$[[DD ||- GG]]$}{Well-formedness}{empty, var}

\drules[swft]{$[[DD |- A]]$}{Well-formedness of type}{top, bot, nat, var, rcd, arrow, all, and}

\drules[S]{$ [[A <: B ~~> c]]  $}{Declarative subtyping}{refl,trans,top,rcd,andl,andr,arr,and,distArr,topArr,distRcd,topRcd,bot,forall,topAll,distAll}

\drules[TL]{$[[ A top  ]]$}{Top-like types}{top,and,arr,rcd,all}

\drules[D]{$[[DD |- A ** B]]$}{Disjointness}{topL, topR, arr, andL, andR, rcdEq, rcdNeq, tvarL, tvarR, forall,ax}

\drules[Dax]{$[[A **a B]]$}{Disjointness Axiom}{intArr, intRcd, intAll, arrAll, arrRcd, allRcd}

\textbf{Note:}   For each form $[[A **a B]]$, we also have a symmetric one $[[B **a A]]$.


\drules[T]{$[[DD; GG |- ee => A ~~> e]]$}{Inference}{top, nat, var, app, merge, anno, tabs, tapp, rcd, proj}

\drules[T]{$[[DD ; GG |- ee <= A ~~> e]]$}{Checking}{abs, sub}

\begin{definition}
  \begin{align*}
    [[ < [] >1 ]] &=  [[top]] \\
    [[ < l , fs >1 ]] &= [[ < fs >1 o id  ]] \\
    [[ < A , fs >1 ]] &= [[(top -> < fs >1) o topArr]] \\
    [[ < X ** A , fs >1 ]] &= [[ \ < fs >1 o topAll ]] \\ \\
    [[ < [] >2 ]] &=  [[id]] \\
    [[ < l , fs >2 ]] &= [[ < fs >2 o id  ]] \\
    [[ < A , fs >2 ]] &= [[(id -> < fs >2) o distArr]] \\
    [[ < X ** A , fs >2 ]] &= [[ \ < fs >2 o distPoly]]
  \end{align*}
\end{definition}

\drules[A]{$[[fs |- A <: B ~~> c]]$}{Algorithmic subtyping}{const, top, bot,and,arr,rcd,forall,arrConst,rcdConst,andConst,allConst}


\drules[CTyp]{$[[CC : (DD ;  GG => A ) ~> (DD' ; GG' => B ) ~~> cc]]$}{Context typing I}{emptyOne, appLOne, appROne, mergeLOne, mergeROne, rcdOne, projOne, annoOne, tabsOne,tappOne}

\drules[CTyp]{$[[CC : ( DD ; GG <= A ) ~> (DD' ; GG' <= B ) ~~> cc]]$}{Context typing II}{emptyTwo, absTwo}

\drules[CTyp]{$[[CC : ( DD ; GG <= A ) ~> (DD' ; GG' => B ) ~~> cc]]$}{Context typing III}{appLTwo, appRTwo, mergeLTwo, mergeRTwo, rcdTwo, projTwo, annoTwo, tabsTwo, tappTwo}

\drules[CTyp]{$[[CC : ( DD ; GG => A ) ~> ( DD' ; GG' <= B ) ~~> cc]]$}{Context typing IV}{absOne}



\section{Full Typing Rules of \tnamee}

\drules[wfe]{$[[ dd |- gg   ]]$}{Well-formedness of value context}{empty, var}

\drules[wft]{$[[ dd |- T   ]]$}{Well-formedness of types}{unit, nat, var, arrow,prod, all}

\drules[ct]{$[[ c |- T1 tri T2  ]]$}{Coercion typing}{refl,trans,top,bot,topArr,arr,pair,distArr,distAll,projl,projr,forall,topAll}

\drules[t]{$[[ dd ; gg |- e : T ]]$}{Static semantics}{unit, nat, var, abs, app, tabs, tapp, pair, capp}

\drules[r]{$[[e --> e']]$}{Single-step reduction}{topArr,topAll,distArr,distAll,id,trans,top,arr,pair,projl,projr,forall,app,tapp,ctxt}

% \section{Well-Foundedness}

% \wellfounded*


\section{Decidability}
\label{appendix:decidable}

\begin{definition}[Size of $[[fs]]$]
  \begin{align*}
    size([[ [] ]]) &=  0 \\
    size([[ fs, l ]]) &= size([[ fs ]]) \\
    size([[ fs, A ]]) &= size([[ fs ]]) + size ([[ A ]]) \\
    size([[ fs, X ** A ]]) &= size([[ fs ]]) + size([[ A ]]) \\
  \end{align*}
\end{definition}

\begin{definition}[Size of types]
  \begin{align*}
    size([[ rho ]]) &= 1 \\
    size([[ A -> B ]]) &= size([[A]]) + size([[B]]) + 1 \\
    size([[ A & B ]]) &= size([[A]]) + size([[B]]) + 1 \\
    size([[ {l:A} ]]) &= size([[A]]) + 1 \\
    size([[ \X ** A. B ]]) &= size([[A]]) + size([[B]]) + 1
  \end{align*}
\end{definition}

% \begin{theorem}[Decidability of Algorithmic Subtyping]
%   \label{lemma:decide-sub}
%   Given $[[fs]]$, $[[A]]$ and $[[B]]$, it is decidable whether there exists
%   $[[c]]$, such that $[[fs |- A <: B ~~> c]]$.
% \end{theorem}
\decidesub*
\proof
Let the judgment $[[fs |- A <: B ~~> c]]$ be measured by $size([[fs]]) +
size([[A]]) + size([[B]])$. For each subtyping rule, we show that every
inductive premise
is smaller than the conclusion.

\begin{itemize}
  \item Rules \rref*{A-const,A-top, A-bot} have no premise.
    \item Case \[ \drule{A-and} \]
      In both premises, they have the same $[[fs]]$ and $[[A]]$, but $[[B1]]$
      and $[[B2]]$ are smaller than $[[B1 & B2]]$.
    \item Case \[\drule{A-arr} \]
      The size of premise is smaller than the conclusion as $size([[B1 -> B2]])
      = size([[B1]]) + size([[B2]]) + 1$.
    \item Case \[ \drule{A-rcd} \]
      In premise, the size is $size([[fs,l]]) + size ([[A]]) + size([[B]]) =
      size([[fs]]) + size([[A]]) + size([[B]])$, which is smaller than
      $size([[fs]]) + size([[A]]) + size([[{l:B}]]) = size([[fs]]) + size([[A]])
      + size([[B]]) + 1$.
    \item Case \[\drule{A-forall} \]
      The size of premise is smaller than the conclusion as $size([[fs]]) +
      size([[A]]) + size([[\X ** B1.B2]])
      = size([[fs]]) + size([[A]]) + size([[B1]]) + size([[B2]]) + 1
      > size([[fs, X ** B1]]) + size([[A]]) + size([[B2]])
      = size([[fs]]) + size([[B1]]) + size([[A]]) + size([[B2]])$.
    \item Case \[\drule{A-arrConst} \]
      In the first premise, the size is smaller than the conclusion because of
      the size of $[[fs]]$ and $[[A2]]$. In the second premise, the size is
      smaller than the conclusion because $size([[A1 -> A2]]) > size([[A2]])$.
    \item Case \[\drule{A-rcdConst} \]
      The size of premise is smaller as $size([[ l, fs ]]) + size([[{l:A}]]) +
      size([[rho]])
      = size([[fs]]) + size([[A]]) + size([[rho]]) + 1
      > size([[fs]]) + size([[A]]) + size([[rho]])$.
    \item Case \[\drule{A-andConst} \]
      The size of premise is smaller as $size([[A1 & A2]]) = size([[A1]]) +
      size([[A2]]) + 1 > size([[Ai]])$.
    \item Case \[\drule{A-allConst} \]
      In the first premise, the size is smaller than the conclusion because of
      the size of $[[fs]]$ and $[[A2]]$. In the second premise, the size is
      smaller than the conclusion because $size([[\Y**A1. A2]]) > size([[A2]])$.
\end{itemize}
\qed

\begin{lemma}[Decidability of Top-like types]
  \label{lemma:decide-top}
  Given a type $[[A]]$, it is decidable whether $[[ A top ]]$.
\end{lemma}
\proof Induction on the judgment $[[A top]]$. For each subtyping rule, we show
that every inductive premise is smaller than the conclusion.
\begin{itemize}
  \item \rref{TL-top} has no premise.
  \item Case \[\drule{TL-and}\]
    $size([[A & B]]) = size([[A]]) + size([[B]]) + 1$, which is greater than
    $size([[A]])$ and $size([[B]])$.
  \item Case \[\drule{TL-arr}\]
    $size([[A -> B]]) = size([[A]]) + size([[B]]) + 1$, which is greater than
    $size([[B]])$.
  \item Case \[\drule{TL-rcd}\]
    $size([[{l:A}]]) = size([[A]]) + 1$, which is greater than
    $size([[A]])$.
  \item Case \[\drule{TL-all}\]
    $size([[\X ** A. B]]) = size([[A]]) + size([[B]]) + 1$, which is greater than
    $size([[B]])$.
\end{itemize}
\qed

\begin{lemma}[Decidability of disjointness axioms checking]
  \label{lemma:decide-disa}
  Given $[[A]]$ and $[[B]]$, it is decidable whether $[[ A **a B ]]$.
\end{lemma}
\proof $[[ A **a B ]]$ is decided by the shape of types, and thus it's
decidable. 
\qed

% \begin{theorem}[Decidability of disjointness checking]
%   \label{lemma:decide-dis}
%   Given $[[DD]]$, $[[A]]$ and $[[B]]$, it is decidable whether $[[ DD |- A ** B ]]$.
% \end{theorem}
\decidedis*
\proof
Let the judgment $[[ DD |- A ** B ]]$ be measured by $ size([[A]]) +
size([[B]])$. For each subtyping rule, we show that every inductive premise
is smaller than the conclusion.
\begin{itemize}
\item Case \[\drule{D-topL}\]
  By \cref{lemma:decide-top}, we know $[[A top]]$ is decidable.
\item Case \[\drule{D-topR}\]
  By \cref{lemma:decide-top}, we know $[[B top]]$ is decidable.
\item Case \[\drule{D-arr}\]
  $size([[A1 -> A2]]) + size ([[B1 -> B2]]) > size([[A2]]) + size([[B2]])$.
\item Case \[\drule{D-andL}\]
  $size([[A1 & A2]]) + size ([[B]])$ is greater than $size([[A1]]) +
  size([[B]])$ and $size([[A2]]) + size([[B]])$.
\item Case \[\drule{D-andR}\]
  $size([[B1 & B2]]) + size ([[A]])$ is greater than $size([[B1]]) +
  size([[A]])$ and $size([[B2]]) + size([[A]])$.
\item Case \[\drule{D-rcdEq}\]
  $size([[{l:A}]]) + size ([[{l:B}]])$ is greater than $size([[A]]) +
  size([[B]])$.
\item Case \[\drule{D-rcdNeq}\]
  It's decidable whether $[[l1]]$ is equal to $[[l2]]$.
\item Case \[\drule{D-tvarL}\]
  By \cref{lemma:decide-sub}, it's decidable whether $[[A<:B]]$.
\item Case \[\drule{D-tvarR}\]
  By \cref{lemma:decide-sub}, it's decidable whether $[[A<:B]]$.
\item Case \[\drule{D-forall}\]
  $size([[\X**A1.B1]]) + size ([[\X**A2.B2]])$ is greater than $size([[B1]]) +
  size([[B2]])$.
\item Case \[\drule{D-ax}\]
  By \cref{lemma:decide-disa} it's decidable whether $[[A **a B]]$.
\end{itemize}
\qed

% \begin{theorem}[Decidability of typing]
%   \label{lemma:decide-typing}
%   Given $[[DD]]$, $[[GG]]$, $[[ee]]$ and $[[A]]$, it is decidable whether $[[DD ; GG  |- ee dirflag A]]$.
% \end{theorem}
% \decidetyp*
% \proof
% The typing judgment $[[DD ; GG  |- ee dirflag A]]$ is syntax-directed.
% And by \cref{lemma:decide-sub} and \cref{lemma:decide-dis}, we know that typing
% is decidable.
% \qed

\section{Circuit Embeddings}
\label{appendix:circuit}

\lstinputlisting[language=haskell,linerange=2-140]{./examples/Scan2.hs}% APPLY:linerange=DSL_FULL



\end{document}
