
\section{Related Work}
\label{sec:related}

\paragraph{Coherence.}

In calculi featuring coercive subtyping, a semantics that interprets the
subtyping judgment by introducing explicit coercions is typically defined on
typing derivations rather than on typing judgments. A natural question that
arises for such systems is whether the semantics is \emph{coherent}, i.e.,
distinct typing derivations of the same typing judgment possess the same
meaning. Since Reynolds~\cite{Reynolds_1991} proved the coherence of a calculus with
intersection types, % based on the denotational semantics for intersection types,
many researchers have studied the problem of coherence in a variety of typed
calculi. Two approaches are commonly found in the literature.
The first approach is to find a normal form for a representation of
the derivation and show that normal forms are unique for a given typing
judgment~\cite{Breazu_Tannen_1991,Curien_1992,SCHWINGHAMMER_2008}.
However, this approach cannot be directly applied to Curry-style
calculi (where the lambda abstractions are not type annotated).
% Also this line of reasoning cannot be used when the calculus has general recursion.
Biernacki and Polesiuk~\cite{biernacki2015logical} considered the coherence
problem of coercion semantics. Their criterion for coherence of the translation
is \emph{contextual equivalence} in the target calculus. % They presented a
% construction of logical relations for establishing so constructed coherence for
% coercion semantics, showing that this approach is applicable in a variety of
% calculi, including delimited continuations and control-effect subtyping.
Inspired by this approach, Bi et al.~\cite{bi_et_al:LIPIcs:2018:9227} proposed the canonicity relation
to prove coherence for a calculus with disjoint intersection types and BCD
subtyping. As we have shown in \cref{sec:coherence:poly}, constructing a
suitable logical relation for \fnamee is challenging. On the one hand, the
original approach by Alpuim et al.~\cite{alpuimdisjoint} in \fname does not work
any more due to the addition of BCD subtyping. On the other hand, simply
combining System F's logical relation with \namee's canonicity relation does not
work as expected, due to the issue of well-foundedness. To solve the problem, we
employ immediate substitutions and a restriction to predicative instantiations.
%Our work is the first to use logical relations to show coherence for a
%rich calculus with
%disjoint intersection types, a merge operator, BCD subtyping and
%parametric polymorphism.



% \bruno{Drop most of what follows and focus
% on summarizing the challenges of adding disjoint quantification.}

% BCD subtyping in our setting poses a non-trivial complication over
% their simple structural subtyping. Indeed, because any two
% coercions between given types are behaviorally equivalent in the target
% language, their coherence reasoning can all take place in the target language.
% This is not true in our setting, where some coercions can be distinguished by
% arbitrary target programs, but not those that are elaborations of source
% programs. % (e.g., $ [[\x . pp1 x]] $ and $ [[ \x . pp2 x]] $ should be equated in
% % our setting.)
% Hence, we have to restrict our reasoning to the latter class,
% which is reflected in a more complicated notion of contextual equivalence and
% our logical relation's non-trivial treatment of pairs.

\paragraph{BCD subtyping and decidability.}

The BCD type system was first introduced by Barendregt et
al.~\cite{Barendregt_1983} to characterize exactly the strongly normalizing
terms. The BCD type system features a powerful subtyping relation, which serves
as a base for our subtyping relation.
% Bessai el at.\cite{DBLP:journals/corr/BessaiDDCd15} show how to type classes and mixins in a
% BCD-style record calculus with a merge-like operator~\cite{bracha1990mixin}
% that only operates on records, and they only study a type assignment system.
The decidability of BCD subtyping has been shown in several
works~\cite{pierce1989decision, Kurata_1995, Rehof_2011, Statman_2015}.
Laurent~\cite{laurent2012intersection} formalized the relation in Coq in order
to eliminate transitivity cuts from it, but his formalization does not deliver
an algorithm. Only recently, Laurent~\cite{Laurent18b} presented a general way of
defining a BCD-like subtyping relation extended with generic
contravariant/covariant type constructors that enjoys the ``sub-formula
property''. Our Coq formalization extends the approach used in \namee,
which follows a different idea based on Pierce's
decision procedure~\cite{pierce1989decision},
with parametric (disjoint) polymorphism and corresponding
distributivity rules. More recently,
Muehlboeck and Tate~\cite{muehlboeck2018empowering} presented a
decidable algorithmic system (proved in Coq)
with union and intersection types. Similar to \fnamee,
their system also has distributive subtyping rules. They also 
discussed the addition of polymorphism, but left a Coq formalization
for future work. In their work they regard intersections of disjoint
types (e.g., $[[str & nat]]$) as uninhabitable, which is different
from our interpretation. 
As a consequence, coherence is a non-issue for them.




\paragraph{Intersection types, the merge operator and polymorphism.}

Forsythe~\cite{reynolds1988preliminary} has intersection types and a merge-like
operator. However to ensure coherence, various restrictions were added to limit
the use of merges. In Forsythe merges cannot contain more than one function.
% whereas our merges do not have this restriction.
Castagna et al.~\cite{Castagna_1992} proposed a coherent calculus $\lambda \&$ to study
overloaded functions. $\lambda \&$ has a special merge operator
that works on functions only. % Like ours, they also impose well-formedness
% conditions on the formation of a (functional) merge. However, those conditions
% operate on function types only, and it is not clear how to generalize them to
% arbitrary types.
Dunfield proposed a calculus~\cite{dunfield2014elaborating}
(which we call \dname) that shows significant expressiveness of type systems
with unrestricted intersection types and an (unrestricted) merge operator.
However, because of his unrestricted merge operator (allowing $[[1,,2]]$), his calculus lacks coherence.
Blaauwbroek's \lname~\cite{lasselambda} enriched \dname with BCD subtyping and
computational effects, but he did not address coherence.
The coherence issue for a calculus similar to \dname was first addressed in
\oname~\cite{oliveira2016disjoint} with the notion of disjointness,
but at the cost of dropping unrestricted intersections, and a strict
notion of coherence (based on $\alpha$-equivalence). Later Bi et
al.~\cite{bi_et_al:LIPIcs:2018:9227} improved calculi with disjoint intersection
types by removing several restrictions, adopted BCD subtyping
and a semantic notion of coherence (based on contextual equivalence)
proved using canonicity. The combination of intersection types, a merge operator and
parametric polymorphism, while achieving coherence was first studied in
\fname~\cite{alpuimdisjoint}, which serves as a foundation for \fnamee. However,
\fname suffered the same problems as \oname. 
Additionally in \fname a bottom type is problematic due
to interactions with disjoint polymorphism and the lack of
unrestricted intersections. The issues can be illustrated with the 
well-typed \fnamee expression $[[ \ X ** Bot . \ x : X . x ,, x     ]]$.
In this expression the type of $[[x ,, x]]$ is
$[[ X & X ]]$. Such a merge does not violate disjointness because
the only types that $[[X]]$ can be instantiated with are top-like, and
top-like types do not introduce incoherence.
In \fname a type variable $[[X]]$ can never be disjoint to another
type that contains $[[X]]$, but (as the previous expression shows)
the addition of a bottom type allows
expressions where such (strict) condition does not hold.
In this work, we removed those
restrictions, extended BCD subtyping with polymorphism, and proposed a more
powerful logical relation for proving coherence. \Cref{fig:compare} summarizes
the main differences between the aforementioned calculi.


\newcommand{\tikzcircle}[2][black,fill=black]{\tikz[baseline=-0.5ex]\draw[#1,radius=#2] (0,0) circle ;}%
\newcommand{\halfcircle}{\tikz[baseline=-0.5ex]{\draw[black, fill=white] (0, 0.07) arc (90:270:0.07); \draw[black, fill=black](0, -0.07) arc (-90:90:0.07);}}
\newcommand{\emptycircle}{\tikzcircle[black,fill=white]{2pt}\xspace}
\newcommand{\fullcircle}{\tikzcircle{2pt}\xspace}

\begin{figure}[t]
  \centering
  \newcolumntype{L}{>{\raggedright\arraybackslash}p{4cm}}
  \newcolumntype{C}{>{\centering\arraybackslash}p{1.2cm}}
\begin{tabular}{LCCCCCC} \toprule
  & \dname~\cite{dunfield2014elaborating} & \oname~\cite{oliveira2016disjoint} & \lname~\cite{lasselambda}  & \namee~\cite{bi_et_al:LIPIcs:2018:9227} & \fname~\cite{alpuimdisjoint}  & \fnamee \\ \midrule
Disjointness &  \emptycircle &  \fullcircle &  \emptycircle  &    \fullcircle   & \fullcircle   &   \fullcircle   \\
Unrestricted intersections &  \fullcircle &  \emptycircle &  \fullcircle  &  \fullcircle &   \emptycircle   &   \fullcircle   \\
BCD subtyping &  \emptycircle & \emptycircle &  \fullcircle  &  \fullcircle             &  \emptycircle         &  \fullcircle  \\
Polymorphism  &  \emptycircle &  \emptycircle  &  \emptycircle  &  \emptycircle &  \fullcircle  &  \fullcircle \\
Coherence &  \emptycircle &  \halfcircle &  \emptycircle  &  \fullcircle &   \halfcircle   &   \fullcircle   \\
Bottom type &  \emptycircle & \emptycircle &  \fullcircle  &  \emptycircle &  \emptycircle & \fullcircle   \\
  \bottomrule
\end{tabular}
\caption{Summary of intersection calculi (\fullcircle $=$ yes, \emptycircle $=$ no, \protect\halfcircle\ $=$ syntactic coherence)}
  \label{fig:compare}
\end{figure}



%%\paragraph{Other forms of intersection types.}

There are also several other calculi with intersections and polymorphism. Pierce
proposed $\mathsf{F}_{\land}$~\cite{pierce1991programming}, a calculus combining
intersection types and bounded quantification. Pierce translates
$\mathsf{F}_{\land}$ to System F extended with products, but he left coherence
as a conjecture. More recently, Castagna et al.~\cite{castagna2014polymorphic}
proposed a polymorphic calculus with set-theoretic type connectives
(intersections, unions, negations). But their calculus does not include a merge
operator. Castagna and Lanvin also proposed a gradual type
system~\cite{Castagna_2017} with intersection and union types, but also without
a merge operator.
% Moreover, their intersections are only used between function types,
% allowing overloading of types. Also they adopted the \emph{semantic} approach
% for defining the subtyping relation. The benefit of this approach, compared with
% the more used \emph{syntactic} approach, is that the subtyping relation is by
% definition \emph{complete}. In that regard, their subtyping relation thus
% completely subsumes BCD subtyping.
%Bi and Oliveira recently proposed
%a typed model~\cite{bi_et_al:LIPIcs:2018:9214} for first-class traits
%using an elaboration into \fname. Our implementation is built on \sedel extended
%with BCD-style subtyping.


% They also
% showed that dynamic inheritance, self-references and abstract methods can all be
% encoded by employing ideas from the denotational model of
% inheritance~\cite{cook1989denotational}.


% \subsection{Intersection Types and Multiple Inheritance}

% Compagnoni and Pierce~\cite{compagnoni1996higher} proposed a lambda calculus
% $\mathsf{F}_{\land}^{\omega}$, an extension of System $\mathsf{F}^{\omega}$ with
% intersection types to model multiple inheritance. $\mathsf{F}_{\land}^{\omega}$
% allows arbitrary finite intersections, where all the type members must have the
% same kind. On the language side, modern object-oriented languages such as Scala, TypeScript,
% Flow, Ceylon, and Grace have adopted some form of intersection types. Notably,
% the DOT calculus~\cite{amin2012dependent,Rompf_2016}---a new type-theoretic
% foundation for Scala---has a native support for intersection types. Generally
% speaking, the most significant difference between our calculi and those
% languages/calculi is that they do not have an explicit introduction form of
% intersection types, like our merge operator. The lack of a native merge operator
% leads to some awkward and type-unsafe solutions for defining a merge operator in
% those languages. As noted by \cite{alpuimdisjoint}, one important use of
% intersection types in TypeScript is the following function:
% \begin{lstlisting}[language=JavaScript]
% function extend<T, U>(first: T, second : U) : T & U {...}
% \end{lstlisting}
% which is analogous to our merge operator in that it takes two objects and
% produces an object with the intersection of the types of the argument objects.
% The implementation of \lstinline{extend} relies on low-level (and type-unsafe)
% features of JavaScript. Similar encodings have also been proposed for Scala to
% enable applications where the merge operator plays a fundamental
% role~\cite{oliveira2013feature, rendel14attributes}. As we have shown in
% \cref{sec:poly:motivation}, with disjointness constraints and a built-in merge
% operator, a type-safe and conflict-free \lstinline{extend} function can be
% naturally defined.



\paragraph{Row polymorphism and bounded polymorphism.}

Row polymorphism was originally proposed by Wand~\cite{wand1987complete}
as a mechanism to enable type inference for a simple object-oriented language
based on recursive records. These ideas were later adopted into type systems for
extensible records~\cite{Harper:1991:RCB:99583.99603, leijen2005extensible, gaster1996polymorphic}.
% Cardelli and
% Mitchell~\cite{cardelli1989operations} define three primitive operations on
% records: \emph{selection}, \emph{restriction} and \emph{extension}.
Our merge operator can be seen as a generalization of record
extension/concatenation, and selection is also built-in. In contrast
to most record calculi, restriction is not a primitive operation in \fnamee,
but can be simulated via subtyping. Disjoint quantification can 
simulate the \emph{lacks} predicate often present in systems
with row polymorphism.
Recently Morris and McKinna presented a
typed language~\cite{morrisrow}, generalizing and abstracting existing systems
of row types and row polymorphism.
Alpuim et al.~\cite{alpuimdisjoint}
informally studied the relationship between row polymorphism and disjoint
polymorphism, but it would be interesting to study such relationship more
formally. The work of  Morris and McKinna may be interesting for such
study in that it gives a general framework for row type systems.

Bounded quantification is currently the dominant mechanism in major mainstream object-oriented languages 
supporting both subtyping and polymorphism.
% , and it features in essentially all
% major mainstream OO languages.
\fsub~\cite{cardelli1985understanding} provides a
simple model for bounded quantification, but type-checking in full \fsub is
proved to be undecidable~\cite{pierce1994bounded}.
Pierce's thesis~\cite{pierce1991programming} discussed the relationship between calculi
with simple polymorphism and intersection types and bounded quantification. He
observed that there is a way to ``encode'' many forms of bounded quantification
in a system with intersections and pure (unbounded) second-order
polymorphism. That encoding can be easily adapted to \fnamee:
{\small
\[
[[ \/ X <: A . B ]] \defeq [[ \ X ** Top . (B [ X ~> A & X]) ]]
\]
}%
The idea is to replace bounded quantification by (unrestricted)
universal quantification and all occurrences of $[[X]]$
by $[[A & X]]$ in the body. Such an encoding seems to indicate
that \fnamee could be used as a decidable alternative to (full) \fsub. 
It is worthwhile to note that this encoding does not
work in \fname because $[[A & X]]$ is not well-formed ($[[X]]$ is not
disjoint to $[[A]]$). In other words, the encoding requires
unrestricted intersections.


% Local Variables:
% org-ref-default-bibliography: "../paper.bib"
% TeX-master: "../paper"
% End:
