\renewcommand{\rulehl}[2][gray!40]{%
  \colorbox{#1}{$\displaystyle#2$}}


\section{Row Polymorphism versus Disjoint Polymorphism}

In this section we show how to systematically translate \rname~\cite{Harper:1991:RCB:99583.99603}, a polymorphic
record calculus (with moderate changes)
into \fnamee. The translation itself is interesting in two regards: first, it
shows that disjoint polymorphism can model some form of row polymorphism;
second, it also reveals a significant difference of expressiveness between
disjoint polymorphism and row polymorphism---in particular, we point out that
row polymorphism alone is impossible to encode nested composition, which is
crucial for applications of extensible designs. In what follows, we first review \rname, and
discuss two examples that have no direct counterparts in \fnamee. By a careful comparison of
the two calculi, we identify a small change, when added to
\rname, leading to a type-directed translation to \fnamee. Finally we prove
in Coq that the translation is type safe, i.e., well-typed \rname terms map to
well-typed \fnamee terms. As a by-product, by mechanically formalizing \rname
in Coq, we identified one broken lemma of \rname due to the design of type
equivalence, which is remedied in our presentation.



\subsection{Syntax of \rname}

\begin{figure}[t]
  \centering
\begin{tabular}{llll} \toprule
  Types & $[[rt]]$ & $\Coloneqq$ & $[[base]] \mid [[rt1 -> rt2]] \mid [[\/ a # R .  rt]] \mid [[ r  ]]$\\
  Records & $[[r]]$ & $\Coloneqq$ & $[[a]] \mid [[Empty]] \mid [[ {l : rt}  ]]  \mid [[  r1 || r2 ]] $\\
  Constraint lists & $[[R]]$ & $\Coloneqq$ &  $[[ <>  ]] \mid [[ r , R ]] $ \\
  Terms & $[[re]]$ & $\Coloneqq$ & $[[x]] \mid [[\x : rt . re]] \mid [[re1 re2]] \mid [[rempty]] \mid [[{ l = re }]] \mid [[re1 || re2]] $ \\
          & & & $ [[ re \ l  ]]  \mid [[ re . l  ]] \mid [[ /\ a # R . re  ]] \mid [[  re [ r ]  ]]$ \\
  Type Contexts & $[[Ttx]]$ & $\Coloneqq$ &  $[[ <> ]] \mid [[Ttx , a # R ]] $ \\
  Term Contexts & $[[Gtx]]$ & $\Coloneqq$ &  $[[ <> ]] \mid [[Gtx , x : rt ]] $ \\ \bottomrule
\end{tabular}
  \caption{Syntax of \rname}
  \label{fig:syntax:record}
\end{figure}

To set the stage, we first briefly review the syntax and semantics of \rname.
\Cref{fig:syntax:record} presents its syntax.

\paragraph{Types}

Metavariable $[[rt]]$ ranges over types, which include primitive types
$[[base]]$, function types $[[rt1 -> rt2]]$, constrained quantification $[[ \/ a # R . rt ]]$
and record types $[[r]]$. The record types are built from record type
variables $[[a]]$, empty record $[[Empty]]$, single-field record types $[[ { l : rt}]]$
and record merges $[[ r1 || r2 ]]$.\footnote{The original \rname also include record
  type restrictions $[[r \ l]]$, which, as they later proved, can be systematically
  erased using type equivalence, thus we omit type-level restrictions but keep term-level restrictions.}
A constraint list $[[R]]$ is essentially a list of record types, which is used to constrain instantiations of type variables, and plays
an important role in the calculus, as we will explain shortly.

\paragraph{Terms} Metavariable $[[re]]$ ranges over terms, which include term
variables $[[x]]$, lambda abstractions $[[ \x : rt . re ]]$, function applications $[[re1 re2]]$, empty records $[[rempty]]$,
single-filed records $[[{l = re}]]$, record merges $[[re1 || re2]]$, record restrictions $[[ re \ l ]]$, record projections $[[ re . l  ]]$,
type abstractions $[[  /\ a # R . re ]]$ and type applications $[[ re [ r ]   ]]$.
As a side note, from the syntax of type applications $[[re [ r ] ]]$, it already can be seen that \rname only supports
quantification over \emph{record types}---though a separate form of quantifier that quantifies over \emph{all types} can be added, \cite{Harper:1991:RCB:99583.99603} decided to have only one form of quantifier for the sake of simplicity.

\paragraph{Examples}

Before proceeding to the formal semantics, let us first see some examples that
can be written in \rname, which may be of help in understanding the overall
system better. First of all, when it comes to extension, every record calculus
must decide what to do with duplicate labels. According to
\cite{leijen2005extensible}, record calculi can be divided into those that
support \emph{free} extension, and those that support \emph{strict} extension.
The former allows duplicate labels to coexist, whereas the latter does not. In
that sense, \rname belongs to the strict camp. What makes \rname stand out
from other strict record calculi is its ability to merge records with statically
unknown fields, and a mechanism to ensure the resulting record is conflict-free
(i.e., no duplicate labels). For example, the following function merges two
records:
\[
  \mathsf{mergeRcd} = [[  /\ a1 # Empty . /\ a2 # a1  . \ x1 : a1 . \ x2 : a2 . x1 || x2  ]]
\]
$\mathsf{mergeRcd}$ takes two type variables: the first one has no constraint
($[[Empty]]$) at all and the second one has only one constraint ($[[ a1 ]]$). It
may come as no surprise that $\mathsf{mergeRcd}$ can take any record type as its
first argument, but the second type must be \emph{compatible} with the first. In
other words, the second record cannot have any labels that already exist in the
first. These constraints are enough to ensure that the resulting record $[[x1 ||
x2]]$ has no duplicate labels. If later we want to say that the first record
$[[x1]]$ has \emph{at least} a field $[[l1]]$ of type $[[nat]]$, we can refine
the constraint list of $[[a1]]$ and the type of $[[x1]]$ accordingly:
\[
  [[  /\ a1 # {l1 : nat} . /\ a2 # a1  . \ x1 : a1 || {l1 : nat} . \ x2 : a2 . x1 || x2  ]]
\]
The above examples suggest an important point: the form of constraint used in
\rname can only express \emph{negative} information about record type variables.
Nonetheless, with the help of the merge operator, positive information can be
encoded as merges of record type variables, e.g., the assigned type of $[[x1]]$
illustrates that the missing field $[[ {l1 : nat} ]]$ is merged back into
$[[a1]]$.

The acute reader may have noticed some correspondences between \rname and
\fnamee: for instance, constrained type abstractions $[[ /\ a # R . re ]]$ look
quite alike disjoint type abstractions $[[ \ a ** A . ee ]]$; record merges
$[[x1 || x2]]$ is similar to \fnamee merges $[[ x1 ,, x2 ]]$ (at least
syntactically, although we have yet touched its semantics). Indeed, the very
function can be written in \fnamee almost verbatim (suppose we extend \fnamee
with annotated lambda abstractions):
\[
  \mathsf{mergeAny} = [[\ a1 ** Top . \ a2 ** a1 . \x1 : a1 . \x2 : a2 . x1 ,, x2 ]]
\]
However, as the name suggests, $\mathsf{mergeAny}$ works for \emph{any} two types,
not just record types.

\subsection{Typing Rules of \rname}
\label{sec:typing_rname}

\jeremy{We may only want to show selected rules. }

The type system of \rname consists of several conventional judgments.
\Cref{fig:rname_well_formed} presents the well-formedness rules for types and
contexts. Most of the rules are quite standard, the only rule worth mentioning
is \rref*{wfr-Merge}: a merge $[[r1 || r2]]$ is well-formed in $[[Ttx]]$ if
$[[r1]]$ and $[[r2]]$ are well-formed in $[[Ttx]]$, and moreover, $[[r1]]$ and
$[[r2]]$ are compatible in $[[Ttx]]$---the most important judgment in \rname,
and also its key contribution, as we will explain below.

\begin{figure}[t]
  \centering
\drules[wftc]{$[[  Ttx ok ]]$}{Well-formed type contexts}{Empty, Tvar}

\drules[wfc]{$[[  Ttx |- Gtx ok ]]$}{Well-formed term contexts}{Empty, Var}

\drules[wfrt]{$[[  Ttx |- rt type ]]$}{Well-formed types}{Prim, Arrow, All, Rec}

\drules[wfr]{$[[  Ttx |- r record ]]$}{Well-formed record types}{Var, Base, Empty, Merge}

\drules[wfcl]{$[[  Ttx |- R ok ]]$}{Well-formed constraint lists}{Nil, Cons}
  \caption{Well-formedness rules}
  \label{fig:rname_well_formed}
\end{figure}



\paragraph{Compatibility}

The compatibility relation plays a central role in \rname. It is the underlying
mechanism of deciding when merging two records is ``sensible''. Informally,
$[[Ttx |- r1 # r2]]$ holds if $[[r1]]$ and $[[r2]]$ are mergeable, that is,
$[[r1]]$ lacks every field contained in $[[r2]]$ and vice versa. The
compatibility rules are shown in \cref{fig:compatible}. Compatibility is
symmetric (\rref{cmp-Symm}) and respects type equivalence (\rref{cmp-Eq}).
\Rref{cmp-Base} says that if a record is compatible with a single-field record
$[[{l : t}]]$, then it is compatible with every other single-field record $[[{l : t'}]]$,
as long as they have the same label. A type variable is compatible
with the records in its constraint list (\rref{cmp-Tvar}). Two single-field
records are compatible if they have different labels (\rref{cmp-BaseBase}). The
rest of the rules are self-explanatory, and we refer the reader to their paper
for further explanations. The judgment of constraint list satisfaction $[[Ttx |- r # R]]$
ensures that $[[r]]$ must be compatible with every record in the constraint list $[[R]]$.
With the compatibility rules, let us go back to the definition of $\mathsf{mergeRcd}$
and see why it can type check, i.e.,  why $[[a1]]$ and $[[a2]]$ are compatible---because
$[[a1]]$ is in the constraint list of $[[a2]]$, and by \rref{cmp-Tvar}, they are compatible.


\begin{figure}[t]
  \centering
\drules[cmp]{$[[  Ttx |- r1 # r2 ]]$}{Compatibility}{Eq, Symm, Base, Tvar, MergeEOne,MergeETwo,MergeI,BaseBase,Empty}

\drules[cmpList]{$[[  Ttx |- r # R ]]$}{Constraint list satisfaction}{Nil, Cons}

\caption{Compatibility}
\label{fig:compatible}

\end{figure}

\paragraph{Type Equivalence}

Unlike \fnamee, \rname does not have subtyping. Instead, \rname uses type
equivalence to convert terms of one type to another. A selection of the rules
defining equivalence of types and constraint lists appears in
\cref{fig:type_equivalence}. The relation $[[Ttx |- rt1 ~ rt2]]$ is an
equivalence relation (\rref{teq-Refl,teq-Symm,teq-Trans}), and is a congruence
with respect to the type constructors. Finally merge is associative
(\rref{teq-MergeAssoc}), commutative (\rref{teq-MergeComm}), and has $[[Empty]]$
as its unit (\rref{teq-MergeUnit}). As a consequence, records are identified up
to permutations. Since the quantifier in \rname is constrained, the equivalence
of constrained quantification (\rref{teq-CongAll}) relies on the equivalence of
constraint lists $[[Ttx |- R1 ~ R2]]$. Again, it is an equivalence relation
(\rref{ceq-Refl,ceq-Symm,ceq-Trans}), and respects type equivalence
(\rref{ceq-Inner}). Constraint lists are essentially finite sets, so order and
multiplicity of constraints are irrelevant (\rref{ceq-Swap,ceq-Dupl}). Merges of
constraints can be ``flattened'' (\rref{ceq-Merge}), and occurences of
$[[Empty]]$ may be eliminated (\rref{ceq-Empty}). The last rule \rref*{ceq-Base}
is quite interesting: it says that the types of single-field records are ignored
when comparing two constraint lists. The reason is that when it comes to
compatibility of records (c.f. \rref{cmp-BaseBase}), only labels matter.
Therefore changing the types of two records will not affect their compatibility
relation. We will have more to say about this in \cref{sec:trouble}. In
particular, this is the ``gap'' between the two calculi, as we cannot find a
corresponding rule in \fnamee.

\begin{figure}[t]
  \centering
\drules[teq]{$[[  Ttx |- rt1 ~ rt2 ]]$}{Type equivalence}{Refl,Symm,Trans,CongArrow,CongAll,CongBase,CongMerge,MergeUnit,MergeAssoc,MergeComm}

\drules[ceq]{$[[  Ttx |- R1 ~ R2 ]]$}{Constraint list equivalence}{Refl,Symm,Trans,Inner,Swap,Empty,Merge,Dupl,Base}

\caption{Type equivalence}
\label{fig:type_equivalence}
\end{figure}

\begin{remark}
  The original rules of type equivalence~\cite{Harper:1991:RCB:99583.99603} do
  not have contexts (i.e.,  judgment of the form $[[rt1 ~ rt2]]$). However this is incorrect, as it invalidates one of the key
  lemmas (Lemma 2.3.1.7) in their type system, which says that
    if $[[Ttx |- r1 # r2]]$, then $[[Ttx |- r1 record]]$ and $[[Ttx |- r2 record]]$.
  Consider two types $[[  {l1 : nat}  ]]$ and $[[ {l2 : \/ a # {l : nat} || {l : bool} . nat  }   ]]$.
  According to the original rules, they are compatible because
  \begin{inparaenum}[(1)]
  \item $[[  {l1 : nat} ]]  $ is compatible with $ [[ {l2 : \/ a # {l : nat} , {l : bool} . nat }  ]]$ by \rref{cmp-BaseBase};
  \item $ [[ {l2 : \/ a # {l : nat} , {l : bool} . nat }  ~ {l2 : \/ a # {l : nat} || {l : bool} . nat } ]]$.
  \end{inparaenum}
  Then it follows that $[[ {l2 : \/ a # {l : nat} || {l : bool} . nat } ]]$ is well-formed.
  However, this record type is not well-formed in any context because $[[{l : nat} || {l : bool}]]$
  is not well-formed in any context. The culprit is \rref{ceq-Merge}---the well-formedness of $[[ r1 , (r2, R) ]]$
  does not necessarily entail the well-formedness of $[[ (r1 || r2) ,R]]$, as
  the latter also requires the compatibility of $[[r1]]$ and $[[r2]]$.
  That is why we need to explicitly add contexts to type equivalence
  so that $ [[ {l2 : \/ a # {l : nat} , {l : bool} . nat } ]] $ and $[[ {l2 : \/ a # {l : nat} || {l : bool} . nat } ]]$
  are not considered equivalent in the first place.
\end{remark}


\paragraph{Typing rules}

Lastly, the typing rules of \rname are shown in \cref{fig:typing_rname}. At
first reading, the gray parts can be ignored, which will be covered in
\cref{sec:row_trans}. Most of the typing rules are quite standard. Typing is
invariant under type equivalence (\rref{wtt-Eq}). Two terms can be merged if
their types are compatible (\rref{wtt-Merge}). The type application $[[ re [ r ]  ]]$ is well-typed
if $[[re]]$ has type $[[  \/ a # R . rt   ]]$ and $[[r]]$ satisfies the constraints
associated with the quantifier (\rref{wtt-AllE}).


\begin{remark}
  A few simplifications have been placed compared to the original \rname,
  notably the typing of record selection (\rref{wtt-Select}) and restriction
  (\rref{wtt-Restr}). In the original formulation, both typing rules need a
  partial function $ r \_ l $ which means the type associated with label $[[l]]$
  in $[[r]]$. Instead of using partial functions, here we explicitly expose the
  expected label in a record. It can be shown that if label $[[l]]$ is present
  in record type $[[r]]$, then the fields in $[[r]]$ can be rearranged so that
  $[[l]]$ comes first by type equivalence. This formulation turns out to be more
  amenable to Coq formalization and was actually adopted by
  \cite{leijen2005extensible}.
\end{remark}


\begin{figure}[t]
  \centering
\drules[wtt]{$[[  Ttx ; Gtx |- re : rt ~~> ee  ]]$}{Type-directed translation}{Eq,Var,ArrowI,ArrowE,Base,Restr,Empty,Merge,Select,AllI,AllE}
\caption{Type-directed translation}
\label{fig:typing_rname}
\end{figure}



\renewcommand{\rulehl}[1]{#1}

\subsection{Two Troubling Examples}
\label{sec:trouble}

In \cref{sec:typing_rname} we alluded to the ``gap'' between \rname and \fnamee.
In this section, we will show two examples that are typeable in \rname, but have
no direct counterparts in \fnamee. More specifically, the first example,
although typable in \fnamee, has a type that does not directly correspond to its
\rname type. The second example, however, is entirely ill-typed in \fnamee. We
then pinpoint two rules that cause the discrepancy and propose small changes to
\rname. In \cref{sec:row_trans} we show a type-directed translation from \rname
to \fnamee and prove the translation is type safe.

Before presenting the examples, let us first sketch out the translation scheme.
On the syntactic level, it is pretty straightforward to see a one-to-one
correspondence between \rname terms and \fnamee expressions. For example,
constrained type abstractions $[[/\ a # R . re ]]$ correspond to \fnamee type
abstractions $[[ \ a ** A . ee]]$; record merges can be simulated by the more
general merge operator of \fnamee; record restriction can be modeled as annotate terms, and so on.
On the semantic level, all well-formedness judgments of \rname have corresponding well-formedness judgments
of \fnamee, given a proper translation from \rname types to \fnamee types
(denoted as $[[| rt |]]$, whose definition will be given shortly).
Compatibility relation corresponds to disjointness relation. What might not be
so obvious is that type equivalence can be expressed via subtyping. More
specifically, $[[ rt1 ~ rt2 ]]$ induces mutual subtyping relations
$[[ | rt1 | <: | rt2 | ]]$ and $[[ | rt2 | <: | rt1 | ]]$.
Informally, type-safety of translation is something along the lines of
``if a term has type $[[rt]]$, then its translation has type $[[| rt | ]]$''.
With all these in mind, let us take a close look at the two troubling examples.

\begin{example} \label{eg:1} %
  Consider term $[[ /\ a # {l : nat} . \x : a . x ]]$. It could be
  assigned type (among others) $[[ \/ a # {l : nat} . a -> a ]]$. This seems
  quite obvious, and its translation
  $[[  \ a ** {l : nat} . \ x : a . x  ]]$ has type $[[ \ a ** { l : nat} . a -> a   ]]$, which corresponds very well to
  $[[ \/ a # {l : nat} . a -> a  ]]$. The same term can also be assigned type $[[  \/ a # {l : bool} . a -> a   ]]$, since
  $[[  \/ a # {l : bool} . a -> a   ]]$ is equivalent to $[[  \/ a # {l : nat} . a -> a   ]]$ by \rref{teq-CongAll}. However,
  as far as \fnamee is concerned, these two types  have no relationship at all---$[[  \ a ** {l : nat} . \ x : a . x  ]]$
  does not have type $[[  \ a ** {l : bool} . a -> a   ]]$, and indeed it should not, as these two types have completely different meanings!
\end{example}

\begin{remark}
  Interestingly, the algorithmic system of \rname can only infer
  type $[[ \/ a # {l : nat} . a -> a ]]$ for the aforementioned term.
  To relate to the declarative system (in particular, to prove completeness of the algorithm),
  they show that the type inferred by the algorithm is equivalent
  (in the sense of type equivalence relation) to the assignable type in the declarative system.
  Proving type-safety of translation is, in a sense, like proving completeness. In the same vein,
  maybe we can change the type-safety statement to something like
  ``if a term has type $[[rt]]$, then there exists type $[[rt']]$ such that $[[ rt ~ rt' ]]$ and the
  translation has type $[[| rt' | ]]$''. As the second example shows, this is still incorrect.
\end{remark}

\begin{example} \label{eg:2} %
  Now let us consider term $$[[re]] = [[  /\ a # {l : bool} . \ x : a . \ y : {l : nat} . x || y  ]]$$
  It has type $[[ \/ a # {l : bool} . a -> {l : nat} -> a || {l : nat}    ]]$, and
  its translation is $$[[ee]] = [[ \ a ** {l : bool} . \x  : a . \ y : {l : nat} . x ,, y  ]]$$
  However, the expression $[[ee]]$ is ill-typed in \fnamee
  for the reasons of coherence. To see why, think about
  the result of evaluating $[[ (ee {l : nat} {l = 1} {l = 2}).l ]]$---it could evaluate to $1$ or $2$!
\end{example}

\begin{remark}
  It may be instructive to think about why \rname allows type-checking $[[re]]$.
  Unlike the case of \fnamee, the existence of $[[re]]$ in \rname will not cause
  incoherence because \rname would reject type application $[[re [{l : nat}] ]]$
  in the first place---more generally, $[[re]]$ can only be applied to records
  that do not contain label $[[l]]$ due to the stringency of the compatibility
  relation. This example underlines a crucial difference between the
  compatibility relation and the disjointness relation. The former can only
  relate records with different labels, whereas the latter is
  more fine-grained in the sense that it can also relate records with same labels, as long as
  their types are disjoint. To illustrate, we slightly change $[[re]]$ to
  \[
    [[re']] = [[  /\ a # {l : nat} . \ x : a . \ y : {l : nat} . x || y  ]]
  \]
  In fact, $[[re']]$ and $[[re]]$ are essentially the same program, as the
  constraint lists only care about labels. Now $[[re']]$ has a corresponding translation
  \[
    [[ee']] = [[ \ a ** {l : nat} . \x  : a . \ y : {l : nat} . x ,, y  ]]
  \]
  which is well-typed in \fnamee. For the same reason as $[[re]]$, \rname would
  reject type application $[[ re' [ {l : char} ] ]]$, whereas $[[
  ee' {l : char} ]]$ type checks in \fnamee. In that sense, $[[ee']]$ is more expressive than $[[re']]$.

\end{remark}


\paragraph{Taming \rname}

The above two examples seem to stop us from getting a translation from \rname to
\fnamee. A careful comparison between the two calculi reveals that we should
forbid \rname from changing types arbitrarily. In particular, we identify two
rules that need to be modified. The first is \rref{ceq-Base}:
\[
  \drule{ceq-Base}
\]
which is the cause of \cref{eg:1}. Note that \rref{ceq-Base} is reasonable
in the setting of \rname, as ``the relevant properties of a record, for the
purposes of consistency checking, are its atomic components (i.e., labels and
record type variables)''~\cite{Harper:1991:RCB:99583.99603}. That is why the
types of single-field records are ignored. However, in the setting of \fnamee,
we also care about the types of single-field records. For the purposes of
translation, we adapt \rref{ceq-Base} to the following:
\[
  \drule{ceq-BaseAlt}
\]
The second is \rref{cmp-Base}
\[
  \drule{cmp-Base}
\]
which is the cause of \cref{eg:2} and should be removed. Of course we should ask
what are the consequences of the changes. At first glance, there are fewer
well-typed programs under the new rule. For example, term $[[re]]$ from
\cref{eg:2} no longer type checks. However, term $[[e']]$ still type checks, and
as we mentioned, it is equivalent to $[[e]]$. We hypothesize that there is no
loss of expressivity. Another way to justify these changes is that, as
\cite{Harper:1991:RCB:99583.99603} pointed out, constraint lists may be
normalized into the form $[[l1]], \dots, [[ln]], [[a1]] , \dots, [[am]]$ where
the $[[l]]_i$'s are labels and the $[[a]]_i$'s are record type variables. When
viewed in this way, \rref{ceq-base,cmp-base} are not really needed, and indeed
these two rules do not appear in their algorithmic system.
In summary, the changes we add to \rname are:
\begin{inparaenum}[(1)]
\item replace \rref{ceq-Base} with \rref{ceq-BaseAlt};
\item remove \rref{cmp-Base}.
\end{inparaenum}


\subsection{Type-Directed Translation}
\label{sec:row_trans}

With the modified \rname, we are now ready to explain the gray parts in \cref{fig:typing_rname}. First we
show how to translate \rname types to \fnamee types in
\cref{fig:type_trans_rname}. Most of them are straightforward. Record merges are
translated into intersection types, so are the constraint lists. Next we look at the
translations of terms. Most of the them are quite intuitive. In \rref{wtt-eq},
we put annotation $[[ | rt' | ]]$ around the translation of $[[re]]$. Record
restrictions are translated to annotated terms (\rref{wtt-Restr}) since we
already know the type without label $[[l]]$. Record merges are translated to
general merges (\rref{wtt-Merge}). The translation of record selections (\rref{wtt-Select}) is  a bit
complicated. Note that we cannot simply translate to $[[ ee . l ]]$ because our
typing rule for record selections (\rref{T-proj}) only applies when $[[ee]]$ is a
single-field record. Instead, we need to first transform $[[ee]]$ to a
single-field record by annotation, and then project.

\begin{remark}
  The acute reader may have noticed that in \rref{wtt-AllE}, the translation type
  $| [[r]] |$ could be a quantifier, but our rule of type applications
  (\rref{T-tapp}) only applies to monotypes. The reason is that, for the
  purposes of translation, we lift the monotype restrictions, which does not
  compromise type-safety of \fnamee.
\end{remark}


\paragraph{Type-safety of Translation}

Finally, we prove that our translation in \cref{fig:typing_rname} is type-safe.
As we mentioned in \cref{sec:trouble}, the main idea is to map each judgment in
\rname to a corresponding judgment in \fnamee: well-formedness to
well-formedness, compatibility to disjointness, type-equivalence to subtyping.
The following lemmas are proved by straightforward induction.

\ningning{I moved lemmas to translation}

With all these lemmas ready, we can prove type-safety of translate as follows:

\begin{theorem}
  If $[[ Ttx ; Gtx |- re : rt ~~> ee ]]$ then $[[ < Ttx > ; < Gtx > |-  ee => < rt >  ]]$.
\end{theorem}
\begin{proof}
  By induction on the typing derivation.
\end{proof}




% Local Variables:
% TeX-master: "../paper"
% org-ref-default-bibliography: "../paper.bib"
% End:
