
\section{Conclusion and Future Work}
\label{sec:conclusion}

We have proposed \fnamee, a type-safe and coherent calculus with disjoint
intersection types, BCD subtyping and parametric polymorphism. \fnamee improves
the state-of-art of compositional designs, and enables the development of highly
modular and reusable programs. One interesting and useful further extension
would be implicit polymorphism. For that we want to combine
Dunfield and Krishnaswami's approach~\cite{dunfield2013complete} with our bidirectional type system.
We would also like to study the parametricity of \fnamee. As we have seen in
\cref{sec:failed:lr}, it is not at all obvious how to extend the standard
logical relation of System F to account for disjointness, and avoid potential
circularity due to impredicativity. A promising solution is to use step-indexed
logical relations~\cite{ahmed2006step}. 
% TOM: This sentence is broken. Do we even need it?
% We have yet investigated further on that direction.


\section*{Acknowledgments}

We thank the anonymous reviewers and Yaoda Zhou for their helpful comments.
This work has been sponsored by the Hong Kong Research Grant
Council projects number 17210617 and 17258816, and by the Research Foundation -
Flanders.



%%% Local Variables:
%%% mode: latex
%%% TeX-master: "../paper"
%%% org-ref-default-bibliography: "../paper.bib"
%%% End:
